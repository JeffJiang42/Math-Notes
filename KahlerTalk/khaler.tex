\documentclass[psamsfonts]{amsart}
%
%-------Packages---------
%
\usepackage[h margin=1 in, v margin=1 in]{geometry}
\usepackage{amssymb,amsfonts}
\usepackage[all,arc]{xy}
\usepackage{enumerate}
\usepackage{mathrsfs}
\usepackage{amsthm}
\usepackage{mathpazo}
\usepackage{yfonts}
\usepackage{enumitem}
\usepackage{mathrsfs}
\usepackage{fourier-orns}
\usepackage[all]{xy}
\usepackage{hyperref}
\usepackage{cite}
\usepackage{url}
\usepackage{mathtools}
\usepackage{graphicx}
\usepackage{pdfsync}
\usepackage{mathdots}
%
\usepackage{tgpagella}
\usepackage[T1]{fontenc}
%
\usepackage{listings}
\usepackage{color}

\definecolor{dkgreen}{rgb}{0,0.6,0}
\definecolor{gray}{rgb}{0.5,0.5,0.5}
\definecolor{mauve}{rgb}{0.58,0,0.82}

\lstset{frame=tb,
  language=Matlab,
  aboveskip=3mm,
  belowskip=3mm,
  showstringspaces=false,
  columns=flexible,
  basicstyle={\small\ttfamily},
  numbers=none,
  numberstyle=\tiny\color{gray},
  keywordstyle=\color{blue},
  commentstyle=\color{dkgreen},
  stringstyle=\color{mauve},
  breaklines=true,
  breakatwhitespace=true,
  tabsize=3
  }
%
%--------Theorem Environments--------
%
\newtheorem{thm}{Theorem}[section]
\newtheorem*{thm*}{Theorem}
\newtheorem{cor}[thm]{Corollary}
\newtheorem{prop}[thm]{Proposition}
\newtheorem{lem}[thm]{Lemma}
\newtheorem*{lem*}{Lemma}
\newtheorem{conj}[thm]{Conjecture}
\newtheorem{quest}[thm]{Question}
%
\theoremstyle{definition}
\newtheorem{defn}[thm]{Definition}
\newtheorem*{defn*}{Definition}
\newtheorem{defns}[thm]{Definitions}
\newtheorem{con}[thm]{Construction}
\newtheorem{exmp}[thm]{Example}
\newtheorem{exmps}[thm]{Examples}
\newtheorem{notn}[thm]{Notation}
\newtheorem{notns}[thm]{Notations}
\newtheorem{addm}[thm]{Addendum}
\newtheorem{exer}[thm]{Exercise}
%
\theoremstyle{remark}
\newtheorem{rem}[thm]{Remark}
\newtheorem*{claim}{Claim}
\newtheorem*{aside*}{Aside}
\newtheorem*{rem*}{Remark}
\newtheorem*{hint*}{Hint}
\newtheorem*{note}{Note}
\newtheorem{rems}[thm]{Remarks}
\newtheorem{warn}[thm]{Warning}
\newtheorem{sch}[thm]{Scholium}
%
%--------Macros--------
\renewcommand{\qedsymbol}{$\blacksquare$}
\renewcommand{\hom}{\mathsf{Hom}}
\renewcommand{\emptyset}{\varnothing}
\newcommand{\R}{\mathbb{R}}
\newcommand{\ib}[1]{\textbf{\textit{#1}}}
\newcommand{\Q}{\mathbb{Q}}
\newcommand{\Z}{\mathbb{Z}}
\newcommand{\N}{\mathbb{N}}
\newcommand{\C}{\mathbb{C}}
\newcommand{\A}{\mathbb{A}}
\newcommand{\F}{\mathbb{F}}
\newcommand{\M}{\mathcal{M}}
\renewcommand{\S}{\mathbb{S}}
\newcommand{\V}{\vec{v}}
\newcommand{\RP}{\mathbb{RP}}
\newcommand{\CP}{\mathbb{CP}}
\newcommand{\B}{\mathcal{B}}
\newcommand{\GL}{\mathsf{GL}}
\newcommand{\SL}{\mathsf{SL}}
\newcommand{\SP}{\mathsf{SP}}
\newcommand{\SO}{\mathsf{SO}}
\newcommand{\SU}{\mathsf{SU}}
\newcommand{\gl}{\mathfrak{gl}}
\newcommand{\g}{\mathfrak{g}}
\newcommand{\inv}{^{-1}}
\newcommand{\bra}[2]{ \left[ #1, #2 \right] }
\newcommand{\ind}{\lambda \in \Lambda}
\newcommand{\set}[1]{\left\lbrace #1 \right\rbrace}
\newcommand{\abs}[1]{\left\lvert#1\right\rvert}
\newcommand{\norm}[1]{\left\lVert#1\right\rVert}
\newcommand{\transv}{\mathrel{\text{\tpitchfork}}}
\newcommand{\enumbreak}{\ \\ \vspace{-\baselineskip}}
\let\oldexists\exists
\renewcommand\exists{\oldexists~}
\let\oldL\L
\renewcommand\L{\mathfrak{L}}
\makeatletter
\newcommand{\tpitchfork}{%
  \vbox{
    \baselineskip\z@skip
    \lineskip-.52ex
    \lineskiplimit\maxdimen
    \m@th
    \ialign{##\crcr\hidewidth\smash{$-$}\hidewidth\crcr$\pitchfork$\crcr}
  }%
}
\makeatother
\newcommand{\bd}{\partial}
\newcommand{\lang}{\begin{picture}(5,7)
\put(1.1,2.5){\rotatebox{45}{\line(1,0){6.0}}}
\put(1.1,2.5){\rotatebox{315}{\line(1,0){6.0}}}
\end{picture}}
\newcommand{\rang}{\begin{picture}(5,7)
\put(.1,2.5){\rotatebox{135}{\line(1,0){6.0}}}
\put(.1,2.5){\rotatebox{225}{\line(1,0){6.0}}}
\end{picture}}
\DeclareMathOperator{\id}{id}
\DeclareMathOperator{\im}{Im}
\DeclareMathOperator{\grap}{graph}
\DeclareMathOperator{\codim}{codim}
\DeclareMathOperator{\supp}{supp}
\DeclareMathOperator{\inter}{Int}
\DeclareMathOperator{\sign}{sign}
\DeclareMathOperator{\sgn}{sgn}
\DeclareMathOperator{\indx}{ind}
\DeclareMathOperator{\alt}{Alt}
\DeclareMathOperator{\Aut}{Aut}
\DeclareMathOperator{\trace}{trace}
\DeclareMathOperator{\ad}{ad}
\DeclareMathOperator{\End}{End}
\DeclareMathOperator{\Ad}{Ad}
\DeclareMathOperator{\Lie}{Lie}
\DeclareMathOperator{\spn}{span}
\DeclareMathOperator{\dv}{div}
\DeclareMathOperator{\grad}{grad}
\newcommand*\myhrulefill{%
   \leavevmode\leaders\hrule depth-2pt height 2.4pt\hfill\kern0pt}
\newcommand\niceending[1]{%
  \begin{center}%
    \LARGE \myhrulefill \hspace{0.2cm} #1 \hspace{0.2cm} \myhrulefill%
  \end{center}}
\newcommand*\sectionend{\niceending{\decofourleft\decofourright}}
\newcommand*\subsectionend{\niceending{\decosix}}
\def\upint{\mathchoice%
    {\mkern13mu\overline{\vphantom{\intop}\mkern7mu}\mkern-20mu}%
    {\mkern7mu\overline{\vphantom{\intop}\mkern7mu}\mkern-14mu}%
    {\mkern7mu\overline{\vphantom{\intop}\mkern7mu}\mkern-14mu}%
    {\mkern7mu\overline{\vphantom{\intop}\mkern7mu}\mkern-14mu}%
  \int}
\def\lowint{\mkern3mu\underline{\vphantom{\intop}\mkern7mu}\mkern-10mu\int}
%
%--------Hypersetup--------
%
\hypersetup{
    colorlinks,
    citecolor=black,
    filecolor=black,
    linkcolor=blue,
    urlcolor=black
}
%
%--------Solution--------
%
\newenvironment{solution}
  {\begin{proof}[Solution]}
  {\end{proof}}
%
%--------Graphics--------
%
%\graphicspath{ {images/} }

\begin{document}
\author{Jeffrey Jiang}
\title{Symplectic Geometry and Kh\"aler Manifolds: An Introduction}
%
\maketitle
%
\section*{Linear Algebra}
%
As typical when studying smooth manifolds, we first look at the linear algebra that will be non-linearized. To discuss Kh\"aler manifolds (and consequently, Kh\"aler vector spaces), we will need to assemble several mutually compatible structures -- an inner product, a symplectic form, and a complex structure. 
%
\begin{defn}
An \ib{inner product space} is a vector space $V$ equipped with a bilinear map $g: V\times V \to V$ that is symmetric and positive definite. In the case that $V$ is a complex vector space, we replace the bilinear condition with the \ib{sesquilinear} condition -- conjugate linear in the first term, and linear in the second term. In this case, $g$ is often referred to as a \ib{hermitian inner product}.
\end{defn}
You're probably familiar with these, so we won't delve into them.
%
\begin{defn}
A \ib{symplectic vector space} is a vector space $V$ equipped with a nondegenerate skew symmetric bilinear form $\omega : V \times V \to V$.
\end{defn}
%
Nondegeneracy here means that if for some $v$, $\omega(v,w) = 0$ for all $w \in V$, then $v = 0$.  In other words, the form $\omega$ gives an isomorphism $V \to V^*$ via the mapping
$$v \mapsto \omega(v,\cdot) $$
We note that the positive definite condition on an inner product $g$ implies nondegeneracy, so that also defines an analogous isomorphism. An equivalent formulation of nondegeneracy for $\omega$ is that the wedge product $\omega \wedge \ldots \wedge \omega$ with itself $n/2$ times gives a nonzero volume form for $V$.
\begin{exmp}
$\R^{2n}$ with coordinates $(x_i,y_i)$ comes with a symplectic structure given by
$$\tilde{\omega} = \sum_i dx^i \wedge dy^i  $$
which can be expressed in coordinates by 
$$\tilde{\omega}(v,w) = v^T \Omega ~w $$
where 
$$\Omega =  \begin{pmatrix}
0 & \id_{\R^n} \\
-\id_{\R^n} & 0
\end{pmatrix} $$
\end{exmp}
In some sense, this is the \emph{only} symplectic vector space, which should be made clear very soon.
\begin{thm}
Every symplectic vector space $(V,\omega)$ admits a \ib{symplectic basis} $\set{e_i, f_i}$ where
$$\omega(e_i,e_j) = 0 \qquad \omega(e_i,f_i) = \delta_{ij}$$
\end{thm}
%
\begin{cor}
Every symplectic vector space is \ib{symplectomorphic} to $(\R^{2n}, \tilde{\omega})$. That is, there exists a linear isomorphism $\varphi: (V,\omega) \to (\R^{2n},\tilde{\omega})$ where $\varphi^*\tilde{\omega} = \omega$.
\end{cor}
%
\begin{proof}
Let $\set{f_i,g_i}$ be a symplectic basis for $V$ on $\R^n$ and consider the map $\varphi: V \to \R^{2n}$ given by mapping $f_i \mapsto e_i$ and $g_i \mapsto e_{n + i}$
\end{proof}
%
\begin{defn}
A \ib{complex structure} on a vector space $V$ is an automorphism $J: V \to V$ such that $J^2 = -\id_V$
\end{defn}
%
Given a complex structure $J$ and an $\R$-vector space $V$, we can make $V$ a $\C$-vector space by defining the action of $i \in \C$ by $i \cdot v = Jv$. We can then extend this to arbitrary complex numbers $\alpha + \beta i$ by $(\alpha + \beta i)\cdot v = \alpha v + \beta Jv$.

If you know some linear algebra, you might know that there is another way to turn a $\R$-vector space into a complex one. 
\begin{defn}
Given an $\R$-vector space $V$, define its complexification $V_\C$ as $V \otimes_\R \C$.
\end{defn}
Note that adding a complex structure to a vector space is \emph{not} the same as complexifying it. If we find some $J \in GL(V)$ such that $J^2 = -\id_V$, we haven't changed the dimension of $V$, but complexifying $V$ doubles its dimension over $\R$. Despite this, the concepts are quite similar. If we have a vector space $V$ with complex structure $J$, then if we complexify $V$, then $J$ extends to a map $V_\C \to V_\C$, where $J(v + i w) = Jv + iJw$. This gives a decomposition
$$V_\C = V^+ \oplus V^- $$
where $V^+ = \set{v + iJv ~:~v \in V}$ and $V^- = \set{v - iJv ~:~ v \in V}$, and we get an isomorphism of $\C$-vector spaces $V \to V^\pm$ where $v \mapsto v \pm iJv$, using the fact that $J^2v = -v$.

Now that we have defined all of these, we would like to define when these structures are compatible, and see what results
%
\begin{defn}
A complex structure $J$ is \ib{compatible} with a symplectic form $\omega$ if $\omega(Jv,Jw) = \omega(v,w)$.
\end{defn}
%
An analogous definition is used for an inner product $g$.
%
\begin{thm}
Given a vector space $V$ with complex structure $J$, given a $J$-compatible inner product $g$, we obtain a symplectic form $\omega$ where $\omega(v,w) = g(Jv,w)$. Likewise, given a $J$-compatible symplectic form $\omega$, we obtain an inner product $g$ where $g(v,w) = \omega(Jv,w)$. Symplectic forms/inner products obtained in this manner are said to be \ib{compatible} with the other.
\end{thm}
%
\begin{defn}
A vector space $V$ is \ib{K\"ahler} if it has compatible structures $J,g,\omega$.
\end{defn}
%
One final way to obtain these compatible structures is to start with a Hermitian vector space $(V,h)$, and let $g(v,w) = \text{Re}~h(v,w)$ and $\omega(v,w) = - \text{Im}~h(v,w)$.
%
\setcounter{section}{1}
%
\section*{K\"ahler Manifolds}
%
With the linear algebra set up, we can move on to talking about the manifolds. We'll first translate the linear algebraic concepts to ones on a manifold.
%
\begin{defn}
A \ib{Riemannian metric} on a smooth manifold $M$ is a smoothly symmetric positive definite $2$-tensor field $g: M \to \mathcal{T}^2(M)$.
\end{defn}
Smooth here mean one of several equivalent things:
\begin{enumerate}
\item The component functions of the matrix $g_{ij} = \langle \partial_i, \partial_j \rangle$ are smooth in any chart.
\item $g$ is a smooth map from $M$ into the tensor bundle $\mathcal{T}^2(M)$.
\item For any vector fields $X,Y \in \mathcal{X}(M)$, the function given by $g(X,Y)$ is smooth.
\end{enumerate}
We note that a Riemannian metric on a manifold is analogous to an inner product on a vector space -- it a smooth assignment of an inner product to each tangent space $T_pM$. Also, we have that $g_p$ being an inner product on $T_pM$ implies that it is nondegenerate, so $g$ induces an isomorphism $TM \to T^*M$.
\begin{defn}
A \ib{symplectic forn} on a smooth manifold $M$ is a smooth nondegenerate closed skew-symmetric $2$-form $\omega: M \to \Lambda^2(M)$.
\end{defn}
Here closed means that the exterior derivative $d\omega = 0$. A symplectic form makes every tangent space $T_pM$ a symplectic vector space.
%
\begin{exmp}
GIven any manifold $M$, we can define a symplectic structure on its cotangent bundle $T^*M$ as follows. Define the \ib{tautological $1$-form} $\alpha \in \Omega^1(T^*M)$ by $\alpha_(p, \xi) = d\pi_p^*\xi$ where $\pi: T^*M \to M$ is the projection. Then the $2$-form $\omega = d\alpha$ is symplectic. If $(p^i)$ denote coordinates on $M$, and $(p^i,q^i)$ denote the induced coordinates given by the local trivialization for $T^*M$, we have that 
$$\omega = \sum_i dq^i \wedge dp^i $$
which looks exactly like the standard form!
\end{exmp}
%
As it turns out, every symplectic manifold has the same local structure, what this means is that every symplectic manifold $(M,\omega)$ is locally symplectomorphic to $\R^{2n}$ (now thought of as a manifold, rather than a vector space). The same does not hold for Riemannian manifolds. It is not the case that every Riemannian manifold is locally isometric to $\R^n$ with the standard Euclidean metric. 

Another difference is that every manifold admits a Riemannian metric. The same is not true in the symplectic case. 
\begin{prop}
The only sphere that admits a symplectic structure is $S^2$
\end{prop}
%
\begin{proof}
We can give $S^2$ a symplectic structure with a choice of volume form, which is closed and skew symmetric. Then let $n \in 2\Z$ with $n \neq 2$. We claim that there exists no symplectic structure on $S^n$. We know that $H^2_{dR}(M) = 0$, so every $2$-form is exact. This tells us that the $n/2$-fold wedge of $\omega$ with itself is also exact, so it is equal to $d\beta$ for some $\beta$. Then by Stokes' Theorem
$$\int_{S^n} \omega^{n/2} = \int_{S^n} d\beta = 0 $$
so $\omega^{n/2}$ is not a volume form.
\end{proof}
%
\begin{defn}
An \ib{almost complex structure} on a smooth manifold is smoothly varying $J$ where at each $p \in M$, $J_p^2: T_pM \to T_pM$ is equal to $\id_{T_pM}$
\end{defn}
Since the endomorphism is acting on a different vector space at each point, what does smooth mean? We have a canonical isomorphism $V^* \otimes V \to \End(V)$, so we identify $J$ as a smoothly varying $(1,1)$ tensor field, and ask that the map to the tensor bundle $J: M \to \mathcal{T}^1_1(M)$ is smooth.
%
\begin{defn}
A \ib{complex manifold} is a smooth manifold with an atlas of charts $\varphi: U \to \C^n$ where the transition maps $\varphi \circ \psi\inv : \psi(U) \cap \varphi(V) \to \C^n$ are holomorphic.
\end{defn}
%
As it turns out, an almost complex structure isn't quite enough for our needs -- it will need to satisfy some integrability condition. If you know about the Frobenius theorem regarding when vector fields determine a subbundle of the tangent bundle, this is the same concept.
%
\begin{defn}
A \ib{K\"ahler manifold} is a complex manifold $M$ with a Riemannian metric $g$ and symplectic form $\omega$ such that at each $T_pM$, $g_p$ and $\omega_p$ are compatible with the complex structure given by multiplication by $i$, i.e.
$$g_p(i\cdot v,w) = \omega(v,w) \qquad \omega_p(i\cdot v,w) = g(v,w) $$
\end{defn}
%
We end with some examples
\begin{prop}
Complex projective space $\CP^n$ is K\"ahler.
\end{prop}
%
\begin{proof}
To see this, we note that the unitary group $U(n+1)$ acts transitively on $\CP^n$ thought of as lines in $\C^{n+1}$, and $\CP^n$ will be both homogeneous and isotropic under this action. Using this, we can define a $U(n+1)$ invariant metric on $\CP^n$, from which we can recover a hermitian form on $\CP^n$, which we can then use the construct the symplectic form. The explicit computations get quite nasty though.
\end{proof}
%
\begin{prop}
Any torus $T^{2n} = \C^n / \Lambda^n$ (where $\Lambda^n \cong \Z^{2n}$ is a lattice) is K\"ahler.
\end{prop}
%
\begin{proof}
The standard hermitian metric on $\C^n$ descends to a hermitian metric on $T^{2n}$. We can then use this to define the Riemannian metric and symplectic form.
\end{proof}
%
\end{document}