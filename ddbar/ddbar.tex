\documentclass[psamsfonts, 12pt]{amsart}
%
%-------Packages---------
%
\usepackage[h margin=1 in, v margin=1 in]{geometry}
\usepackage{amssymb,amsfonts}
\usepackage[all,arc]{xy}
\usepackage{tikz-cd}
\usepackage{enumerate}
\usepackage{mathrsfs}
\usepackage{amsthm}
\usepackage{mathpazo}
\usepackage{float}
\usepackage{spectralsequences}
%\usepackage{charter} %another font
%\usepackage{eulervm} %Vakil font
\usepackage{yfonts}
\usepackage{mathtools}
\usepackage{enumitem}
\usepackage{mathrsfs}
\usepackage{fourier-orns}
\usepackage[all]{xy}
\usepackage{hyperref}
\usepackage{url}
\usepackage{mathtools}
\usepackage{graphicx}
\usepackage{pdfsync}
\usepackage{mathdots}
\usepackage{calligra}
\usepackage{import}
\usepackage{xifthen}
\usepackage{pdfpages}
\usepackage{transparent}

\newcommand{\incfig}[2]{%
    \fontsize{48pt}{50pt}\selectfont
    \def\svgwidth{\columnwidth}
    \scalebox{#2}{\input{#1.pdf_tex}}
}
%
\usepackage{tgpagella}
\usepackage[T1]{fontenc}
%
\usepackage{listings}
\usepackage{color}

\definecolor{dkgreen}{rgb}{0,0.6,0}
\definecolor{gray}{rgb}{0.5,0.5,0.5}
\definecolor{mauve}{rgb}{0.58,0,0.82}

\lstset{frame=tb,
  language=Matlab,
  aboveskip=3mm,
  belowskip=3mm,
  showstringspaces=false,
  columns=flexible,
  basicstyle={\small\ttfamily},
  numbers=none,
  numberstyle=\tiny\color{gray},
  keywordstyle=\color{blue},
  commentstyle=\color{dkgreen},
  stringstyle=\color{mauve},
  breaklines=true,
  breakatwhitespace=true,
  tabsize=3
  }
%
%--------Theorem Environments--------
%
\newtheorem{thm}{Theorem}[section]
\newtheorem*{thm*}{Theorem}
\newtheorem{cor}[thm]{Corollary}
\newtheorem{prop}[thm]{Proposition}
\newtheorem{lem}[thm]{Lemma}
\newtheorem*{lem*}{Lemma}
\newtheorem{conj}[thm]{Conjecture}
\newtheorem{quest}[thm]{Question}
%
\theoremstyle{definition}
\newtheorem{defn}[thm]{Definition}
\newtheorem*{defn*}{Definition}
\newtheorem{defns}[thm]{Definitions}
\newtheorem{con}[thm]{Construction}
\newtheorem{exmp}[thm]{Example}
\newtheorem{exmps}[thm]{Examples}
\newtheorem{notn}[thm]{Notation}
\newtheorem{notns}[thm]{Notations}
\newtheorem{addm}[thm]{Addendum}
\newtheorem{exer}[thm]{Exercise}
%
\theoremstyle{remark}
\newtheorem{rem}[thm]{Remark}
\newtheorem*{claim}{Claim}
\newtheorem*{aside*}{Aside}
\newtheorem*{rem*}{Remark}
\newtheorem*{hint*}{Hint}
\newtheorem*{note}{Note}
\newtheorem{rems}[thm]{Remarks}
\newtheorem{warn}[thm]{Warning}
\newtheorem{sch}[thm]{Scholium}
%
%--------Macros--------
\renewcommand{\qedsymbol}{$\blacksquare$}
\renewcommand{\sl}{\mathfrak{sl}}
\newcommand{\Bord}{\mathsf{Bord}}
\renewcommand{\hom}{\mathsf{Hom}}
\renewcommand{\emptyset}{\varnothing}
\renewcommand{\O}{\mathscr{O}}
\newcommand{\R}{\mathbb{R}}
\newcommand{\ib}[1]{\textbf{\textit{#1}}}
\newcommand{\Q}{\mathbb{Q}}
\newcommand{\Z}{\mathbb{Z}}
\newcommand{\N}{\mathbb{N}}
\newcommand{\C}{\mathbb{C}}
\newcommand{\A}{\mathbb{A}}
\newcommand{\F}{\mathbb{F}}
\newcommand{\M}{\mathcal{M}}
\newcommand{\dbar}{\overline{\partial}}
\newcommand{\zbar}{\overline{z}}
\renewcommand{\S}{\mathbb{S}}
\renewcommand{\P}{\mathbb{P}}
\newcommand{\V}{\vec{v}}
\newcommand{\RP}{\mathbb{RP}}
\newcommand{\CP}{\mathbb{CP}}
\newcommand{\B}{\mathcal{B}}
\newcommand{\GL}{\mathrm{GL}}
\newcommand{\PGL}{\mathrm{PGL}}
\newcommand{\SL}{\mathrm{SL}}
\newcommand{\PSL}{\mathrm{PSL}}
\newcommand{\SP}{\mathrm{SP}}
\newcommand{\SO}{\mathrm{SO}}
\newcommand{\SU}{\mathrm{SU}}
\newcommand{\gl}{\mathfrak{gl}}
\newcommand{\g}{\mathfrak{g}}
\newcommand{\Bun}{\mathsf{Bun}}
\newcommand{\inv}{^{-1}}
\newcommand{\bra}[2]{ \left[ #1, #2 \right] }
\newcommand{\set}[1]{\left\lbrace #1 \right\rbrace}
\newcommand{\abs}[1]{\left\lvert#1\right\rvert}
\newcommand{\norm}[1]{\left\lVert#1\right\rVert}
\newcommand{\transv}{\mathrel{\text{\tpitchfork}}}
\newcommand{\defeq}{\vcentcolon=}
\newcommand{\enumbreak}{\ \\ \vspace{-\baselineskip}}
\let\oldexists\exists
\renewcommand\exists{\oldexists~}
\let\oldL\L
\renewcommand\L{\mathfrak{L}}
\makeatletter
\newcommand{\tpitchfork}{%
  \vbox{
    \baselineskip\z@skip
    \lineskip-.52ex
    \lineskiplimit\maxdimen
    \m@th
    \ialign{##\crcr\hidewidth\smash{$-$}\hidewidth\crcr$\pitchfork$\crcr}
  }%
}
\makeatother
\newcommand{\bd}{\partial}
\newcommand{\lang}{\begin{picture}(5,7)
\put(1.1,2.5){\rotatebox{45}{\line(1,0){6.0}}}
\put(1.1,2.5){\rotatebox{315}{\line(1,0){6.0}}}
\end{picture}}
\newcommand{\rang}{\begin{picture}(5,7)
\put(.1,2.5){\rotatebox{135}{\line(1,0){6.0}}}
\put(.1,2.5){\rotatebox{225}{\line(1,0){6.0}}}
\end{picture}}
\DeclareMathOperator{\id}{id}
\DeclareMathOperator{\im}{Im}
\DeclareMathOperator{\codim}{codim}
\DeclareMathOperator{\coker}{coker}
\DeclareMathOperator{\supp}{supp}
\DeclareMathOperator{\inter}{Int}
\DeclareMathOperator{\sign}{sign}
\DeclareMathOperator{\Stab}{Stab}
\DeclareMathOperator{\sgn}{sgn}
\DeclareMathOperator{\indx}{ind}
\DeclareMathOperator{\alt}{Alt}
\DeclareMathOperator{\Aut}{Aut}
\DeclareMathOperator{\trace}{trace}
\DeclareMathOperator{\ad}{ad}
\DeclareMathOperator{\End}{End}
\DeclareMathOperator{\Ad}{Ad}
\DeclareMathOperator{\Lie}{Lie}
\DeclareMathOperator{\spn}{span}
\DeclareMathOperator{\dv}{div}
\DeclareMathOperator{\grad}{grad}
\DeclareMathOperator{\Sym}{Sym}
\DeclareMathOperator{\sheafhom}{\mathscr{H}\text{\kern -3pt {\calligra\large om}}\,}
\newcommand*\myhrulefill{%
   \leavevmode\leaders\hrule depth-2pt height 2.4pt\hfill\kern0pt}
\newcommand\niceending[1]{%
  \begin{center}%
    \LARGE \myhrulefill \hspace{0.2cm} #1 \hspace{0.2cm} \myhrulefill%
  \end{center}}
\newcommand*\sectionend{\niceending{\decofourleft\decofourright}}
\newcommand*\subsectionend{\niceending{\decosix}}
\def\upint{\mathchoice%
    {\mkern13mu\overline{\vphantom{\intop}\mkern7mu}\mkern-20mu}%
    {\mkern7mu\overline{\vphantom{\intop}\mkern7mu}\mkern-14mu}%
    {\mkern7mu\overline{\vphantom{\intop}\mkern7mu}\mkern-14mu}%
    {\mkern7mu\overline{\vphantom{\intop}\mkern7mu}\mkern-14mu}%
  \int}
\def\lowint{\mkern3mu\underline{\vphantom{\intop}\mkern7mu}\mkern-10mu\int}
%
%--------Hypersetup--------
%
\hypersetup{
    colorlinks,
    citecolor=black,
    filecolor=black,
    linkcolor=blue,
    urlcolor=blacksquare
}
%
%--------Solution--------
%
\newenvironment{solution}
  {\begin{proof}[Solution]}
  {\end{proof}}
%
%--------Graphics--------
%
%\graphicspath{ {images/} }

\begin{document}
%
\author{Jeffrey Jiang}
%
\title{The Fr\"olicher Spectral Sequence and the $\partial\dbar$ Lemma}
%
\setcounter{section}{1}
%
\maketitle
%
For any complex manifold $X$, we have the Fr\"olicher spectral sequence, which
computes the de Rham cohomology of a complex manifold in terms of the
$\partial$ and $\dbar$ cohomology. On the $E_0$ page, it is given
by the Dolbeault cohomology of the holomorphic vector bundles $\Omega^{p,0}$, i.e.
the item in the $(p,q)$ position is the space $\mathcal{A}^{p,q}$ of smooth $(p,q)$
forms, and the differential on $E_0$ is just $\dbar$. For example, a small section of
the $E_0$ page would be: \\
%
\begin{sseqdata}[name=Frolicher0, classes = {draw = none}, yscale=2, xscale=2]
\class["\mathcal{A}^{0,0}"](0,0)
\class["\mathcal{A}^{1,0}"](1,0)
\class["\mathcal{A}^{2,0}"](2,0)
\class["\mathcal{A}^{3,0}"](3,0)
\class["\mathcal{A}^{0,1}"](0,1)
%
\class["\mathcal{A}^{1,1}"](1,1)
\class["\mathcal{A}^{2,1}"](2,1)
\class["\mathcal{A}^{3,1}"](3,1)
\d["\dbar"']0(0,0)(0,1)
\d["\dbar"']0(1,0)(1,1)
\d["\dbar"']0(2,0)(2,1)
\d["\dbar"']0(3,0)(3,1)
%
\class["\mathcal{A}^{0,2}"](0,2)
\class["\mathcal{A}^{1,2}"](1,2)
\class["\mathcal{A}^{2,2}"](2,2)
\class["\mathcal{A}^{3,2}"](3,2)
\d["\dbar"']0(0,1)(0,2)
\d["\dbar"']0(1,1)(1,2)
\d["\dbar"']0(2,1)(2,2)
\d["\dbar"']0(3,1)(3,2)
\end{sseqdata}
\begin{center}
\printpage[name=Frolicher0, page = 0]
\end{center}
%
Then in the $E_1$ page, we have $E_1^{p,q}$ is the cohomology in the $(p,q)$ slot
in the $E_0$ page, which is just $H^{p,q}(X)$ (abbreviated to $H^{p,q}$) by Hodge
theory in the compact case. The differentials going from left to right are the
operators $\partial$, which descends to cohomology because
$\partial\dbar = -\dbar\partial$. A small section of the $E_1$ page would be : \\
%
\begin{sseqdata}[name=Frolicher1, classes = {draw = none}, yscale=2, xscale=2]
\class["H^{0,0}"](0,0)
\class["H^{1,0}"](1,0)
\class["H^{2,0}"](2,0)
\class["H^{3,0}"](3,0)
\class["H^{0,1}"](0,1)
%
\class["H^{1,1}"](1,1)
\class["H^{2,1}"](2,1)
\class["H^{3,1}"](3,1)
\d["\partial"']1(0,0)(1,0)
\d["\partial"']1(1,0)(2,0)
\d["\partial"']1(2,0)(3,0)
%
\class["H^{0,2}"](0,2)
\class["H^{1,2}"](1,2)
\class["H^{2,2}"](2,2)
\class["H^{3,2}"](3,2)
\d["\partial"']1(0,1)(1,1)
\d["\partial"']1(1,1)(2,1)
\d["\partial"']1(2,1)(3,1)
%
\d["\partial"']1(0,2)(1,2)
\d["\partial"']1(1,2)(2,2)
\d["\partial"']1(2,2)(3,2)
\end{sseqdata}
\begin{center}
\printpage[name=Frolicher1, page = 1]
\end{center}
%
For a general compact complex manifold $X$, this continues to the $E_2$ page,
where the differential ``rotates," and the $(p,q)$ slot is the cohomology in
the $(p,q)$ slot of the $E_1$ page. The big theorem we want to prove is:
%
\begin{thm}
For a compact K\"ahler manifold $X$, the Fr\"olicher spectral sequence degenerates at
the $E_1$ page, i.e. all the differentials are $0$.
\end{thm}
%
In other words, we can terminate our spectral sequence computations at $E_1$.
Going to the $E_1$ page is easy, since all the computations are done with the
operators $\partial$ and $\dbar$. In practice, continuing on to further pages
is difficult. The entire spectral sequence story seems very difficult (and it is),
but in the compact K\"ahler story, it reduces to a simple lemma.
%
\begin{thm}[\ib{The $\partial\dbar$ lemma}]
Let $X$ be a K\"ahler manifold, and $\eta$ a complex $k$-form that is $\partial$
and $\dbar$-closed. Then if $\eta$ is $d$, $\partial$, or $\dbar$-exact, there exists
a form $\xi$ such that $\eta = \partial\dbar\xi$.
\end{thm}
%
The proof of this lemma requires the following results from Hodge Theory:
%
\begin{thm}[\ib{Comparison of the Laplacians}]
Let $X$ be a compact K\"ahler manifold. Then
\[
\Delta = 2\Delta_\partial = 2\Delta_{\dbar}
\]
where $\Delta$, $\Delta_\partial$, and $\Delta_{\dbar}$ are the Laplacians
\begin{align*}
\Delta &= dd^* + d^*d \\
\Delta_\partial &= \partial\partial^* + \partial^*\partial \\
\Delta_{\dbar} &= \dbar\dbar^* + \dbar^*\dbar
\end{align*}
\end{thm}
%
The proof of this theorem requires certain commutation relations to hold, called
the \ib{K\"ahler identities}. This identity is not true in general for an
arbitrary compact complex manifold. The other result we need is
%
\begin{thm}[\ib{The Hodge Decomposition}]
Any complex valued form $\alpha \in \Omega^{p,q}$ can be written as
\[
\alpha = \beta + \Delta\gamma
\]
where $\beta$ is harmonic, i.e. $\Delta\beta = 0$.
\end{thm}
%
This theorem is true for general compact complex manifolds, not just K\"ahler manifolds.
%
\begin{proof}[Proof of the $\partial\dbar$ lemma]
The proof for all three cases is much the same, so we do just the one where
$\eta = \dbar\alpha$ is $\dbar$-exact. By the Hodge decomposition, we write
$\alpha = \beta + \Delta\gamma$, with $\beta$ harmonic. Since
$\Delta = 2\Delta_{\dbar}$, and $\beta$ is $\Delta_{\dbar}$-harmonic if and only
if $\dbar\beta = \dbar^*\beta = 0$, we have that $\dbar\beta = 0$. We then compute
\begin{align*}
\eta &= \dbar\alpha \\
&= \dbar(\beta + \Delta\gamma) \\
&= \dbar\beta + 2\dbar(\Delta_\partial\gamma) \\
&= 0 + 2\dbar(\partial\partial^*\gamma + \partial^*\partial\gamma) \\
&= 2\dbar\partial\partial^*\gamma - 2\partial^*\dbar\partial\gamma \\
&= -2\partial\dbar\partial^*\gamma - 2\partial^*\dbar\partial\gamma
\end{align*}
Then since $\eta$ is $\partial$-closed, we have that $\partial^*\dbar\partial\gamma$
must also be $\partial$-closed. By orthogonality of the image of $\partial^*$ with
the kernel of $\partial$, we have that $\partial^*\dbar\partial\gamma = 0$,
so $\eta = -2\partial\dbar\partial^*\gamma = 2\dbar\partial\partial^*\gamma$,
so letting $\xi = \partial^*\gamma$, we are done.
\end{proof}
%
We now use this to prove Theorem 1.1.
%
\begin{proof}[Proof of 1.1]
We want to show that all the differentials on the $E_1$ page are $0$, i.e.
for a cohomology class $[\alpha] \in H^{p,q}$, $[\partial\alpha] = 0$. Since
$[\alpha]$ is a Dolbeault cohomology class, we know that $\alpha$ is $\dbar$-closed.
Therefore, $\partial\alpha$ is both $\dbar$ and $\partial$ closed, since $\partial$
and $\dbar$ anticommute. Then by the $\partial\dbar$ lemma, we have that
$\partial\alpha = \partial\dbar\eta$ for some $\eta$. Therefore, using the
fact that $\partial$ and $\dbar$ anticommute one final time, we find that
$\partial\alpha$ is $\dbar$-exact, i.e. $[\partial\alpha] = 0$
\end{proof}
%
The idea of the Fr\"olicher spectral sequence is that we can compute the
cohomology of $d = \partial + \dbar$ in terms of the cohomology of $\partial$
and $\dbar$ themselves. However, for general complex manifolds, this is hard,
since the differentials and cohomology beyond the $E_1$ page become much
more complicated than the ones on the $E_0$ and $E_1$ pages, which are all
objects we are familiar with, and are relatively easy to compute with. What
Theorem 1.1 tells us is that in the case that $X$ is compact K\"ahler, this
is enough.
%
\end{document}
