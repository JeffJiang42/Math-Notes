\documentclass[psamsfonts, 12pt]{amsart}
%
%-------Packages---------
%
\usepackage[h margin=1 in, v margin=1 in]{geometry}
\usepackage{amssymb,amsfonts}
\usepackage{amsmath}
\usepackage{accents}
\usepackage[all,arc]{xy}
\usepackage{tikz-cd}
\usepackage{enumerate}
\usepackage{mathrsfs}
\usepackage{amsthm}
\usepackage{mathpazo}
\usepackage{float}
%\usepackage[backend=biber]{biblatex}
%\addbibresource{bibliography.bib}
%\usepackage{charter} %another font
%\usepackage{eulervm} %Vakil font
\usepackage{yfonts}
\usepackage{mathtools}
\usepackage{enumitem}
\usepackage{mathrsfs}
\usepackage{fourier-orns}
\usepackage[all]{xy}
\usepackage{hyperref}
\usepackage{url}
\usepackage{mathtools}
\usepackage{graphicx}
\usepackage{pdfsync}
\usepackage{mathdots}
\usepackage{calligra}
\usepackage{import}
\usepackage{xifthen}
\usepackage{pdfpages}
\usepackage{transparent}

\usepackage{tgpagella}
\usepackage[T1]{fontenc}
%
\usepackage{listings}
\usepackage{color}

\definecolor{dkgreen}{rgb}{0,0.6,0}
\definecolor{gray}{rgb}{0.5,0.5,0.5}
\definecolor{mauve}{rgb}{0.58,0,0.82}

\lstset{frame=tb,
  language=Matlab,
  aboveskip=3mm,
  belowskip=3mm,
  showstringspaces=false,
  columns=flexible,
  basicstyle={\small\ttfamily},
  numbers=none,
  numberstyle=\tiny\color{gray},
  keywordstyle=\color{blue},
  commentstyle=\color{dkgreen},
  stringstyle=\color{mauve},
  breaklines=true,
  breakatwhitespace=true,
  tabsize=3
  }
%
%--------Theorem Environments--------
%
\newtheorem{thm}{Theorem}[section]
\newtheorem*{thm*}{Theorem}
\newtheorem{cor}[thm]{Corollary}
\newtheorem{prop}[thm]{Proposition}
\newtheorem{lem}[thm]{Lemma}
\newtheorem*{lem*}{Lemma}
\newtheorem{conj}[thm]{Conjecture}
\newtheorem{quest}[thm]{Question}
%
\theoremstyle{definition}
\newtheorem{defn}[thm]{Definition}
\newtheorem*{defn*}{Definition}
\newtheorem{defns}[thm]{Definitions}
\newtheorem{con}[thm]{Construction}
\newtheorem{exmp}[thm]{Example}
\newtheorem{exmps}[thm]{Examples}
\newtheorem{notn}[thm]{Notation}
\newtheorem{notns}[thm]{Notations}
\newtheorem{addm}[thm]{Addendum}
\newtheorem{exer}[thm]{Exercise}
%
\theoremstyle{remark}
\newtheorem{rem}[thm]{Remark}
\newtheorem*{claim}{Claim}
\newtheorem*{aside*}{Aside}
\newtheorem*{rem*}{Remark}
\newtheorem*{hint*}{Hint}
\newtheorem*{note}{Note}
\newtheorem{rems}[thm]{Remarks}
\newtheorem{warn}[thm]{Warning}
\newtheorem{sch}[thm]{Scholium}
%
%--------Macros--------
\renewcommand{\qedsymbol}{$\blacksquare$}
\renewcommand{\sl}{\mathfrak{sl}}
\newcommand{\Bord}{\mathsf{Bord}}
\renewcommand{\hom}{\mathrm{Hom}}
\renewcommand{\emptyset}{\varnothing}
\renewcommand{\O}{\mathcal{O}}
\newcommand{\R}{\mathbb{R}}
\newcommand{\ib}[1]{\textbf{\textit{#1}}}
\newcommand{\Q}{\mathbb{Q}}
\newcommand{\Z}{\mathbb{Z}}
\newcommand{\N}{\mathbb{N}}
\newcommand{\C}{\mathbb{C}}
\newcommand{\A}{\mathbb{A}}
\newcommand{\F}{\mathbb{F}}
\newcommand{\M}{\mathcal{M}}
\newcommand{\dbar}{\overline{\partial}}
\newcommand{\zbar}{\overline{z}}
\renewcommand{\S}{\mathbb{S}}
\newcommand{\V}{\vec{v}}
\newcommand{\RP}{\mathbb{RP}}
\newcommand{\CP}{\mathbb{CP}}
\newcommand{\B}{\mathcal{B}}
\newcommand{\GL}{\mathrm{GL}}
\newcommand{\SL}{\mathrm{SL}}
\newcommand{\SP}{\mathrm{SP}}
\newcommand{\SO}{\mathrm{SO}}
\newcommand{\SU}{\mathrm{SU}}
\newcommand{\gl}{\mathfrak{gl}}
\newcommand{\g}{\mathfrak{g}}
\newcommand{\Bun}{\mathsf{Bun}}
\newcommand*{\dt}[1]{%
   \accentset{\mbox{\large\bfseries .}}{#1}}
\newcommand{\inv}{^{-1}}
\newcommand{\bra}[2]{ \left[ #1, #2 \right] }
\newcommand{\set}[1]{\left\lbrace #1 \right\rbrace}
\newcommand{\abs}[1]{\left\lvert#1\right\rvert}
\newcommand{\norm}[1]{\left\lVert#1\right\rVert}
\newcommand{\transv}{\mathrel{\text{\tpitchfork}}}
\newcommand{\defeq}{\vcentcolon=}
\newcommand{\enumbreak}{\ \\ \vspace{-\baselineskip}}
\let\oldexists\exists
\renewcommand\exists{\oldexists~}
\let\oldL\L
\renewcommand\L{\mathfrak{L}}
\makeatletter
\newcommand{\incfig}[2]{%
    \fontsize{48pt}{50pt}\selectfont
    \def\svgwidth{\columnwidth}
    \scalebox{#2}{\input{#1.pdf_tex}}
}
%
\newcommand{\tpitchfork}{%
  \vbox{
    \baselineskip\z@skip
    \lineskip-.52ex
    \lineskiplimit\maxdimen
    \m@th
    \ialign{##\crcr\hidewidth\smash{$-$}\hidewidth\crcr$\pitchfork$\crcr}
  }%
}
\makeatother
\newcommand{\bd}{\partial}
\newcommand{\lang}{\begin{picture}(5,7)
\put(1.1,2.5){\rotatebox{45}{\line(1,0){6.0}}}
\put(1.1,2.5){\rotatebox{315}{\line(1,0){6.0}}}
\end{picture}}
\newcommand{\rang}{\begin{picture}(5,7)
\put(.1,2.5){\rotatebox{135}{\line(1,0){6.0}}}
\put(.1,2.5){\rotatebox{225}{\line(1,0){6.0}}}
\end{picture}}
\DeclareMathOperator{\id}{id}
\DeclareMathOperator{\im}{Im}
\DeclareMathOperator{\codim}{codim}
\DeclareMathOperator{\coker}{coker}
\DeclareMathOperator{\supp}{supp}
\DeclareMathOperator{\inter}{Int}
\DeclareMathOperator{\sign}{sign}
\DeclareMathOperator{\sgn}{sgn}
\DeclareMathOperator{\indx}{ind}
\DeclareMathOperator{\alt}{Alt}
\DeclareMathOperator{\Aut}{Aut}
\DeclareMathOperator{\trace}{trace}
\DeclareMathOperator{\ad}{ad}
\DeclareMathOperator{\End}{End}
\DeclareMathOperator{\Ad}{Ad}
\DeclareMathOperator{\Lie}{Lie}
\DeclareMathOperator{\spn}{span}
\DeclareMathOperator{\dv}{div}
\DeclareMathOperator{\grad}{grad}
\DeclareMathOperator{\Sym}{Sym}
\DeclareMathOperator{\tr}{tr}
\DeclareMathOperator{\sheafhom}{\mathscr{H}\text{\kern -3pt {\calligra\large om}}\,}
\newcommand*\myhrulefill{%
   \leavevmode\leaders\hrule depth-2pt height 2.4pt\hfill\kern0pt}
\newcommand\niceending[1]{%
  \begin{center}%
    \LARGE \myhrulefill \hspace{0.2cm} #1 \hspace{0.2cm} \myhrulefill%
  \end{center}}
\newcommand*\sectionend{\niceending{\decofourleft\decofourright}}
\newcommand*\subsectionend{\niceending{\decosix}}
\def\upint{\mathchoice%
    {\mkern13mu\overline{\vphantom{\intop}\mkern7mu}\mkern-20mu}%
    {\mkern7mu\overline{\vphantom{\intop}\mkern7mu}\mkern-14mu}%
    {\mkern7mu\overline{\vphantom{\intop}\mkern7mu}\mkern-14mu}%
    {\mkern7mu\overline{\vphantom{\intop}\mkern7mu}\mkern-14mu}%
  \int}
\def\lowint{\mkern3mu\underline{\vphantom{\intop}\mkern7mu}\mkern-10mu\int}
%
%--------Hypersetup--------
%
\hypersetup{
    colorlinks,
    citecolor=black,
    filecolor=black,
    linkcolor=blue
}
%
%--------Solution--------
%
\newenvironment{solution}
  {\begin{proof}[Solution]}
  {\end{proof}}
%
%--------Graphics--------
%
%\graphicspath{ {images/} }
%
\begin{document}
%
\author{Jeffrey Jiang}
%
\title{The Yoneda Lemma}
%
\maketitle
%
Let $\mathcal{C}$ be a category. Any object $X \in \mathcal{C}$ determines a
covariant functor $h_X : \mathcal{C} \to \mathsf{Set}$ where given an object
$Y \in \mathcal{C}$, we let $h_X(Y) = \mathrm{Map}_\mathcal{C}(X,Y)$. Given a morphism
$f : Y \to Z$, we let
$h_X(f):\mathrm{Map}_{\mathcal{C}}(X,Y)\to\mathrm{Map}_{\mathcal{C}}(X,Z)$ be defined
by $h_X(f)(\varphi) = f \circ \varphi$. This defines a contravariant functor
$h : \mathcal{C} \to \mathrm{Fun}(\mathcal{C}, \mathsf{Set})$ assigning
each object $X$ the functor $h_X$, and to each morphism $f : X \to Y$ a natural
transformation $h_Y \to h_X$ that maps a morphism
$\varphi \in \mathrm{Map}_{\mathcal{C}}(Y,Z)$ to
$\varphi \circ f \in \mathrm{Map}_{\mathcal{C}}(X,Z)$
for all objects $Z \in \mathcal{C}$.
%
\begin{thm}[\ib{The Yoneda Lemma}]
The functor $h : \mathcal{C} \to \mathrm{Fun}(\mathcal{C}, \mathsf{Set})$ is
fully faithful, i.e. for any objects $X,Y \in \mathcal{C}$, the map
$\mathrm{Map}_\mathcal{C}(X,Y)\to
\mathrm{Map}_{\mathrm{Fun}(\mathcal{C},\mathsf{Set})}(h_Y, h_X)$ is bijective.
\end{thm}
%
\begin{proof}
Let $X,Y \in \mathcal{C}$. Suppose two functions $f,g : X \to Y$ induce
the same natural transformation $h_Y \to h_X$. The mappings
$\mathrm{Map}_{\mathcal{C}}(Y,Y) \to \mathrm{Map}_{\mathcal{C}}(X,Y)$ given by
precomposition with $f$ and $g$ agree, so in particular, we must have that
$\id_Y \circ f = \id_Y \circ g$, so $f = g$. This shows that the map
$\mathrm{Map}_\mathcal{C}(X,Y)\to
\mathrm{Map}_{\mathrm{Fun}(\mathcal{C},\mathsf{Set})}(h_Y, h_X)$
is injective. For surjectivity, we want to show that any natural transformation
$\eta : h_Y \to h_X$ is given by precomposition by some morphism $X \to Y$. Let
$\eta_Y : \mathrm{Map}_{\mathcal{C}}(Y,Y) \to \mathrm{Map}_{\mathcal{C}}(X,Y)$
be the map given by $\eta$. Then we claim that $\eta_Y$ is given by precomposition
with the morphism $\eta_Y(\id_Y)$. Let $\varphi : X \to Y$ be a morphism.
Then $\eta$ being a natural transformations gives us the commutative
diagram
\[\begin{tikzcd}
\mathrm{Map}_{\mathcal{C}}(Y,Y) \ar[d,"\eta_Y"']
& \ar[l, "\varphi \circ (-)"'] \mathrm{Map}_{\mathcal{C}}(Y,Y) \ar[d, "\eta_Y"]\\
\mathrm{Map}_{\mathcal{C}}(X,Y)
& \mathrm{Map}_{\mathcal{C}}(X,Y) \ar[l,"\varphi \circ (-)"]
\end{tikzcd}\]
Starting with $\id_Y$ in the top right we get
\[
\eta_Y(\varphi \circ \id_Y) = \varphi \circ \eta_Y(\id_Y)
\]
which is the desired result. Then given any $Z \in \mathcal{C}$, we want to show
that $\eta_Z$ is also given by precomposition with $\eta_Y(\id_Y)$. Given
a morphism $\varphi : Y \to Z$, we again get the diagram
\[\begin{tikzcd}
\mathrm{Map}_{\mathcal{C}}(Y,Z) \ar[d,"\eta_Z"']
& \ar[l, "\varphi \circ (-)"'] \mathrm{Map}_{\mathcal{C}}(Y,Y) \ar[d, "\eta_Y"]\\
\mathrm{Map}_{\mathcal{C}}(X,Z)
& \mathrm{Map}_{\mathcal{C}}(X,Y) \ar[l,"\varphi \circ (-)"]
\end{tikzcd}\]
Then the fact that this commutes tells us that
\[
\eta_Z(\varphi \circ \id_Y) = \varphi \circ \eta_Y(\id_Y)
\]
which is the desired result. Therefore, the map $\mathrm{Map}_\mathcal{C}(X,Y)\to
\mathrm{Map}_{\mathrm{Fun}(\mathcal{C},\mathsf{Set})}(h_Y, h_X)$
is surjective
\end{proof}
%
\end{document}