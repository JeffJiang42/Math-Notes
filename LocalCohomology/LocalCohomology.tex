\documentclass[psamsfonts, 12pt]{amsart}
%
%-------Packages---------
%
\usepackage[h margin=1 in, v margin=1 in]{geometry}
\usepackage{amssymb,amsfonts}
\usepackage{amsmath}
\usepackage{accents}
\usepackage[all,arc]{xy}
\usepackage{tikz-cd}
\usepackage{enumerate}
\usepackage{mathrsfs}
\usepackage{amsthm}
\usepackage{mathpazo}
\usepackage{float}
%\usepackage[backend=biber]{biblatex}
%\addbibresource{bibliography.bib}
%\usepackage{charter} %another font
%\usepackage{eulervm} %Vakil font
\usepackage{yfonts}
\usepackage{mathtools}
\usepackage{enumitem}
\usepackage{mathrsfs}
\usepackage{fourier-orns}
\usepackage[all]{xy}
\usepackage{hyperref}
\usepackage{url}
\usepackage{mathtools}
\usepackage{graphicx}
\usepackage{pdfsync}
\usepackage{mathdots}
\usepackage{calligra}
\usepackage{import}
\usepackage{xifthen}
\usepackage{pdfpages}
\usepackage{transparent}

\usepackage{tgpagella}
\usepackage[T1]{fontenc}
%
\usepackage{listings}
\usepackage{color}

\definecolor{dkgreen}{rgb}{0,0.6,0}
\definecolor{gray}{rgb}{0.5,0.5,0.5}
\definecolor{mauve}{rgb}{0.58,0,0.82}

\lstset{frame=tb,
  language=Matlab,
  aboveskip=3mm,
  belowskip=3mm,
  showstringspaces=false,
  columns=flexible,
  basicstyle={\small\ttfamily},
  numbers=none,
  numberstyle=\tiny\color{gray},
  keywordstyle=\color{blue},
  commentstyle=\color{dkgreen},
  stringstyle=\color{mauve},
  breaklines=true,
  breakatwhitespace=true,
  tabsize=3
  }
%
%--------Theorem Environments--------
%
\newtheorem{thm}{Theorem}[section]
\newtheorem*{thm*}{Theorem}
\newtheorem{cor}[thm]{Corollary}
\newtheorem{prop}[thm]{Proposition}
\newtheorem{lem}[thm]{Lemma}
\newtheorem*{lem*}{Lemma}
\newtheorem{conj}[thm]{Conjecture}
\newtheorem{quest}[thm]{Question}
%
\theoremstyle{definition}
\newtheorem{defn}[thm]{Definition}
\newtheorem*{defn*}{Definition}
\newtheorem{defns}[thm]{Definitions}
\newtheorem{con}[thm]{Construction}
\newtheorem{exmp}[thm]{Example}
\newtheorem{exmps}[thm]{Examples}
\newtheorem{notn}[thm]{Notation}
\newtheorem{notns}[thm]{Notations}
\newtheorem{addm}[thm]{Addendum}
\newtheorem{exer}[thm]{Exercise}
%
\theoremstyle{remark}
\newtheorem{rem}[thm]{Remark}
\newtheorem*{claim}{Claim}
\newtheorem*{aside*}{Aside}
\newtheorem*{rem*}{Remark}
\newtheorem*{hint*}{Hint}
\newtheorem*{note}{Note}
\newtheorem{rems}[thm]{Remarks}
\newtheorem{warn}[thm]{Warning}
\newtheorem{sch}[thm]{Scholium}
%
%--------Macros--------
\renewcommand{\qedsymbol}{$\blacksquare$}
\renewcommand{\sl}{\mathfrak{sl}}
\newcommand{\Bord}{\mathsf{Bord}}
\renewcommand{\hom}{\mathsf{Hom}}
\renewcommand{\emptyset}{\varnothing}
\renewcommand{\O}{\mathcal{O}}
\newcommand{\R}{\mathbb{R}}
\newcommand{\ib}[1]{\textbf{\textit{#1}}}
\newcommand{\Q}{\mathbb{Q}}
\newcommand{\Z}{\mathbb{Z}}
\newcommand{\N}{\mathbb{N}}
\newcommand{\C}{\mathbb{C}}
\newcommand{\A}{\mathbb{A}}
\newcommand{\F}{\mathbb{F}}
\newcommand{\M}{\mathcal{M}}
\newcommand{\dbar}{\overline{\partial}}
\newcommand{\zbar}{\overline{z}}
\renewcommand{\S}{\mathbb{S}}
\newcommand{\V}{\vec{v}}
\newcommand{\RP}{\mathbb{RP}}
\newcommand{\CP}{\mathbb{CP}}
\newcommand{\B}{\mathcal{B}}
\newcommand{\GL}{\mathrm{GL}}
\newcommand{\SL}{\mathrm{SL}}
\newcommand{\SP}{\mathrm{SP}}
\newcommand{\SO}{\mathrm{SO}}
\newcommand{\SU}{\mathrm{SU}}
\newcommand{\gl}{\mathfrak{gl}}
\newcommand{\g}{\mathfrak{g}}
\newcommand{\Bun}{\mathsf{Bun}}
\newcommand*{\dt}[1]{%
   \accentset{\mbox{\large\bfseries .}}{#1}}
\newcommand{\inv}{^{-1}}
\newcommand{\bra}[2]{ \left[ #1, #2 \right] }
\newcommand{\set}[1]{\left\lbrace #1 \right\rbrace}
\newcommand{\abs}[1]{\left\lvert#1\right\rvert}
\newcommand{\norm}[1]{\left\lVert#1\right\rVert}
\newcommand{\transv}{\mathrel{\text{\tpitchfork}}}
\newcommand{\defeq}{\vcentcolon=}
\newcommand{\enumbreak}{\ \\ \vspace{-\baselineskip}}
\let\oldexists\exists
\renewcommand\exists{\oldexists~}
\let\oldL\L
\renewcommand\L{\mathfrak{L}}
\makeatletter
\newcommand{\incfig}[2]{%
    \fontsize{48pt}{50pt}\selectfont
    \def\svgwidth{\columnwidth}
    \scalebox{#2}{\input{#1.pdf_tex}}
}
%
\newcommand{\tpitchfork}{%
  \vbox{
    \baselineskip\z@skip
    \lineskip-.52ex
    \lineskiplimit\maxdimen
    \m@th
    \ialign{##\crcr\hidewidth\smash{$-$}\hidewidth\crcr$\pitchfork$\crcr}
  }%
}
\makeatother
\newcommand{\bd}{\partial}
\newcommand{\lang}{\begin{picture}(5,7)
\put(1.1,2.5){\rotatebox{45}{\line(1,0){6.0}}}
\put(1.1,2.5){\rotatebox{315}{\line(1,0){6.0}}}
\end{picture}}
\newcommand{\rang}{\begin{picture}(5,7)
\put(.1,2.5){\rotatebox{135}{\line(1,0){6.0}}}
\put(.1,2.5){\rotatebox{225}{\line(1,0){6.0}}}
\end{picture}}
\DeclareMathOperator{\id}{id}
\DeclareMathOperator{\im}{Im}
\DeclareMathOperator{\codim}{codim}
\DeclareMathOperator{\coker}{coker}
\DeclareMathOperator{\supp}{supp}
\DeclareMathOperator{\inter}{Int}
\DeclareMathOperator{\sign}{sign}
\DeclareMathOperator{\sgn}{sgn}
\DeclareMathOperator{\indx}{ind}
\DeclareMathOperator{\alt}{Alt}
\DeclareMathOperator{\Aut}{Aut}
\DeclareMathOperator{\trace}{trace}
\DeclareMathOperator{\ad}{ad}
\DeclareMathOperator{\End}{End}
\DeclareMathOperator{\Ad}{Ad}
\DeclareMathOperator{\Lie}{Lie}
\DeclareMathOperator{\spn}{span}
\DeclareMathOperator{\dv}{div}
\DeclareMathOperator{\grad}{grad}
\DeclareMathOperator{\Sym}{Sym}
\DeclareMathOperator{\Spec}{Spec}
\DeclareMathOperator{\tr}{tr}
\DeclareMathOperator{\sheafhom}{\mathscr{H}\text{\kern -3pt {\calligra\large om}}\,}
\newcommand*\myhrulefill{%
   \leavevmode\leaders\hrule depth-2pt height 2.4pt\hfill\kern0pt}
\newcommand\niceending[1]{%
  \begin{center}%
    \LARGE \myhrulefill \hspace{0.2cm} #1 \hspace{0.2cm} \myhrulefill%
  \end{center}}
\newcommand*\sectionend{\niceending{\decofourleft\decofourright}}
\newcommand*\subsectionend{\niceending{\decosix}}
\def\upint{\mathchoice%
    {\mkern13mu\overline{\vphantom{\intop}\mkern7mu}\mkern-20mu}%
    {\mkern7mu\overline{\vphantom{\intop}\mkern7mu}\mkern-14mu}%
    {\mkern7mu\overline{\vphantom{\intop}\mkern7mu}\mkern-14mu}%
    {\mkern7mu\overline{\vphantom{\intop}\mkern7mu}\mkern-14mu}%
  \int}
\def\lowint{\mkern3mu\underline{\vphantom{\intop}\mkern7mu}\mkern-10mu\int}
%
%--------Hypersetup--------
%
\hypersetup{
    colorlinks,
    citecolor=black,
    filecolor=black,
    linkcolor=blue,
    urlcolor=blacksquare
}
%
%--------Solution--------
%
\newenvironment{solution}
  {\begin{proof}[Solution]}
  {\end{proof}}
%
%--------Graphics--------
%
%\graphicspath{ {images/} }
%
\begin{document}
%
\author{Jeffrey Jiang}
%
\title{Local Cohomology}
%
\maketitle
%
\setcounter{section}{1}
%
\begin{defn}
An $R$-module $I$ is \ib{injective} if for every injection $M \hookrightarrow N$,
any homomorphism $M \to I$ extends to a homomorphism $N \to I$.
\[\begin{tikzcd}
M \ar[dr]\ar[rr, hookrightarrow] && N \ar[dl, dashed] \\
& I
\end{tikzcd}\]
Equivalently, the functor $\hom_R(\cdot,I)$ is exact.
\end{defn}
%
Over a PID, injective modules have a relatively nice characterization.
%
\begin{prop}
Let $R$ be a PID and $M$ an $R$-module. Then $M$ is injective if and only if it
is \ib{divisible}, i.e. given a non-zero divisor (nzd) $x$ and $u \in M$, there exists
$v \in M$ such that $xv = u$.
\end{prop}
%
\begin{cor} \enumbreak
\begin{enumerate}
  \item $\Q$ is an injective $\Z$-module.
  \item Let $k$ be a field. Then $k(x)$ is an injective $k[x]$-module.
\end{enumerate}
\end{cor}
%
\begin{defn}
Let $I \subset R$ be an ideal, and $M$ an $R$ module. The \ib{sections of $M$
supported on $V(I)$} is the submodule
\[
\Gamma_I(M) \defeq \set{m \in M ~:~ I^nm = 0 \text{ for some $n$}}
\]
\end{defn}
%
Recall that for a ring $R$, the sets $V(I) = \set{P \in \Spec R ~:~ I \subset P}$
are the closed sets in the Zariski topology on $R$, and an $R$-module can be thought
of as a sheaf over $\Spec R$ (e.g. sections of a vector bundle). In this perspective,
the elements of $M$ are the global sections, and the elements of $\Gamma_I(M)$ tell
us the elements supported on the closed set $V(I)$. Using $\Gamma_I$, we can define a
functor from $\mathsf{Mod}_R \to \mathsf{Mod}_R$ where $M\mapsto \Gamma_I(M)$, and
given $\varphi : M \to N$, $\Gamma_I(\varphi)$ is just $\varphi$ restricted to the
submodule $\Gamma_I(M)$. From this it is clear that $\Gamma_I$ is left-exact, since
it clearly preserves injective maps.
%
\begin{defn}
The \ib{local cohomology modules} $H^i_I(M)$ are obtained from the right derived functors
of $\Gamma_I$, i.e. $H^i_I(M) = R^i\Gamma_I(M)$.
\end{defn}
%
The idea here is that $H^i_I(M)$ should be like the relative cohomology with
respect closed set $V(I)$, we'll explore this analogy more later. \\

The modules $H^i_I(M)$ can be computed by taking an injective resolution
\[\begin{tikzcd}
0 \ar[r] & M \ar[r] & I^1 \ar[r] & \cdots
\end{tikzcd}\]
and taking the cohomology of the complex
\[\begin{tikzcd}
0 \ar[r] & \Gamma_I(M) \ar[r] & \Gamma_I(I^1) \ar[r] & \cdots
\end{tikzcd}\]
%
As with other derived functors, we will always have $\Gamma_I(M) = H^0_I(M)$.
%
\begin{exmp}
Let $I = (p) \subset \Z$. Then we have the injective resolution
\[\begin{tikzcd}
0 \ar[r] & \Z \ar[r] & \Q \ar[r] & \Q/\Z \ar[r] & 0
\end{tikzcd}\]
So we want to compute the cohomology of the sequence
\[\begin{tikzcd}
0 \ar[r] & \Gamma_{(p)}(\Z) \ar[r] & \Gamma_{(p)}(\Q) \ar[r] & \Gamma_{(p)}(\Q/\Z)
\ar[r] & 0
\end{tikzcd}\]
We note that $\Gamma_(p)$ simply picks out the submodule of elements of $p^n$ torsion.
Therefore, we have that $\Gamma_{(p)}(\Z) = H^0_{(p)}(\Z) = 0$. In addition, since
$\Q$ is torsion-free, we also have that $\Gamma_{(p)}(\Q) = 0$. Therefore, we
get that $H^1_{(p)}(\Z) = \Gamma_{(p)}(\Q/\Z)$. By factoring the numerator and
denominator of a rational number into products of prime powers, we get the identification
$H^1_{(p)}(\Z) = \Z[p\inv]/\Z$.
\end{exmp}
%
\begin{exmp}
Let $R = k[x]$ and $I = (x)$, where $k$ is a field. We have the injective resolution
\[\begin{tikzcd}
0 \ar[r] & R \ar[r] & k(x) \ar[r] & k(x)/R \ar[r] & 0
\end{tikzcd}\]
So we want to compute the cohomology of
\[\begin{tikzcd}
0 \ar[r] & \Gamma_{(x)}(R) \ar[r] & \Gamma_{(x)}(k(x)) \ar[r] & \Gamma_{(x)}(k(x)/R)
\ar[r] & 0
\end{tikzcd}\]
Since $R$ and $k(x)$ are both domains, $\Gamma_{(x)}(R) = \Gamma_{(x)}(k(x)) = 0$,
so we have that $H^0_{(x)}(R) = 0$ and $H^1_{(x)}(R) = \Gamma_{(x)}(k(x)/R)$.
Which is similar to the case with $\Q/\Z$. Factoring the numerators and denominators,
into products of irreducibles, we get an identification $H^1_{(x)}(R) = k[x,x\inv]/k[x]$.
\end{exmp}
%
In general, injective modules are hard to write down, so we want an easier way to compute
local cohomology. In the case that $R$ is Noetherian, we can do this using Koszul
complexes. For $x \in R$, construct
the complex
\[\begin{tikzcd}
K^\bullet(x,R) \defeq ~ 0 \ar[r] & R \ar[r] & R_x \ar[r] & 0
\end{tikzcd}\]
where $R$ is in degree $0$, and the map $R \to R_x$ is the localization map. Given
multiple elements $x_1,\ldots x_n \in R$, we let
$K^\bullet(x_1,\ldots x_n, R) \defeq
K^\bullet(x_1,R) \otimes \cdots \otimes K^\bullet(x_n,R)$. We then define
$K^\bullet(x_1,\ldots x_n, M) \defeq K^\bullet(x_1,\ldots x_n, R) \otimes M$. We
denote the cohomology of this complex by $H^i(x_1,\ldots x_n, M)$.
%
\begin{prop}
Let $R$ be Noetherian, and $I \subset R$ an ideal with
$\sqrt{I} = \sqrt{(x_1,\ldots x_n)}$. Then
\[
H^i_I(M) = H^i(x_1,\ldots x_n, M)
\]
\end{prop}
%
This can be interpreted geometrically as follows. The complement of $V(I)$ is covered
by the open sets $D(x_i) = \set{P \in \Spec R ~:~ x_i \notin P}$, where $x \in I$. The
sections of $M$ over an open set $D(x)$ are given by the localized module
$M_x = R_x \otimes_R M$. Then the complex $K^\bullet(x_1,\ldots x_n, M)$ is
the \v{C}ech complex for the open covering $\set{D(x_i)}$. This shows that what we are
doing is computing the sheaf cohomology of $M$ restricted to the complement of $V(I)$.
In addition, this perspective sheds light on how $H^i_I(M)$ behaves like relative
cohomology. If we look at the first terms in the Koszul complex, we have
\[\begin{tikzcd}
0 \ar[r] & M \ar[r] &  \bigoplus\limits_{x \notin \sqrt{I}} M_x
\ar[r] & \bigoplus\limits_{x,y \notin \sqrt{I}} M_{x,y} \ar[r] & \cdots
\end{tikzcd}\]
the kernel of
$\bigoplus_{x \notin \sqrt{I}} R_x \to \bigoplus_{x,y \notin \sqrt{I}} R_{x,y}$
consists of tuples of elements of $M_x$ that map to the same element in $M_{x,y}$,
which should be thought of as sections agreeing on intersections. The image
of the first map should be thought of as sections that extend over $V(I)$ to a
global section. In this way, we see that this is analogous to relative cohomology
with respect to the closed subspace $V(I)$. \\

The Koszul complex perspective also allows us to prove results about base change.
%
\begin{prop}
Let $R$ be Noetherian, $I \subset R$ an ideal, and $M$ an $R$-module.
\begin{enumerate}
  \item Let $\varphi : R \to S$ be ring homomorphism such that $S$ is a flat $R$-module.
  Then
  \[
  H^i_I(M) \otimes_R S \cong H^i_{IS}(M \otimes_R S)
  \]
  In particular, taking local cohomology commutes with localization and completion.
  \item For any ring homomorphism $\varphi : R \to S$ and an $S$-module $N$,
  \[
  H^i_I(N) \cong H^i_{IS}(N)
  \]
\end{enumerate}
\end{prop}
%
\begin{proof}
The first claim is easy. Let $I = (x_1,\ldots x_n)$ We have that
\[
K^\bullet(x_1,\ldots x_n, M) \otimes_R S =
K^\bullet(\varphi(x_1),\ldots \varphi(x_n), M \otimes_R S)
\]
Then since $S$ is flat, tensoring with $S$ is exact, so it commutes with taking
cohomology. \\

The second claim is also easy. This follows from
\begin{align*}
K^\bullet(x_1,\ldots x_n, N) &= K^\bullet(x_1,\ldots x_n, R) \otimes_R N \\
&= K^\bullet(x_1,\ldots x_n,R) \otimes_R S \otimes_S N \\
&= K^\bullet(\varphi(x_1),\ldots \varphi(x_n), S) \otimes_R N \\
&= K^\bullet(\varphi(x_1),\ldots \varphi(x_n), N)
\end{align*}
The first equality comes from the definition of the Koszul complex. The second comes
from the  definition of the $R$-module structure, the third comes from the first part,
and the last comes from the the definition of the $R$-module structure
as well as the definition of the Koszul complex.
\end{proof}
%
For another geometric perspective, there's a ``Mayer-Vietoris"-like sequence in local
cohomology. To do this, we need another perspective for local cohomology over Noetherian
rings. Given a Noetherian ring $R$ and an $R$-module $M$, define the ideal quotients
\[
0:_M I^n \defeq \set{m \in M ~:~ I^nm =0}
\]
Then we have that $\Gamma_I$ is the direct limit of the $0:_M I^n$, which
it can be identified with the quotient
\[
\bigoplus_{n \in \Z^{\geq 0}} (0:_MI^n) / \sim
\]
where we identify $x \in (0 :_M I^n) \in \bigoplus_{n \in \Z^{\geq 0}} (0:_MI^n)$ with
its copies lying in $0 :_M I^m$ for $m < n$. Heuristically, it is best to think
of the limit as the ``union" $\cup (0 :_M I^n)$. Furthermore, we can make
the identification $0 :_M I^n \cong \hom_R(R/I^n, M)$, where we identify
$m \in 0:_M I^n$ with the homomorphism determined by $1 \mapsto m$. This gives us
\[
\Gamma_I(M) = \lim_{\longrightarrow}\hom_R(R/I^n,M)
\]
Then if we take an injective resolution
\[\begin{tikzcd}
0 \ar[r] & M \ar[r] & E^1 \ar[r] & \cdots
\end{tikzcd}\]
and apply $\Gamma_I$, we get
\[\begin{tikzcd}
0 \ar[r] & \lim\limits_{\longrightarrow}\hom(R/I^n, M) \ar[r] &
\lim\limits_{\longrightarrow}\hom(R/I^n, E^1) \ar[r] & \cdots
\end{tikzcd}\]
By abstract nonsense, we can commute the limit taking cohomology, giving us the
identification $H^i_I(M) = \lim\limits_{\longrightarrow}\mathrm{Ext}^i_R(R/I^n, M)$.
In fact, we see that we can do this for a slightly wider class of collections
of ideals. Suppose we had some collection $\set{J_n}$ that is \ib{cofinal}
with respect to the collection $\set{I^n}$, i.e. given $I^n$, it contains
some $J_m$. Then the limits $\lim_{\longrightarrow}R/I^n$ and
$\lim_{\longrightarrow}R/J_n$ coincide, so we can compute local cohomology with
the $J_n$ as well. \\

This interpretation of local cohomology gives us a ``Mayer-Vietoris" sequence.
%
\begin{thm}
Let $I,J \subset R$ be ideals in a Noetherian ring $R$. Then there is a long
exact sequence in local cohomology
\[\begin{tikzcd}
0 \ar[r] & H^0_{I+J}(M) \ar[r] & H^0_I(M) \oplus H^0_J(M) \ar[r] &
H^0_{I \cap J}(M) \ar[dll]\\
& H^1_{I+J}(M)\ar[r] & H^1_I(M) \oplus H^1_J(M) \ar[r] & H^1_{I \cap J}(M) \ar[dll]\\
& \ldots
\end{tikzcd}\]
\end{thm}
%
\begin{proof}
The ideals $\set{I^n + J^n}$ are cofinal with respect to $\set{(I+J)^n}$ and
the $\set{I^n \cap J^n}$ are cofinal with $\set{(I\cap J )^n}$. We have a short exact
sequence
\[\begin{tikzcd}
0 \ar[r] & R/(I^n \cap J^n) \ar[r] & R/I^n \oplus R/J^n \ar[r] & R/(I^n+J^n) \ar[r] & 0
\end{tikzcd}\]
Applying $\hom(\cdot, M)$, we get a long exact sequence
\[\begin{tikzcd}
0 \ar[r] & \hom(R/(I^n \cap J^n, M) \ar[r] & \hom(R/I^n, M) \hom(R/J^n, M) \ar[r] &
\hom(R/(I^n+J^n),M) \ar[dll] \\
& \mathrm{Ext^1_R}(R/(I^n \cap J^n), M) \ar[r] & \mathrm{Ext}_R^1(R/I^n, M)
\mathrm{Ext}^1_R(R/J^n, M) \ar[r] & \mathrm{Ext}_R^1(R/(I^n+J^n),M) \ar[dll] \\
& \ldots
\end{tikzcd}\]
taking the limit then completes the proof.
\end{proof}
%
Geometrically, recall
\begin{align*}
V(I + J) &= V(I) \cap V(J) \\
V(I \cap J) &= V(I) \cup V(J)
\end{align*}
So, just like Mayer-Vietoris for topological spaces, this allows us to compute
local cohomology supported in the union $V(I) \cup V(J)$ by knowing
the local cohomology on $V(I)$, $V(J)$, and the intersection. In particular,
if $I+J = R$, we have that $\Gamma_{I+J}(M) = \Gamma(M)$, so this allows us to express
the sheaf cohomology of $M$ in terms of local cohomology.
%
\begin{defn}
Let $M$ be an $R$-module. An \ib{associated prime} of $M$ is a prime ideal $P \subset R$
of the form $P = \mathrm{Ann}(m)$ for some $m \in M$.
\end{defn}
%
We can recover the old notion of associated prime for an ideal $I$ by taking the
associated primes of $R/I$.
%
\section{Matlis Duality}
%
\begin{defn}
Let $N$ be a submodule of $M$. $M$ is \ib{essential} over $N$ if every nonzero
submodule of $M$ intersects $N$ nontrivially.
\end{defn}
%
In some sense, injective modules are ``large." On the other hand, essential modules
are ``small" in the sense that they are close to a submodule.
%
\begin{defn}
A maximal essential extension of an $R$-module $M$ is the same as a minimal injective
module containing $M$. In fact, any essential extension that is injective is maximal.
Such a module is called the \ib{injective hull} of $M$, and we denote the injective hull
of an $R$-module $M$ by $E_R(M)$.
\end{defn}
%
\begin{thm}[\ib{Matlis}]
Let $R$ be Noetherian.
\begin{enumerate}
  \item An $R$-module $E$ is an indecomposable injective if and only if
  $E$ is of the injective hull $E_R(R/P)$ for some $P \in \Spec R$.
  \item Every finitely generated submodule of $E_R(R/P)$ has $P$ as the unique
  associated prime.
  \item Every injective module is a direct sum of indecomposable injective modules.
\end{enumerate}
\end{thm}
%
From now on, we will assume that $(R,m)$ is a Noetherian local ring. We let $k$
denote the residue field $R/m$, and we let $E$ denote the injective hull of
$k$, thought of as an $R$-module.
%
\begin{defn}
Let $M$ be an $R$-module. The \ib{Matlis dual} of $M$, denoted $M^\vee$, is the
module $M^\vee \defeq \hom_R(M, E)$.
\end{defn}
%
\begin{prop}
Let $M$ and $N$ be $R$ modules. Then
\[
\mathrm{Tor}^R_i(M,N)^\vee \cong \mathrm{Ext}_R^i(M,N^\vee)
\]
\end{prop}
%
\begin{proof}
Let
\[\begin{tikzcd}
0 \ar[r] & M \ar[r] & F^1 \ar[r] & \cdots
\end{tikzcd}\]
be a free resolution of $M$. Then $\mathrm{Tor}^R_i(M,N)$ is computed as
the homology of the complex
\[\begin{tikzcd}
0 \ar[r] & M \otimes_R N \ar[r] & F^1 \otimes_R N \ar[r] & \cdots
\end{tikzcd}\]
Taking Matlis duals of each term, which commutes with taking homology since $E$ is
injective, so $\mathrm{Tor}^R_i(M,N)^\vee$ is the homology of the complex
\[\begin{tikzcd}
0 \ar[r] & \hom_R(M \otimes_R N, E) \ar[r] & \hom_R(F^1\otimes_R N, E) \ar[r] & \cdots
\end{tikzcd}\]
Applying Tensor-Hom adjunction, this becomes
\[\begin{tikzcd}
0 \ar[r] & \hom_R(M, \hom_R(N,E)) \ar[r] & \hom_R(F^1, \hom_R(N,E)) \ar[r] & \cdots
\end{tikzcd}\]
And the homology of this sequence is exactly $\mathrm{Ext}^i_R(M,N^\vee)$.
\end{proof}
%
If we assume that $M$ is finitely generated, we get a similar dual statement where
\[
\mathrm{Ext}^i_R(M,N)^\vee \cong \mathrm{Tor}^R_i(M,N^\vee)
\]
%
\begin{thm}[\ib{Matlis duality}]
Let $\hat{R} = \lim\limits_{\longleftarrow} R/m^n$ be the completion of $R$ with
respect to the maximal ideal $m$. Then we have
\begin{enumerate}
  \item Any Artinian $R$-module is isomorphic to a submodule of $E^r$ for some $r$.
  \item Taking Matlis duals gives a bijective correspondence
  \[
  \set{\text{finitely generated } \hat{R}\text{-modules}} \longleftrightarrow
  \set{\text{Artinian } R\text{-modules}}
  \]
  In addition, $(N^\vee)^\vee  = N$.
\end{enumerate}
\end{thm}
%

%
\end{document}