\documentclass[psamsfonts]{amsart}
%
%-------Packages---------
%
\usepackage[h margin=1 in, v margin=1 in]{geometry}
\usepackage{amssymb,amsfonts}
\usepackage{rank-2-roots}
\usepackage[all,arc]{xy}
\usepackage{tikz-cd}
\usepackage{enumerate}
\usepackage{mathrsfs}
\usepackage{amsthm}
\usepackage{mathpazo}
\usepackage{yfonts}
\usepackage{enumitem}
\usepackage{mathrsfs}
\usepackage{fourier-orns}
\usepackage[all]{xy}
\usepackage{hyperref}
\usepackage{cite}
\usepackage{url}
\usepackage{mathtools}
\usepackage{graphicx}
\usepackage{pdfsync}
\usepackage{mathdots}
\usepackage{calligra}
%
\usepackage{tgpagella}
\usepackage[T1]{fontenc}
%
\usepackage{listings}
\usepackage{color}

\definecolor{dkgreen}{rgb}{0,0.6,0}
\definecolor{gray}{rgb}{0.5,0.5,0.5}
\definecolor{mauve}{rgb}{0.58,0,0.82}

\lstset{frame=tb,
  language=Matlab,
  aboveskip=3mm,
  belowskip=3mm,
  showstringspaces=false,
  columns=flexible,
  basicstyle={\small\ttfamily},
  numbers=none,
  numberstyle=\tiny\color{gray},
  keywordstyle=\color{blue},
  commentstyle=\color{dkgreen},
  stringstyle=\color{mauve},
  breaklines=true,
  breakatwhitespace=true,
  tabsize=3
  }
%
%--------Theorem Environments--------
%
\newtheorem{thm}{Theorem}[section]
\newtheorem*{thm*}{Theorem}
\newtheorem{cor}[thm]{Corollary}
\newtheorem{prop}[thm]{Proposition}
\newtheorem{lem}[thm]{Lemma}
\newtheorem*{lem*}{Lemma}
\newtheorem{conj}[thm]{Conjecture}
\newtheorem{quest}[thm]{Question}
%
\theoremstyle{definition}
\newtheorem{defn}[thm]{Definition}
\newtheorem*{defn*}{Definition}
\newtheorem{defns}[thm]{Definitions}
\newtheorem{con}[thm]{Construction}
\newtheorem{exmp}[thm]{Example}
\newtheorem{exmps}[thm]{Examples}
\newtheorem{notn}[thm]{Notation}
\newtheorem{notns}[thm]{Notations}
\newtheorem{addm}[thm]{Addendum}
\newtheorem{exer}[thm]{Exercise}
%
\theoremstyle{remark}
\newtheorem{rem}[thm]{Remark}
\newtheorem*{claim}{Claim}
\newtheorem*{aside*}{Aside}
\newtheorem*{rem*}{Remark}
\newtheorem*{hint*}{Hint}
\newtheorem*{note}{Note}
\newtheorem{rems}[thm]{Remarks}
\newtheorem{warn}[thm]{Warning}
\newtheorem{sch}[thm]{Scholium}
%
%--------Macros--------
\renewcommand{\qedsymbol}{$\blacksquare$}
\renewcommand{\sl}{\mathfrak{sl}}
\renewcommand{\hom}{\mathsf{Hom}}
\renewcommand{\emptyset}{\varnothing}
\renewcommand{\O}{\mathscr{O}}
\newcommand{\R}{\mathbb{R}}
\newcommand{\ib}[1]{\textbf{\textit{#1}}}
\newcommand{\Q}{\mathbb{Q}}
\newcommand{\Z}{\mathbb{Z}}
\newcommand{\N}{\mathbb{N}}
\newcommand{\C}{\mathbb{C}}
\newcommand{\A}{\mathbb{A}}
\newcommand{\F}{\mathbb{F}}
\newcommand{\M}{\mathcal{M}}
\renewcommand{\S}{\mathbb{S}}
\newcommand{\V}{\vec{v}}
\newcommand{\RP}{\mathbb{RP}}
\newcommand{\CP}{\mathbb{CP}}
\newcommand{\B}{\mathcal{B}}
\newcommand{\GL}{\mathsf{GL}}
\newcommand{\SL}{\mathsf{SL}}
\newcommand{\SP}{\mathsf{SP}}
\newcommand{\SO}{\mathsf{SO}}
\newcommand{\SU}{\mathsf{SU}}
\newcommand{\gl}{\mathfrak{gl}}
\newcommand{\g}{\mathfrak{g}}
\newcommand{\h}{\mathfrak{h}}
\newcommand{\inv}{^{-1}}
\newcommand{\bra}[2]{ \left[ #1, #2 \right] }
\newcommand{\set}[1]{\left\lbrace #1 \right\rbrace}
\newcommand{\abs}[1]{\left\lvert#1\right\rvert}
\newcommand{\norm}[1]{\left\lVert#1\right\rVert}
\newcommand{\transv}{\mathrel{\text{\tpitchfork}}}
\newcommand{\enumbreak}{\ \\ \vspace{-\baselineskip}}
\let\oldexists\exists
\renewcommand\exists{\oldexists~}
\let\oldL\L
\renewcommand\L{\mathfrak{L}}
\makeatletter
\newcommand{\tpitchfork}{%
  \vbox{
    \baselineskip\z@skip
    \lineskip-.52ex
    \lineskiplimit\maxdimen
    \m@th
    \ialign{##\crcr\hidewidth\smash{$-$}\hidewidth\crcr$\pitchfork$\crcr}
  }%
}
\makeatother
\newcommand{\bd}{\partial}
\newcommand{\lang}{\begin{picture}(5,7)
\put(1.1,2.5){\rotatebox{45}{\line(1,0){6.0}}}
\put(1.1,2.5){\rotatebox{315}{\line(1,0){6.0}}}
\end{picture}}
\newcommand{\rang}{\begin{picture}(5,7)
\put(.1,2.5){\rotatebox{135}{\line(1,0){6.0}}}
\put(.1,2.5){\rotatebox{225}{\line(1,0){6.0}}}
\end{picture}}
\DeclareMathOperator{\id}{id}
\DeclareMathOperator{\im}{Im}
\DeclareMathOperator{\codim}{codim}
\DeclareMathOperator{\coker}{coker}
\DeclareMathOperator{\supp}{supp}
\DeclareMathOperator{\inter}{Int}
\DeclareMathOperator{\sign}{sign}
\DeclareMathOperator{\sgn}{sgn}
\DeclareMathOperator{\indx}{ind}
\DeclareMathOperator{\alt}{Alt}
\DeclareMathOperator{\Aut}{Aut}
\DeclareMathOperator{\trace}{trace}
\DeclareMathOperator{\ad}{ad}
\DeclareMathOperator{\End}{End}
\DeclareMathOperator{\Ad}{Ad}
\DeclareMathOperator{\Lie}{Lie}
\DeclareMathOperator{\spn}{span}
\DeclareMathOperator{\dv}{div}
\DeclareMathOperator{\grad}{grad}
\DeclareMathOperator{\Sym}{Sym}
\DeclareMathOperator{\sheafhom}{\mathscr{H}\text{\kern -3pt {\calligra\large om}}\,}
\newcommand*\myhrulefill{%
   \leavevmode\leaders\hrule depth-2pt height 2.4pt\hfill\kern0pt}
\newcommand\niceending[1]{%
  \begin{center}%
    \LARGE \myhrulefill \hspace{0.2cm} #1 \hspace{0.2cm} \myhrulefill%
  \end{center}}
\newcommand*\sectionend{\niceending{\decofourleft\decofourright}}
\newcommand*\subsectionend{\niceending{\decosix}}
\def\upint{\mathchoice%
    {\mkern13mu\overline{\vphantom{\intop}\mkern7mu}\mkern-20mu}%
    {\mkern7mu\overline{\vphantom{\intop}\mkern7mu}\mkern-14mu}%
    {\mkern7mu\overline{\vphantom{\intop}\mkern7mu}\mkern-14mu}%
    {\mkern7mu\overline{\vphantom{\intop}\mkern7mu}\mkern-14mu}%
  \int}
\def\lowint{\mkern3mu\underline{\vphantom{\intop}\mkern7mu}\mkern-10mu\int}
%
%--------Hypersetup--------
%
\hypersetup{
    colorlinks,
    citecolor=black,
    filecolor=black,
    linkcolor=blue,
    urlcolor=blacksquare
}
%
%--------Solution--------
%
\newenvironment{solution}
  {\begin{proof}[Solution]}
  {\end{proof}}
%
%--------Graphics--------
%
%\graphicspath{ {images/} }

\begin{document}
%
\author{Jeffrey Jiang}
%
\title{Differential Geometry Lecture 2}
%
\setcounter{section}{0}
%
\maketitle
%
\section{Vector Fields}
%
Recall that given a smooth $n$ dimensional manifold $M$, we get the
\ib{tangent bundle} $TM$, which is a $2n$ dimensional smooth manifold, equipped
with a map $\pi : TM \to M$, where the fiber $\pi\inv(p)$ over a point $p \in M$
is the tangent space $T_pM$. Also recall that a \ib{section} of a $\pi$
is a map $\sigma : U \to TM$ such that $\pi \circ \sigma = \id_U$. If $U = M$,
we call $\sigma$ a global section.
%
\begin{defn}
A \ib{vector field} on a smooth manifold is a global section $X : M \to TM$.
The set of vector fields on $M$ is denoted $\mathfrak{X}(M)$.
\end{defn}
%
If you unwrap the definitions, we see that a section is exactly the data we
want -- for every point $p \in M$, we are assigning to it a vector in $T_pM$.
We will often denote the vector $X(p)$ at $p$ by $X_p$. \\

If we fix a chart $U$ with coordinates $x^i$, we get the \ib{coordinate vector fields}
$\partial / \partial x^i$, where
\[
\left( \frac{\partial}{\partial x^i} \right)_p = \frac{\partial}{\partial x^i}\bigg\vert_p
\]
Then given an arbitrary smooth vector field $X$, we have that $X_p$ is a linear
combination of the coordinate vector fields. Smoothness of $X$ then tells us
that we can write $X$ locally as
\[
X = X^i\frac{\partial}{\partial x^i}
\]
for smooth functions $X^i : U \to \R$. We also note that $\mathfrak{X}(M)$
admits more structure than that of a set.
%
\begin{prop} \enumbreak
\begin{enumerate}
  \item Let $X,Y \in \mathfrak{X}(M)$. Then $fX + gY$ defined a smooth vector
  field, for $f,g \in C^\infty(M)$.
  \item $\mathfrak{X}(M)$ is a $C^\infty(M)$ module.
\end{enumerate}
\end{prop}
%
In more sheafy language, this tells us that the sheaf of vector fields on $M$ is a
sheaf of modules over the sheaf of smooth functions.  In fact, we can say even
more about the algebraic structure of vector fields. Given a chart $U$ with
coordinates $x^i$, we know we can write any vector field as $X^i\partial_i$.
This tells us that the $\partial_i$ form a basis for the local vector fields
$\mathfrak{X}(U)$ as a $C^\infty(U)$ module, i.e. locally, the vector fields
form a free module over $C^\infty(U)$. Note that this might not hold globally
though.
%
\begin{defn}
A \ib{local frame} for $M$ is an collection of smooth vector fields $E_i$ defined
on an open set $U \subset M$ such that for each $p \in U$, we have that the
$E_i\vert_p$ form a basis for $T_pM$. If $U = M$, we say that the $E_i$
form a \ib{global frame}.
\end{defn}
%
We've already seen a local frame, the coordinate vector fields $\partial_i$. \\

Recall that a vector $v \in T_pM$ acts on functions $f \in C^\infty(M)$ -- it
takes as an input a smooth function, and produces a real number. We see then
that vector fields act on smooth functions as well, where we define the action
pointwise to produce a new function. Explicity, given $X \in \mathfrak{X}(M)$
and $f \in C^\infty(M)$, we have
\[
(Xf)(p) = X_pf
\]
In this way, we see that a vector field $X$ determines a linear map
$C^\infty(M) \to C^\infty(M)$. In fact, it defines a \ib{derivation},
i.e.
\[
X(fg) = fXg + gXf
\]
since each vector $X_p$ is a derivation at $p$.
%
\begin{prop}
Vector fields on $M$ are in bijection with derivations $D: C^\infty(M) \to C^\infty(M)$,
where the mapping is given by $X \mapsto D_X$ where $D_Xf = Xf$.
\end{prop}
%
Given a vector field $X \in \mathfrak{X}(M)$ and a smooth map $F : M \to N$,
we can apply the differential $dF_p$ pointwise to $X$, but the result may
not be well defined. For example, if $F$ is not injective, there will
exists at least two points $p,q$ such that $F(p) = F(q) = y$. Then if we want
to use $F$ and $X$ to define a vector field on $Y$, we have a conundrum --
what vector should we assign to $y$? Should it be $dF_p(X_p)$ or $dF_q(X_q)$?
Therefore in order for $X$ to push forward to a vector field on $Y$, we must
impose the condition on each fiber $F\inv(y)$ that for all $p \in F\inv(y)$,
we have $dF_p(X_p)$ is the same.
%
\begin{defn}
Given smooth manifolds $M$ and $N$, a smooth map $F: M \to N$, and vector
fields $X \in \mathfrak{X}(M)$ and $Y \in \mathfrak{X}(N)$. We say that
$X$ and $Y$ are \ib{$F$-related} if for all $q \in N$ and all $p \in F\inv(q)$,
we have $dF_p(X_p) = Y_q$.
\end{defn}
%
Given an arbitrary vector field $X$ and a smooth map $F: M \to N$, it's not true
in general that an $F$-related vector field exists in $\mathfrak{X}(N)$, however,
in the case that $F$ is a diffeomorphism, a unique $F$-related vector field
exists, called the pushforward $F_*X$. In order for a vector field to be $F$-related
to $X$, we see that it must be define by
\[
(F_*X)_p = dF_{F\inv(p)}(X_{F\inv(p)})
\]
which is well defined since $F$ is invertible.
%
The set of vector fields $\mathfrak{X}(M)$ already carries a great deal of
rich algebraic structure. It is a vector space, a $C^\infty(M)$ module, and
the space of derivations on $C^\infty(M)$. It also has another kind of
algebraic structure, that of a \emph{Lie algebra}.
%
\begin{defn}
The \ib{Lie bracket} of vector fields is the bilinear map
\[
[\cdot,\cdot] : \mathfrak{X}(M) \times \mathfrak{X}(M) \to \mathfrak{X}(M)
\]
where given $X,Y \in \mathfrak{X}(M)$, the vector field $[X,Y]$ is defined by
\[
[X,Y]f = XYf - YXf
\]
\end{defn}
%
There's a little to unpack with the defintion, namely, what do the terms
$XYf$ and $YXf$ actually mean? Recall that vector fields eat functions and
produce new ones, so $Yf$ is some smooth function. Therefore, we can feed this
function into $X$ to get another function. Doing this in the opposite order
gives us $YXf$ and their difference is the action of the Lie bracket of $X$
and $Y$. If we have local coordinates $x^i$, then the vector fields $X$
and $Y$ have the coordinate formulas
\[
X = X^i \frac{\partial}{\partial x^i} \qquad Y = Y^i \frac{\partial }{\partial x^i}
\]
Then the Lie bracket has the coordinate representation
\[
[X,Y] = \left( X^i\frac{\partial Y^j}{\partial x^i} - Y^i\frac{\partial X^j}{\partial x^i}
\right) \frac{\partial}{\partial x^j} = \left( XY^j - YX^j \right)
\frac{\partial}{\partial x^j}
\]
One thing to note is that the coordinate vector fields satisfy
$[\partial_i,\partial_j] = 0$, since all the component functions are constant.
In some sense, this is the defining feature of the coordinate vector fields.
%
\begin{thm}
The Lie bracket is natural in the following sense. Let $F : M \to N$
be a smooth map, $X_1, X_2 \in \mathfrak{X}(M)$ and $Y_1,Y_2 \in mathfrak{X}(N)$
such that $Y_1$ is $F$-related to $X_1$ and $Y_2$ is $F$-related to $X_2$. Then
$[Y_1,Y_2]$ is $F$-related to $[X_1,F_2]$. In the case that $F$ is a diffeomorphism,
this says that $[\cdot,\cdot]$ commutes with pushforward, i.e.
\[
F_*[X,Y] = [F_*X,F_*Y]
\]
\end{thm}
%
\section{Flows}
%
In $\R^n$, when we have a vector field $X$, we can ``integrate" the to produce
curves. The intution here is that a vector field gives us infinitesimal directions
of how to move (like a current in a stream). At a point $p$, the vector
vector $X_p$ tells us which direction to move. After taking a small step, we
arrive at a new point $q$, and then look at $X_q$ for the new direction to step
in. This intuition shows us that integrating vector fields to curves is a
matter of differential equations. We want a function $f$ such that
when we differentiate it, we recover the vector field $X$. One important
thing to note here is that solving differential equations is a \emph{local}
condition. To integrate a vector field $X$ near $p$, we don't need
to know the behavior of $X$ outside of some small neighborhood of $p$. Therefore,
translating this to manifolds should go without a hitch. To find integral curves
of $X \in \mathfrak{X}(M)$, we can pull the picture back to Eucldiean space
with charts, and then the solutions back up the manifold after using our
knowledge of differential equations in $\R^n$.
%
\begin{defn}
Given a vector field $V \in \mathfrak{X}(M)$, a curve $\gamma : I \to M$
is an \ib{integral curve} of $V$ if for all $t \in I$, we have
\[
\dot{\gamma}(t) = V_{\gamma(t)}
\]
We often call $V$ the \ib{velocity vector field} of $\gamma$.
\end{defn}
%
If we write the vector field $V$ and the curve $\gamma$ in coordinates $x^i$,
then $\dot{\gamma}(t) = V_{\gamma(t)}$ if and only if
\[
\dot{\gamma}^i \frac{\partial}{\partial x^i}\bigg\vert_{\gamma(t)} =
V^i(\gamma(t)) \frac{\partial}{\partial x^i}\bigg\vert_{\gamma(t)}
\]
Therefore, finding such a $\gamma$ in these coordinates is equivalent to
solving the system of differential equations $\dot{\gamma}^i(t) =
V^i(\gamma^1(t), \ldots, \gamma^n(t))$.
%
From the theory of differential equations, we know this is possible in a
small neighborhood of $p$.
%
\begin{thm}[\ib{Existence and Uniqueness Theorem for ODEs}]
Let $U \subset \R^n$ be open, and let $V : U \to \R^n$ be a smooth function.
This determines an initial value problem $\dot{\gamma}(t) =
V^i(\gamma^1(t), \ldots, \gamma^n(t))$ with intial condition $\gamma(t_0) = c$
for fixed constants $t_0 \in \R$ and $c \in U$. Then for any $t_0 \in \R$ and
$x_0 \in U$, there exists some interval $J$ and a smaller neighborhood $V \subset U$
of $x_0$ such that there exists a continuously differentiable solution
$\gamma : J_0 \to U$ for any intial condition $\gamma(t_0) = c$ for $c \in V$.
In addition, this solution is unique, and the map $\theta : J \times V \to U$
defined by $\theta(t,x) = \gamma(t)$ where $\gamma$ is the unique solution with
initial condition $\gamma(t_0) = x$ is smooth.
\end{thm}
%
\begin{cor}
Given smooth vector field $V \in \mathfrak{X}(M)$, for each point $p \in M$ there
exists some $\varepsilon > 0$ and a smoooth curve
$\gamma : (-\varepsilon, \varepsilon) \to M$ that is an integral curve of $V$
starting at $p$.
\end{cor}
%
As you can imagine, $F$-related vector fields generate ``$F$-related" flows.
%
\begin{prop}
Given a smooth map $F: M \to N$, vector fields $X \in \mathfrak{X}(M)$ and
$Y \in \mathfrak{X}(N)$ are $F$-related if an only if $F$ maps integral
curves of $X$ to integral curves of $Y$, i.e. given an integral curve $\gamma$
for $X$, $F \circ \gamma$ is an integral curve of $Y$.
\end{prop}
%
\begin{defn}
A \ib{global flow} on a manifold $M$ is a smooth map $\theta : \R \times M \to M$
such that
\begin{enumerate}
  \item $\theta(t, \theta(s,p)) = \theta(t+s, p)$ \\
  \item $\theta(0,p) = p$
\end{enumerate}
in other words, it is a group action of the additive group $\R$ on $M$.
\end{defn}
%
The intuition here is that given a point $p$, we can take the integral curve $\gamma$
starting at $p$ and flow for time $t$. Following this by following with the integral curve
at $\gamma(t)$ for time $s$, this shoulds be the same as flowing along $\gamma$
for time $t+s$. \\

Given a global flow $\theta$, we have two different times to obtain a map. If we fix
a point $p\in M$, we obtain the orbit map $\theta^{(p)} : \R \to M$ given
by $t \mapsto \theta(t,p)$, and given a number $t \in \R$, we can
define the map $\theta_t$ by $\theta_t(p) = \theta(t,p)$. Using this,
we can define a vector field $V$ by
\[
V_p = \dot{\theta}^{(p)}(0)
\]
You would like to say that vector fields and flows are inverses to each other --
Given a flow, we can define a vector field by differentiating integral curves,
and given a vector field, we can integrate it into a flow. Unfortunately,
this isn't exactly true. When we integrate a vector field it's not true
that the flow is defined for all time. For a simple example, the vector
field $\partial_x$ on$\R^2 - \set{0}$ is not complete, if we start flowing
at a point $p$ on the negative $x$-axis, then we cannot flow past $(0,0)$. That
being said, we won't concern ourselves too much about this, and assume that
flows are global. \\

The naturality of integral curves also tells us that flows are natural in the
follwing sense
%
\begin{thm}[\ib{Naturality of Flows}]
Let $F : M \to N$, and let $X \in \mathfrak{X}(M)$ and $Y \in \mathfrak{X}(N)$
be $F$-related vector fields that generate global flows $\theta$ and $\eta$
respectively. Then for every $t \in \R$,the following diagram commutes
%
\[\begin{tikzcd}
M \ar[r, "F"] \ar[d, "\theta_t"']& N \ar[d, "\eta_t"]\\
M \ar[r, "F"'] & N
\end{tikzcd}\]
%
\end{thm}
%
This also tells us how flows transform under diffeomorphism.
%
\begin{cor}
Let $F : M \to N$ be a diffeomorphism, and let $X \in \mathfrak{X}(M)$ with
flow $\theta$. Then the flow of the pushforward $F_*X$ is
$\eta_t = F \circ \theta_t \circ F\inv$.
\end{cor}
%
Given two vector fields $V$ and $W$, we can ask how the vector field $W$ changes
along the flow of $V$. This is the notion of a Lie derivative.
%
\begin{defn}
Let $V,W \in \mathfrak{X}(M)$. Then the \ib{Lie derivative} of $W$ with respect
to $V$ is another vector field denoted $\mathcal{L}_VW$, defined by
\[
\mathcal{L}_VW\vert_p = \frac{d}{dt}\bigg\vert_{t = 0}
d(\theta_{-t})_{\theta_t(p)}(W_{\theta_t(p)})
\]
where $\theta$ denotes the flow of $W$.
\end{defn}
%
Luckily, computing the Lie derivative is easy!
%
\begin{thm}
\[
\mathcal{L}_VW = [V,W]
\]
\end{thm}
%
We mentioned before that the fact that $[\partial_i, \partial_j] = 0$ is in some
sense, the defining feature of the coordinate vector fields. It captures the
idea that mixe partial derivates are the same, independent of the order of
differentiation.
%
\begin{thm}
Suppose we have a collection $\set{E_i}$ of vector fields that form a local
frame for $U \subset M$ and $[E_i,E_j] = 0$ for all $i,j$. Then there exist
functions $x^i : U \to \R$ such that $E_i = \partial_i$.
\end{thm}
%
There's a higher dimensional analogue of vector fields and flows.
%
\begin{defn}
A \ib{a distribution} $D$ is a rank $k$ subbundle of the tangent bundle $TM$, i.e.
a smoothly varying family of $k$-dimensional subspaces of the tangent spaces.
We denote the subspace over a point $p$ by $D_p$.
\end{defn}
%
From the definition, we see that a vector field is a special case of a distribution --
it is a rank $1$ distribution. With vector fields, we could integrate them into
flows, giving a family of integral curves, i.e. $1$-dimensional submanifolds of
$M$ whose tangent spaces are the span of the vector field. We can ask a similar
question for general distributions.
%
\begin{defn}
Let $D$ be a rank $k$ distribution. Then a submanifold $N \subset M$ is an
\ib{integral manifold} of $D$ if for each point $p$, we have that $T_pN = D_p$.
\end{defn}
%
However, unlike vector fields, integral manifolds need not exist, even in small
neighborhoods. A distribution $D$ is said to be \ib{integrable} if
integral manifolds exist. The obstructure here is something called
\emph{involutivity}.
%
\begin{defn}
A distribution $D$ is \ib{involutive} if for any vector fields $X,Y$ lying
entirely in $D$ (i.e. $X_p \in D_p$ and $Y_p \in D_p$ for all $p$), we
have that $[X,Y]$ lies entirely in $D$
\end{defn}
%
The intuition here is that if a distribution has an integral manifold, the restriction
of the distribution to the integral manifolds is isomorphic to the tangent
bundle of the integral manifold, so vector fields should be closed
under the bracket. A theorem of Frobenius then tells us that this is a necessary
and sufficient condition.
%
\begin{thm}[\ib{The Frobenius Theorem}]
A distribution $D$ is integrable if and only if it is involutive.
\end{thm}
%
\end{document}
