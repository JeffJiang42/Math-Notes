\documentclass[psamsfonts]{amsart}
%
%-------Packages---------
%
\usepackage[h margin=1 in, v margin=1 in]{geometry}
\usepackage{amssymb,amsfonts}
\usepackage{rank-2-roots}
\usepackage[all,arc]{xy}
\usepackage{tikz-cd}
\usepackage{enumerate}
\usepackage{mathrsfs}
\usepackage{amsthm}
\usepackage{mathpazo}
\usepackage{yfonts}
\usepackage{enumitem}
\usepackage{mathrsfs}
\usepackage{fourier-orns}
\usepackage[all]{xy}
\usepackage{hyperref}
\usepackage{cite}
\usepackage{url}
\usepackage{mathtools}
\usepackage{graphicx}
\usepackage{pdfsync}
\usepackage{mathdots}
\usepackage{calligra}
%
\usepackage{tgpagella}
\usepackage[T1]{fontenc}
%
\usepackage{listings}
\usepackage{color}

\definecolor{dkgreen}{rgb}{0,0.6,0}
\definecolor{gray}{rgb}{0.5,0.5,0.5}
\definecolor{mauve}{rgb}{0.58,0,0.82}

\lstset{frame=tb,
  language=Matlab,
  aboveskip=3mm,
  belowskip=3mm,
  showstringspaces=false,
  columns=flexible,
  basicstyle={\small\ttfamily},
  numbers=none,
  numberstyle=\tiny\color{gray},
  keywordstyle=\color{blue},
  commentstyle=\color{dkgreen},
  stringstyle=\color{mauve},
  breaklines=true,
  breakatwhitespace=true,
  tabsize=3
  }
%
%--------Theorem Environments--------
%
\newtheorem{thm}{Theorem}[section]
\newtheorem*{thm*}{Theorem}
\newtheorem{cor}[thm]{Corollary}
\newtheorem{prop}[thm]{Proposition}
\newtheorem{lem}[thm]{Lemma}
\newtheorem*{lem*}{Lemma}
\newtheorem{conj}[thm]{Conjecture}
\newtheorem{quest}[thm]{Question}
%
\theoremstyle{definition}
\newtheorem{defn}[thm]{Definition}
\newtheorem*{defn*}{Definition}
\newtheorem{defns}[thm]{Definitions}
\newtheorem{con}[thm]{Construction}
\newtheorem{exmp}[thm]{Example}
\newtheorem{exmps}[thm]{Examples}
\newtheorem{notn}[thm]{Notation}
\newtheorem{notns}[thm]{Notations}
\newtheorem{addm}[thm]{Addendum}
\newtheorem{exer}[thm]{Exercise}
%
\theoremstyle{remark}
\newtheorem{rem}[thm]{Remark}
\newtheorem*{claim}{Claim}
\newtheorem*{aside*}{Aside}
\newtheorem*{rem*}{Remark}
\newtheorem*{hint*}{Hint}
\newtheorem*{note}{Note}
\newtheorem{rems}[thm]{Remarks}
\newtheorem{warn}[thm]{Warning}
\newtheorem{sch}[thm]{Scholium}
%
%--------Macros--------
\renewcommand{\qedsymbol}{$\blacksquare$}
\renewcommand{\sl}{\mathfrak{sl}}
\renewcommand{\hom}{\mathsf{Hom}}
\renewcommand{\emptyset}{\varnothing}
\renewcommand{\O}{\mathscr{O}}
\newcommand{\R}{\mathbb{R}}
\newcommand{\ib}[1]{\textbf{\textit{#1}}}
\newcommand{\Q}{\mathbb{Q}}
\newcommand{\Z}{\mathbb{Z}}
\newcommand{\N}{\mathbb{N}}
\newcommand{\C}{\mathbb{C}}
\newcommand{\A}{\mathbb{A}}
\newcommand{\F}{\mathbb{F}}
\newcommand{\M}{\mathcal{M}}
\renewcommand{\S}{\mathbb{S}}
\newcommand{\V}{\vec{v}}
\newcommand{\RP}{\mathbb{RP}}
\newcommand{\CP}{\mathbb{CP}}
\newcommand{\B}{\mathcal{B}}
\newcommand{\GL}{\mathsf{GL}}
\newcommand{\SL}{\mathsf{SL}}
\newcommand{\SP}{\mathsf{SP}}
\newcommand{\SO}{\mathsf{SO}}
\newcommand{\SU}{\mathsf{SU}}
\newcommand{\gl}{\mathfrak{gl}}
\newcommand{\g}{\mathfrak{g}}
\newcommand{\h}{\mathfrak{h}}
\newcommand{\inv}{^{-1}}
\newcommand{\bra}[2]{ \left[ #1, #2 \right] }
\newcommand{\set}[1]{\left\lbrace #1 \right\rbrace}
\newcommand{\abs}[1]{\left\lvert#1\right\rvert}
\newcommand{\norm}[1]{\left\lVert#1\right\rVert}
\newcommand{\transv}{\mathrel{\text{\tpitchfork}}}
\newcommand{\enumbreak}{\ \\ \vspace{-\baselineskip}}
\let\oldexists\exists
\renewcommand\exists{\oldexists~}
\let\oldL\L
\renewcommand\L{\mathfrak{L}}
\makeatletter
\newcommand{\tpitchfork}{%
  \vbox{
    \baselineskip\z@skip
    \lineskip-.52ex
    \lineskiplimit\maxdimen
    \m@th
    \ialign{##\crcr\hidewidth\smash{$-$}\hidewidth\crcr$\pitchfork$\crcr}
  }%
}
\makeatother
\newcommand{\bd}{\partial}
\newcommand{\lang}{\begin{picture}(5,7)
\put(1.1,2.5){\rotatebox{45}{\line(1,0){6.0}}}
\put(1.1,2.5){\rotatebox{315}{\line(1,0){6.0}}}
\end{picture}}
\newcommand{\rang}{\begin{picture}(5,7)
\put(.1,2.5){\rotatebox{135}{\line(1,0){6.0}}}
\put(.1,2.5){\rotatebox{225}{\line(1,0){6.0}}}
\end{picture}}
\DeclareMathOperator{\id}{id}
\DeclareMathOperator{\im}{Im}
\DeclareMathOperator{\codim}{codim}
\DeclareMathOperator{\coker}{coker}
\DeclareMathOperator{\supp}{supp}
\DeclareMathOperator{\inter}{Int}
\DeclareMathOperator{\sign}{sign}
\DeclareMathOperator{\sgn}{sgn}
\DeclareMathOperator{\indx}{ind}
\DeclareMathOperator{\alt}{Alt}
\DeclareMathOperator{\Aut}{Aut}
\DeclareMathOperator{\trace}{trace}
\DeclareMathOperator{\ad}{ad}
\DeclareMathOperator{\End}{End}
\DeclareMathOperator{\Ad}{Ad}
\DeclareMathOperator{\Lie}{Lie}
\DeclareMathOperator{\spn}{span}
\DeclareMathOperator{\dv}{div}
\DeclareMathOperator{\grad}{grad}
\DeclareMathOperator{\Sym}{Sym}
\DeclareMathOperator{\sheafhom}{\mathscr{H}\text{\kern -3pt {\calligra\large om}}\,}
\newcommand*\myhrulefill{%
   \leavevmode\leaders\hrule depth-2pt height 2.4pt\hfill\kern0pt}
\newcommand\niceending[1]{%
  \begin{center}%
    \LARGE \myhrulefill \hspace{0.2cm} #1 \hspace{0.2cm} \myhrulefill%
  \end{center}}
\newcommand*\sectionend{\niceending{\decofourleft\decofourright}}
\newcommand*\subsectionend{\niceending{\decosix}}
\def\upint{\mathchoice%
    {\mkern13mu\overline{\vphantom{\intop}\mkern7mu}\mkern-20mu}%
    {\mkern7mu\overline{\vphantom{\intop}\mkern7mu}\mkern-14mu}%
    {\mkern7mu\overline{\vphantom{\intop}\mkern7mu}\mkern-14mu}%
    {\mkern7mu\overline{\vphantom{\intop}\mkern7mu}\mkern-14mu}%
  \int}
\def\lowint{\mkern3mu\underline{\vphantom{\intop}\mkern7mu}\mkern-10mu\int}
%
%--------Hypersetup--------
%
\hypersetup{
    colorlinks,
    citecolor=black,
    filecolor=black,
    linkcolor=blue,
    urlcolor=blacksquare
}
%
%--------Solution--------
%
\newenvironment{solution}
  {\begin{proof}[Solution]}
  {\end{proof}}
%
%--------Graphics--------
%
%\graphicspath{ {images/} }

\begin{document}
%
\author{Jeffrey Jiang}
%
\title{Differential Geometry Lecture 2}
%
\setcounter{section}{0}
%
\maketitle
%
\section{Vector Fields}
%
Recall that given a smooth $n$ dimensional manifold $M$, we get the
\ib{tangent bundle} $TM$, which is a $2n$ dimensional smooth manifold, equipped
with a map $\pi : TM \to M$, where the fiber $\pi\inv(p)$ over a point $p \in M$
is the tangent space $T_pM$. Also recall that a \ib{section} of a $\pi$
is a map $\sigma : U \to TM$ such that $\pi \circ \sigma = \id_U$. If $U = M$,
we call $\sigma$ a global section.
%
\begin{defn}
A \ib{vector field} on a smooth manifold is a global section $X : M \to TM$.
The set of vector fields on $M$ is denoted $\mathfrak{X}(M)$.
\end{defn}
%
If you unwrap the definitions, we see that a section is exactly the data we
want -- for every point $p \in M$, we are assigning to it a vector in $T_pM$.
We will often denote the vector $X(p)$ at $p$ by $X_p$. \\

If we fix a chart $U$ with coordinates $x^i$, we get the \ib{coordinate vector fields}
$\partial / \partial x^i$, where
\[
\left( \frac{\partial}{\partial x^i} \right)_p = \frac{\partial}{\partial x^i}\bigg\vert_p
\]
Then given an arbitrary smooth vector field $X$, we have that $X_p$ is a linear
combination of the coordinate vector fields. Smoothness of $X$ then tells us
that we can write $X$ locally as
\[
X = X^i\frac{\partial}{\partial x^i}
\]
for smooth functions $X^i : U \to \R$. We also note that $\mathfrak{X}(M)$
admits more structure than that of a set.
%
\begin{prop} \enumbreak
\begin{enumerate}
  \item Let $X,Y \in \mathfrak{X}(M)$. Then $fX + gY$ defined a smooth vector
  field, for $f,g \in C^\infty(M)$.
  \item $\mathfrak{X}(M)$ is a $C^\infty(M)$ module.
\end{enumerate}
\end{prop}
%
In more sheafy language, this tells us that the sheaf of vector fields on $M$ is a
sheaf of modules over the sheaf of smooth functions.  In fact, we can say even
more about the algebraic structure of vector fields. Given a chart $U$ with
coordinates $x^i$, we know we can write any vector field as $X^i\partial_i$.
This tells us that the $\partial_i$ form a basis for the local vector fields
$\mathfrak{X}(U)$ as a $C^\infty(U)$ module, i.e. locally, the vector fields
form a free module over $C^\infty(U)$. Note that this might not hold globally
though.
%
\begin{defn}
A \ib{local frame} for $M$ is an collection of smooth vector fields $E_i$ defined
on an open set $U \subset M$ such that for each $p \in U$, we have that the
$E_i\vert_p$ form a basis for $T_pM$. If $U = M$, we say that the $E_i$
form a \ib{global frame}.
\end{defn}
%
We've already seen a local frame, the coordinate vector fields $\partial_i$. \\

Recall that a vector $v \in T_pM$ acts on functions $f \in C^\infty(M)$ -- it
takes as an input a smooth function, and produces a real number. We see then
that vector fields act on smooth functions as well, where we define the action
pointwise to produce a new function. Explicity, given $X \in \mathfrak{X}(M)$
and $f \in C^\infty(M)$, we have
\[
(Xf)(p) = X_pf
\]
In this way, we see that a vector field $X$ determines a linear map
$C^\infty(M) \to C^\infty(M)$. In fact, it defines a \ib{derivation},
i.e.
\[
X(fg) = fXg + gXf
\]
since each vector $X_p$ is a derivation at $p$.
%
\begin{prop}
Vector fields on $M$ are in bijection with derivations $D: C^\infty(M) \to C^\infty(M)$,
where the mapping is given by $X \mapsto D_X$ where $D_Xf = Xf$.
\end{prop}
%
Given a vector field $X \in \mathfrak{X}(M)$ and a smooth map $F : M \to N$,
we can apply the differential $dF_p$ pointwise to $X$, but the result may
not be well defined. For example, if $F$ is not injective, there will
exists at least two points $p,q$ such that $F(p) = F(q) = y$. Then if we want
to use $F$ and $X$ to define a vector field on $Y$, we have a conundrum --
what vector should we assign to $y$? Should it be $dF_p(X_p)$ or $dF_q(X_q)$?
Therefore in order for $X$ to push forward to a vector field on $Y$, we must
impose the condition on each fiber $F\inv(y)$ that for all $p \in F\inv(y)$,
we have $dF_p(X_p)$ is the same.
%
\begin{defn}
Given smooth manifolds $M$ and $N$, a smooth map $F: M \to N$, and vector
fields $X \in \mathfrak{X}(M)$ and $Y \in \mathfrak{X}(N)$. We say that
$X$ and $Y$ are \ib{$F$-related} if for all $q \in N$ and all $p \in F\inv(q)$,
we have $dF_p(X_p) = Y_q$.
\end{defn}
%
Given an arbitrary vector field $X$ and a smooth map $F: M \to N$, it's not true
in general that an $F$-related vector field exists in $\mathfrak{X}(N)$, however,
in the case that $F$ is a diffeomorphism, a unique $F$-related vector field
exists, called the pushforward $F_*X$. In order for a vector field to be $F$-related
to $X$, we see that it must be define by
\[
(F_*X)_p = dF_{F\inv(p)}(X_{F\inv(p)})
\]
which is well defined since $F$ is invertible.
%
The set of vector fields $\mathfrak{X}(M)$ already carries a great deal of
rich algebraic structure. It is a vector space, a $C^\infty(M)$ module, and
the space of derivations on $C^\infty(M)$. It also has another kind of
algebraic structure, that of a \emph{Lie algebra}.
%
\begin{defn}
The \ib{Lie bracket} of vector fields is the bilinear map
\[
[\cdot,\cdot] : \mathfrak{X}(M) \times \mathfrak{X}(M) \to \mathfrak{X}(M)
\]
where given $X,Y \in \mathfrak{X}(M)$, the vector field $[X,Y]$ is defined by
\[
[X,Y]f = XYf - YXf
\]
\end{defn}
%
There's a little to unpack with the defintion, namely, what do the terms
$XYf$ and $YXf$ actually mean? Recall that vector fields eat functions and
produce new ones, so $Yf$ is some smooth function. Therefore, we can feed this
function into $X$ to get another function. Doing this in the opposite order
gives us $YXf$ and their difference is the action of the Lie bracket of $X$
and $Y$. If we have local coordinates $x^i$, then the vector fields $X$
and $Y$ have the coordinate formulas
\[
X = X^i \frac{\partial}{\partial x^i} \qquad Y = Y^i \frac{\partial }{\partial x^i}
\]
Then the Lie bracket has the coordinate representation
\[
[X,Y] = \left( X^i\frac{\partial Y^j}{\partial x^i} - Y^i\frac{\partial X^j}{\partial x^i}
\right) \frac{\partial}{\partial x^j} = \left( XY^j - YX^j \right)
\frac{\partial}{\partial x^j}
\]
One thing to note is that the coordinate vector fields satisfy
$[\partial_i,\partial_j] = 0$, since all the component functions are constant.
In some sense, this is the defining feature of the coordinate vector fields.
%
\begin{thm}
The Lie bracket is natural in the following sense. Let $F : M \to N$
be a smooth map, $X_1, X_2 \in \mathfrak{X}(M)$ and $Y_1,Y_2 \in mathfrak{X}(N)$
such that $Y_1$ is $F$-related to $X_1$ and $Y_2$ is $F$-related to $X_2$. Then
$[Y_1,Y_2]$ is $F$-related to $[X_1,F_2]$. In the case that $F$ is a diffeomorphism,
this says that $[\cdot,\cdot]$ commutes with pushforward, i.e.
\[
F_*[X,Y] = [F_*X,F_*Y]
\]
\end{thm}
%
\section{Flows}
%
In $\R^n$, when we have a vector field $X$, we can ``integrate" the to produce
curves. The intution here is that a vector field gives us infinitesimal directions
of how to move (like a current in a stream). At a point $p$, the vector
vector $X_p$ tells us which direction to move. After taking a small step, we
arrive at a new point $q$, and then look at $X_q$ for the new direction to step
in. This intuition shows us that integrating vector fields to curves is a
matter of differential equations. We want a function $f$ such that
when we differentiate it, we recover the vector field $X$. One important
thing to note here is that solving differential equations is a \emph{local}
condition. To integrate a vector field $X$ near $p$, we don't need
to know the behavior of $X$ outside of some small neighborhood of $p$. Therefore,
translating this to manifolds should go without a hitch. To find integral curves
of $X \in \mathfrak{X}(M)$, we can pull the picture back to Eucldiean space
with charts, and then the solutions back up the manifold after using our
knowledge of differential equations in $\R^n$.
%
\begin{defn}
Given a vector field $V \in \mathfrak{X}(M)$, a curve $\gamma : I \to M$
is an \ib{integral curve} of $V$ if for all $t \in I$, we have
\[
\gamma'(t) = V_{\gamma(t)}
\]
We often call $V$ the \ib{velocity vector field} of $\gamma$.
\end{defn}
%
\end{document}
