\documentclass[psamsfonts]{amsart}
%
%-------Packages---------
%
\usepackage[h margin=1 in, v margin=1 in]{geometry}
\usepackage{amssymb,amsfonts}
\usepackage{rank-2-roots}
\usepackage[all,arc]{xy}
\usepackage{tikz-cd}
\usepackage{enumerate}
\usepackage{mathrsfs}
\usepackage{amsthm}
\usepackage{mathpazo}
\usepackage{yfonts}
\usepackage{enumitem}
\usepackage{mathrsfs}
\usepackage{fourier-orns}
\usepackage[all]{xy}
\usepackage{hyperref}
\usepackage{cite}
\usepackage{url}
\usepackage{mathtools}
\usepackage{graphicx}
\usepackage{pdfsync}
\usepackage{mathdots}
\usepackage{calligra}
%
\usepackage{tgpagella}
\usepackage[T1]{fontenc}
%
\usepackage{listings}
\usepackage{color}

\definecolor{dkgreen}{rgb}{0,0.6,0}
\definecolor{gray}{rgb}{0.5,0.5,0.5}
\definecolor{mauve}{rgb}{0.58,0,0.82}

\lstset{frame=tb,
  language=Matlab,
  aboveskip=3mm,
  belowskip=3mm,
  showstringspaces=false,
  columns=flexible,
  basicstyle={\small\ttfamily},
  numbers=none,
  numberstyle=\tiny\color{gray},
  keywordstyle=\color{blue},
  commentstyle=\color{dkgreen},
  stringstyle=\color{mauve},
  breaklines=true,
  breakatwhitespace=true,
  tabsize=3
  }
%
%--------Theorem Environments--------
%
\newtheorem{thm}{Theorem}[section]
\newtheorem*{thm*}{Theorem}
\newtheorem{cor}[thm]{Corollary}
\newtheorem{prop}[thm]{Proposition}
\newtheorem{lem}[thm]{Lemma}
\newtheorem*{lem*}{Lemma}
\newtheorem{conj}[thm]{Conjecture}
\newtheorem{quest}[thm]{Question}
%
\theoremstyle{definition}
\newtheorem{defn}[thm]{Definition}
\newtheorem*{defn*}{Definition}
\newtheorem{defns}[thm]{Definitions}
\newtheorem{con}[thm]{Construction}
\newtheorem{exmp}[thm]{Example}
\newtheorem{exmps}[thm]{Examples}
\newtheorem{notn}[thm]{Notation}
\newtheorem{notns}[thm]{Notations}
\newtheorem{addm}[thm]{Addendum}
\newtheorem{exer}[thm]{Exercise}
%
\theoremstyle{remark}
\newtheorem{rem}[thm]{Remark}
\newtheorem*{claim}{Claim}
\newtheorem*{aside*}{Aside}
\newtheorem*{rem*}{Remark}
\newtheorem*{hint*}{Hint}
\newtheorem*{note}{Note}
\newtheorem{rems}[thm]{Remarks}
\newtheorem{warn}[thm]{Warning}
\newtheorem{sch}[thm]{Scholium}
%
%--------Macros--------
\renewcommand{\qedsymbol}{$\blacksquare$}
\renewcommand{\sl}{\mathfrak{sl}}
\renewcommand{\hom}{\mathsf{Hom}}
\renewcommand{\emptyset}{\varnothing}
\renewcommand{\O}{\mathscr{O}}
\newcommand{\R}{\mathbb{R}}
\newcommand{\ib}[1]{\textbf{\textit{#1}}}
\newcommand{\Q}{\mathbb{Q}}
\newcommand{\Z}{\mathbb{Z}}
\newcommand{\N}{\mathbb{N}}
\newcommand{\C}{\mathbb{C}}
\newcommand{\A}{\mathbb{A}}
\newcommand{\F}{\mathbb{F}}
\newcommand{\M}{\mathcal{M}}
\renewcommand{\S}{\mathbb{S}}
\newcommand{\V}{\vec{v}}
\newcommand{\RP}{\mathbb{RP}}
\newcommand{\CP}{\mathbb{CP}}
\newcommand{\B}{\mathcal{B}}
\newcommand{\GL}{\mathsf{GL}}
\newcommand{\SL}{\mathsf{SL}}
\newcommand{\SP}{\mathsf{SP}}
\newcommand{\SO}{\mathsf{SO}}
\newcommand{\SU}{\mathsf{SU}}
\newcommand{\gl}{\mathfrak{gl}}
\newcommand{\g}{\mathfrak{g}}
\newcommand{\h}{\mathfrak{h}}
\newcommand{\inv}{^{-1}}
\newcommand{\bra}[2]{ \left[ #1, #2 \right] }
\newcommand{\set}[1]{\left\lbrace #1 \right\rbrace}
\newcommand{\abs}[1]{\left\lvert#1\right\rvert}
\newcommand{\norm}[1]{\left\lVert#1\right\rVert}
\newcommand{\transv}{\mathrel{\text{\tpitchfork}}}
\newcommand{\enumbreak}{\ \\ \vspace{-\baselineskip}}
\let\oldexists\exists
\renewcommand\exists{\oldexists~}
\let\oldL\L
\renewcommand\L{\mathfrak{L}}
\makeatletter
\newcommand{\tpitchfork}{%
  \vbox{
    \baselineskip\z@skip
    \lineskip-.52ex
    \lineskiplimit\maxdimen
    \m@th
    \ialign{##\crcr\hidewidth\smash{$-$}\hidewidth\crcr$\pitchfork$\crcr}
  }%
}
\makeatother
\newcommand{\bd}{\partial}
\newcommand{\lang}{\begin{picture}(5,7)
\put(1.1,2.5){\rotatebox{45}{\line(1,0){6.0}}}
\put(1.1,2.5){\rotatebox{315}{\line(1,0){6.0}}}
\end{picture}}
\newcommand{\rang}{\begin{picture}(5,7)
\put(.1,2.5){\rotatebox{135}{\line(1,0){6.0}}}
\put(.1,2.5){\rotatebox{225}{\line(1,0){6.0}}}
\end{picture}}
\DeclareMathOperator{\id}{id}
\DeclareMathOperator{\im}{Im}
\DeclareMathOperator{\codim}{codim}
\DeclareMathOperator{\coker}{coker}
\DeclareMathOperator{\supp}{supp}
\DeclareMathOperator{\inter}{Int}
\DeclareMathOperator{\sign}{sign}
\DeclareMathOperator{\sgn}{sgn}
\DeclareMathOperator{\indx}{ind}
\DeclareMathOperator{\alt}{Alt}
\DeclareMathOperator{\Aut}{Aut}
\DeclareMathOperator{\trace}{trace}
\DeclareMathOperator{\ad}{ad}
\DeclareMathOperator{\End}{End}
\DeclareMathOperator{\Ad}{Ad}
\DeclareMathOperator{\Lie}{Lie}
\DeclareMathOperator{\spn}{span}
\DeclareMathOperator{\dv}{div}
\DeclareMathOperator{\grad}{grad}
\DeclareMathOperator{\Sym}{Sym}
\DeclareMathOperator{\sheafhom}{\mathscr{H}\text{\kern -3pt {\calligra\large om}}\,}
\newcommand*\myhrulefill{%
   \leavevmode\leaders\hrule depth-2pt height 2.4pt\hfill\kern0pt}
\newcommand\niceending[1]{%
  \begin{center}%
    \LARGE \myhrulefill \hspace{0.2cm} #1 \hspace{0.2cm} \myhrulefill%
  \end{center}}
\newcommand*\sectionend{\niceending{\decofourleft\decofourright}}
\newcommand*\subsectionend{\niceending{\decosix}}
\def\upint{\mathchoice%
    {\mkern13mu\overline{\vphantom{\intop}\mkern7mu}\mkern-20mu}%
    {\mkern7mu\overline{\vphantom{\intop}\mkern7mu}\mkern-14mu}%
    {\mkern7mu\overline{\vphantom{\intop}\mkern7mu}\mkern-14mu}%
    {\mkern7mu\overline{\vphantom{\intop}\mkern7mu}\mkern-14mu}%
  \int}
\def\lowint{\mkern3mu\underline{\vphantom{\intop}\mkern7mu}\mkern-10mu\int}
%
%--------Hypersetup--------
%
\hypersetup{
    colorlinks,
    citecolor=black,
    filecolor=black,
    linkcolor=blue,
    urlcolor=blacksquare
}
%
%--------Solution--------
%
\newenvironment{solution}
  {\begin{proof}[Solution]}
  {\end{proof}}
%
%--------Graphics--------
%
%\graphicspath{ {images/} }

\begin{document}
%
\author{Jeffrey Jiang}
%
\title{Differential Geometry Preliminaries}
%
\setcounter{section}{-1}
%
\maketitle
%
\section{Preliminaries}
%
These are some preliminary details regarding multivariable calculus. Don't
worry if you're not 100\% familiar with them, but we will build upon these
concepts when we study manifolds. Since the goal of differential geometry
is to use calculus to study nonlinear spaces via linear approximation,
it's good to have a foundation in calculus in the standard setting.\\

Recall that given a differentiable function $F: \R^n \to \R^m$, where
$F = (F^1, \ldots, F^m)$ for functions $F^i : \R^n \to \R$, its derivative
is given by it's \ib{Jacobian matrix} $DF$, which is the matrix
$$DF = \begin{pmatrix}
\frac{\partial F^1}{\partial x^1} &
\ldots & \frac{\partial F^1}{\partial x^n} \\
\vdots & \vdots & \vdots \\
\frac{\partial F^m}{\partial x^1} &
\ldots & \frac{\partial F^m}{\partial x^n}
\end{pmatrix} $$
We denote the Jacobian evaluated at a point $p$ to be $DF_p$. Given maps
$$\begin{tikzcd}
\R^k \ar[r,"F"] & \R^n \ar[r, "G"] & \R^m
\end{tikzcd}$$
We have that $D(G \circ F)_p = DG_{F(p)} \circ DF_p$. This is the
\ib{multivariable chain rule}. Concisely, ``the differential of the
composition is the composition of the differentials."
%
\begin{defn}
	A map $F : U \to V$ where $U \subset \R^n$ and $V \subset \R^m$ is
	\ib{smooth} if all of its component functions are infinitely
	differentiable, i.e. the partial derivatives of all ordersexist and are
	continuous.
\end{defn}
%
Note that smoothness is a local condition -- to know a function is smooth
at a point, it suffices to check in a small neighborhood. A function is
smooth if and only if it is smooth at every point.
%
\begin{defn}
	Let $U \subset \R^n$ and $V \subset \R^m$ be open subsets. A map
	$F : U \to V$ is called a \ib{diffeomorphism} if $F$ is smooth,
	bijective, and admits a smooth inverse $F\inv : V \to U$.
\end{defn}
%
Note that a smooth bijection is not sufficient for a map to be a
diffeomorphism. The map $f : \R \to \R$ given by $f(x) = x^3$ is a smooth
bijection, but it's inverse $x \mapsto x^{1/3}$ is not smooth at $0$, so
$f$ is not a diffeomorphism. Since smoothness is a local property, we
also have the notion of a local diffeomorphism.
%
\begin{defn}
	Let $U \subset \R^n$ and $V \subset \R^m$. A smooth function
	$F : U \to V$ is a \ib{local diffeomorphism} if for every $p \in U$,
	there exists an open neighborhood $U_p \subset U$ such that
	$F\vert_{U_p}$ is a diffeomorphism onto its image.
\end{defn}
%
It is clear that every diffeomorphism is a local diffeomorphism, but
the converse is not true, since local diffeomorphisms can fail to be
bijective. An example of such a map is the polar transformation
$p : \R^2-\set{0} \to \R^2-\set{0}$ given by
$p(r,\theta) = (r\cos\theta,r\sin\theta)$. An important theorem tells us
that the Jacobian matrix captures all the local information we need.
%
\begin{thm}[\ib{The Inverse Function Theorem}]
	Let $F : U \to V$ be smooth. Then $F$ is a local diffeomorphism at $p$ if
	and only if $dF_p$ is an isomorphism.
\end{thm}
%
Finally, there's a widespread convention in differential geometry
regarding summations, called the \ib{Einstein summation convention}. We
can compactly represent the summation $\sum_i v^ie_i$ as just $v^ie_i$.
The rule is that if you see an upper index and a lower index, there is
an implicit summation. For example, matrix multiplication in this
notation is given by $(AB)^i_j = A^i_kB^k_j$. Where the upper index denotes
the row, and the lower index denotes the column, and the implicit
summation is over the dummy index $k$. This convention becomes
useful when there are a lot of indices floating around, which is
a common occurence in differential geometry.
%
\end{document}
