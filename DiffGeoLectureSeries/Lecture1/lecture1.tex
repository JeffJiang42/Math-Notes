\documentclass[psamsfonts]{amsart}
%
%-------Packages---------
%
\usepackage[h margin=1 in, v margin=1 in]{geometry}
\usepackage{amssymb,amsfonts}
\usepackage{rank-2-roots}
\usepackage[all,arc]{xy}
\usepackage{tikz-cd}
\usepackage{enumerate}
\usepackage{mathrsfs}
\usepackage{amsthm}
\usepackage{mathpazo}
\usepackage{yfonts}
\usepackage{enumitem}
\usepackage{mathrsfs}
\usepackage{fourier-orns}
\usepackage[all]{xy}
\usepackage{hyperref}
\usepackage{cite}
\usepackage{url}
\usepackage{mathtools}
\usepackage{graphicx}
\usepackage{pdfsync}
\usepackage{mathdots}
\usepackage{calligra}
%
\usepackage{tgpagella}
\usepackage[T1]{fontenc}
%
\usepackage{listings}
\usepackage{color}

\definecolor{dkgreen}{rgb}{0,0.6,0}
\definecolor{gray}{rgb}{0.5,0.5,0.5}
\definecolor{mauve}{rgb}{0.58,0,0.82}

\lstset{frame=tb,
  language=Matlab,
  aboveskip=3mm,
  belowskip=3mm,
  showstringspaces=false,
  columns=flexible,
  basicstyle={\small\ttfamily},
  numbers=none,
  numberstyle=\tiny\color{gray},
  keywordstyle=\color{blue},
  commentstyle=\color{dkgreen},
  stringstyle=\color{mauve},
  breaklines=true,
  breakatwhitespace=true,
  tabsize=3
  }
%
%--------Theorem Environments--------
%
\newtheorem{thm}{Theorem}[section]
\newtheorem*{thm*}{Theorem}
\newtheorem{cor}[thm]{Corollary}
\newtheorem{prop}[thm]{Proposition}
\newtheorem{lem}[thm]{Lemma}
\newtheorem*{lem*}{Lemma}
\newtheorem{conj}[thm]{Conjecture}
\newtheorem{quest}[thm]{Question}
%
\theoremstyle{definition}
\newtheorem{defn}[thm]{Definition}
\newtheorem*{defn*}{Definition}
\newtheorem{defns}[thm]{Definitions}
\newtheorem{con}[thm]{Construction}
\newtheorem{exmp}[thm]{Example}
\newtheorem{exmps}[thm]{Examples}
\newtheorem{notn}[thm]{Notation}
\newtheorem{notns}[thm]{Notations}
\newtheorem{addm}[thm]{Addendum}
\newtheorem{exer}[thm]{Exercise}
%
\theoremstyle{remark}
\newtheorem{rem}[thm]{Remark}
\newtheorem*{claim}{Claim}
\newtheorem*{aside*}{Aside}
\newtheorem*{rem*}{Remark}
\newtheorem*{hint*}{Hint}
\newtheorem*{note}{Note}
\newtheorem{rems}[thm]{Remarks}
\newtheorem{warn}[thm]{Warning}
\newtheorem{sch}[thm]{Scholium}
%
%--------Macros--------
\renewcommand{\qedsymbol}{$\blacksquare$}
\renewcommand{\sl}{\mathfrak{sl}}
\renewcommand{\hom}{\mathsf{Hom}}
\renewcommand{\emptyset}{\varnothing}
\renewcommand{\O}{\mathscr{O}}
\newcommand{\R}{\mathbb{R}}
\newcommand{\ib}[1]{\textbf{\textit{#1}}}
\newcommand{\Q}{\mathbb{Q}}
\newcommand{\Z}{\mathbb{Z}}
\newcommand{\N}{\mathbb{N}}
\newcommand{\C}{\mathbb{C}}
\newcommand{\A}{\mathbb{A}}
\newcommand{\F}{\mathbb{F}}
\newcommand{\M}{\mathcal{M}}
\renewcommand{\S}{\mathbb{S}}
\newcommand{\V}{\vec{v}}
\newcommand{\RP}{\mathbb{RP}}
\newcommand{\CP}{\mathbb{CP}}
\newcommand{\B}{\mathcal{B}}
\newcommand{\GL}{\mathsf{GL}}
\newcommand{\SL}{\mathsf{SL}}
\newcommand{\SP}{\mathsf{SP}}
\newcommand{\SO}{\mathsf{SO}}
\newcommand{\SU}{\mathsf{SU}}
\newcommand{\gl}{\mathfrak{gl}}
\newcommand{\g}{\mathfrak{g}}
\newcommand{\h}{\mathfrak{h}}
\newcommand{\inv}{^{-1}}
\newcommand{\bra}[2]{ \left[ #1, #2 \right] }
\newcommand{\set}[1]{\left\lbrace #1 \right\rbrace}
\newcommand{\abs}[1]{\left\lvert#1\right\rvert}
\newcommand{\norm}[1]{\left\lVert#1\right\rVert}
\newcommand{\transv}{\mathrel{\text{\tpitchfork}}}
\newcommand{\enumbreak}{\ \\ \vspace{-\baselineskip}}
\let\oldexists\exists
\renewcommand\exists{\oldexists~}
\let\oldL\L
\renewcommand\L{\mathfrak{L}}
\makeatletter
\newcommand{\tpitchfork}{%
  \vbox{
    \baselineskip\z@skip
    \lineskip-.52ex
    \lineskiplimit\maxdimen
    \m@th
    \ialign{##\crcr\hidewidth\smash{$-$}\hidewidth\crcr$\pitchfork$\crcr}
  }%
}
\makeatother
\newcommand{\bd}{\partial}
\newcommand{\lang}{\begin{picture}(5,7)
\put(1.1,2.5){\rotatebox{45}{\line(1,0){6.0}}}
\put(1.1,2.5){\rotatebox{315}{\line(1,0){6.0}}}
\end{picture}}
\newcommand{\rang}{\begin{picture}(5,7)
\put(.1,2.5){\rotatebox{135}{\line(1,0){6.0}}}
\put(.1,2.5){\rotatebox{225}{\line(1,0){6.0}}}
\end{picture}}
\DeclareMathOperator{\id}{id}
\DeclareMathOperator{\im}{Im}
\DeclareMathOperator{\codim}{codim}
\DeclareMathOperator{\coker}{coker}
\DeclareMathOperator{\supp}{supp}
\DeclareMathOperator{\inter}{Int}
\DeclareMathOperator{\sign}{sign}
\DeclareMathOperator{\sgn}{sgn}
\DeclareMathOperator{\indx}{ind}
\DeclareMathOperator{\alt}{Alt}
\DeclareMathOperator{\Aut}{Aut}
\DeclareMathOperator{\trace}{trace}
\DeclareMathOperator{\ad}{ad}
\DeclareMathOperator{\End}{End}
\DeclareMathOperator{\Ad}{Ad}
\DeclareMathOperator{\Lie}{Lie}
\DeclareMathOperator{\spn}{span}
\DeclareMathOperator{\dv}{div}
\DeclareMathOperator{\grad}{grad}
\DeclareMathOperator{\Sym}{Sym}
\DeclareMathOperator{\sheafhom}{\mathscr{H}\text{\kern -3pt {\calligra\large om}}\,}
\newcommand*\myhrulefill{%
   \leavevmode\leaders\hrule depth-2pt height 2.4pt\hfill\kern0pt}
\newcommand\niceending[1]{%
  \begin{center}%
    \LARGE \myhrulefill \hspace{0.2cm} #1 \hspace{0.2cm} \myhrulefill%
  \end{center}}
\newcommand*\sectionend{\niceending{\decofourleft\decofourright}}
\newcommand*\subsectionend{\niceending{\decosix}}
\def\upint{\mathchoice%
    {\mkern13mu\overline{\vphantom{\intop}\mkern7mu}\mkern-20mu}%
    {\mkern7mu\overline{\vphantom{\intop}\mkern7mu}\mkern-14mu}%
    {\mkern7mu\overline{\vphantom{\intop}\mkern7mu}\mkern-14mu}%
    {\mkern7mu\overline{\vphantom{\intop}\mkern7mu}\mkern-14mu}%
  \int}
\def\lowint{\mkern3mu\underline{\vphantom{\intop}\mkern7mu}\mkern-10mu\int}
%
%--------Hypersetup--------
%
\hypersetup{
    colorlinks,
    citecolor=black,
    filecolor=black,
    linkcolor=blue,
    urlcolor=blacksquare
}
%
%--------Solution--------
%
\newenvironment{solution}
  {\begin{proof}[Solution]}
  {\end{proof}}
%
%--------Graphics--------
%
%\graphicspath{ {images/} }

\begin{document}
%
\author{Jeffrey Jiang}
%
\title{Differential Geometry Lecture 1}
%
\setcounter{section}{0}
%
\maketitle
%
\section{Introduction to Manifolds}
%
\begin{defn}
A \ib{topological manifold} is a Hausdorff space $X$ such there exists a countable open cover $\set{U_\alpha}$
of $X$, along with homeomorphisms $\varphi_\alpha : U_\alpha \to V_\alpha$, where $V_\alpha$ is an open subset
of $\R^n$. Given a chart $(U, \varphi)$, we can write $\varphi$ in terms of its component functions
$$\varphi(p) = (x^1(p), \ldots, x^n(p)) $$
The functions $x^i$ are called \ib{local coordinates} on $U$.
\end{defn}
%
In this way, we see that a manifold is a topological space that is locally topologically indistinguishable from
Euclidean space. Both modifiers are important here -- there could be global and geometric properties that
differ from $\R^n$.
%
\begin{exmp}\enumbreak
	\begin{enumerate}
		\item $\R^n$ is a topological manifold -- it admits a global chart $(\R^n, \id_{\R^n})$.
		\item The $2$-sphere $S^2 = \set{(x,y,z) \in \R^3 ~:~ x^2 + y^2 + z^2 = 1}$ is a topological
		manifold. If we let $N$ and $S$ denote the North and South poles repsectively, then we can
		construct charts using \ib{stereographic projection}. Stereographic projection from the North
		pole is a map $\varphi_N : S^2 - \set{N} \to \R^n$, where given a point $p \in S^2 - \set{N}$,
		we take the line cotaining both $N$ and $p$, which intersects the $z = 0 $ plane at one point
		$q$. We then define $\varphi_N(p) = q$. Stereographic projection $\varphi_S$ from the South
		pole is defined in an analogous manner. An explicit formula for $\varphi_N$ is
		$$\varphi_N(x,y,z) = \left(\frac{x}{1 - z}, \frac{y}{1 - z} \right) $$
	\end{enumerate}
\end{exmp}
%
\begin{exer}\enumbreak
	\begin{enumerate}
		\item $S^2$ does not admit a global chart. Why?
		\item Give an inverse map $\R^n \to \S^n - \set{N}$ for stereographic projection from the
			North Pole
		\item Generalize stereographic projection to the $n$-sphere
		$S^n = \set{p \in \R^{n+1} ~:~ \norm{p} = 1}$
	\end{enumerate}
\end{exer}
%
One of the main appeals of Euclidean space is that we have a lot of tools at our disposal, like linear algebra
and calculus. We would like to generalize these notions to manifolds. One of the wonderful properties of
manifolds is that they locally look like $\R^n$, so we can use the charts to translate concepts we know in
$\R^n$ to concepts on the manifold.
%
We know what it means for a function on $\R^n$ to be smooth, how do we
translate this to a manifold $X$? Given a map $F : X \to \R^m$, what does it mean for $F$ to be smooth? Our
first guess is to use our charts. Given a map $F: X \to \R^m$, we want
to say that it is smooth if for every $p \in X$, there exists a chart
$(U_p, \varphi_p)$ such that the composition
$F \circ \varphi_p\inv : U_p \to \R^m$ is smooth. The intuition here
is correct, there there are some technicalities that need to be addressed.
 Namely, suppose $p \in X$ lies in the domain of two charts
 $(U_1, \varphi_1)$, and $(U_2, \varphi_2)$. Then what if
 $F \circ \varphi_1\inv$ was smooth, but $F \circ \varphi_2\inv$ wasn't?
 This suggests that we need a compatibility condition for our charts
 if we want a notion of smoothness. We say two charts $(U_1, \varphi_1)$
 and $(U_2, \varphi_2)$ are \ib{smoothly compatible} if both
 $\varphi_1 \circ \varphi_2\inv$ and $\varphi_2 \circ \varphi_1\inv$ are
 smooth maps.
 \begin{defn}
	 A collection $\mathcal{A} = \set{(U_\alpha, \varphi_\alpha)}$ of
	 smoothly compatible charts such that the $U_\alpha$ cover $X$ is called
	 an \ib{atlas}. An atlas is \ib{maximal} if it is not property contained in any other atlas.
 \end{defn}
%
\begin{defn}
	A \ib{smooth manifold} is a topological manifold $X$ equipped with a maximal atlast $\mathcal{A}$.
\end{defn}
%
Now if we have a smooth manifold $X$, we have a notion of smooth maps
$X \to \R^n$
%
\begin{defn}
	A map $F : X \to \R^k$ is \ib{smooth} if for every chart
	$(U_\alpha, \varphi_\alpha$), we have that $F \circ \varphi_\alpha\inv$ is
	smooth in the Euclidean sense. We call $F \circ \varphi_\alpha\inv$ a local
	\ib{coordinate representation} of the map $F$.
\end{defn}
%
We also know what if means for maps between manifolds to be smooth.
%
\begin{defn}
	A map $F : X \to Y$ of smooth manifolds is $\ib{smooth}$ if for every $p$,
	there exists charts $(U, \varphi)$ and $(V, \psi)$ of $p$ and $F(p)$
	respectively, such that (after appropriate shrinking of $U$ and $V$), we
	have that $\psi \circ F \circ \varphi\inv$ is smooth in the Euclidean sense.
\end{defn}
%
For convenience, we will make the a distinction between the words map
and function, which is also common in the literature. A \emph{map} will denote
an arbitrary mapping $X \to Y$, where $X$ and $Y$ can denote any kind of set.
A \emph{function} on a space $X$ is a mapping $X \to \R$. We let $C^\infty(X)$
denote the vector space of functions of $X$, which forms a commutative ring
under pointwise multiplication and addition.\\

This is nice, but we've excluded a huge class of spaces. For example, the
unit interval isn't a manifold, since there exist no charts about the
endpoints satisfying our definition (why?). To include the spaces, we
need another definition. Let $\mathbb{H}^n$ denote
\ib{Euclidean half space}, where
$$\mathbb{H}^n = \set{(x^1, \ldots x^n) \in \R^n ~:~ x^n \geq 0} $$
%
\begin{defn}
	A \ib{manifold with boundary} is a Hausdorff space $X$ such that there
	exists a countable cover by charts $(U_\alpha, \varphi_\alpha)$ where
	the $\varphi_\alpha$ are homeomorphisms to open sets in $\mathbb{H}^n$.
	Given a point $p \in X$, we say that $p$ is in the \ib{boundary} of
	$X$ if there exists a chart $(U, \varphi)$ containing $p$ such that
	$\varphi(p) = (x^1, \ldots, x^n)$, and $x^n = 0$. We say that $p$ is in
	the \ib{interior} of $X$ if no such chart exists. We denote the boundary
	and interior of $X$ as $\bd X$ and $\inter X$ respectively.
\end{defn}
%
Under this definition, $\mathbb{H}^n$ is a smooth manifold with boundary,
with a global chart given by $(\mathbb{H}^n, \id_{\mathbb{H}^n})$, and
$\bd \mathbb{H}^n = \set{(x^1, \ldots x^n) \in \mathbb{H}^n ~:~ x^n = 0}$
%
\begin{exer}
	Let $X$ be a smooth $n$-manifold with boundary
	\begin{enumerate}
		\item Show that $\bd X$ is
		well defined. Namely, show that if we have a point $p \in X$ such that
		there exists a chart $(U, \varphi)$ where $\varphi_1(p) \in \bd
		\mathbb{H}^n$, show that any other chart $(U', \varphi')$ for $p$ must
		also have $\varphi'(p) \in \bd \mathbb{H^n}$. (Hint: Use the inverse
		function theorem to show there exists no diffeomorphism from an open set
		in $\R^n$ to an open set in $\R^{n-1}$.)
		\item Prove that $\bd X$ is a smooth $(n-1)$-manifold without boundary.
	\end{enumerate}
\end{exer}
%
\section{Tangent Spaces and the Tangent Bundle}
%
Recall from multivariable calculus that the derivative is the best linear
approximation to a map. Because of this, we need to introduce some sort
of linear structure on our smooth manifold in order to make sense of
derivatives of maps between manifolds. In the case where our manifold $M$
is embedded in $\R^n$, this is easy, since we can take a chart
$(U, \varphi)$, and treat is as a map $\R^k \to \R^n$. Then
the image of derivative $d\varphi\inv_p : \R^k \to \R^n$ is the best
linear approximation to our manifold $M$ at the point $p \in M \subset
\R^n$. This is nice, but is not an intrinsic definition, since it relies
on our  manifold lying in some ambient Euclidean space. While this is
always possible thanks to the Whitney Embedding theorem, there's a more
natural way to talk about tangent vectors to an abstract manifold, that
need not be embedded in $\R^n$. \\

We'll provide some motivation in the Euclidean case first. We naturally
think of tangent vectors at a point $p \in \R^n$ as another copy of $\R^n$
based at $p$. However, there is a different way to think about tangent
vectors. Let $e_i$ denote the standard basis for $\R^n$, and let $v \in
\R^n$. Then there exists a linear differential operator $D_v$ on
$\C^\infty(\R^n)$ that takes a function $f: \R^n \to \R$, and takes its
directional derivative in the direction of $v$. If we write $v = v^ie_i$,
then
$$D_v\vert_p = v^i\frac{\partial}{\partial x^i}\bigg\vert_p $$
We also note that $D_v$ satisfies to product rule (also referred to as
the Leibniz rule), i.e. for $f,g \in C^\infty(\R^n)$, we have that
$$D_v\vert_p(fg) = f(p)D_vg + g(p)D_vf$$
\begin{defn}
	A linear map $D : C^\infty(\R^n) \to \R$ satisfying the product rule at $p \in \R^n$,
	i.e.
	$$D(fg) = f(p)Dg + g(p)Df$$
	is called a \ib{derivation at p}
\end{defn}
%
\begin{thm}
	There is a bijection
	$$\set{\text{Derivations at } p} \leftrightarrow
	\set{\text{tangent vectors based at } p}  $$
	given by the mapping $v \mapsto D_v$
\end{thm}
%
This theorem tells us that thinking about derivations is equivalent to thinking
of tangent vectors based at a point $p$. In addition, this definition gives
more structure to our original idea of tangent vectors, since derivations act
on smooth functions, rather than just being some arrow. This definition also
generalizes nicely to abstract manifolds, since the only real tools we have
as of now are notions of smoothness.
%
\begin{defn}
	Let $M$ be a smooth manifold and $p \in M$. Then the \ib{tangent space} at
	$p$, denoted $T_pM$, is the vector space of derivations at $p$.
\end{defn}
%
Then given a smooth map of manifolds $F: M \to N$, we want the derivative of $F$
at $p$ to be a linear map $dF_p : T_pM \to T_{f(p)}N$. Given some derivation
$v \in T_pM$, how should $dF_p(v)$ act on some function $g \in C^\infty(N)$?
We don't have too much at our disposal, so the only reasonable choice here is
to have
$$dF_p(v)g = v(g \circ F) $$
The derivative is also referred to as the \ib{pushforward}, and some choose
to denote it as $F_*$.\\

Now that we have an abstract definition of tangent spaces, we'd like to see how
these objects behave in local coordinates. Let $M$ be a smooth $n$-manifold,
and let $(U,\varphi)$ be a chart for $p$ with $\varphi = (x^1 ,\ldots x^n)$.
Then the \ib{coordinate vectors} at $p$ for the chart $(U, \varphi)$ are the
derivations $\partial/\partial x^i\vert_p$. The action of the coordinate
vectors is given by
$$\frac{\partial}{\partial x^i}f =
\frac{\partial (f \circ \varphi\inv)}{\partial x^i}\bigg\vert_{\varphi(p)} $$
So the coordinate vectors are literally the partial derivative operations on
the coordinate representation of a function, justifying the notation. We often
abbreviate them as just $\partial_i$ when the choice of coordinates are clear.
%
\begin{thm}
	The coordinate vectors $\partial/ \partial x^i \vert_p$ form a basis
	for $T_pM$.
\end{thm}
%
The following exercise should help you build intuition about the derivative,
since it will resemble something you might already know. In addition, it's
a good exercise to make sure you understand the definition of vectors as
derivations and the derivative.
%
\begin{exer}
Let $M$ and $N$ be smooth manifolds, and let $F : M \to N$ be a smooth map.
Fix $p \in M$ and charts $(U, \varphi)$ and $(V, \psi)$ for $p$ and $F(p)$
respectively. Show that $dF_p$ is given in local coordinates by the Jacobian
matrix of $\psi \circ F \circ \varphi\inv$. (Hint: To do this, observe
the action of $dF_p(\partial /\partial x^i\vert_p)$ on an arbitrary smooth
function $f \in C^\infty(N)$).
\end{exer}
%
Local coordinates provide an excellent tool for giving concrete descriptions
of functions and objects of a manifold, but working in coordinates has a few
nuances. If we define some quantity (a function, tensor, etc.) using local
coordinates, we need to ensure that our definition still works under a change
of coordinates. Otherwise, the quantity we attempted to define is not
well-defined at all! Because of this, it's in our best interest to understand
how tangent vectors behave under coordinate transformation. \\

Let $(U, \varphi)$ and $(V, \psi)$ be two charts about a point $p \in M$, with
$\varphi = (x^1, \ldots x^n)$ and $\psi = (y^1, \ldots, y^n)$.
We want to express $\partial/\partial x^i\vert_p$ in terms of the
$\partial/\partial y^i\vert_p$. To do this, we want to compute the image of
$\partial / \partial x^i$ under the derivative $d(\psi \circ \varphi\inv)_p$.
The computation yields
$$\frac{\partial}{\partial x^i}\bigg\vert_p =
\frac{\partial y^j}{\partial x^i}(\psi(p)) \frac{\partial}{\partial y^j}\bigg\vert_p$$
which looks just like the chain rule from multivariable calculus. Likewise,
for an arbitrary vector $v = v^i \partial / \partial x^i$, it's componenets
$\tilde{v}^i$ in the coordinate system given by the $y^i$ is
$$\tilde{v}^j = v^i\frac{\partial y^j}{\partial x^i}(\psi(p)) $$
%
It's important to note that each point $p \in M$ gets its own tangent
space -- there's no natural identification between tangent spaces, and
no way to transport information between then (at least not yet!).
However, we can construct a new object using the technology we have
now -- the tangent bundle.
\begin{defn}
	Let $M$ be a smooth manifold. Then the \ib{tangent bundle} of $M$ is
	the set
	$$TM = \coprod_{p \in M} T_pM = \set{(p,v) ~:~ p \in M, v \in T_pM} $$
	The tangent bundle comes equipped with a natural map $\pi : TM \to M$
	given by $\pi(p,v) = p$ called the \ib{projection}.
\end{defn}
%
We only said set in the definition, but there's more structure here. As
you might expect, $TM$ is a smooth manifold.
%
\begin{thm}
	The tangent bundle $TM$ is a smooth manifold, and has a natural smooth
	structure such that the projection $\pi : TM \to M$ is smooth.
\end{thm}
%
\begin{proof}
For a point $p \in M$, fix a chart $(U, \varphi)$, where
$\varphi = (x^1, \ldots, x^n)$. We then use this to
construct what will be chart for $\pi\inv(U)$, and we will use these
charts to define the topology. We know that for each point in $U$, the
vectors $\partial_i$ form a basis for the tangent space at this point,
so each pair $(q,v) \in \pi\inv(U)$, we can write it uniquely in
coordinates as $(x^1(q), \ldots x^n(q), v^1, \ldots v^n)$, where the
$v^i$ are the components of $v$ in terms of the $\partial_i$. This gives
a bijection $\Phi_p : \pi\inv(U) \to U \times \R^n$. Doing this for
all $p \in M$ gives a cover of $TM$, and these sets satisfies the
conditions for a basis, so we can take the topology generated by these
sets. In addition, the transition maps $\Phi_p \circ \Phi_q$ are smooth,
so they determine a smooth structure on $TM$, making a smooth manifold
of dimension $2n$.
\end{proof}
%
In addition, given a smooth map $F : M \to N$, this induces a map
$dF : TM \to TN$, where in local coordinates, $dF(p,v) = (F(p), dF_p(v))$.
%
\begin{exer}
Show that the assignments $M \mapsto TM$ and $F \mapsto dF$ determines
a covariant functor from the category of smooth manifolds to itself,
called the \ib{Tangent Functor}. Can you recognize this statement in
terms a familiar statement in calculus?
\end{exer}
%
Note that the charts we constructed for $TM$ are of a particular form,
namely they are isomorphisms to $U \times \R^n$, so $TM$ locally looks
like a product manifold, but globally there may be some twisting. In
addition, each fiber $\pi\inv(p)$ has the structure of a real vector
space. These are defining features of a geometric object called a
\ib{vector bundle}, which are central objects in differential geometry.
%
\section{A Short Introduction to Vector Bundles}
%
\begin{defn}
Let $M$ be a smooth manifold. A \ib{vector bundle} over $M$ is
another smooth manifold $E$ equipped with a smooth map $\pi : E \to M$
satisfying the following properties.
\begin{enumerate}
	\item $\pi$ is surjective
	\item Each fiber $\pi\inv(p)$ has the structure of a real vector space.
	\item For every $p \in M$, there exists a neighborhood $U$ of $p$ and
	a diffeomorphism $\Phi : \pi\inv(U) \to U \times \R^k$ such that we get
	the commutative diagram
	$$\begin{tikzcd}
	\pi\inv(U) \ar[dr, "\pi"'] \ar[rr, "\Phi"] &&
	U \times \R^k \ar[dl, "p"] \\
	& U
	\end{tikzcd}$$
	where $p$ denotes the standard projection $U \times \R^k \to U$. The
	map $\Phi$ is called a \ib{local trivialization}.
\end{enumerate}
The dimension $k$ of each fiber $\pi\inv(p)$ is called the \ib{rank} of
the vector bundle. We may just call $E$ the vector bundle, and the fiber
$\pi\inv(p)$ as $E_p$ where the map $\pi$ is implicit.
\end{defn}
%
\begin{exmp} \enumbreak
	\begin{enumerate}
		\item Given any manifold $M$ and a vector space $V$, the product
		manifold $M \times V$ equppied with the projection $M \times V \to M$
		is a vector bundle, called the \ib{trivial bundle}.
		\item The tangent bundle $TM$ of a smooth $n$-manifold $M$ is a
		vector bundle of rank $n$.
		\item Consider $\RP^n$, the space of lines in $\R^{n+1}$. A point
		in $\RP^n$ is a $1$-dimensional subspace of $\R^{n+1}$, so we can
		construct the vector bundle $E$ where the fiber over a point
		$\ell \in \RP^n$ is the the $1$-dimensional subspace $\ell$.
		This is called the \ib{tautological bundle} over $\RP^n$, and
		exists over more general spaces called \ib{Grassmannians}.
	\end{enumerate}
\end{exmp}
%
\begin{exer}
	Let $M$ and $N$ be smooth manifolds, $\pi : E \to N$ a vector bundle,
	and $F : M \to N$ a smooth map. Let
	$$F^*E = \set{(p, v) ~:~ p \in M, ~v \in \pi\inv(f(p))} $$
	Show that $F^*E$ is a vector bundle over $M$, called the
	\ib{pullback bundle}.
\end{exer}
%
\begin{defn}
Let $E \to M$ be a vector bundle. Then a \ib{local section} of $E$ is
a right inverse $\sigma:U \to E$ for some open subset $U \subset M$, i.e.
$\pi \circ \sigma = \id_U$. In the case that $U = M$, we call $\sigma$ a
\ib{global section}.
\end{defn}
%
For a vector bundle $E \to M$, there is a canonical global section called
the \ib{zero section}, which is the map $p \mapsto (p, 0)$. where $0$
denotes the $0$ vector in $E_p$. A lot of the concepts in differential
geometry can be expressed in terms of vector bundles and their sections,
especially when it comes to vector and tensor fields on manifolds.
%
\begin{exer}
	Prove that the tangent bundle $TS^2$ is not isomorphic to a trivial
	bundle, i.e. there exists no diffeomorphism $TS^2 \to S^2 \times \R^2$
	such that the diagram
	$$\begin{tikzcd}
	TS^2 \ar[dr]\ar[rr] && S^2 \times \R^2 \ar[dl]\\
	& S^2
	\end{tikzcd}$$
	commutes. (Hint : Use the Hairy Ball Theorem, which states that
	there does not exist an everywhere nonzero vector field on $S^2$.
	What would triviality of $TS^2$ imply about vector fields on $S^2$?)
\end{exer}
%
\end{document}
%
