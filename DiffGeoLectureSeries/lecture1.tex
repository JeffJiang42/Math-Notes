\section{Introduction to Manifolds}
%
\begin{defn}
A \ib{topological manifold} is a Hausdorff space $X$ such there exists a countable open cover $\set{U_\alpha}$ of $X$, along with homeomorphisms $\varphi_\alpha : U_\alpha \to V_\alpha$, where $V_\alpha$ is an open subset of $\R^n$.
\end{defn}
%
In this way, we see that a manifold is a topological space that is locally topologically indistinguishable from
Euclidean space. Both modifiers are important here -- there could be global and geometric properties that 
differ from $\R^n$.
%
\begin{exmp}\enumbreak 
	\begin{enumerate}
		\item $\R^n$ is a topological manifold -- it admits a global chart $(\R^n, \id_{\R^n})$.
		\item The $2$-sphere $S^2 = \set{(x,y,z) \in \R^3 ~:~ x^2 + y^2 + z^2 = 1}$ is a topological
		manifold. If we let $N$ and $S$ denote the North and South poles repsectively, then we can 
		construct charts using \ib{stereographic projection}. Stereographic projection from the North
		pole is a map $\varphi_N : S^2 - \set{N} \to \R^n$, where given a point $p \in S^2 - \set{N}$,
		we take the line cotaining both $N$ and $p$, which intersects the $z = 0 $ plane at one point 
		$q$. We then define $\varphi_N(p) = q$. Stereographic projection $\varphi_S$ from the South 
		pole is defined in an analogous manner. An explicit formula for $\varphi_N$ is 
		$$\varphi_N(x,y,z) = \left(\frac{x}{1 - z}, \frac{x}{1 - z} \right) $$
	\end{enumerate}
\end{exmp}
%
\begin{exer}\enumbreak
	\begin{enumerate}
		\item $S^2$ does not admit a global chart. Why?
		\item Give an inverse map $\R^n \to \S^n - {N}$ for stereographic projection from the 
			North Pole
		\item Generalize stereographic projection to the $n$-sphere 
		$S^n = \set{p \in \R^{n+1} ~:~ \norm{p} = 1}$
	\end{enumerate}
\end{exer}
