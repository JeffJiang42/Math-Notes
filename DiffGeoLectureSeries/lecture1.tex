\section{Introduction to Manifolds}
%
\begin{defn}
A \ib{topological manifold} is a Hausdorff space $X$ such there exists a countable open cover $\set{U_\alpha}$
of $X$, along with homeomorphisms $\varphi_\alpha : U_\alpha \to V_\alpha$, where $V_\alpha$ is an open subset
of $\R^n$. Given a chart $(U, \varphi)$, we can write $\varphi$ in terms of its component functions
$$\varphi(p) = (x^1(p), \ldots, x^n(p)) $$
The functions $x^i$ are called \ib{local coordinates} on $U$.
\end{defn}
%
In this way, we see that a manifold is a topological space that is locally topologically indistinguishable from
Euclidean space. Both modifiers are important here -- there could be global and geometric properties that
differ from $\R^n$.
%
\begin{exmp}\enumbreak
	\begin{enumerate}
		\item $\R^n$ is a topological manifold -- it admits a global chart $(\R^n, \id_{\R^n})$.
		\item The $2$-sphere $S^2 = \set{(x,y,z) \in \R^3 ~:~ x^2 + y^2 + z^2 = 1}$ is a topological
		manifold. If we let $N$ and $S$ denote the North and South poles repsectively, then we can
		construct charts using \ib{stereographic projection}. Stereographic projection from the North
		pole is a map $\varphi_N : S^2 - \set{N} \to \R^n$, where given a point $p \in S^2 - \set{N}$,
		we take the line cotaining both $N$ and $p$, which intersects the $z = 0 $ plane at one point
		$q$. We then define $\varphi_N(p) = q$. Stereographic projection $\varphi_S$ from the South
		pole is defined in an analogous manner. An explicit formula for $\varphi_N$ is
		$$\varphi_N(x,y,z) = \left(\frac{x}{1 - z}, \frac{x}{1 - z} \right) $$
	\end{enumerate}
\end{exmp}
%
\begin{exer}\enumbreak
	\begin{enumerate}
		\item $S^2$ does not admit a global chart. Why?
		\item Give an inverse map $\R^n \to \S^n - {N}$ for stereographic projection from the
			North Pole
		\item Generalize stereographic projection to the $n$-sphere
		$S^n = \set{p \in \R^{n+1} ~:~ \norm{p} = 1}$
	\end{enumerate}
\end{exer}
%
One of the main appeals of Euclidean space is that we have a lot of tools at our disposal, like linear algebra
and calculus. We would like to generalize these notions to manifolds. One of the wonderful properties of
manifolds is that they locally look like $\R^n$, so we can use the charts to translate concepts we know in
$\R^n$ to concepts on the manifold.
%
We know what it means for a function on $\R^n$ to be smooth, how do we
translate this to a manifold $X$? Given a map $F : X \to \R^m$, what does it mean for $F$ to be smooth? Our
first guess is to use our charts. Given a map $F: X \to \R^m$, we want
to say that it is smooth if for every $p \in X$, there exists a chart
$(U_p, \varphi_p)$ such that the composition
$F \circ \varphi_p\inv : U_p \to \R^m$ is smooth. The intuition here
is correct, there there are some technicalities that need to be addressed.
 Namely, suppose $p \in X$ lies in the domain of two charts
 $(U_1, \varphi_1)$, and $(U_2, \varphi_2)$. Then what if
 $F \circ \varphi_1\inv$ was smooth, but $F \circ \varphi_2\inv$ wasn't?
 This suggests that we need a compatibility condition for our charts
 if we want a notion of smoothness. We say two charts $(U_1, \varphi_1)$
 and $(U_2, \varphi_2)$ are \ib{smoothly compatible} if both
 $\varphi_1 \circ \varphi_2\inv$ and $\varphi_2 \circ \varphi_1\inv$ are
 smooth maps.
 \begin{defn}
	 A collection $\mathcal{A} = \set{(U_\alpha, \varphi_\alpha)}$ of
	 smoothly compatible charts such that the $U_\alpha$ cover $X$ is called
	 an \ib{atlas}. An atlas is \ib{maximal} if it is not property contained in any other atlas.
 \end{defn}
%
\begin{defn}
	A \ib{smooth manifold} is a topological manifold $X$ equipped with a maximal atlast $\mathcal{A}$.
\end{defn}
%
Now if we have a smooth manifold $X$, we have a notion of smooth maps
$X \to \R^n$
%
\begin{defn}
	A map $F : X \to \R^k$ is \ib{smooth} if for every chart
	$(U_\alpha, \varphi_\alpha$), we have that $F \circ \varphi_\alpha\inv$ is
	smooth in the Euclidean sense. We call $F \circ \varphi_\alpha\inv$ a local
	\ib{coordinate representation} of the map $F$.
\end{defn}
%
We also know what if means for maps between manifolds to be smooth.
%
\begin{defn}
	A map $F : X \to Y$ of smooth manifolds is $\ib{smooth}$ if for every $p$,
	there exists charts $(U, \varphi)$ and $(V, \psi)$ of $p$ and $F(p)$
	respectively, such that (after appropriate shrinking of $U$ and $V$), we
	have that $\psi \circ F \circ \varphi\inv$ is smooth in the Euclidean sense.
\end{defn}
%
For convenience, we will make the a distinction between the words map
and function, which is also common in the literature. A \emph{map} will denote
an arbitrary mapping $X \to Y$, where $X$ and $Y$ can denote any kind of set.
A \emph{function} on a space $X$ is a mapping $X \to \R$. We let $C^\infty(X)$
denote the vector space of functions of $X$, which forms a commutative ring
under pointwise multiplication and addition.\\

This is nice, but we've excluded a huge class of spaces. For example, the
unit interval isn't a manifold, since there exist no charts about the
endpoints satisfying our definition (why?). To include the spaces, we
need another definition. Let $\mathbb{H}^n$ denote
\ib{Euclidean half space}, where
$$\mathbb{H}^n = \set{(x^1, \ldots x^n) \in \R^n ~:~ x^n \geq 0} $$
%
\begin{defn}
	A \ib{manifold with boundary} is a Hausdorff space $X$ such that there
	exists a countable cover by charts $(U_\alpha, \varphi_\alpha)$ where
	the $\varphi_\alpha$ are homeomorphisms to open sets in $\mathbb{H}^n$.
	Given a point $p \in X$, we say that $p$ is in the \ib{boundary} of
	$X$ if there exists a chart $(U, \varphi)$ containing $p$ such that
	$\varphi(p) = (x^1, \ldots, x^n)$, and $x^n = 0$. We say that $p$ is in
	the \ib{interior} of $X$ if no such chart exists. We denote the boundary
	and interior of $X$ as $\bd X$ and $\inter X$ respectively.
\end{defn}
%
Under this definition, $\mathbb{H}^n$ is a smooth manifold with boundary,
with a global chart given by $(\mathbb{H}^n, \id_{\mathbb{H}^n})$, and
$\bd \mathbb{H}^n = \set{(x^1, \ldots x^n) \in \mathbb{H}^n ~:~ x^n = 0}$
%
\begin{exer}
	Let $X$ be a smooth $n$-manifold with boundary
	\begin{enumerate}
		\item Show that $\bd X$ is
		well defined. Namely, show that if we have a point $p \in X$ such that
		there exists a chart $(U, \varphi)$ where $\varphi_1(p) \in \bd
		\mathbb{H}^n$, show that any other chart $(U', \varphi')$ for $p$ must
		also have $\varphi'(p) \in \bd \mathbb{H^n}$. (Hint: Use the inverse
		function theorem to show there exists no diffeomorphism from an open set
		in $\R^n$ to an open set in $\R^{n-1}$)
		\item Prove that $\bd X$ is a smooth $(n-1)$-manifold without boundary
	\end{enumerate}
\end{exer}
%
