\section{Introduction to Manifolds}
%
\begin{defn}
A \ib{topological manifold} is a Hausdorff space $X$ such there exists a countable open cover $\set{U_\alpha}$
of $X$, along with homeomorphisms $\varphi_\alpha : U_\alpha \to V_\alpha$, where $V_\alpha$ is an open subset 
of $\R^n$. Given a chart $(U, \varphi)$, we can write $\varphi$ in terms of its component functions 
$$\varphi(p) = (x^1(p), \ldots, x^n(p)) $$
The functions $x^i$ are called \ib{local coordinates} on $U$.

\end{defn}
%
In this way, we see that a manifold is a topological space that is locally topologically indistinguishable from
Euclidean space. Both modifiers are important here -- there could be global and geometric properties that 
differ from $\R^n$.
%
\begin{exmp}\enumbreak 
	\begin{enumerate}
		\item $\R^n$ is a topological manifold -- it admits a global chart $(\R^n, \id_{\R^n})$.
		\item The $2$-sphere $S^2 = \set{(x,y,z) \in \R^3 ~:~ x^2 + y^2 + z^2 = 1}$ is a topological
		manifold. If we let $N$ and $S$ denote the North and South poles repsectively, then we can 
		construct charts using \ib{stereographic projection}. Stereographic projection from the North
		pole is a map $\varphi_N : S^2 - \set{N} \to \R^n$, where given a point $p \in S^2 - \set{N}$,
		we take the line cotaining both $N$ and $p$, which intersects the $z = 0 $ plane at one point 
		$q$. We then define $\varphi_N(p) = q$. Stereographic projection $\varphi_S$ from the South 
		pole is defined in an analogous manner. An explicit formula for $\varphi_N$ is 
		$$\varphi_N(x,y,z) = \left(\frac{x}{1 - z}, \frac{x}{1 - z} \right) $$
	\end{enumerate}
\end{exmp}
%
\begin{exer}\enumbreak
	\begin{enumerate}
		\item $S^2$ does not admit a global chart. Why?
		\item Give an inverse map $\R^n \to \S^n - {N}$ for stereographic projection from the 
			North Pole
		\item Generalize stereographic projection to the $n$-sphere 
		$S^n = \set{p \in \R^{n+1} ~:~ \norm{p} = 1}$
	\end{enumerate}
\end{exer}
%
One of the main appeals of Euclidean space is that we have a lot of tools at our disposal, like linear algebra
and calculus. We would like to generalize these notions to manifolds. One of the wonderful properties of 
manifolds is that they locally look like $\R^n$, so we can use the charts to translate concepts we know in 
$\R^n$ to concepts on the manifold. 
%
\begin{defn}
	A map $F : U \to V$ where $U \subset \R^n$ and $V \subset \R^m$ is \ib{smooth} if all of its 
	component functions are infinitely differentiable, i.e. the partial derivatives of all orders 
	exist and are continuous.
\end{defn}
%
Note that smoothness is a local condition -- to know a function is smooth at a point, it suffices to check 
in a small neighborhood. Now that we know what it means for a function on $\R^n$ to be smooth, how do we 
translate this to a manifold $X$? Given a map $F : X \to \R^m$, what does it mean for $F$ to be smooth? Our 
first guess is to use our charts. Given 

