\documentclass{article}

% Expand margins
\addtolength{\oddsidemargin}{-.75in}
\addtolength{\evensidemargin}{-.75in}
\addtolength{\textwidth}{1.5in}
\addtolength{\topmargin}{-.75in}
\addtolength{\textheight}{1.5in}
\linespread{1}

% Packages
\usepackage{amsmath}
\usepackage{amssymb}
\usepackage{amsthm}
\usepackage{comment}
\usepackage{enumitem}
\usepackage{graphicx}
\usepackage{parskip} % automatically removes paragraph indents
\usepackage{tabu} % allows for creation of tables inside math mode
\usepackage{thmtools}
\usepackage{tikz}
\usepackage{tikz-cd}
\usepackage{mathpazo}

\usetikzlibrary{matrix, calc, arrows}

% Blackboard bold
\newcommand{\C}{\mathbb{C}}
\newcommand{\F}{\mathbb{F}}
\newcommand{\N}{\mathbb{N}}
\newcommand{\Q}{\mathbb{Q}}
\newcommand{\R}{\mathbb{R}}
\newcommand{\Z}{\mathbb{Z}}
\newcommand{\RP}{\mathbb{RP}}
\DeclareMathOperator{\Span}{Span}
% New theorems
\newtheorem{definition}{Definition}[subsection]
\newtheorem{theorem}{Theorem}[section]
\newtheorem*{corollary}{Corollary}
\newtheorem{proposition}{Proposition}[subsection]
\newtheorem{lemma}{Lemma}[subsection]
\newtheorem*{remark}{Remark}

\newcommand{\Partialx}{\frac{\partial }{\partial x}}
\newcommand{\Partialy}{\frac{\partial }{\partial y}}

\DeclareMathOperator{\GL}{GL}
\DeclareMathOperator{\tr}{tr}
\DeclareMathOperator{\id}{id}

\begin{document}
\title{Clifford Algebras}
\author{Jeffrey Jiang}
\date{}
\maketitle
%
An important group in mathematics is the \textbf{orthogonal group} $O_n$, which consists of linear transformations $T: \R^n \to \R^n$ such that $\langle Tv, Tw \rangle = \langle v, w \rangle$ for all $v, w \in \R^n$ (such transformations are said to preserve the inner product). This inner product induces a norm $| \cdot |$ on $\R^n$ given by $|v| = \langle v,v \rangle^{1/2}$, which in turn introduces a notion of length in $\R^n$. This also gives us a notion of the angle $\theta$ between two vectors $v,w \in \R^n$ via
$$\theta = \arccos\left( \frac{|\langle v,w \rangle|}{|v| |w|} \right).$$
Since $O_n$ preserves the inner product, it also preserves these notions of angle and length. The induced norm makes $\R^n$ into a metric space with distance function $d : \R^n \times \R^n \to \R$ given by $d(v,w) = |v - w|$. It follows that $O_n$ is the group of linear isometries of the metric space $(\R^n, d)$. 

The orthogonal group plays a large role in physics, since many of the linear transformations we want to observe in the natural world preserve our perceptions of angle and length. For intuition, try thinking about what kind of shape these groups can have. In one dimension, we don't have too much to work with since any linear map preserving the absolute value must be multiplication by $\pm 1$. Therefore, we find that $O_1 = \{\pm \id\}$. 

Things are a little more interesting in two dimensions. Note first of all that any $A \in O_2$ preserving length and angle must necessarily preserve the unit circle $S^1 \subset \R^2$. Any such transformation is then (almost) uniquely determined by the angle by which it rotates a single vector, as well as whether it flips the orientation of $\R^2$. By orientation, we can say in a heuristic sense that it is a choice of clockwise or counterclockwise. A rotation by some angle $\theta$ preserves orientation, but a reflection, say across the $y$-axis, flips the orientation. From our description, we now see that $O_2$ should look like two disjoint circles: one circle for all the rotations, and another separate circle of rotations following a reflection. We call the orientation preserving component $SO_2$, which you can think of as the group of rotations. The components of $O_2$ form a group, which we denote $\pi_0(O_2)$. This follows from the fact that the composition of two orientation reversing transformations preserves orientation, which gives a group isomorphism $\pi_0(O_2) \cong \Z / 2\Z$.

Going back a bit, why did we say that the transformation is \emph{almost} uniquely determined by the angle? The reason is that a rotation by angle $\theta$ is the same as a rotation by $2\pi - \theta$. This means that $SO_2$ might not be parameterized by a circle as we might have thought! In fact, though, $SO_2$ is isomorphic to a circle (why?) -- a fact that does not generalize to higher dimensions. Indeed, the same observation as above in the three-dimensional case gives us that $SO_3$ is isomorphic not to $S^2$ but rather to $\RP^3$! It's a good mental exercise to figure out what happens here, and why the two- and three-dimensional cases are different. You might want the think of $\RP^3$ as the unit sphere $S^3 \subset \R^4$ with antipodal points identified via $v \sim -v$. Can you see why this is the right picture? If not, don't worry: after some work we will have a more rigorous explanation.

Summarizing our results from tinkering around with orthogonal groups in low dimensions:

\begin{enumerate}
\item The composition of an even number of orientation reversing maps is orientation preserving, i.e. $\pi_0(O_n) \cong \Z/2\Z$.
\item Using spheres to describe orthogonal transformations has redundancies, which can be attributed to antipodal points. Somehow, $v$ and $-v$ encode the same data for an orthogonal transformation.
\end{enumerate}

Before constructing the Clifford algebra, we need a definition.

\vspace{0.5em}

\begin{definition}
Let $V$ be a real vector space of dimension $n$. A \textbf{hyperplane} in $V$ is an $(n-1)$-dimensional subspace $P \subset V$.
\end{definition}

\vspace{0.5em}

If $V$ has an inner product then any hyperplane $P$ is determined by the line given by its orthogonal complement $P^\perp$. In addition, since we have a notion of length, a line is determined by the unique vector $v \in P$ with norm $1$ (called the unit normal). If $V$ doesn't have an inner product then there is in general no way to make these distinctions. But wait, something is wrong with what we just said! There isn't a unique vector with norm $1$ in $P$: $-v$ is just as good. This is one of the key points we observed about the orthogonal group, which suggests that something deeper is going on here. We can capture this deeper relation by taking a hyperplane $P \subset \R^n$ and defining a map $R_P : \R^n \to \R^n$ that reflects $\R^n$ about $P$ (e.g. reflection about a line in $\R^2$ or reflection about a plane in $\R^3$). If we let $v$ be one of the two unit normal vectors then $R_P$ is given by the formula
$$R_P(w) = w -2\langle w, v \rangle v.$$
How do we interpret the formula? A hyperplane reflection should not change the components of a vector that lie in $P$, instead flipping only the component orthogonal to $P$. This is accomplished exactly by subtracting $2\langle w,v \rangle v$ from $w$. 

We're now ready for the final piece of the puzzle.

\vspace{0.5em}

\begin{theorem}[\textbf{Cartan-Dieudonn\'e}]
Any orthogonal transformation $A \in O_n$ can be written as a composition of $\leq n$ hyperplane reflections
\end{theorem}

\begin{proof}
The proof is by induction on $n$. The case $n=1$ is easy since the only hyperplane in $\R^1$ is the zero vector. The only elements of $O_1$ are $\pm \id$. $-\id$ is the composition of a single hyperplane reflection (itself), while $\id$ is by convention a composition of no hyperplane reflections.

Assume now that the result holds for $n-1$. Fix $A \in O_n$ and a nonzero vector $v \in \R^n$. We want to find a hyperplane reflection $R : \R^n \to \R^n$ such that $RAv = v$ (the reason why will soon be apparent). To do this, let $R$ be the hyperplane reflection about the plane bisecting $v$ and $Av$. Explicitly, $R$ is given by the formula
$$Rw = w - 2\frac{\langle Av - v, v \rangle}{\langle Av -v, Av - v\rangle}v.$$
Then we have that $RA$ is an orthogonal transformation that fixes $v$, since 
\begin{align*}
Rv &= v - \frac{2 \langle Av - v, v \rangle}{\langle Av -v, Av -v \rangle} v \\
&= v - \frac{2\langle Av, v \rangle - 2\langle v, v \rangle}{\langle Av, Av \rangle - 2\langle Av, v \rangle + \langle v,v \rangle}v \\
&= v - \frac{2\langle Av,v \rangle - 2 \langle v,v \rangle}{2 \langle Av,v \rangle - 2 \langle v,v \rangle}v \\
&= v - (-1)(Av -v) \\
&= Av.
\end{align*}
Since $R^2 = \id$, $R$ interchanges $v$ and $Av$ as desired. Since $R$ is orthogonal, fixing $v$ is equivalent to fixing its orthogonal complement $v^\perp$ and so $R$ restricts to an orthogonal transformation on the $(n-1)$-dimensional subspace $v^\perp$. By the inductive hypothesis, this is a composition of $\leq n-1$ hyperplane reflections in $v^\perp$, which can be extended to hyperplane reflections in all of $\R^n$ by appending the vector $v$ to a basis for each hyperplane and taking the corresponding reflection, so $RA$ is a composition of $\leq n-1$ hyperplane reflections. Then composing with $R$ again (since $R^2 = \id$) gives us that $R^2A = A$, so $A$ is also a composition of at most $n$ hyperplane reflections.
\end{proof}

% Editor's Note: I'm a bit concerned about the step in the above proof at which you apply the inductive hypothesis to the restriction of R to v^{\perp}. It seems you have implicitly identified this space with \R^{n-1}, and I think the choice of linear isomorphism matters since you need to know that hyperplane reflections get sent to hyperplane reflections.

The stage is now set for the construction of the Clifford algebra.

\vspace{0.5em}

\begin{definition}
Equip $\R^n$ with the standard inner product $\langle \cdot,\cdot \rangle$. The \textbf{Clifford algebra} for $\R^n$ is a unital associative $\R$-algebra $\mathrm{Cliff}(n)$ generated by $\R^n$, subject to the relations
\begin{enumerate}
\item $v^2 = -1$ if $|v| = 1$,
\item $vw = -wv$ if $\langle v,w \rangle = 0$.
\end{enumerate}
\end{definition}

\vspace{0.5em}

By unital, we mean that we have thrown in a new element $e$ that acts as the multiplicative unit. Therefore, we have that for any $\lambda \in \R$ and $v \in \mathrm{Cliff}(n)$, we must have $\lambda e \cdot v = \lambda v$, so by a slight abuse of notation, we follow the standard convention in which the multiplicative unit is denoted $1$. When we say that $\mathrm{Cliff}(n)$ is generated by $\R^n$, we mean that every element of $\mathrm{Cliff}(n)$ can be written as a finite formal sum of products of vectors in $\R^n$ (which then reduces by the relations specified above). Therefore, the standard basis $\{e_i\}$ of $\R^n$ is a generating set for $\mathrm{Cliff}(n)$, yielding the basis
$$\{e_{i_1} \ldots e_{i_k} ~:~ 0 \leq k \leq n, 1 \leq i_1, \ldots i_k \leq n\}.$$
Thus, a basis for $\mathrm{Cliff}(n)$ is just the set of all products of basis vectors with increasing indices, along with $1$. For example, a basis for $\mathrm{Cliff}(3)$ as a vector space is given by
$$\{1, e_1, e_2, e_3, e_1e_2, e_1e_3, e_2e_3, e_1e_2e_3\}.$$
This characterization allows us to see that the dimension of $\mathrm{Cliff}(n)$ as a real vector space is $2^n$. We also note that $\mathrm{Cliff}(n)$ contains a subspace
$$\Span(e_1, \ldots, e_n) \cong \R^n.$$

What is the motivation for this definition? A unit vector $v \in \R^n \subset \mathrm{Cliff}(n)$ should represent a hyperplane reflection about the plane perpendicular to $v$. In particular, since hyperplane reflections square to the identity, we want the same to be true here (but with an added sign).\footnote{Our choice of sign is simply convention: the choice $v^2 = 1$ works just as good. One benefit of our choice is that some formulas will look cleaner.} Another item of motivation is that hyperplane reflections about planes determined by orthogonal vectors commute. Try messing around with such reflections in the two- and three-dimensional settings: pay attention to signs!

We now derive a helpful formula.

\vspace{0.5em}

\begin{lemma}
Let $v,w \in \R^n \subset \mathrm{Cliff}(n)$. Then, $vw + wv = -2\langle v,w \rangle$.
\end{lemma}

\vspace{0.5em}

\begin{proof}
Let $v = \sum_i v^ie_i$ and $w = \sum_j w^je_j$. We compute
\begin{align*}
vw + wv &= \sum_i \sum_j (v^iw^j e_ie_j + v^iw^je_je_i) \\
&= \sum_i\sum_j v^iw^j(e_ie_j + e_je_i) 
\end{align*}
Since $e_i^2 = -1$ and $\langle e_i, e_j \rangle$ is $0$ for $i \neq j$ and $1$ for $i = j$, this sum collapses to 
$$\sum_i v^iw^i(e_i^2 + e_i^2) = -2\sum_iv^iw^i  = -2\langle v,w \rangle.$$
\end{proof}

Another useful observation about $\mathrm{Cliff}(n)$ is that it comes with a linear map $T : \mathrm{Cliff}(n) \to \mathrm{Cliff}(n)$ such that $T^2 = \id$ (called an \textit{involution}). This map is uniquely determined by reversing the order of products of basis vectors \`{a} la
$$T(e_{i_1} \ldots e_{i_k}) = e_{i_k} \ldots e_{i_1} = e_{i_1} \ldots e_{i_k}$$
and then extending linearly to the rest of $\mathrm{Cliff}(n)$. For an arbitrary $g \in \mathrm{Cliff(n)}$, we use the notation $g^T = T(g)$. The similarity to the notation for matrix transposition is not a coincidence!

Consider now the multiplicative group $\mathrm{Cliff}(n)^\times$ of invertible elements of $\mathrm{Cliff}(n)$. Note that we have to be a little careful working with this group since it is nonabelian. There is a nice subgroup $G \subset \mathrm{Cliff}(n)^\times$ generated by the unit vectors of $\R^n$. Identifying $\R^n$ as a subspace of $\mathrm{Cliff}(n)$, we claim that there is a (somewhat) natural left action of $G$ on $\R^n$ given by 
$$g\cdot w = gwg^T $$
for $g \in G$ and $w \in \R^n$. Of course, at the outset we have no reason to even believe that $gwg^T$ is an element of $\R^n$. To verify that we do in fact have a well-defined action of $G$, it suffices to check on the generating set of unit vectors. Let $v \in \R^n$ with $|v| = 1$. Writing $v = \sum_i v^ie_i$, we have
$$v^T = T(v) = T \left( \sum_i v^ie_i \right) = \sum_i v^iT(e_i) = \sum_i v^ie_i = v.$$
Using the lemma, we compute
\begin{align*}
v\cdot w &= vwv \\ 
&= (-2 \langle v,w \rangle - wv)v \\
&= -2\langle v,w \rangle v -wv^2 \\
&= w - 2\langle v,w \rangle v \in \R^n,
\end{align*}
which is exactly \emph{hyperplane reflection} about the hyperplane perpendicular to $v$! Thus, not only do we have a group action on our hands, we also know that the generating set for $G$ acts exactly as the generating set for $O_n$. This gives us a natural group homomorphism $\varphi : G \to O_n$ that sends $g \in G$ to the linear transformation $w \mapsto g\cdot w$. This map is surjective by the Cartan-Dieudonn\'{e} Theorem. 

A little work gives that $\ker \varphi = \{\pm 1\}$ and so $\varphi$ is a 2-to-1 map. 

% Editor's Note: I'm not sure what you mean by -1. My understanding is that 1 denotes the empty product in Cliff(n), and so -1 = 1.

In fact, we can see that $(-v)w(-v) = vwv$, which is exactly the fact that hyperplane reflection about $v^\perp$ and $-v^\perp$ are exactly the same. However, in our larger group $G$, we are able to distinguish between $v$ and $-v$. In addition, we have that any $g \in G$ that is a product of an even number of unit vectors, the the map determined by $g$ will be orientation preserving, since it will be an even number of hyperplane reflections. What did we just discover? We discovered the group $G = \mathrm{Pin}(n)$, and the subgroup generated by even products is $\mathrm{Spin}(n)$, which are the double covers of $O_n$ and $SO_n$ respectively. Thus in one fell swoop, we've addressed two of our earlier observations with the orthogonal group and encoded it in a new mathematical object -- the Clifford algebra. Because of this very geometric interpretation of how multiplication behaves, some call the Clifford algebra Clifford's geometric algebra, as multiplication in $\mathrm{Cliff}(n)$ seems to encode the geometry of $\R^n$.

% Editor's Note: The above paragraph needs further editing.

It's a useful exercise to explore exactly what these Clifford algebras are in more familiar terms. For example, $\mathrm{Cliff}(1)$ is isomorphic to the $\R$-algebra $\C$ of complex numbers and $\mathrm{Cliff}(2)$ is isomorphic to the $\R$-algebra $\mathbb{H}$ of quaternions, both of which have an involution given by conjugation $q \mapsto \bar{q}$. If you're familiar with computer graphics, you might recall that rotations are often more compactly represented as quaternions, where the action of a quaternion $q \in \mathbb{H}$ on $\R^3$ is given by $v \mapsto qv\bar{q}$. This formula should look awfully familiar: it tells us that $\mathrm{Spin}(3)$ is isomorphic the multiplicative group $\mathbb{H}^{\times}$ of unit quaternions. The $2$-to-$1$ map $\varphi$ onto $SO_3$ is just the quotient by antipodal points, giving an isomorphism $SO_3 \cong \RP^3$.

Although the geometric insights involving $\mathrm{Pin}$ and $\mathrm{Spin}$ are nice, there's a very beautiful theory concerning the algebras themselves. If we replace the inner product $\langle \cdot,\cdot \rangle$ with an arbitrary symmetric bilinear form $b : \R^n \times \R^n \to \R$ then we can repeat the same construction as above to give $\mathrm{Cliff}(\R^n ,b)$. Every nondegenerate symmetic bilinear form is uniquely determined by its signature -- the number of $1$'s and $-1$'s on the diagonal of its matrix after diagonalizing -- so we get an infinite family of algebras $\mathrm{Cliff}(p,q)$ (where $p$ denotes the number of $1$'s and $q$ the number of $-1$'s). Amazingly, this collection of Clifford algebras is closed under taking tensor products. Via something called \textbf{Bott periodicity}, it turns out that $\mathrm{Cliff}(0,n)$ and $\mathrm{Cliff}(n,0)$ for $0 \leq n \leq 8$ provide us with enough information to reconstruct \emph{all} Clifford algebras by taking tensor products. 

With that, I leave you with the abstract definition of a Clifford algebra, which is the usual context in which one first sees it. Can you see why this abstract definition agrees with our own definition?

\vspace{0.5em}

\begin{definition}
Given a vector space $V$ and a symmetric bilinear form $b : V \times V \to \R$, the \textbf{Clifford algebra} is the data of a unital associative algebra $\mathrm{Cliff}(V,b)$ along with a linear map $\iota : V \to \mathrm{Cliff}(V,b)$ such that, for any linear map $\varphi : V \to A$ of $V$ into another unital associative algebra $A$ satisfying 
$$\varphi(v)^2 = b(v,v),$$
$\varphi$ factors through $\mathrm{Cliff}(V,b)$ to give a unique map $\tilde{\varphi} : \mathrm{Cliff}(V,b) \to A$ such that the diagram
$$\begin{tikzcd} 
V \ar[d, "\iota"'] \ar[dr, "\varphi"] \\
\mathrm{Cliff}(V,b) \ar[r, "\exists! \tilde{\varphi}"', dashed] & A
\end{tikzcd}$$
commutes.
\end{definition}
%
\end{document}