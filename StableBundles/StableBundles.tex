\documentclass[psamsfonts, 12pt]{amsart}
%
%-------Packages---------
%
\usepackage[h margin=1 in, v margin=1 in]{geometry}
\usepackage{amssymb,amsfonts}
\usepackage{amsmath}
\usepackage{accents}
\usepackage[all,arc]{xy}
\usepackage{tikz-cd}
\usepackage{enumerate}
\usepackage{mathrsfs}
\usepackage{amsthm}
\usepackage{mathpazo}
\usepackage{float}
\usepackage[backend=biber]{biblatex}
\addbibresource{bibliography.bib}
%\usepackage{charter} %another font
%\usepackage{eulervm} %Vakil font
\usepackage{yfonts}
\usepackage{mathtools}
\usepackage{enumitem}
\usepackage{mathrsfs}
\usepackage{fourier-orns}
\usepackage[all]{xy}
\usepackage{hyperref}
\usepackage{url}
\usepackage{mathtools}
\usepackage{graphicx}
\usepackage{pdfsync}
\usepackage{mathdots}
\usepackage{calligra}
\usepackage{import}
\usepackage{xifthen}
\usepackage{pdfpages}
\usepackage{transparent}

\usepackage{tgpagella}
\usepackage[T1]{fontenc}
%
\usepackage{listings}
\usepackage{color}

\definecolor{dkgreen}{rgb}{0,0.6,0}
\definecolor{gray}{rgb}{0.5,0.5,0.5}
\definecolor{mauve}{rgb}{0.58,0,0.82}

\lstset{frame=tb,
  language=Matlab,
  aboveskip=3mm,
  belowskip=3mm,
  showstringspaces=false,
  columns=flexible,
  basicstyle={\small\ttfamily},
  numbers=none,
  numberstyle=\tiny\color{gray},
  keywordstyle=\color{blue},
  commentstyle=\color{dkgreen},
  stringstyle=\color{mauve},
  breaklines=true,
  breakatwhitespace=true,
  tabsize=3
  }
%
%--------Theorem Environments--------
%
\newtheorem{thm}{Theorem}[section]
\newtheorem*{thm*}{Theorem}
\newtheorem{cor}[thm]{Corollary}
\newtheorem{prop}[thm]{Proposition}
\newtheorem{lem}[thm]{Lemma}
\newtheorem*{lem*}{Lemma}
\newtheorem{conj}[thm]{Conjecture}
\newtheorem{quest}[thm]{Question}
%
\theoremstyle{definition}
\newtheorem{defn}[thm]{Definition}
\newtheorem*{defn*}{Definition}
\newtheorem{defns}[thm]{Definitions}
\newtheorem{con}[thm]{Construction}
\newtheorem{exmp}[thm]{Example}
\newtheorem{exmps}[thm]{Examples}
\newtheorem{notn}[thm]{Notation}
\newtheorem{notns}[thm]{Notations}
\newtheorem{addm}[thm]{Addendum}
\newtheorem{exer}[thm]{Exercise}
%
\theoremstyle{remark}
\newtheorem{rem}[thm]{Remark}
\newtheorem*{claim}{Claim}
\newtheorem*{aside*}{Aside}
\newtheorem*{rem*}{Remark}
\newtheorem*{hint*}{Hint}
\newtheorem*{note}{Note}
\newtheorem{rems}[thm]{Remarks}
\newtheorem{warn}[thm]{Warning}
\newtheorem{sch}[thm]{Scholium}
%
%--------Macros--------
\renewcommand{\qedsymbol}{$\blacksquare$}
\renewcommand{\sl}{\mathfrak{sl}}
\newcommand{\Bord}{\mathsf{Bord}}
\renewcommand{\hom}{\mathsf{Hom}}
\renewcommand{\emptyset}{\varnothing}
\renewcommand{\O}{\mathcal{O}}
\newcommand{\R}{\mathbb{R}}
\newcommand{\ib}[1]{\textbf{\textit{#1}}}
\newcommand{\Q}{\mathbb{Q}}
\newcommand{\Z}{\mathbb{Z}}
\newcommand{\N}{\mathbb{N}}
\newcommand{\C}{\mathbb{C}}
\newcommand{\A}{\mathbb{A}}
\newcommand{\F}{\mathbb{F}}
\newcommand{\M}{\mathcal{M}}
\newcommand{\dbar}{\overline{\partial}}
\newcommand{\zbar}{\overline{z}}
\renewcommand{\S}{\mathbb{S}}
\newcommand{\V}{\vec{v}}
\newcommand{\RP}{\mathbb{RP}}
\newcommand{\CP}{\mathbb{CP}}
\newcommand{\B}{\mathcal{B}}
\newcommand{\GL}{\mathrm{GL}}
\newcommand{\SL}{\mathrm{SL}}
\newcommand{\SP}{\mathrm{SP}}
\newcommand{\SO}{\mathrm{SO}}
\newcommand{\SU}{\mathrm{SU}}
\newcommand{\gl}{\mathfrak{gl}}
\newcommand{\g}{\mathfrak{g}}
\newcommand{\Bun}{\mathsf{Bun}}
\newcommand*{\dt}[1]{%
   \accentset{\mbox{\large\bfseries .}}{#1}}
\newcommand{\inv}{^{-1}}
\newcommand{\bra}[2]{ \left[ #1, #2 \right] }
\newcommand{\set}[1]{\left\lbrace #1 \right\rbrace}
\newcommand{\abs}[1]{\left\lvert#1\right\rvert}
\newcommand{\norm}[1]{\left\lVert#1\right\rVert}
\newcommand{\transv}{\mathrel{\text{\tpitchfork}}}
\newcommand{\defeq}{\vcentcolon=}
\newcommand{\enumbreak}{\ \\ \vspace{-\baselineskip}}
\let\oldexists\exists
\renewcommand\exists{\oldexists~}
\let\oldL\L
\renewcommand\L{\mathfrak{L}}
\makeatletter
\newcommand{\incfig}[2]{%
    \fontsize{48pt}{50pt}\selectfont
    \def\svgwidth{\columnwidth}
    \scalebox{#2}{\input{#1.pdf_tex}}
}
%
\newcommand{\tpitchfork}{%
  \vbox{
    \baselineskip\z@skip
    \lineskip-.52ex
    \lineskiplimit\maxdimen
    \m@th
    \ialign{##\crcr\hidewidth\smash{$-$}\hidewidth\crcr$\pitchfork$\crcr}
  }%
}
\makeatother
\newcommand{\bd}{\partial}
\newcommand{\lang}{\begin{picture}(5,7)
\put(1.1,2.5){\rotatebox{45}{\line(1,0){6.0}}}
\put(1.1,2.5){\rotatebox{315}{\line(1,0){6.0}}}
\end{picture}}
\newcommand{\rang}{\begin{picture}(5,7)
\put(.1,2.5){\rotatebox{135}{\line(1,0){6.0}}}
\put(.1,2.5){\rotatebox{225}{\line(1,0){6.0}}}
\end{picture}}
\DeclareMathOperator{\id}{id}
\DeclareMathOperator{\im}{Im}
\DeclareMathOperator{\codim}{codim}
\DeclareMathOperator{\coker}{coker}
\DeclareMathOperator{\supp}{supp}
\DeclareMathOperator{\inter}{Int}
\DeclareMathOperator{\sign}{sign}
\DeclareMathOperator{\sgn}{sgn}
\DeclareMathOperator{\indx}{ind}
\DeclareMathOperator{\alt}{Alt}
\DeclareMathOperator{\Aut}{Aut}
\DeclareMathOperator{\trace}{trace}
\DeclareMathOperator{\ad}{ad}
\DeclareMathOperator{\End}{End}
\DeclareMathOperator{\Ad}{Ad}
\DeclareMathOperator{\Lie}{Lie}
\DeclareMathOperator{\spn}{span}
\DeclareMathOperator{\dv}{div}
\DeclareMathOperator{\grad}{grad}
\DeclareMathOperator{\Sym}{Sym}
\DeclareMathOperator{\tr}{tr}
\DeclareMathOperator{\sheafhom}{\mathscr{H}\text{\kern -3pt {\calligra\large om}}\,}
\newcommand*\myhrulefill{%
   \leavevmode\leaders\hrule depth-2pt height 2.4pt\hfill\kern0pt}
\newcommand\niceending[1]{%
  \begin{center}%
    \LARGE \myhrulefill \hspace{0.2cm} #1 \hspace{0.2cm} \myhrulefill%
  \end{center}}
\newcommand*\sectionend{\niceending{\decofourleft\decofourright}}
\newcommand*\subsectionend{\niceending{\decosix}}
\def\upint{\mathchoice%
    {\mkern13mu\overline{\vphantom{\intop}\mkern7mu}\mkern-20mu}%
    {\mkern7mu\overline{\vphantom{\intop}\mkern7mu}\mkern-14mu}%
    {\mkern7mu\overline{\vphantom{\intop}\mkern7mu}\mkern-14mu}%
    {\mkern7mu\overline{\vphantom{\intop}\mkern7mu}\mkern-14mu}%
  \int}
\def\lowint{\mkern3mu\underline{\vphantom{\intop}\mkern7mu}\mkern-10mu\int}
%
%--------Hypersetup--------
%
\hypersetup{
    colorlinks,
    citecolor=black,
    filecolor=black,
    linkcolor=blue,
    urlcolor=blacksquare
}
%
%--------Solution--------
%
\newenvironment{solution}
  {\begin{proof}[Solution]}
  {\end{proof}}
%
%--------Graphics--------
%
%\graphicspath{ {images/} }
%
\begin{document}
%
\author{Jeffrey Jiang}
%
\title{Stable Vector Bundles on Riemann Surfaces}
%
\maketitle
%
\section{The Harder-Narasimhan Filtration}
%
\begin{defn}
Let $M$ be a compact Riemann surface and $E \to M$ a holomorphic vector bundle.
The complex structure of $M$ induces a natural orientation on $M$, giving an
isomorphism $H^2(M,\Z) \cong \Z$. The \ib{degree} of $E$, denoted
$\mathrm{deg}(E)$, is the image of the first Chern class $c_1(E)$ under this
isomorphism. The \ib{slope} of $E$, denoted $\mu(E)$, is the ratio
\[
\mu(E) = \frac{\mathrm{deg}(E)}{\mathrm{rank}(E)}
\]
\end{defn}
%
\begin{rem*}
Note that for a holomorphic vector bundle $E \to M$ of rank $k$, we have
\[
\deg(E) = \deg(\Lambda^kE)
\]
\end{rem*}
%
One important property of degree is that it places heavy limitations on the existence
of holomorphic sections.
%
\begin{prop}
Let $L \to M$ be a holomorphic line bundle over a compact Riemann surface $M$ with
$\deg(L) < 0$. Then $H^0(M,L) = 0$.
\end{prop}
%
\begin{proof}
Since $L$ is a negative line bundle, we have that $L^*$ is a positive line bundle.
Kodaira vanishing gives us that $H^1(M, L^* \otimes K_M) = 0$, where
$K_M \defeq \Omega^{1,0}_M$ is the canonical bundle. Serre duality then
tells us that $H^0(M, L) = 0$.
\end{proof}
%
A different, lower tech proof follows almost immediately from an alternate
characterization of degree.
%
\begin{prop}
Let $L \to M$ be a holomorphic line bundle over a compact Riemann surface $M$
of degree $n$. Then for any nonzero meromorphic section $\sigma$ of $L$,
the number of zeroes minus the number of poles, counted with multiplicity,
is equal to $n$.
\end{prop}
%
\begin{cor}
Let $L_1,L_2 \to M$ be holomorphic line bundles over a compact Riemann surface $M$.
If there exists a nonzero holomorphic bundle homomorphism $\varphi : L_1 \to L_2$, then
$\deg(L_1) \leq \deg(L_2)$, with equality if and only if $\varphi$ is an isomorphism.
\end{cor}
%
\begin{proof}
We use the bundle isomorphism $\hom(L_1,L_2) \cong L_1^* \otimes L_2$. Since
degree is additive on tensor products, we have
$\deg(L_1^* \otimes L_2) = \deg(L_2) - \deg(L_2)$. Therefore, if
$\deg(L_1)$ is less than $\deg(L_2)$, we know that $\hom(L_1,L_2)$ is of negative
degree, so it admits no nonzero global sections. \\

For the other part, we note that if the degrees of $L_1$ and $L_2$ are equal,
then $\hom(L_1,L_2)$ is a degree $0$ bundle. We then note that if $\hom(L_1,L_2)$
admits a nonzero holomorphic section, it must necessarily have no zeroes,
which implies that it is isomorphic to the trivial line bundle $\O_M$, which
implies that any global section of $\hom(L_1,L_2)$ is an isomorphism.
\end{proof}
%
\begin{cor}
Let $E,F \to M$ be holomorphic vector bundles of the same rank over a compact Riemann
surface $M$. Then If there exists a nonzero holomorphic bundle homomorphism $E \to F$,
we must have $\deg(E) \leq \deg(F)$.
\end{cor}
%
\begin{proof}
Apply the above corollary to the induced map $\Lambda^k E \to \Lambda^kF$, noting
that the degree of $E$ and $F$ are equal to the degrees of their respective
determinant line bundles.
\end{proof}
%
\begin{prop}
Let $E \to M$ be a holomorphic vector bundle over a compact Riemann surface $M$.
Then there exists an integer $q > 0$ such that for any line bundle $L$
with $\deg(L) > q$, the bundle $\hom(L,E)$ has no nonzero holomorphic sections.
\end{prop}
%
\begin{proof}
Since $M$ is projective, we can find an ample line bundle $H \to M$, which is
necessarily of positive degree. Then by Serre Vanishing, we know that for
a sufficiently large $m$, the bundle $H^m \otimes E^* \otimes K_M$ has no higher
cohomology. Serre Duality then lets us conclude that $H^0(M,H^{-m}\otimes E) = 0$.
Using the isomorphism $H^{-m}\otimes E \cong \hom(H^m, E)$, this tells us that there
exist no nonzero holomorphic bundle homomorphisms $H^m \to E$. Riemann-Roch then
tells us that for any line bundle $L$, we have
\[
h^0(M,H^{-m} \otimes L) - h^0(M, H^m \otimes L^*\otimes K_M)
= \mathrm{deg}(H^{-m} \otimes L) + 1 - g
\]
where $K_M$ is the canonical bundle of $M$. Since degree is additive on tensor
products, this tell us that
\[
h^0(M,H^{-m}\otimes L) \geq -m\deg(H) + \deg(L) + 1 - g
\]
Therefore, if $\deg(L)$ is sufficiently large, $H^0(M,H^{-m}\otimes L)$ is nonzero.
Then using the fact that $H^{-m} \otimes L \cong \hom(H^m, L)$, this tells us
that there exists a nonzero holomorphic bundle homomorphism $H^m \to L$. Therefore,
when $L$ is of sufficiently high degree, $\hom(L, E)$ admits no global sections,
since otherwise, this would imply the existence of a nonzero map
$H^m \to E$, which is impossible since $H^0(M,\hom(H^m,E)) = 0$.
\end{proof}
%
The previous proposition allows us to give a uniform bound on the slopes of
holomorphic subbundles of a holomorphic vector bundle.
%
\begin{prop}
Let $E \to M$ be a holomorphic vector bundle over a compact Riemann surface $M$.
Then there exists a nonnegative integer $m \geq 0$ such that
\[
\mu(F) \leq m
\]
for all proper nontrivial holomorphic subbundles $F \subset E$.
\end{prop}
%
\begin{proof}
Consider the bundles $\Lambda^k E$. The previous proposition gives nonnegative
integers $m(k)$ such that any holomorphic line bundle $L \to M$ with degree
greater than $m(k)$ admits no nonzero holomorphic bundle homomorphisms to $\Lambda^kE$.
Then let $F \subset E$ be a proper nontrivial holomorphic subbundle of rank $k$.
The inclusion $F \hookrightarrow E$ induces an inclusion
$\Lambda^k F \hookrightarrow \Lambda^kE$, which tells us that
$\deg(F) = \deg(\Lambda^kF) \leq m(k)$. Therefore, any holomorphic subbundle
$F \subset E$ of rank $k$ must satisfy $\mu(F) \leq m(k)/k$. Taking the
maximum among the $m(k)/k$ over all $k$ less than the rank of $E$ then gives the
desired slope bound.
\end{proof}
%
We now collect a few facts about slopes.
%
\begin{lem}
Let
\[\begin{tikzcd}
0 \ar[r] & E \ar[r] & F \ar[r] & G \ar[r] & 0
\end{tikzcd}\]
be an exact sequence of holomorphic vector bundles over a compact Riemann surface
$M$. Then
\[
\mu(F) = \frac{\deg(E) + \deg(G)}{\mathrm{rank}(E) + \mathrm{rank}(G)}
\]
\end{lem}
%
\begin{proof}
Since the degree is purely topological, we may forget the holomorphic structures,
and treat the exact sequence as an exact sequence of smooth vector bundles.
Any such exact sequence splits in the smooth category, giving us a direct
sum decomposition $F = E \oplus G$ as smooth vector bundles. Then since the
first Chern class is additive on direct sums, we get that
$\deg(F) = \deg(E) + \deg(G)$. Rank is also clearly additive on exact sequences,
and putting these two together gives the desired formula for the slope of $F$.
\end{proof}
%
\begin{cor}
Let
\[\begin{tikzcd}
0 \ar[r] & E \ar[r] & F \ar[r] & G \ar[r] & 0
\end{tikzcd}\]
be an exact sequence of holomorphic vector bundles over a compact Riemann surface
$M$. Then if $\mu(E) \geq \mu(F)$, we have $\mu(F) \geq \mu(G)$. Likewise, if
$\mu(E) \leq \mu(F)$, then $\mu(F) \leq \mu(G)$.
\end{cor}
%
\begin{defn}
A holomorphic vector bundle $E \to M$ over a compact Riemann surface is \ib{stable}
if for all proper nontrivial vector subbundles $F \subset E$, we have the
strict inequality
\[
\mu(F) < \mu(E)
\]
We say $E$ is \ib{semistable} if instead we have the inequality
\[
\mu(F) \leq \mu(E)
\]
\end{defn}
%
To iteratively construct the Harder-Narasimhan filtration for a holomorphic
vector bundle, we will need to find some sort of maximal subbundles.
%
\begin{prop}
Let $E \to M$ be a holomorphic vector bundle over a compact Riemann surface.
Then there exists a unique holomorphic subbundle $E_1 \subset E$ such that
for every holomorphic subbundle $F \subset E$, we have
\begin{enumerate}
  \item $\mu(F) \leq \mu(E_1)$
  \item $\mathrm{rank}(F) \leq \mathrm{rank}(E_1)$ if $\mu(F) = \mu(E_1)$.
\end{enumerate}
Furthermore, $E_1$ is semistable, and is called the \ib{maximal semistable subbundle}.
\end{prop}
%
\begin{proof}
If $E$ is already semistable, then taking $E_1 = E$ works, so we assume that
$E$ is not semistable. From the previous proposition, we know that the slopes
$\mu(F)$ of all proper nontrivial subbundles is bounded by some integer $m$. We
then claim that there exists a subbundle of maximal slope. To see this, we note
that by definition, the slope of a proper subbundle $F \subset E$ of rank $k$ is
given by
\[
\mu(F) \defeq \frac{\mathrm{deg}(F)}{k} \leq m
\]
Since the degree of $F$ is a nonnegative integer, this tells us that $\mu(F)$
is contained in the finite set $\set{0, 1/k, \ldots, mk/k}$. Therefore, the set
of all possible slopes is a finite set, so there exists bundles that attain this
maximum. We then choose $E_1$ to be a bundle with maximal rank among those with
maximal slope. Then $E_1$ satisfies the two specified conditions, and is semistable
by the first condition. \\

We then want to show uniqueness of $E_1$. Let $F$ be another subbundle satisfying the
specified properties. If $F \subset E_1$, the the second property guarantees that
$F = E_1$, so we assume that $F \not\subset E_1$. Then consider the holomorphic
bundle $E/E_1$, and let $\varphi : F \to E/E_1$ be the restriction of the quotient
map to $F$. By assumption, $\varphi$ is a nonzero holomorphic map. This
gives us the exact sequence of coherent sheaves
\[\begin{tikzcd}
0 \ar[r] & \ker\varphi \ar[r] & F \ar[r,"\varphi"] & \varphi_*F \ar[r] & 0
\end{tikzcd}\]
We note that we must pass to sheaves, since the kernel and image of $\varphi$ may
fail to be vector bundles. However, we may return to the land of vector bundles
via the following observation : the stalks of the structure sheaf $\O_M$ are all
principal ideal domains, so the stalks of $\ker\varphi$ and $\varphi_*F$ have
direct sum decompositions into a torsion submodule and a free module, the
free components define vector bundles $A \supset \ker\varphi$ and
$B \supset \varphi_*F$, giving us the exact sequence of vector bundles
\[\begin{tikzcd}
0 \ar[r] & A \ar[r] & F \ar[r, "\varphi"] & B \ar[r] & 0
\end{tikzcd}\]
More explicitly, $A$ and $B$ may be realized as the smallest holomorphic subbundles
containing $\ker\varphi$ and $\im\varphi$ respectively. To see this,
let $k$ be the rank of $\varphi_*F$ as an $\O_M$-module, and let $\set{s_i}$ be
a local holomorphic frame for $F$. Generically, we have that the set
$\set{\varphi(s_i)}$ spans a $k$-dimensional subbundle of $E/E_1$, except when
all the sections of $\Lambda^k(E/E_1)$ of the form
\[
\varphi(s_{i_1}) \wedge \cdots \wedge \varphi(s_{i_k})
\]
simultaneously vanish, i.e. there exists no linearly independent subset of
$\set{\varphi(s_i)}$ of cardinality $k$. Then in a sufficiently small neighborhood
of a point where all these sections vanish, we may fix local holomorphic coordinates
and a holomorphic trivialization of $\Lambda^k(E/E_1)$ such that the sections
$\varphi(s_{i_1})\wedge\cdots\wedge\varphi(s_{i_k})$ are identified with holomorphic
functions $U \to \C$, where $U$ is an open subset in $\C$. Then we can factor out
all of the zeroes of these sections to obtain a set of nonvanishing sections of
$\Lambda^k(E/E_1)$, which then locally determine a rank $k$ subbundle of $E/E_1$.
Gluing these local definitions together then gives us the bundle $B$. We construct
$A$ from $\ker\varphi$ via a similar procedure. In total, we get the commutative
diagram
\[\begin{tikzcd}
0 \ar[r] & A \ar[r] & F \ar[r] \ar[d,"\varphi"]
& F/A \ar[r]\ar[d,"\widetilde{\varphi}"] & 0 \\
0 & (E/E_1)/B \ar[l] & E/E_1 \ar[l] & B \ar[l] & 0 \ar[l]
\end{tikzcd}\]
Where $\widetilde{\varphi}$ is the natural map induced by $\varphi$. We then
make the following observations:
\begin{enumerate}
  \item Since $F$ is semistable, we have that $\mu(A) \leq \mu(F)$. Then since
  the top row is exact. this tells us that $\mu(F) \leq \mu(F/A)$.
  \item We know that $F/A$ is the same rank as $B$, and by assumption, the map
  $\widetilde{\varphi}$ is nonzero, so $\deg(F/A) \leq \deg(B)$. Therefore,
  $\mu(F/A) \leq \mu(B)$.
  \item We note that since $E_1$ is of maximal slope, we in particular have that
  for any holomorphic subbundle $Q \subset E$ with $E_1 \subsetneq Q$, then
  $\mu(Q) \leq \mu(E_1)$. Then if we consider the exact sequence
  \[\begin{tikzcd}
  0 \ar[r] & E_1 \ar[r] & Q \ar[r] & Q/E_1 \ar[r] & 0
  \end{tikzcd}\]
  this tells us that the slope of $\mu(Q/E_1) \leq \mu(Q)$. Since we may
  regard subbundles of $E/E_1$ as quotients $Q/E_1$, this tells us that for
  any holomorphic subbundle of $E/E_1$, the slope is bounded by $\mu(E_1)$. In
  our case, this implies that $\mu(B) \leq \mu(E_1)$.
\end{enumerate}
We now put everything together. Our observations give us the chain of inequalities
\[
\mu(F) \leq \mu(F/A) \leq \mu(B) \leq \mu(E_1)
\]
We then observe that we have the exact sequence
\[\begin{tikzcd}
0 \ar[r] & E_1 \ar[r] & \varphi\inv(B) \ar[r] & B \ar[r] & 0
\end{tikzcd}\]
By the first property of $E_1$, we have that $\mu(\varphi\inv(B)) \leq \mu(E_1)$.
However, by assumption the rank of $\varphi\inv(B)$ is larger than the rank of $E_1$,
so the second property guarantees that we have the strict inequality
$\mu(\varphi\inv(B)) < \mu(E_1)$. The exact sequence then tells us that
$\mu(\varphi\inv(B)) \geq \mu(B)$, so $\mu(B) < \mu(E_1)$. However,
this tells us that our above chain of inequalities is actually
\[
\mu(F) \leq \mu(F/A) \leq \mu(B) < \mu(E_1)
\]
which in particular implies that $\mu(F) < \mu(E_1)$, contradicting maximality of
$\mu(F)$.
\end{proof}
%
\begin{thm}[\ib{Harder-Narasimhan Filtration}]
Any holomorphic vector bundle $E \to M$ over a compact Riemann surface $M$ admits
a filtration
\[
0 = E_0 \subset E_1 \subset \cdots \subset E_n = E
\]
by holomorphic subbundles $E_i$ such that $E_i/E_{i-1}$ is semistable and
\[
\mu(E_1/E_0) > \mu(E_2/E_1) > \cdots > \mu(E_n/E_{n-1})
\]
\end{thm}
%
\begin{proof}
Let $E_1$ be the maximal semistable subbundle of $E$. Then let $E_2$ be
the preimage of the maximal semistable subbundle of $E/E_1$ under the quotient
map. We then iteratively construct the filtration by taking $E_i$ to be the
preimage of the maximal semistable subbundle of $E/E_{i-1}$. By construction,
the quotients $E_i/E_{i-1}$ are semistable. We then want to show the monotonicity
of the slopes. We also note that since the slopes are strictly decreasing,
this construction will eventually terminate. To show this, we use the exact sequence
\[\begin{tikzcd}
0 \ar[r] & E_{i+1}/E_i \ar[r] & E/E_i \ar[r] & E/E_{i+1} \ar[r] & 0
\end{tikzcd}\]
By constuction, we have that $E_{i+1}/E_i$ is the maximal semistable bundle of
$E/E_i$. Therefore, any subbundle of $E/E_{i+1}$ has a strictly smaller slope than
$E_{i+1}/E_i$, in particular, this gives us that the slope of $E_{i+2}/E_{i+1}$
is less than the slope of $E_{i+1}/E_i$, which is what we wanted to show.
\end{proof}
%
\newpage
%
\nocite{*}
%
\printbibliography
%
\end{document}