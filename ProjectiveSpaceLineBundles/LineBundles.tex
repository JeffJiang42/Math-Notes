\documentclass[psamsfonts, 12pt]{amsart}
%
%-------Packages---------
%
\usepackage[h margin=1 in, v margin=1 in]{geometry}
\usepackage{amssymb,amsfonts}
\usepackage[all,arc]{xy}
\usepackage{tikz-cd}
\usepackage{enumerate}
\usepackage{mathrsfs}
\usepackage{amsthm}
\usepackage{mathpazo}
\usepackage{float}

%\usepackage{charter} %another font
%\usepackage{eulervm} %Vakil font
\usepackage{yfonts}
\usepackage{mathtools}
\usepackage{enumitem}
\usepackage{mathrsfs}
\usepackage{fourier-orns}
\usepackage[all]{xy}
\usepackage{hyperref}
\usepackage{url}
\usepackage{mathtools}
\usepackage{graphicx}
\usepackage{pdfsync}
\usepackage{mathdots}
\usepackage{calligra}
\usepackage{import}
\usepackage{xifthen}
\usepackage{pdfpages}
\usepackage{transparent}

\newcommand{\incfig}[2]{%
    \fontsize{48pt}{50pt}\selectfont
    \def\svgwidth{\columnwidth}
    \scalebox{#2}{\input{#1.pdf_tex}}
}
%
\usepackage{tgpagella}
\usepackage[T1]{fontenc}
%
\usepackage{listings}
\usepackage{color}

\definecolor{dkgreen}{rgb}{0,0.6,0}
\definecolor{gray}{rgb}{0.5,0.5,0.5}
\definecolor{mauve}{rgb}{0.58,0,0.82}

\lstset{frame=tb,
  language=Matlab,
  aboveskip=3mm,
  belowskip=3mm,
  showstringspaces=false,
  columns=flexible,
  basicstyle={\small\ttfamily},
  numbers=none,
  numberstyle=\tiny\color{gray},
  keywordstyle=\color{blue},
  commentstyle=\color{dkgreen},
  stringstyle=\color{mauve},
  breaklines=true,
  breakatwhitespace=true,
  tabsize=3
  }
%
%--------Theorem Environments--------
%
\newtheorem{thm}{Theorem}[section]
\newtheorem*{thm*}{Theorem}
\newtheorem{cor}[thm]{Corollary}
\newtheorem{prop}[thm]{Proposition}
\newtheorem{lem}[thm]{Lemma}
\newtheorem*{lem*}{Lemma}
\newtheorem{conj}[thm]{Conjecture}
\newtheorem{quest}[thm]{Question}
%
\theoremstyle{definition}
\newtheorem{defn}[thm]{Definition}
\newtheorem*{defn*}{Definition}
\newtheorem{defns}[thm]{Definitions}
\newtheorem{con}[thm]{Construction}
\newtheorem{exmp}[thm]{Example}
\newtheorem{exmps}[thm]{Examples}
\newtheorem{notn}[thm]{Notation}
\newtheorem{notns}[thm]{Notations}
\newtheorem{addm}[thm]{Addendum}
\newtheorem{exer}[thm]{Exercise}
%
\theoremstyle{remark}
\newtheorem{rem}[thm]{Remark}
\newtheorem*{claim}{Claim}
\newtheorem*{aside*}{Aside}
\newtheorem*{rem*}{Remark}
\newtheorem*{hint*}{Hint}
\newtheorem*{note}{Note}
\newtheorem{rems}[thm]{Remarks}
\newtheorem{warn}[thm]{Warning}
\newtheorem{sch}[thm]{Scholium}
%
%--------Macros--------
\renewcommand{\qedsymbol}{$\blacksquare$}
\renewcommand{\sl}{\mathfrak{sl}}
\newcommand{\Bord}{\mathsf{Bord}}
\renewcommand{\hom}{\mathsf{Hom}}
\renewcommand{\emptyset}{\varnothing}
\renewcommand{\O}{\mathscr{O}}
\newcommand{\R}{\mathbb{R}}
\newcommand{\ib}[1]{\textbf{\textit{#1}}}
\newcommand{\Q}{\mathbb{Q}}
\newcommand{\Z}{\mathbb{Z}}
\newcommand{\N}{\mathbb{N}}
\newcommand{\C}{\mathbb{C}}
\newcommand{\A}{\mathbb{A}}
\newcommand{\F}{\mathbb{F}}
\newcommand{\M}{\mathcal{M}}
\newcommand{\dbar}{\overline{\partial}}
\newcommand{\zbar}{\overline{z}}
\renewcommand{\S}{\mathbb{S}}
\renewcommand{\P}{\mathbb{P}}
\newcommand{\V}{\vec{v}}
\newcommand{\RP}{\mathbb{RP}}
\newcommand{\CP}{\mathbb{CP}}
\newcommand{\B}{\mathcal{B}}
\newcommand{\GL}{\mathrm{GL}}
\newcommand{\PGL}{\mathrm{PGL}}
\newcommand{\SL}{\mathrm{SL}}
\newcommand{\PSL}{\mathrm{PSL}}
\newcommand{\SP}{\mathrm{SP}}
\newcommand{\SO}{\mathrm{SO}}
\newcommand{\SU}{\mathrm{SU}}
\newcommand{\gl}{\mathfrak{gl}}
\newcommand{\g}{\mathfrak{g}}
\newcommand{\Bun}{\mathsf{Bun}}
\newcommand{\inv}{^{-1}}
\newcommand{\bra}[2]{ \left[ #1, #2 \right] }
\newcommand{\set}[1]{\left\lbrace #1 \right\rbrace}
\newcommand{\abs}[1]{\left\lvert#1\right\rvert}
\newcommand{\norm}[1]{\left\lVert#1\right\rVert}
\newcommand{\transv}{\mathrel{\text{\tpitchfork}}}
\newcommand{\defeq}{\vcentcolon=}
\newcommand{\enumbreak}{\ \\ \vspace{-\baselineskip}}
\let\oldexists\exists
\renewcommand\exists{\oldexists~}
\let\oldL\L
\renewcommand\L{\mathfrak{L}}
\makeatletter
\newcommand{\tpitchfork}{%
  \vbox{
    \baselineskip\z@skip
    \lineskip-.52ex
    \lineskiplimit\maxdimen
    \m@th
    \ialign{##\crcr\hidewidth\smash{$-$}\hidewidth\crcr$\pitchfork$\crcr}
  }%
}
\makeatother
\newcommand{\bd}{\partial}
\newcommand{\lang}{\begin{picture}(5,7)
\put(1.1,2.5){\rotatebox{45}{\line(1,0){6.0}}}
\put(1.1,2.5){\rotatebox{315}{\line(1,0){6.0}}}
\end{picture}}
\newcommand{\rang}{\begin{picture}(5,7)
\put(.1,2.5){\rotatebox{135}{\line(1,0){6.0}}}
\put(.1,2.5){\rotatebox{225}{\line(1,0){6.0}}}
\end{picture}}
\DeclareMathOperator{\id}{id}
\DeclareMathOperator{\im}{Im}
\DeclareMathOperator{\codim}{codim}
\DeclareMathOperator{\coker}{coker}
\DeclareMathOperator{\supp}{supp}
\DeclareMathOperator{\inter}{Int}
\DeclareMathOperator{\sign}{sign}
\DeclareMathOperator{\Stab}{Stab}
\DeclareMathOperator{\sgn}{sgn}
\DeclareMathOperator{\indx}{ind}
\DeclareMathOperator{\alt}{Alt}
\DeclareMathOperator{\Aut}{Aut}
\DeclareMathOperator{\trace}{trace}
\DeclareMathOperator{\ad}{ad}
\DeclareMathOperator{\End}{End}
\DeclareMathOperator{\Ad}{Ad}
\DeclareMathOperator{\Lie}{Lie}
\DeclareMathOperator{\spn}{span}
\DeclareMathOperator{\dv}{div}
\DeclareMathOperator{\grad}{grad}
\DeclareMathOperator{\Sym}{Sym}
\DeclareMathOperator{\sheafhom}{\mathscr{H}\text{\kern -3pt {\calligra\large om}}\,}
\newcommand*\myhrulefill{%
   \leavevmode\leaders\hrule depth-2pt height 2.4pt\hfill\kern0pt}
\newcommand\niceending[1]{%
  \begin{center}%
    \LARGE \myhrulefill \hspace{0.2cm} #1 \hspace{0.2cm} \myhrulefill%
  \end{center}}
\newcommand*\sectionend{\niceending{\decofourleft\decofourright}}
\newcommand*\subsectionend{\niceending{\decosix}}
\def\upint{\mathchoice%
    {\mkern13mu\overline{\vphantom{\intop}\mkern7mu}\mkern-20mu}%
    {\mkern7mu\overline{\vphantom{\intop}\mkern7mu}\mkern-14mu}%
    {\mkern7mu\overline{\vphantom{\intop}\mkern7mu}\mkern-14mu}%
    {\mkern7mu\overline{\vphantom{\intop}\mkern7mu}\mkern-14mu}%
  \int}
\def\lowint{\mkern3mu\underline{\vphantom{\intop}\mkern7mu}\mkern-10mu\int}
%
%--------Hypersetup--------
%
\hypersetup{
    colorlinks,
    citecolor=black,
    filecolor=black,
    linkcolor=blue,
    urlcolor=blacksquare
}
%
%--------Solution--------
%
\newenvironment{solution}
  {\begin{proof}[Solution]}
  {\end{proof}}
%
%--------Graphics--------
%
%\graphicspath{ {images/} }

\begin{document}
%
\author{Jeffrey Jiang}
%
\title{Line Bundles on $\CP^n$}
%
\setcounter{section}{1}
%
\maketitle
%
\begin{defn}
\ib{Complex projective space}, denoted $\CP^n$ is the set of $1$-dimensional
subspaces of $\C^{n+1}$.
\end{defn}
%
Given a line $\ell \in \CP^n$, it can be recovered from any nonzero vector
$v \in \ell$ by taking the span of $v$. If $v \in \ell$ has coordinates
\[
v = (z_0, \ldots, z_n)
\]
we denote the line $\ell$ with the notation
\[
\ell = [z_0 : \ldots : z_n]
\]
where it is understood that the coordinates in the square brackets are only determined
up to scaling, since $\lambda v$ determines the same line as $v$ for any
$\lambda \in \C^\times$. These are called \ib{homogeneous coordinates} for $\CP^n$.
%
\begin{prop}
$\CP^n$ can be endowed with the structure of an $n$-dimensional complex manifold.
\end{prop}
%
\begin{proof}
It suffices to provide a covering of $\CP^n$ by charts with holomorphic
transition functions. Define the set $U_i \subset \CP^n$ by
\[
U_i \defeq \set{[z_0: \cdots z_n] ~:~ z_i \neq 0}
\]
note that this is well defined, since if the $i^{th}$ component of a
vector $v \in \ell$ is $0$, then the $i^{th}$ component of $\lambda v$ will be
as well. Then define the maps
\begin{align*}
\varphi_i : U_i &\to \C^n \\
[z_0: \ldots : z_n] &\mapsto \left( \frac{z_0}{z_i},\ldots,
\widehat{\frac{z_i}{z_i}}, \ldots \frac{z_n}{z_i} \right)
\end{align*}
where $\widehat{\frac{z_i}{z_i}}$ denotes the fact that the term is missing.
We note that this is well defined since we have that $z_i \neq 0$ for any line
in $U_i$. The maps $\varphi_i$ are bijections, with inverses given by
\begin{align*}
\varphi_i\inv : \C^n &\to U_i \\
(z_1,\ldots z_n) &\mapsto [z_1:\ldots: z_{i-1}: 1 : z_i \ldots : z_n]
\end{align*}
i.e. we insert a $1$ in the $i^{th}$ slot in homogeneous coordinates. We then check
the transition functions. We compute for $i \neq j$
\[
(\varphi_j \circ \varphi_i\inv)(z_1,\ldots z_n)
= \begin{cases}
\left(\frac{z_1}{z_{j-1}}, \ldots ,\frac{z_{i-1}}{z_{j-1}}, \frac{1}{z_{j-1}},
\frac{z_i}{z_{j-1}}, \ldots ,\widehat{\frac{z_{j-1}}{z_{j-1}}} ,
\ldots,\frac{z_n}{z_{j-1}}\right) & j > i \\[20pt]
\left(\frac{z_1}{z_j},\ldots,\widehat{\frac{z_j}{z_j}},\ldots\frac{z_{i-1}}{z_j},
\ldots,\frac{1}{z_j},\frac{1}{z_j},\frac{z_i}{z_j},\ldots,\frac{z_n}{z_j} \right)
& j < i
\end{cases}
\]
which is visibly holomorphic, since in either case, the functions $1/z_j$ or $1/z_{j-1}$
are holomorphic if $z_j$ or $z_{j-1}$ are nonzero.
\end{proof}
%
\begin{defn}
Let $X$ be a complex manifold. A \ib{holomorphic line bundle} over $X$ is a complex
manifold $L$ equipped with a holomorphic map $\pi : L \to X$ such that for every
$x \in X$,  the fiber $L_x \defeq \pi\inv(x)$ has the structure of a $1$-dimensional
complex vector space. In addition, there exists an open set $U$ about $x$ and
a map $\psi : \pi\inv(U) \to U \times \C$ such that $\psi\vert_{L_x}$ is a linear
map and the following diagram commutes
\[\begin{tikzcd}
\pi\inv(U) \ar[dr, "\pi"'] \ar[rr, "\psi"] & & U \times \C \ar[dl, "\pi_U"]\\
& U
\end{tikzcd}\]
where the map $\pi_U : U \times \C \to U$ is projection onto the first factor.
The maps $\psi$ are called \ib{local trivializations}.
\end{defn}
%
One example of a holomorphic line bundle is the product $X \times \C \to X$, where
the bundle projection is just projection onto the first factor, which
is called the \ib{trivial bundle}. A bundle isomorphic to the trivial bundle
is said to be \ib{trivial} or \ib{trivializable}.
One way to tell if a line bundle $L \to X$ is trivial is to provide a global
nonvanishing section $\sigma : X \to L$. This determines a global trivialization of $L$
in the following way: At each $x$, we have that $\sigma(x) \neq 0$ determines a basis for
the fiber $L_x$. Therefore, for any element $v$ of the fiber $L_x$, we can map it
to $(x, \lambda) \in X \times \C$, where $\lambda$ is the component of $v$ in the basis
$\set{\sigma(x)}$ for $L_x$. This determines an bundle isomorphism
$L \cong X \times \C$. \\

Suppose we have a holomorphic line bundle $\pi : L \to X$, and two local trivializations
$\psi_i : \pi\inv(U_i) \to U_i \times \C$ and $\psi_j : \pi\inv(U_j) \to U_j \to \C$.
Then if we consider the map $(\psi_i \circ \psi_j\inv)$ after appropriately restricting
and shrinking the domain and codomains, if we fix a $x \in U_i \cap U_j$,
the map $(\psi_i \circ \psi_j\inv)\vert_{\set{x} \times \C}$ is a linear function
from $\set{x} \times \C$ to itself. Therefore, the maps $(\psi_i \circ \psi_j)\inv$
determines holomorphic maps $\psi_{ij} : U_i \cap U_j \to \GL_1\C$. Holomorphicity of the
$\psi_{ij}$ comes from the fact that $L$ is a holomorphic line bundle. We can go the
other direction as well. One way to construct line bundles is to use trivial bundles over
sets in an open cover, and to glue them together by specifying the transition functions.
%
\begin{thm}[\ib{The gluing construction}]
Let $X$ be a complex manifold, and $\set{U_i}$ an open cover of $X$. Then
for each $U_i  \cap U_j$, let $\psi_{ij} : U_i\cap U_j \to \GL_1\C$ be holomorphic
maps satisfying the \ib{cocycle condition}
\[
\psi_{ij}\psi_{jk} = \psi_{ik}
\]
Then the set
\[
L = \coprod_{i} U_i \times \C/\sim
\]
where the equivalence relation $\sim$ is defined by
\[
(x,\lambda) \sim (y,\mu) \iff x=y \text{ and there exist } i,j \text{ such that }
\psi_{ij}(x)(\lambda) = \mu
\]
equipped with the natural projection $\pi : L \to X$ is a holomorphic line bundle.
\end{thm}
%
With respect to a cover $\set{U_i}$, the pairs $(U_i, \psi_{ij})$ are called
\ib{cocycles}, though the set $U_i$ is sometimes omitted, and the transition
functions themselves are called coycles. The gluing construction tells us
that the cocycles determine the line bundle up to isomorphism. In
particular, if we have that all the cocycles are given by the constant map
$\psi_{ij}(x) = \lambda$, then we know that the bundle is trivial. In a similar
vein, if we have two bundles $L,L' \to X$, and we know that their transition
functions differ by a constant, then we can conclude that $L$ and $L'$ are
isomorphic as holomorphic line bundles.
%
\begin{defn}
Let $\pi : L \to X$ be a holomorphic line bundle. A \ib{local section} is
a holomorphic map $\sigma : U \to L$  from an open set $U \subset X$ such that
$\pi \circ \sigma = \id_U$. If $U = X$, then $\sigma$ is said to be a
\ib{global section}. If it is not specified whether a section is local
or global, it is implicitly assumed to be global.
\end{defn}
%
\begin{rem*}
The idea of a section is to generalize the notion of a function. A section
of the trivial bundle $X \times \C$ is just a holomorphic function on $X$. However,
sections of nontrivial holomorphic line bundles are maps into holomorphically
varying complex lines, and could possibly fail to exist.
\end{rem*}
%
The gluing construction gives an alternate characterization of sections of a line bundle.
%
\begin{prop}
Let $\pi : L \to X$ be a holomorphic line bundle, and $\set{U_i}$ an open cover of
$X$ in which $L$ is trivialized over each $U_i$ with transition functions $\psi_{ij}$.
Then the data of a section $\sigma : X \to L$ is equivalent to the data
of holomorphic functions $\sigma_i : U_i \to \C$ with the compatibility condition
$\sigma_i = \psi_{ij}\sigma_j$.
\end{prop}
%
\begin{proof}
Given a section $\sigma : X \to L$, by composing with the trivialization
$\pi\inv(U_i) \to U_i \times \C$, we obtain smooth functions $\sigma_i : U \to \C$,
and the fact that they satisfy the compatibility condition can be easily checked.
In the other direction, given a a collection of compatible maps $\sigma_i$,
we can define $\sigma$ by specifying $\sigma$ on each $U_i$ to be
the map $\sigma_i$ composed with the inverse of the local trivialization.
The fact that these local definitions glue to a well-defined holomorphic map
$\sigma : X \to L$ is easily checked to be exactly the condition that they are
compatible.
\end{proof}
%
\begin{defn}
The \ib{tautological bundle} over $\CP^n$, denoted $\O(-1)$, is the bundle
where the fiber over $\ell \in \CP^n$ is the line $\ell$, i.e. as a set,
\[
\O(-1) \defeq \set{(\ell, v) ~:~ \ell \in \CP^n, v \in \ell}
\]
\end{defn}
%
\begin{prop}
$\O(-1)$ actually is a holomorphic line bundle.
\end{prop}
%
\begin{proof}
Let $\pi : \O(-1) \to \CP^n$ denote the bundle projection.
It suffices to produce local trivializations and then check that the transitions
functions are holomorphic. Let $\varphi_i : \C^n \to U_i$ be the charts on
$\CP^n$ defined earlier. These coordinates induce trivializations
$\psi_i : \pi\inv(U_i) \to U_i \times \C$ as follows: \\

Let $\ell = [z_0 : \ldots : z_n] \in \CP^n$. Then
\[
\varphi(\ell) = \left(\frac{z_0}{z_i},\ldots,\widehat{\frac{z_i}{z_i}},\ldots
\frac{z_n}{z_i} \right)
\]
this determines a vector $v^i_\ell \in \ell$ via the mapping
\[
\left(\frac{z_0}{z_i},\ldots,\widehat{\frac{z_i}{z_i}},\ldots
\frac{z_n}{z_i} \right) \mapsto v_\ell\defeq
\left(\frac{z_0}{z_i},\ldots,\frac{z_{i-1}}{z_i},1,\frac{z_{i+1}}{z_i},\ldots
\frac{z_n}{z_i} \right)
\]
Therefore, we can define the map $\psi_i$ by declaring $\psi_i(\ell,v^i_\ell) = (\ell,1)$
and extending linearly to the rest of the fiber over $\ell$. This clearly makes
the diagram
\[\begin{tikzcd}
\pi\inv(U_i) \ar[dr, "\pi"']\ar[rr, "\psi_i"] & & U_i \times \C \ar[dl, "\pi_{U_i}"] \\
& U_i
\end{tikzcd}\]
commute, so we now check the transition functions. To do this, we compute the
action of $(\psi_i\circ\psi_j\inv)$ on $(\ell, 1) \in U_i\cap U_j \times \C$. We
compute, letting $\ell = [z_0:,\ldots : z_n]$
\begin{align*}
(\psi_i\circ\psi_j\inv)(\ell, 1)
&= \psi_i\left(\ell, \left( \frac{z_0}{z_j},\ldots
\frac{z_{j-1}}{z_j},1,\frac{z_{j+1}}{z_j},\ldots {z_n}{z_j} \right)\right) \\
&= \left(\ell, \frac{z_i}{z_j}\right)
\end{align*}
where the last equality comes from the fact that the component of the vector
\[
v^j_\ell \defeq \left( \frac{z_0}{z_j},\ldots
\frac{z_{j-1}}{z_j},1,\frac{z_{j+1}}{z_j},\ldots \frac{z_n}{z_j} \right)
\]
with respect to the basis $\set{v^i_\ell}$ for $\ell$ where
\[
v^i_\ell \defeq \left( \frac{z_0}{z_i},\ldots
\frac{z_{i-1}}{z_i},1,\frac{z_{i+1}}{z_i},\ldots \frac{z_n}{z_i} \right)
\]
is just $z_i/z_j$. Therefore, we have that the transition functions $\psi_{ij}$
are given by
\[
\psi_{ij}(\ell) = \begin{pmatrix}
\frac{z_i}{z_j}
\end{pmatrix}
\]
where $\ell = [z_0: \ldots : z_n]$. Since these functions are all holomorphic,
this shows that $\O(-1)$ is a holomorphic line bundle.
\end{proof}
%
To discuss all of the line bundles over $\CP^n$, we need to discuss how to
construct new line bundles from existing ones.
%
\begin{prop}[\ib{New bundles from old}]
Let $L,L'\to X$ be holomorphic line bundles over $X$. Then the following linear algebraic
operations, done fiberwise, define another holomorphic line bundle.
\begin{enumerate}
  \item $L \otimes L' \to X$, where the fiber over $x$ is $L_x \otimes L'_x$.
  \item $L^* \to X$, where the fiber over $x$ is the dual space $L^*_x$.
\end{enumerate}
\end{prop}
%
\begin{proof}
It suffices to provide local trivializations and check the transition functions.
By potentially shrinking sets, we can find an open cover $\set{U_i}$ of $X$
such that both $L$ and $L'$ are trivialized over the $U_i$. Let
$\varphi_i$ and $\varphi_i'$ denote the local trivializations for $L$ and $L'$
respectively, and let $\psi_{ij}$, $\psi_{ij}'$ denote their respective transition
functions.
\begin{enumerate}
  \item The local trivializations for $L \otimes L$ are just given by
  $\varphi_i \otimes \varphi_i'$, where if $\varphi_i(\ell) = (x,\lambda)$
  and $\varphi_i'(\ell) = (x,\lambda')$, we have
  \[
  (\varphi_i \otimes \varphi_i')(\ell) = (x, \lambda \otimes \lambda')
  \]
  the transition functions are given by $\psi_{ij} \otimes \psi'_{ij}$, where
  \[
  (\psi_{ij}\otimes\psi'_{ij})(\ell) = \psi_{ij}(\ell) \otimes \psi'_{ij}(\ell)
  \]
  where the $\otimes$ in the right hand side denotes the standard notion of
  the tensor product of matrices/linear maps.
  \item The trivializations for $L^*$ are just the trivializations
  for $\varphi_i$ for $L$ composed with the isomorphism $\C \to \C^*$. The
  transition functions are given by $(\psi_{ij}\inv)^T$, where $T$ denotes
  the transpose (\emph{not} the conjugate transpose), where the inverse
  comes into play due to the contravariance of taking dual spaces. In the
  case of the line bundle, transposing a $1\times 1$ matrix is pointless, so
  the transition functions are $\psi_{ij}\inv$ (i.e. inverting the matrix,
  not inverting the function $\psi_{ij}$).
\end{enumerate}
\end{proof}
%
\begin{prop}
For a holomorphic line bundle $L \to X$, The bundle $L \otimes L^*$ is isomorphic to
the trivial bundle $X \times \C$.
\end{prop}
%
\begin{proof}
We know for a vector space $V$ that $V^*\otimes V$ is isomorphic to $\End(V)$.
Therefore, the bundle $L \otimes L$ is isomorphic to the bundle
$\End(L)$, where the fiber over $x \in X$ is the space of endomorphisms of $L_x$.
To show that this is trivial, it suffices to provide a global nonvanishing section,
$\sigma$, which we can do by defining $\sigma(x) = \id_{L_x}$.
\end{proof}
%
\begin{defn}
For a complex manifold $X$, the \ib{Picard group} of $X$, denoted
$\mathrm{Pic}(X)$ is the group of isomorphism classes of holomorphic line bundles over
$X$, where the group operation is given by the tensor product. The operation is
associative, since tensor products are associative, and the trivial bundle is the
identity with respect to the tensor product. Finally, for any bundle
$L \in \mathrm{Pic}(X)$, it has an inverse given by the dual bundle $L^*$.
\end{defn}
%
We now restrict our attention again to $\CP^n$.
%
\begin{defn}
Define the bundle $\O(1) \defeq \O(-1)^*$. It is often referred to as the
\ib{hyperplane bundle}. The define the bundle $\O(k) \defeq \O(1)^{\otimes^k}$
and $\O(-k) \defeq \O(-1)^{\otimes^{-k}}$ for $k \in \Z^{>0}$. We let
$\O(0)\defeq \CP^n \times \C$.
\end{defn}
%
A fact that we will not prove is that all line bundles over $\CP^n$ are of
the form $\O(k)$ for some integer $k$. Using our observations in Proposition 1.7,
we can identify the transition functions for the line bundles $\O(k)$ with
respect to the local trivializations defined for $\O(-1)$ in the open cover
$\set{U_i}$. \\

For the following discussion, fix $\ell \in \CP^n$ with $\ell = [z_0: \ldots : z_n]$.
Then since the transition functions for the dual bundle are the inverses of the
transition functions, we have that the transition functions for $\O(1)$ are
$\psi_{ij}(\ell) = z_j/z_i$. Then for the tensor powers, the transition functions
for $\O(k)$ where $k \geq 0$ are $\psi_{ij}(\ell) = (z_j/z_i)^{\otimes^k}$, and
using the identification $\C^{\otimes^k}$ with $\C$, under the mapping
\[
\lambda_1 \otimes \cdots \otimes \lambda_k \mapsto \lambda_1\lambda_2\cdots\lambda_k
\]
this is the same as
$\psi_{ij}(\ell) = z_j^k/z_i^k$. Likewise, the transition functions for $\O(-k)$
are $\psi_{ij}(\ell) = z_i^k/z_j^k$. \\

A natural question to ask is which of these bundles admit global sections. We work
through the case of $\O(-1)$ explicitly.
%
\begin{thm}
The bundle $\O(-1)$ admits no nonzero global holomorphic sections, i.e.
\[
\Gamma_{\CP^n}(\O(-1)) = 0
\]
\end{thm}
%
We give two proofs of this fact.
%
\begin{proof}[Proof 1]
The line bundle naturally lives as a subbundle of the trivial bundle
$\CP^n \times \C^{n+1}$, since each element $\ell \in \CP^n$ is a line in
$\C^{n+1}$. Therefore, a global holomorphic section $\sigma : \CP^n \to \O(-1)$
is also a global holomorphic section $\CP^n \to \CP^n \times \C^{n+1}$ by composing
with the inclusion $\O(-1) \hookrightarrow \CP^n \times \C^{n+1}$. However,
a a holomorphic section of $\CP^n \times \C^{n+1}$ is just a holomorphic map
$\CP^n \to \C^{n+1}$. Then since $\CP^n$ is compact and connected, the map is constant by
the maximum principle for holomorphic maps. However, $\sigma(\ell) \in \C^{n+1}$ must
be a point in $\ell$ for every $\ell$, and the only point that lies in all lines
in $\C^{n+1}$ is $0$. Therefore, $\sigma = 0$.
\end{proof}
%
\begin{proof}[Proof 2]
The transition functions for $\O(-1)$ are given by $\psi_{ij}(\ell) = z_i/z_j$
with respect to the standard open cover $\set{U_i}$ of $\CP^n$. Since $\O(-1)$ is
trivialized over the $U_i$, the data of a section $\sigma$ is the same as specifying
a function $f_i : U_i \to \C$ with a compatibility condition on the intersections
specified by the transition functions $\psi_{ij}$. To determine this compatibility
condition, we have that by composing with the chart maps $\varphi_i\inv : \C^n \to U_i$,
the function $f_i$ on the intersection $U_i \cap U_j$ can be interpreted as
a holomorphic function of the variables $x_k \defeq z_k/z_i$ (where $x_i$ is missing).
Likewise, $f_j$ can be interpreted as a holomorphic function of the variable
$y_k \defeq z_k/z_j$ (where $y_j$ is missing). The compatibility condition is that these
two functions are related by the transition function $\psi_{ij}$, i.e. at the point
$\ell = [z_0: \ldots z_n] \in U_i \cap U_j$,
\[
f_i(x_0,\ldots,\widehat{x_i},\ldots, x_n) =
\psi_{ij}(\ell)f_j(y_1,\ldots, \widehat{y_j},\ldots, y_n)
\]
But since $\psi_{ij}(\ell) = z_i/z_j = y_i = 1/x_j$, this condition is asking that
\[
f_i(x_0,\ldots,\widehat{x_i},\ldots, x_n) =
\frac{f_j(y_1, \ldots y_{i-1}, 1/x_j, y_{i+1}, \ldots y_n)}{x_j}
\]
However this is not possible for any nonzero functions $f_i$ and $f_j$. To see
this, we can fix all the variables other than $x_j$ and $y_i = 1/x_j$, and take
series expansions of $f_i$ and $f_j$. Then the expansion for $f_i$ would be a power
series in $x_j$ and the expansion for $f_j$ would be a Laurent series in $x_j$ with
only negative degrees of $x_j$ (i.e. a power series in $1/x_j$), and the only way for
the identity above to hold on the level of series is for $f_i = f_j = 0$, since
holomorphic functions are determined by their series expansions.
\end{proof}
%
However, not all is lost!
%
\begin{thm}
The bundle $\O(1)$ admits global sections, and
\[
\Gamma_{\CP^n}(\O(1)) \cong \C^{n+1}
\]
\end{thm}
%
We again give two proofs of this fact, mirroring our proofs for Theorem 1.11 to
illustrate the differences between $\O(-1)$ and $\O(1)$, which are a bit subtle at
first.
%
\begin{proof}[Proof 1]
Since $\O(-1)$ is a subbundle $\O(-1) \hookrightarrow \CP^n \times \C^{n+1}$, we
have that the dual bundle $\O(-1)$ is a subbundle $\O(-1) \hookrightarrow (\C^{n+1})^*$.
Again, any holomorphic section of $\O(1)$ is a holomorphic function
$\CP^n \to \C^{n+1}$, which is constant by the maximum principle. However, any
linear functional $\omega \in (\C^{n+1})^*$ determines a linear functional
all the lines $\ell \in \CP^n$ by taking the restriction $\omega|_\ell$. Therefore,
the space of sections is $(\C^{n+1})^* \cong \C^{n+1}$.
\end{proof}
%
\begin{proof}[Proof 2]
The transition functions for $\O(1)$ are given by $\psi_{ij}(\ell) = z_j/z_i$. Again,
using the trivialization of $\O(1)$ in the standard open cover $\set{U_i}$ for
$\CP^n$, we have that the data of a section $\CP^n \to \O(1)$ is equivalent to
the data of holomorphic functions $f_i : U_i \to \C$, where in $U_i \cap U_j$, we again
interpret $f_i$ as a holomorphic function of the variables  $x_k \defeq z_k/z_i$
(where $x_i$ is missing) and $f_j$ as a holomorphic function of the variables
$y_k \defeq z_k/z_j$ (where $y_j$ is missing). However, we now have that the transition
functions are given by
\[
\psi_{ij} = z_j/z_i = x_j = 1/y_i
\]
We then have that the compatibility condition prescribed by the transition function
is
\[
f_i(x_1,\ldots,\widehat{x_i},\ldots, x_n) =
x_jf_j(y_1,\ldots y_{i-1}, 1/x_j, y_{i+1}, \ldots y_n)
\]
We again take the series expansions of $f_i$ and $f_j$ in the variables $x_j$
and $1/x_j$ (holding the rest fixed)  and we find that if the identity were to hold,
$f_j$ must be at most degree $1$ in $y_i$, i.e. at least degree $-1$ in $x_j$,
and $f_i$ must be at most degree $1$ in $x_j$, since $f_j$ is holomorphic as a function
of $y_i = 1/x_j$ and $f_i$ is holomorphic as a function of $x_j$. Doing this over
all pairs $f_i$ and $f_j$, we find that the compatibility conditions force the
function $f_i$ to be a polynomial of at most degree $1$ in all the
$x_1, \ldots \widehat{x_i}, \ldots x_n$, and a choice of $f_i$ determines all the other
functions $f_j$, since they must have the same coefficients (rearranged appropriately)
as $f_i$. Therefore, the space of sections is isomorphic to the subspace of
$\C[x_1,\ldots,\widehat{x_i},\ldots x_n]$ consisting of linear polynomials in the
variables $x_0, \ldots, \widehat{x_i} x_n$, which is isomorphic to $\C^{n+1}$, or more
suggestively, remembering that $x_i \defeq z_i/z_j$,  the subspace of the space
\[
\C\left[\frac{z_1}{z_i},\ldots \widehat{\frac{z_i}{z_i}}\ldots,\frac{z_n}{z_i}\right]
\]
of linear polynomials in the variables $z_1/z_i,\ldots \widehat{z_i/z_i},\ldots ,z_n/z_i$,
which is isomorphic to the space of degree $1$ homogeneous polynomials in the
$z_0,\ldots z_n$ via multiplication by $z_i$.
\end{proof}
%
Using similar methods, we can determine the sections for $\O(k)$ for any
$k \in \Z$, using either proof method. If $k > 0$, we have that
$\O(-k)$ admits no holomorphic sections, which can be verified by checking the
transition functions, or noting that $\O(-k)$ embeds as a subbundle of
$\CP^n \times (\C^{n+1})^{\otimes^k}$, and repeating the maximum principle argument.
In the case of $\O(k)$, we are in a similar situation as $\O(1)$, where by
looking at how the transition functions interact with the series expansions of
functions, we have that the space of sections is isomorphic to the
space of polynomials in $n$ variables of degree $\leq k$, which by multiplying through
by a the $k^{th}$ power of variable, is isomorphic to the space of homogeneous
polynomials in $n+1$ variables of degree $k$. \\

There is another, more geometric way to interpret why the space of sections of
the bundle $\O(k)$ is the space of degree $k$ homogeneous polynomials, which
echoes the discussion in the first proofs we gave for Theorem 1.11 and 1.12.
A degree $k$ homogeneous polynomial $p \in \C[z_0,\ldots,z_n]_k$ determines is
exactly a linear maps $(\C^{n+1})^{\otimes^k} \to \C$, so the
constant maps $\CP^n \to \C[z_0,\ldots,z_n]_k$ mapping all of $\CP^n$ to a
homogeneous polynomial $p$ are exactly the holomorphic sections of the
trivial bundle $\CP^n \times [(\C^{n+1})^{\otimes^k}]^*$. Restricting
these linear maps to $\ell^{\otimes^k} \subset (\C^{n+1})^{\otimes^k}$, for
$\ell \in \CP^n$ this determines a section of $\O(k)$. \\

As a final remark, note that while we started by working with complex geometry
and holomorphic maps, we were quickly reduced to studying polynomials, which
at the closely intertwined nature of complex and algebraic geometry.
%
\end{document}