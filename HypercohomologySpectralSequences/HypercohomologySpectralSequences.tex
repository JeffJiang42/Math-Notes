\documentclass[psamsfonts, 12pt]{amsart}
%
%-------Packages---------
%
\usepackage[h margin=1 in, v margin=1 in]{geometry}
\usepackage{amssymb,amsfonts}
\usepackage[all,arc]{xy}
\usepackage{tikz-cd}
\usepackage{enumerate}
\usepackage{mathrsfs}
\usepackage{amsthm}
\usepackage{mathpazo}
\usepackage{float}
\usepackage[backend=biber]{biblatex}
\addbibresource{bibliography.bib}
%\usepackage{spectralsequences}
%\usepackage{charter} %another font
%\usepackage{eulervm} %Vakil font
\usepackage[mathcal]{eucal}
\usepackage{yfonts}
\usepackage{mathtools}
\usepackage{enumitem}
\usepackage{mathrsfs}
\usepackage{fourier-orns}
\usepackage[all]{xy}
\usepackage{hyperref}
\usepackage{url}
\usepackage{mathtools}
\usepackage{graphicx}
\usepackage{pdfsync}
\usepackage{mathdots}
\usepackage{calligra}
\usepackage{import}
\usepackage{xifthen}
\usepackage{pdfpages}
\usepackage{transparent}

\usepackage{tgpagella}
\usepackage[T1]{fontenc}
%
\usepackage{listings}
\usepackage{color}

\definecolor{dkgreen}{rgb}{0,0.6,0}
\definecolor{gray}{rgb}{0.5,0.5,0.5}
\definecolor{mauve}{rgb}{0.58,0,0.82}

\lstset{frame=tb,
  language=Matlab,
  aboveskip=3mm,
  belowskip=3mm,
  showstringspaces=false,
  columns=flexible,
  basicstyle={\small\ttfamily},
  numbers=none,
  numberstyle=\tiny\color{gray},
  keywordstyle=\color{blue},
  commentstyle=\color{dkgreen},
  stringstyle=\color{mauve},
  breaklines=true,
  breakatwhitespace=true,
  tabsize=3
  }
%
%--------Theorem Environments--------
%
\newtheorem{thm}{Theorem}[section]
\newtheorem*{thm*}{Theorem}
\newtheorem{cor}[thm]{Corollary}
\newtheorem{prop}[thm]{Proposition}
\newtheorem{lem}[thm]{Lemma}
\newtheorem*{lem*}{Lemma}
\newtheorem{conj}[thm]{Conjecture}
\newtheorem{quest}[thm]{Question}
%
\theoremstyle{definition}
\newtheorem{defn}[thm]{Definition}
\newtheorem*{defn*}{Definition}
\newtheorem{defns}[thm]{Definitions}
\newtheorem{con}[thm]{Construction}
\newtheorem{exmp}[thm]{Example}
\newtheorem{exmps}[thm]{Examples}
\newtheorem{notn}[thm]{Notation}
\newtheorem{notns}[thm]{Notations}
\newtheorem{addm}[thm]{Addendum}
\newtheorem{exer}[thm]{Exercise}
%
\theoremstyle{remark}
\newtheorem{rem}[thm]{Remark}
\newtheorem*{claim}{Claim}
\newtheorem*{aside*}{Aside}
\newtheorem*{rem*}{Remark}
\newtheorem*{hint*}{Hint}
\newtheorem*{note}{Note}
\newtheorem{rems}[thm]{Remarks}
\newtheorem{warn}[thm]{Warning}
\newtheorem{sch}[thm]{Scholium}
%
%--------Macros--------
\renewcommand{\qedsymbol}{$\blacksquare$}
\renewcommand{\sl}{\mathfrak{sl}}
\newcommand{\Bord}{\mathsf{Bord}}
\renewcommand{\hom}{\mathrm{Hom}}
\renewcommand{\emptyset}{\varnothing}
\renewcommand{\O}{\mathcal{O}}
\newcommand{\R}{\mathbb{R}}
\newcommand{\ib}[1]{\textbf{\textit{#1}}}
\newcommand{\Q}{\mathbb{Q}}
\newcommand{\Z}{\mathbb{Z}}
\newcommand{\N}{\mathbb{N}}
\newcommand{\C}{\mathbb{C}}
\newcommand{\A}{\mathbb{A}}
\newcommand{\F}{\mathbb{F}}
\newcommand{\M}{\mathcal{M}}
\newcommand{\dbar}{\overline{\partial}}
\newcommand{\zbar}{\overline{z}}
\renewcommand{\S}{\mathbb{S}}
\newcommand{\V}{\vec{v}}
\newcommand{\RP}{\mathbb{RP}}
\newcommand{\CP}{\mathbb{CP}}
\newcommand{\B}{\mathcal{B}}
\newcommand{\GL}{\mathrm{GL}}
\newcommand{\SL}{\mathrm{SL}}
\newcommand{\SP}{\mathrm{SP}}
\newcommand{\SO}{\mathrm{SO}}
\newcommand{\SU}{\mathrm{SU}}
\newcommand{\gl}{\mathfrak{gl}}
\newcommand{\g}{\mathfrak{g}}
\newcommand{\Bun}{\mathsf{Bun}}
\newcommand{\inv}{^{-1}}
\newcommand{\bra}[2]{ \left[ #1, #2 \right] }
\newcommand{\set}[1]{\left\lbrace #1 \right\rbrace}
\newcommand{\abs}[1]{\left\lvert#1\right\rvert}
\newcommand{\norm}[1]{\left\lVert#1\right\rVert}
\newcommand{\transv}{\mathrel{\text{\tpitchfork}}}
\newcommand{\defeq}{\vcentcolon=}
\newcommand{\enumbreak}{\ \\ \vspace{-\baselineskip}}
\let\oldexists\exists
\renewcommand\exists{\oldexists~}
\let\oldL\L
\renewcommand\L{\mathfrak{L}}
\makeatletter
\newcommand{\incfig}[2]{%
    \fontsize{48pt}{50pt}\selectfont
    \def\svgwidth{\columnwidth}
    \scalebox{#2}{\input{#1.pdf_tex}}
}
%
\newcommand{\tpitchfork}{%
  \vbox{
    \baselineskip\z@skip
    \lineskip-.52ex
    \lineskiplimit\maxdimen
    \m@th
    \ialign{##\crcr\hidewidth\smash{$-$}\hidewidth\crcr$\pitchfork$\crcr}
  }%
}
\makeatother
\newcommand{\bd}{\partial}
\newcommand{\lang}{\begin{picture}(5,7)
\put(1.1,2.5){\rotatebox{45}{\line(1,0){6.0}}}
\put(1.1,2.5){\rotatebox{315}{\line(1,0){6.0}}}
\end{picture}}
\newcommand{\rang}{\begin{picture}(5,7)
\put(.1,2.5){\rotatebox{135}{\line(1,0){6.0}}}
\put(.1,2.5){\rotatebox{225}{\line(1,0){6.0}}}
\end{picture}}
\DeclareMathOperator{\id}{id}
\DeclareMathOperator{\im}{Im}
\DeclareMathOperator{\codim}{codim}
\DeclareMathOperator{\coker}{coker}
\DeclareMathOperator{\supp}{supp}
\DeclareMathOperator{\inter}{Int}
\DeclareMathOperator{\sign}{sign}
\DeclareMathOperator{\sgn}{sgn}
\DeclareMathOperator{\indx}{ind}
\DeclareMathOperator{\alt}{Alt}
\DeclareMathOperator{\Aut}{Aut}
\DeclareMathOperator{\trace}{trace}
\DeclareMathOperator{\ad}{ad}
\DeclareMathOperator{\End}{End}
\DeclareMathOperator{\Ad}{Ad}
\DeclareMathOperator{\Lie}{Lie}
\DeclareMathOperator{\spn}{span}
\DeclareMathOperator{\dv}{div}
\DeclareMathOperator{\grad}{grad}
\DeclareMathOperator{\Sym}{Sym}
\DeclareMathOperator{\sheafhom}{\mathscr{H}\text{\kern -3pt {\calligra\large om}}\,}
\newcommand*\myhrulefill{%
   \leavevmode\leaders\hrule depth-2pt height 2.4pt\hfill\kern0pt}
\newcommand\niceending[1]{%
  \begin{center}%
    \LARGE \myhrulefill \hspace{0.2cm} #1 \hspace{0.2cm} \myhrulefill%
  \end{center}}
\newcommand*\sectionend{\niceending{\decofourleft\decofourright}}
\newcommand*\subsectionend{\niceending{\decosix}}
\def\upint{\mathchoice%
    {\mkern13mu\overline{\vphantom{\intop}\mkern7mu}\mkern-20mu}%
    {\mkern7mu\overline{\vphantom{\intop}\mkern7mu}\mkern-14mu}%
    {\mkern7mu\overline{\vphantom{\intop}\mkern7mu}\mkern-14mu}%
    {\mkern7mu\overline{\vphantom{\intop}\mkern7mu}\mkern-14mu}%
  \int}
\def\lowint{\mkern3mu\underline{\vphantom{\intop}\mkern7mu}\mkern-10mu\int}
%
%--------Hypersetup--------
%
\hypersetup{
    colorlinks,
    citecolor=black,
    filecolor=black,
    linkcolor=blue,
    urlcolor=blacksquare
}
%
%--------Solution--------
%
\newenvironment{solution}
  {\begin{proof}[Solution]}
  {\end{proof}}
%
%--------Graphics--------
%
%\graphicspath{ {images/} }
%
\begin{document}
%
\author{Jeffrey Jiang}
%
\title{Hypercohomology and Spectral Sequences}
%
\maketitle
%
\tableofcontents
%
\section*{Conventions and Notation}
%
We will use cohomological indexing -- for a complex $(A^\bullet, d)$, the differential
increases the degree. We fix an abelian category $\mathcal{A}$ with enough
injectives -- for our purposes it is harmless to assume that $\mathcal{A}$ is
one of $\mathsf{Ab}$, $\mathsf{Mod}_R$, or a category of sheaves over a space
(e.g. $\mathrm{Sh}_{\mathsf{Ab}}(X)$ or $\mathrm{QCoh}(X))$, because of this,
we will be relatively cavalier about using elements when doing homological algebra.
%
\section{Derived Functors and Hypercohomology}
%
\begin{defn}
A \ib{double complex} is a collection of objects $K^{p,q} \in \mathcal{A}$ with
$p,q \in \Z$ equipped with differentials
\begin{align*}
D_1 : K^{p,q} &\to K^{p+1,q} \\
D_2 : K^{p,q} &\to K^{p,q+1}
\end{align*}
such that $D_i^2 = 0$ and $D_1 \circ D_2 = D_2 \circ D_1$.
\end{defn}
%
There is a sign convention with double complexes -- some prefer to replace
the condition that the differentials commute with the condition that they
\emph{anticommute}, i.e. $D_1 \circ D_2 = -D_2\circ D_1$. \\

The way to visualize a double complex is to arrange the objects $K^{p,q}$ in a
grid, with $K^{p,q}$ in the position with coordinate $(p,q)$.
\[\begin{tikzcd}
K^{0,3} \ar[r, "D_1"] & K^{1,3} \ar[r, "D_1"] & K^{2,3} \ar[r, "D_1"] & K^{3,3} \\
K^{0,2} \ar[u, "D_2"] \ar[r, "D_1"] & K^{1,2} \ar[u, "D_2"] \ar[r, "D_1"]
& K^{2,2} \ar[u, "D_2"] \ar[r, "D_1"] & K^{3,2} \ar[u, "D_2"] \\
K^{0,1} \ar[u, "D_2"] \ar[r, "D_1"] & K^{1,1} \ar[u, "D_2"] \ar[r, "D_1"]
& K^{2,1} \ar[u, "D_2"] \ar[r, "D_1"] & K^{3,1} \ar[u, "D_2"] \\
K^{0,0} \ar[u, "D_2"] \ar[r, "D_1"] & K^{1,0} \ar[u, "D_2"] \ar[r, "D_1"]
& K^{2,0} \ar[u, "D_2"] \ar[r, "D_1"] & K^{3,0} \ar[u, "D_2"]
\end{tikzcd}\]
Because of this, we refer to $D_1$ as the \ib{horizontal differential} and
$D_2$ as the \ib{vertical differential}. One thing to note is that the defintion
of a double complex has no finiteness or positivity assumption -- we can
have a nonzero objects $K^{p,q}$ for arbitrarily large $p$ and $q$, and we may also
have nonzero $K^{p,q}$ for $p,q < 0$.
%
\begin{exmp}
Given a complex manifold $X$, we have the sheaves $\mathcal{A}^{p,q}_X$
of smooth $(p,q)$-forms. These come equipped with differentials
$\partial : \mathcal{A}^{p,q}_X \to \mathcal{A}^{p+1,q}_X$ and
$\dbar : \mathcal{A}^{p,q}_X \to \mathcal{A}^{p,q+1}_X$. The operators $\partial$
and $\dbar$ anticommute, so with our sign convention, we need to introduce
the sign $(-1)^p$ to $\dbar$ to get the double complex of sheaves
$(\mathcal{A}^{p,q}_X, \partial, (-1)^p\dbar)$.
\end{exmp}
%
In the case of some finiteness conditions, we can produce a chain complex from
a double complex.
%
\begin{defn}
Let $(K^{p,q},D_1,D_2)$ be a double complex such that $K^{p,q} = 0$ when
$p > p_0$ or $q > q_0$, for some fixed $p_0,q_0 \in \Z$ (i.e. all the objects
become zero once we move far enough down and to the left). Then the
\ib{total complex} (also called the \ib{simple complex}) associated to the double
complex is the complex ($K^\bullet, D$) where
\[
K^n \defeq \bigoplus_{p+q=n}K^{p,q}
\]
and the differential $D$ is given by
\[
D \defeq \bigoplus_{p+q=n}(D_1 \oplus (-1)^pD_2)
\]
\end{defn}
%
For convenience, we will call a double complex that satisfies the
conditions above to be \ib{bounded below}. Given a complex $(A^\bullet, d)$, we
will also say that it is bounded below if there are only finitely many
terms in negative degree. By shifting degrees, we may as well assume that
a bounded below double complex begins indexing at $(0,0)$, and that a bounded
below complex begins at index $0$.
%
\begin{prop}
The total complex of a bounded below double complex actually forms a complex.
\end{prop}
%
\begin{proof}
It suffices to verify this on the component $K^{p,q} \subset K^n$.
The differential $D^2$ on this component maps to
\[
K^{p+q+2} = K^{p+2} \oplus K^{p+1,q+1} \oplus K^{p,q+2}
\]
and is given component wise by the map
$(D_1^2, (-1)^{p+1}(D_2\circ D_1) + (-1)^p(D_1 \circ D_2), D_2^2)$.
Then since $D_i^2 = 0$ and the differentials commute, we have that this is the
zero map.
\end{proof}
%
To make use of double complexes, we first prove a useful fact.
%
\begin{prop}
Let $(I^{p,q},D_1,D_2)$ be a double complex with total complex $(I^\bullet, D)$,
and let $(M^\bullet, d)$ be another complex in $\mathcal{A}$. Then if there
exist injective maps $i^p : M^p \to I^{p,0}$ such that $(I^{p,\bullet},D_2)$ is a
resolution of $M^p$ (i.e. the complex $M^p \to I^{p,\bullet}$ is exact
for all $p$), then the morphism $M^\bullet \to I^\bullet$ by the $i^p$ induces
an isomorphism from the cohomology of $(M^\bullet,d)$ to the cohomology of the
total complex $(I^\bullet, D)$.
\end{prop}
%
Pictorially, the setup is given by the diagram
\[\begin{tikzcd}
\vdots & \vdots & \vdots & \vdots \\
I^{0,2} \ar[u] \ar[r, "D_1"] & I^{1,2} \ar[u] \ar[r, "D_1"]
& I^{2,2} \ar[u] \ar[r, "D_1"] & I^{3,2} \ar[u] \ar[r] & \cdots \\
I^{0,0} \ar[u, "D_2"] \ar[r, "D_1"] & I^{1,0} \ar[u, "D_2"] \ar[r, "D_1"]
& I^{2,0} \ar[u, "D_2"] \ar[r, "D_1"] & I^{3,0} \ar[u, "D_2"] \ar[r] & \cdots \\
M^0 \ar[u, "i^0"] \ar[r, "d"] & M^1 \ar[u, "i^1"] \ar[r, "d"]
& M^2 \ar[u, "i^2"] \ar[r, "d"] & M^3 \ar[u, "i^3"] \ar[r] & \cdots
\end{tikzcd}\]
Where the columns are all exact.
%
\begin{proof}[Proof of proposition]
We prove this for abelian groups. The proof for $R$-modules is near identical,
and the proof can also be adapted for an abelian category of sheaves by looking
at the induced maps on stalks. \\

Let $\alpha \in I^k \defeq \bigoplus_{p+q=k}I^{p,q}$ such that $D\alpha = 0$, and
let $\alpha_{p,q} \in I^{p,q}$ denote the component of $\alpha$ lying in $I^{p,q}$,
so $\alpha = \sum_{p,q}\alpha_{p,q}$. Then since $D\vert_{I^{p,q}}$ is given
by $D = D_1 + (-1)^pD_2$, we have that $D_2\alpha_{0,k} = 0$, since no cancellation
can occur, since you cannot obtain an element in $I^{0,k+1}$ by appplying $D_1$.
Furthermore, we know that for $q < k$ we have
$D_1\alpha_{p,q} + (-1)^p\alpha_{p+1,q-1} = 0$. By assumption, the complex
$(I^{0,\bullet},D_2)$ is exact, since it is a resolution of $M^0$, so we have that
$\alpha_{0,k} = D_2\beta$ for some $\beta \in I^{0,k-1}$. We then consider the
element $\alpha' \defeq \alpha - D\beta$, which defines the same element
as $\alpha$ in $H^k(I^\bullet)$. We note that $\alpha'$ has the property that
$\alpha'_{0,k} = 0$. Because of this, the fact that $D\alpha' = 0$ implies
that $D_2\alpha'_{1,k-1} = 0$, since the only way to get cancellation is
by applying $D_1$ to $\alpha_{0,k}$, which we just noted was $0$. Repeating
this argument, we obtain an element $\tilde{\alpha} \in I^{k,0}$ that defines
the same element in $H^k(I^\bullet)$ as our original element $\alpha$. We
then note that $D\tilde{\alpha} = 0$ implies that $D_i\tilde{\alpha} = 0$,
again since there is no way to obtain cancellation. Then since
$I^{k\bullet}$ is a resolution of $M^k$ via $i^k$, the fact that
$D_2\tilde{\alpha} = 0$ implies by exactness that  $\tilde{\alpha} = i^k(\beta)$
for some $\beta \in M^k$. The fact that $D_1\tilde{\alpha} = 0$ implies that
$d\beta = 0$ by commutativity of the square
\[\begin{tikzcd}
I^{k,0} \ar[r, "D_1"] & I^{k+1,0} \\
M^k \ar[u, "i^k"]\ar[r,"d"'] & M^{k+1} \ar[u, "i^{k+1}"']
\end{tikzcd}\]
Therefore, we have that in the induced map on cohomology, the cohomology class
$[\alpha] = [\tilde{\alpha}] \in H^k(I^\bullet)$ is the image of the class
$[\beta] \in H^k(M^\bullet)$, so the map $H^k(M^\bullet) \to H^k(I^\bullet)$ is
surjective. \\

For injectivity, suppose we have that a cohomology class $[\alpha] \in H^k(M^\bullet)$
maps to $0$, so $i^k(\alpha) = D\beta$ for some $\beta \in I^{k-1}$. We note
that $(D\beta)_{0,k} = 0$ since $i^k$ is a map $M^k \to I^{k,0}$. Furthermore,
we note that $(D\beta)_{0,k} = D_2\beta_{0,k-1}$. Therefore, we can find a
$\gamma \in I^{0,k-2}$ such that $\beta_{0,k-1} = D_2\gamma$. Then setting
$\beta' = \beta - D\gamma$, we have that $D\beta' = D\beta = i^k(\alpha)$ but
$\beta'$ has the property that $\beta'_{0,k-1} = 0$. By repeating this argument,
we may assume that $i^k(\alpha) = D\beta$, where $\beta \in I^{k-1,0}$. Then
since $D_2\beta \in I^{k-1,1}$ and there is no possibility of cancellation, we have
that $D_2\beta = 0$ and $D_1\beta = \alpha$. In the case $k=1$,
we can immediately write $\alpha = D\beta$ with $\beta \in I^{0,0}$, and the
same reasoning as above lets us conclude that $D_2\beta = 0$ and $D_1\beta = \alpha$.
By exactness, $D_2\beta = 0$ implies that $\beta = i^{k-1}(\gamma)$ for some
$\gamma \in M^{k-1}$. Then commutativity of the square
\[\begin{tikzcd}
I^{k-1,0} \ar[r, "D_1"] & I^{k,0} \\
M^{k-1} \ar[u, "i^{k-1}"]\ar[r,"d"'] & M^k \ar[u, "i^k"']
\end{tikzcd}\]
this implies that $d\gamma = \alpha$, so $[\alpha] = 0$. Therefore, the
induced map on cohomology is injective.
\end{proof}
%
We will use this to prove the next proposition.
%
\begin{prop}
let $(M^\bullet,d)$ be a bounded below complex in $\mathcal{A}$. Then there exists
another bounded below complex $(I^\bullet,D)$ in $\mathcal{A}$ and a quasi-isomorphism
$\varphi : M^\bullet \to I^\bullet$ such that
\begin{enumerate}
  \item Each of the $I^k$ is an injective object in $\mathcal{A}$.
  \item Each $\varphi^k : M^k \to I^k$ is injective.
\end{enumerate}
\end{prop}
%
\begin{proof}
Our goal will be to construct a double complex of injective objects satisfying the
conditions of the previous proposition, i.e. a double complex $(I^{p,q},D_1,D_2)$
that is a levelwise resolution  of $M^\bullet$. The desired complex will then be
the total complex of the double complex $I^{p,q}$. To do this, it suffices to
produce a chain map $i^\bullet : M^\bullet \to I^{1,\bullet}$ such that each $i^k$
is an injection into an injective object $I^{1,k}$. Once we do this, we can iteratively
construct the double complex by repeating the argument with
$M^\bullet = \coker i^\bullet$ to obtain a chain map
$\coker i^\bullet \to I^{2,\bullet}$, and then defining the map
$I^{1,\bullet} \to I^{2,\bullet}$ to be the composition  of the quotient map
$I^{1,\bullet} \to \coker i^\bullet$ with the map
$\coker i^\bullet \to I^{2,\bullet}$. \\

We construct the complex $I^{1,\bullet}$ iteratively. To start, we pick
an arbitrary injection $i^0 : M^0 \hookrightarrow  I^{0,0}$ from $M^0$ to an
injective object $I^{0,0}$. We have an injective map $M^0 \to I^{0,0} \oplus M^1$
given by $(i^0,-d)$. Let $j : M^1 \to I^{0,0} \oplus M^1$ and
$k : I^{0,0} \to I^{0,0} \oplus M^1$ be the inclusions, and let
$\pi : I^{0,0} \oplus M^1 \to \coker(I^0,-d)$ be the quotient map. Then fix
an injection $\eta : \coker(i^0,-d) \to I^{1,0}$ for an arbitrary injective object
$I^{1,0}$. Then define $i^1 : M^1 \to I^{1,0}$ by $i^1 \defeq \eta \circ \pi \circ j$,
and define $D_1 : I^{0,0} \to I^{1,0}$ by $D_1 \defeq \eta \circ \pi \circ k$.
We claim that the map $i^1$ is injective. By the definition of the cokernel,
an element $\alpha \in M^1$ maps to $0$ if it is contained in the image of
$(i^0,-d) : M^0 \to I^{0,0} \oplus M^1$. However, since $i^0$ is injective,
the only element with a $0$ in the first component is $(0,0)$. Therefore,
since $j$ and $\eta$ are injective, $i^1(\alpha) = 0$ if and only if $\alpha = 0$.
We then claim that $D_1 \circ i^0 = i^1 \circ d$. We note that for $\alpha \in M^0$
we have
\begin{align*}
i^1(d\alpha) &= (\eta \circ \pi)(0,d\alpha) \\
D_1(i^0(\alpha)) &= (\eta \circ \pi)(i^0(\alpha),0)
\end{align*}
So if suffices to show that $\pi(0,d\alpha) = \pi(i^0(\alpha),0)$, but we have that
$(i^0(\alpha),0) - (0,d_\alpha) = (i^0(\alpha),-d\alpha) \in \coker(i^0,-d)$, so
this is true. In summary, what we have constructed so far can be summarized by the
diagram
\[\begin{tikzcd}
I^{0,0} \ar[r, "D_1"] & I^{1,0} \\
M^0 \ar[u, "i^0"]\ar[r, "d"] & M^1 \ar[r, "d"] \ar[u, "i^1"]& M^2
\end{tikzcd}\]
To continue this, let $\overline{i^1} : M^1 \to \coker D_1$ denote the composition
of $i^1$ with the quotient map. This gives us a map
$(\overline{i^1},-d) : M^1 \to \coker D_1 \oplus M^2$. As we did before,
fix an injection $\eta : \coker(\overline{i^1},-d) \to I^{2,0}$, and define the
maps $i^2$ and $D_1$ analogously to the previous construction. The map $i^2$ is
injective and $D_1 \circ i^1 = i^2 \circ d$ by near identical reasoning to before.
What's left to show is that $D_1^2 = 0$, but this comes immediately from
the fact that $\eta \circ \pi$ vanishes on the image of $(\overline{i^1}, -d)$
by construction. Continuing this process then constructs the desired complex
$I^{\bullet,0}$ of injectives.
\end{proof}
%
A consequence of this proposition is that for the purposes of cohomology,
we can always assume we are computing the cohomology of a complex of injective
objects. In more modern language, this says that any bounded below complex $M^\bullet$
is isomorphic to a complex of injectives in the derived category $D(\mathcal{A})$. \\

The goal of the preceding discussion is to extend the classical notion of
the right derived functors of a left exact functor $F: \mathcal{A} \to \mathcal{B}$.
The classical examples (e.g. $\mathrm{Ext}$ and $\mathrm{Tor}$) are defined
on objects of $\mathcal{A}$. However, the preceding discussion allows us
to extend the functors $R^iF$ to bounded below complexes $M^\bullet$ in $\mathcal{A}$.
Echoing the classical definition, we would like to define $R^iF(M^\bullet)$ on
a bounded below complex $M^\bullet$ to be the cohomology groups $H^i(F(I^\bullet))$,
where $I^\bullet$ is a complex of injective objects admitting a quasi-isomorphism
$i^\bullet : M^\bullet \to I^\bullet$, with each $i^k$ injective (which always exists
by the previous discussion). Indeed, this is the correct definition, but there is work
to be done to show that it is well defined. We fix once and for all a left
exact functor $F$. We first collect some lemmas.
%
\begin{lem}
Let $i^\bullet : M^\bullet \to I^\bullet$
be a quasi-isomorphism of a bounded below complex $M^\bullet$ into a complex $I^\bullet$
of injective objects with each $i^k$ injective. Then for another complex
of injective objects $J^\bullet$ with a quasi-isomorphism
$j^\bullet : M^\bullet \to J^\bullet$ (where each $j^k$ need not be injective),
there exists a chain map $\varphi^\bullet I^\bullet \to J^\bullet$, well defined
up to homotopy, such that $\varphi^\bullet \circ j^\bullet$.
\end{lem}
%
\begin{proof}
Let $d_M$, $d_I$, and $d_J$ denote the differentials for $M^\bullet$, $I^\bullet$,
and $J^\bullet$ respectively. Since $i^0 : M^0 \to I^0$ is an injective map,
and $J^0$ is an injective object, the map $j^0 : M^0 \to J^0$ extends to a map
$\varphi^0 : I^0 \to J^0$ such that $j^0 = \varphi^0 \circ i^0$. We then define
$\varphi^1 : I^1 \to J^1$. To do this, we first claim that the kernel
of $d_I + i^1 : I^0 \oplus M^1 \to I^1$ is equal to the image of
$(i^0,-d_M) : M^0 \to I^0 \oplus M^1$. The inclusion
$\im(i^0,-d_M) \subset \ker(d_I + i^1)$ follows immediately from $i^\bullet$
being a chain map. For the opposite inclusion, let $(x,m) \in \ker(d_I + i^1)$,
so $i^1(m) = -d_Ix$. Therefore, $i^1(m)$ is $0$ in $H^1(I^\bullet)$. Then since
$i^\bullet$ is a quasi-isomorphism, the induced map $H^1(M^\bullet) \to H^1(I^\bullet)$
is injective, so we know that $m = d_Mm_0$ for some $m_0 \in M^0$.
Then since $H^0(M^\bullet) \to H^0(I^\bullet)$ is surjective, and there exist
no terms in negative degree, we have that $x = i^0(n)$ for some $n \in M^0$.
Therefore, what remains is to show that $n = m_0$, which again follows immediately
from the fact that there is no term in negative degree. Then if we
consider the map $(d_J \circ \varphi^0) + j^1 : I^0\oplus M^1 \to J^1$,
we have that $d_J \circ \varphi^0 \circ i^0 = d_J \circ j^0$, so we
have that $(d_J \circ \varphi^0) + j^1$ vanishes on $\im(i^0,-d_M)$ since
$j^\bullet$ is a chain map. Then since $d_I + i^1$ also vanishes on $\im(i^0,-d_M)$,
we have that $(d_J \circ \varphi^0) + j^1$ factors through the
inclusion $\im(d_I + i^1) \hookrightarrow I^1$. Then since $J^1$ is
injective, this extends to a map $\varphi^1: I^1 \to J^1$. We construct
the rest of the maps $\varphi^k$ in the same fashion, and omit the proof
of the uniqueness of $\varphi^\bullet$ up to homotopy out of laziness.
\end{proof}
%
With this, we can prove that our tentative definition of $R^iF(M^\bullet)$ is
reasonable.
%
\begin{prop}
The definition of $R^iF(M^\bullet)$ is independent (up to canonical isomorphism)
from the choice of injective quasi-isomorphism $i^\bullet : M^\bullet \to I^\bullet$.
In other words, given another injective quasi-isomorphism
$j^\bullet : M^\bullet \to J^\bullet$, there exists a canonical isomorphism
$H^i(F(I^\bullet)) \to H^i(F(J^\bullet)$).
\end{prop}
%
\begin{proof}
By the preceding lemma, we get chain maps $\varphi^\bullet : I^\bullet \to J^\bullet$
and $\psi^\bullet : J^\bullet \to I^\bullet$ such that
$j^\bullet = \varphi^\bullet \circ i^\bullet$ and
$i^\bullet = \psi^\bullet \circ i^\bullet$. We have that
$\id_{I^\bullet} - (\psi^\bullet \circ \varphi^\bullet)$ is zero on $M^\bullet$,
regarded as asubcomplex  of $I^\bullet$ and similarly for
$\id_{J^\bullet} - (\varphi^\bullet\circ \psi^\bullet)$,
which show that $\varphi$ and $\psi$ induce isomorphisms on cohomology.
In fact, it can be shown that the compositions are chain homotopic to the
identity, so applying the functor $F$ gives a chain homotopy
from $F(I^\bullet) \to F(J^\bullet)$, giving us the desired isomorphism
$H^i(F(I^\bullet)) \to H^i(F(J^\bullet))$.
\end{proof}
%
This definition is perfectly fine, but we want to weaken the assumptions for
the quasi-isomorphism $i^\bullet : M^\bullet \to I^\bullet$ used to define
$R^iF(M^\bullet)$. The first thing is to remove the requirement that the
$i^k : M^k \to I^k$ be injective. To prove this, we first introduce some new defintions.
%
\begin{defn}
Let $(A^\bullet, d_A)$ and $(B^\bullet, d_B)$ be complexes in $\mathcal{A}$,
and let $\varphi^\bullet : A^\bullet \to B^\bullet$ be a chain map.
The \ib{cone} of $\varphi^\bullet$, is the complex $(C^\bullet, d_C)$ where
$C^k = A^k \oplus B^{k-1}$ and the differntial $d_C^k : C^k \to C^{k+1}$ is given by
\[
d_C^k \defeq \begin{pmatrix}
d_A^k & (-1)^k\varphi^k \\
0 & d^{k-1}_B
\end{pmatrix}
\]
\end{defn}
%
\begin{defn}
Given a chain complex $A^\bullet$, the \ib{shift} of $A$, denoted $A[n]^\bullet$
is the chain complex such that $A[n]^k = A^{k-n}$.
\end{defn}
%
Given a chain map $\varphi^\bullet : A^\bullet \to B^\bullet$ with cone $C^\bullet$,
we get a split short exact sequence of chain complexes
\[\begin{tikzcd}
0 \ar[r] & B[1]^\bullet \ar[r] & C^\bullet \ar[r] & A^\bullet \ar[r] & 0
\end{tikzcd}\]
where the map $B[1]^k \to C^k$ is given by the inclusion
$B[1]^k \defeq B^{k-1} \hookrightarrow C^k \defeq A^k \oplus B^{k-1}$ and
the map $C^k \to A^k$ is given by projection onto $A^k$.
%
\begin{lem}
Let $\varphi^\bullet : A^\bullet \to B^\bullet$ be a quasi-isomorphism, and let
$C^\bullet$ be the cone of $\varphi^\bullet$. Then $H^i(C^\bullet) = 0$
\end{lem}
%
\begin{proof}
We use the long exact sequence induced by the short exact sequence
\[\begin{tikzcd}
0 \ar[r] & B[1]^\bullet \ar[r] & C^\bullet \ar[r] & A^\bullet \ar[r] & 0
\end{tikzcd}\]
A diagram chase reveals that the connecting homomorphism
$H^k(A^\bullet) \to H^{k+1}(B[1]^\bullet) = H^k(B^\bullet)$ is the map induced
by $\varphi^\bullet$, which is an isomorphism since $\varphi^\bullet$ is a
quasi-isomorphism. Therefore, by exactness we can conclude that
the cohomology groups of $C^\bullet$ vanish.
\end{proof}
%
We will also use (but won't prove) a standard fact about injective objects
%
\begin{prop}
Any exact complex $I^\bullet$ of injective objects is homotopy equivalent to the
trivial complex, i.e. There exists a chain homotopy $H^k : I^k \to I^{k-1}$ such that
$H^{k+1} \circ d^k + d^{k-1}+ \circ H^k = \id_{I^k}$.
\end{prop}
%
\begin{cor}
For an exact bounded below complex $I^\bullet$ of injective objects,
the complex $F(I^\bullet)$ is exact.
\end{cor}
%
\begin{proof}
Appyling the functor $F$ to to the chain homotopy $H^k : I^k \to I^{k-1}$
gives the desired chain homotopy for $F(I^\bullet)$ to the trivial complex in
$\mathcal{B}$, so $F(I^\bullet)$ is exact.
\end{proof}
%
\begin{prop}
Let $\alpha^\bullet : M^\bullet \to I^\bullet$ be any quasi-isomorphism, where
$I^\bullet$ is a complex of injective objects. Then $\alpha^\bullet$ induces
an isomorphsim  $R^iF(M^\bullet) \cong H^i(F(I^\bullet))$.
\end{prop}
%
\begin{proof}
Let $i^\bullet : M^\bullet \to J^\bullet$ be a quasi-isomorphism, with
each $i^k$ injective, so $R^iF(M^\bullet) = H^i(F(J^\bullet))$. Then by
Lemma 1.7, we have that there exists a chain map
$\varphi^\bullet : J^\bullet \to I^\bullet$ such that
$\alpha^\bullet = \varphi^\bullet \circ i^\bullet$. Since $\alpha^\bullet$ and
$i^\bullet$ are quasi-isomorphisms, $\varphi^\bullet$ is necessarily a
quasi-isomorphism. Then consider the cone $C^\bullet$ of $\varphi^\bullet$.
By Lemma 1.11, we have that $C^\bullet$ has trivial cohomology, and
consists of injective objects, since $I^\bullet$ and $J^\bullet$ are complexes
of injective objects. Applying $F$ to the short exact sequence gives us
\[\begin{tikzcd}
0 \ar[r] & F(I[1]^\bullet) \ar[r] & F(C^\bullet) \ar[r] & F(J^\bullet) \ar[r] & 0
\end{tikzcd}\]
which remains exact, since the original short exact sequence was split. Then
taking the long exact sequence in cohomology, we use the Corollary 1.13 to
conclude that $F(C^\bullet)$ has no cohomology, so the map
$H^k(F(J^\bullet)) \to H^k(F(I^\bullet)) = R^iF(M^\bullet)$ induced by
$F(\varphi^\bullet)$ is an isomorphism.
\end{proof}
%
As a consequence, we get a functoriality property for the functors $R^iF$.
%
\begin{cor}
Let $\alpha^\bullet : M^\bullet \to N^\bullet$ be a quasi-isomorphism
of bounded below complexes. Then for any choice of injective quasi-isomorphisms
$i^\bullet : M^\bullet \to I^\bullet$ and $j^\bullet : N^\bullet \to J^\bullet$
into complexes $I^\bullet$ and $J^\bullet$ of injective objects, we get a
canonical map
\[
R^iF(M^\bullet) = H^k(F(I^\bullet)) \to H^k(J^\bullet) = R^iF(N^\bullet)
\]
\end{cor}
%
\begin{proof}
Apply the previous proposition to the quasi-isomorphism $j^\bullet \circ \alpha^\bullet$.
\end{proof}
%
We close with some functoriality properties, but omit the proofs out of laziness.
%
\begin{prop}
Let $\varphi^\bullet : A^\bullet \to B^\bullet$ be a chain map, and
let $i^\bullet : A^\bullet \to I^\bullet$ and $j^\bullet : B^\bullet \to J^\bullet$
be quasi-isomorphisms to injective complexes with $i^\bullet$ injective. Then
there exists a chain map $\psi^\bullet : I^\bullet \to J^\bullet$, unique
up to homotopy, such that the following diagram commutes:
\[\begin{tikzcd}
A^\bullet\ar[d, "i^\bullet"']\ar[r, "\varphi^\bullet"]&B^\bullet\ar[d, "j^\bullet"] \\
I^\bullet \ar[r, "\psi^\bullet"'] & J^\bullet
\end{tikzcd}\]
\end{prop}
%
\begin{cor}
Let $\varphi^\bullet : A^\bullet \to B^\bullet$ be any chain map. Then
for injective quasi-isomorphisms $i^\bullet : A^\bullet \to I^\bullet$
and $j^\bullet : B^\bullet \to J^\bullet$ into complexes of injectives,
we get a canonical map
\[
R^iF(A^\bullet) = H^i(F(I^\bullet)) \to H^i(F(J^\bullet)) = R^iF(B^\bullet)
\]
\end{cor}
%
Using this, we can show that we can replace the injective complexes with
$F$-acyclic objects, i.e. complexes $N^\bullet$ where $F(N^i) = 0$.
%
\begin{prop}
Let $\varphi^\bullet : M^\bullet \to N^\bullet$ be a quasi-isomorphism
from $M^\bullet$ to a complex of $N^\bullet$ of $F$-acyclic objects. Then
$\varphi^\bullet$ induces an isomorphism $R^iF(M^\bullet) \to H^i(F(N^\bullet))$.
\end{prop}
%
In summary, we have extended the notion of $R^iF$ to bounded below complexes --
we can recover the original notion by considering an object $A$ in $\mathcal{A}$
as a complex with a single term in degree $0$, and zeroes elsewhere.
In this case, the new definition coincides with the original, since we may
regard an injective resolution $0 \to A \to I^\bullet$ of $A$ as an
injective quasi-isomorphism
\[\begin{tikzcd}
A \ar[d]\ar[r] & 0 \ar[d]\ar[r] & 0 \ar[d]\ar[r] & \cdots \\
I^1 \ar[r] & I^2 \ar[r] & I^3 \ar[r] & \cdots
\end{tikzcd}\]
%
\section{The Holomorphic de Rham Theorem}
%
We now apply our foray into derived functors to sheaves. Let
$\mathcal{A}$ denote some abelian category of sheaves over a space $X$
(e.g. $\mathrm{Sh}_{\mathsf{Ab}}(X)$, $\mathrm{Mod}_{\O_X}$, etc.).
Let $\Gamma$ denote the global sections functor for $\mathcal{A}$.
%
\begin{defn}
For a bounded below complex $\mathcal{F}^\bullet$ of sheaves, the \ib{hypercohomology}
of $\mathcal{F}^\bullet$, denoted $\mathbb{H}^k(X,\mathcal{F}^\bullet)$,
is
\[
\mathbb{H}^k(X,\mathcal{F}^\bullet) \defeq R^k\Gamma(\mathcal{F}^\bullet)
\]
\end{defn}
%
An resolution $0 \to \mathcal{F} \to \mathcal{G}^\bullet$ is
the same thing as a quasi-isomorphism from $\mathcal{F}$, regarded as a complex
with the complex $\mathcal{G}^\bullet$, giving us that
$H^k(X,\mathcal{F}) \cong \mathbb{H}^k(X,\mathcal{G}^\bullet)$. \\

We now restrict to the case where $X$ is a complex manifold of complex dimension $n$.
Let $\O_X$ denote the sheaf of holomorphic functions on $X$, and let $\Omega^{p,0}_X$
denote the sheaves of holomorphic $p$-forms on $X$. The de Rham differential
$d : \mathcal{A}^k_X \to \mathcal{A}^{k+1}_X$ on the sheaves $\mathcal{A}^k_X$
of smooth $k$-forms splits as $d = \partial + \dbar$, where
$\partial : \mathcal{A}^{p,q}_X \to \mathcal{A}^{p+1}$ and
$\dbar : \mathcal{A}^{p,q} \to \mathcal{A}^{p,q+1}_X$ that anticommute. Furthermore,
the operator $\partial$ preserves holomorphic forms, i.e. it maps
$\Omega^{p,0}_X \subset \mathcal{A}^{p,0}_X$ into
$\Omega^{p+1,0}_X \subset \mathcal{A}^{p+1,0}$ since the holomorphic $p$-forms
are exactly those annihilated by $\dbar$. This gives us the
\ib{holomorphic de Rham complex}:
\[\begin{tikzcd}
0 \ar[r]  & \O_X \ar[r, "\partial"] & \Omega^{1,0} \ar[r, "\partial"] &
\cdots \ar[r, "\partial"] & \Omega^{n,0} \ar[r] & 0
\end{tikzcd}\]
There is an inclusion $i : \underline{\C} \hookrightarrow \O_X$ of the locally
constant functions into the sheaf of holomorphic functions.
%
\begin{prop}[\ib{The holomorphic Poincar\'e Lemma}]
The map $i : \underline{\C} \to \Omega^{\bullet,0}$ is a resolution of the constant
sheaf $\underline{\C}$, i.e. the complex
\[\begin{tikzcd}
0 \ar[r] & \underline{\C} \ar[r, "i"] & \O_X \ar[r, "\partial"] &
\Omega^{1,0} \ar[r, "\partial"] & \cdots \ar[r, "\partial"] & \Omega^{n,0} \ar[r] & 0
\end{tikzcd}\]
is exact
\end{prop}
%
\begin{proof}
Consider the double complex $(\mathcal{A}^{p,q}_X, \partial, (-1)^p\dbar)$.
We have inclusions $\Omega^{\bullet,0}_X \hookrightarrow \mathcal{A}^{\bullet,q}_X$,
giving us
\[\begin{tikzcd}
\mathcal{A}^{0,n} \ar[r, "\partial"] & \mathcal{A}^{1,n} \ar[r,"\partial"]
& \cdots \ar[r, "\partial"] & \mathcal{A}^{n,n} \\
\vdots \ar[u, "\dbar"]& \vdots \ar[u, "-\dbar"]
& \vdots \ar[u, "(-1)^p\dbar"] & \vdots \ar[u, "(-1)^n\dbar"]\\
\mathcal{A}^{0,1} \ar[r, "\partial"] \ar[u, "\dbar"]
& \mathcal{A}^{1,1} \ar[r,"\partial"] \ar[u, "-\dbar"]
& \cdots \ar[r, "\partial"] \ar[u,"(-1)^p\dbar"]
& \mathcal{A}^{n,1} \ar[u, "(-1)^n\dbar"]\\
C^\infty_X(\C) \ar[r, "\partial"] \ar[u, "\dbar"]
& \mathcal{A}^{1,0} \ar[r,"\partial"] \ar[u, "-\dbar"]
& \cdots \ar[r, "\partial"] \ar[u, "(-1)^p\dbar"]
& \mathcal{A}^{n,0} \ar[u, "(-1)^n\dbar"]\\
\O_X \ar[r, "\partial"] \ar[u, hookrightarrow]
& \Omega^{1,0}_X \ar[r, "\partial"] \ar[u, hookrightarrow]
& \cdots \ar[r," \partial"] \ar[u, hookrightarrow] & \Omega^{n,0} \ar[u, hookrightarrow]
\end{tikzcd}\]
The columns are exact by the $\dbar$-Poincar\'e lemma, so by Proposition 1.5,
the chain map from the holomorphic de Rham complex to the total complex of
$(\mathcal{A}^{p,q}, \partial, (-1)^p\dbar)$ is a quasi-isomorphism.
We then note that the total complex is nothing but the smooth de Rham
complex $(\mathcal{A}^\bullet, d)$, which is exact by the smooth Poincar\'e lemma,
and that the map to the total complex induced by the inclusions
$\Omega^{p,0} \hookrightarrow \mathcal{A}^{p,0}$ is just the inclusion
$\Omega^{p,0} \hookrightarrow \mathcal{A}^k$. Furthermore, the $0^{th}$ cohomology
sheaf of the smooth de Rham complex is the constant sheaf $\underline{\C}$.
Therefore, we have that the holomorphic de Rham complex is a resolution of
$\underline{\C}$.
\end{proof}
%
\begin{cor}[\ib{The Holomorphic de Rham Theorem}]
There is an isomorphism
\[
H^k(X,\C) \cong \mathbb{H}^k(X, \Omega_X^{\bullet,0})
\]
\end{cor}
%
\newpage
%
\nocite{*}
%
\printbibliography
%
\end{document}