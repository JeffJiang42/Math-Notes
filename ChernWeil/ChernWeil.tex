\documentclass[psamsfonts, 12pt]{amsart}
%
%-------Packages---------
%
\usepackage[h margin=1 in, v margin=1 in]{geometry}
\usepackage{amssymb,amsfonts}
\usepackage{amsmath}
\usepackage{accents}
\usepackage[all,arc]{xy}
\usepackage{tikz-cd}
\usepackage{enumerate}
\usepackage{mathrsfs}
\usepackage{amsthm}
\usepackage{mathpazo}
\usepackage{float}
%\usepackage[backend=biber]{biblatex}
%\addbibresource{bibliography.bib}
%\usepackage{charter} %another font
%\usepackage{eulervm} %Vakil font
\usepackage{yfonts}
\usepackage{mathtools}
\usepackage{enumitem}
\usepackage{mathrsfs}
\usepackage{fourier-orns}
\usepackage[all]{xy}
\usepackage{hyperref}
\usepackage{url}
\usepackage{mathtools}
\usepackage{graphicx}
\usepackage{pdfsync}
\usepackage{mathdots}
\usepackage{calligra}
\usepackage{import}
\usepackage{xifthen}
\usepackage{pdfpages}
\usepackage{transparent}

\usepackage{tgpagella}
\usepackage[T1]{fontenc}
%
\usepackage{listings}
\usepackage{color}

\definecolor{dkgreen}{rgb}{0,0.6,0}
\definecolor{gray}{rgb}{0.5,0.5,0.5}
\definecolor{mauve}{rgb}{0.58,0,0.82}

\lstset{frame=tb,
  language=Matlab,
  aboveskip=3mm,
  belowskip=3mm,
  showstringspaces=false,
  columns=flexible,
  basicstyle={\small\ttfamily},
  numbers=none,
  numberstyle=\tiny\color{gray},
  keywordstyle=\color{blue},
  commentstyle=\color{dkgreen},
  stringstyle=\color{mauve},
  breaklines=true,
  breakatwhitespace=true,
  tabsize=3
  }
%
%--------Theorem Environments--------
%
\newtheorem{thm}{Theorem}[section]
\newtheorem*{thm*}{Theorem}
\newtheorem{cor}[thm]{Corollary}
\newtheorem{prop}[thm]{Proposition}
\newtheorem{lem}[thm]{Lemma}
\newtheorem*{lem*}{Lemma}
\newtheorem{conj}[thm]{Conjecture}
\newtheorem{quest}[thm]{Question}
%
\theoremstyle{definition}
\newtheorem{defn}[thm]{Definition}
\newtheorem*{defn*}{Definition}
\newtheorem{defns}[thm]{Definitions}
\newtheorem{con}[thm]{Construction}
\newtheorem{exmp}[thm]{Example}
\newtheorem{exmps}[thm]{Examples}
\newtheorem{notn}[thm]{Notation}
\newtheorem{notns}[thm]{Notations}
\newtheorem{addm}[thm]{Addendum}
\newtheorem{exer}[thm]{Exercise}
%
\theoremstyle{remark}
\newtheorem{rem}[thm]{Remark}
\newtheorem*{claim}{Claim}
\newtheorem*{aside*}{Aside}
\newtheorem*{rem*}{Remark}
\newtheorem*{hint*}{Hint}
\newtheorem*{note}{Note}
\newtheorem{rems}[thm]{Remarks}
\newtheorem{warn}[thm]{Warning}
\newtheorem{sch}[thm]{Scholium}
%
%--------Macros--------
\renewcommand{\qedsymbol}{$\blacksquare$}
\renewcommand{\sl}{\mathfrak{sl}}
\newcommand{\Bord}{\mathsf{Bord}}
\renewcommand{\hom}{\mathsf{Hom}}
\renewcommand{\emptyset}{\varnothing}
\renewcommand{\O}{\mathcal{O}}
\newcommand{\R}{\mathbb{R}}
\newcommand{\ib}[1]{\textbf{\textit{#1}}}
\newcommand{\Q}{\mathbb{Q}}
\newcommand{\Z}{\mathbb{Z}}
\newcommand{\N}{\mathbb{N}}
\newcommand{\C}{\mathbb{C}}
\newcommand{\A}{\mathbb{A}}
\newcommand{\F}{\mathbb{F}}
\newcommand{\M}{\mathcal{M}}
\newcommand{\dbar}{\overline{\partial}}
\newcommand{\zbar}{\overline{z}}
\renewcommand{\S}{\mathbb{S}}
\newcommand{\V}{\vec{v}}
\newcommand{\RP}{\mathbb{RP}}
\newcommand{\CP}{\mathbb{CP}}
\newcommand{\B}{\mathcal{B}}
\newcommand{\GL}{\mathrm{GL}}
\newcommand{\SL}{\mathrm{SL}}
\newcommand{\SP}{\mathrm{SP}}
\newcommand{\SO}{\mathrm{SO}}
\newcommand{\SU}{\mathrm{SU}}
\newcommand{\gl}{\mathfrak{gl}}
\newcommand{\g}{\mathfrak{g}}
\newcommand{\Bun}{\mathsf{Bun}}
\newcommand*{\dt}[1]{%
   \accentset{\mbox{\large\bfseries .}}{#1}}
\newcommand{\inv}{^{-1}}
\newcommand{\bra}[2]{ \left[ #1, #2 \right] }
\newcommand{\set}[1]{\left\lbrace #1 \right\rbrace}
\newcommand{\abs}[1]{\left\lvert#1\right\rvert}
\newcommand{\norm}[1]{\left\lVert#1\right\rVert}
\newcommand{\transv}{\mathrel{\text{\tpitchfork}}}
\newcommand{\defeq}{\vcentcolon=}
\newcommand{\enumbreak}{\ \\ \vspace{-\baselineskip}}
\let\oldexists\exists
\renewcommand\exists{\oldexists~}
\let\oldL\L
\renewcommand\L{\mathfrak{L}}
\makeatletter
\newcommand{\incfig}[2]{%
    \fontsize{48pt}{50pt}\selectfont
    \def\svgwidth{\columnwidth}
    \scalebox{#2}{\input{#1.pdf_tex}}
}
%
\newcommand{\tpitchfork}{%
  \vbox{
    \baselineskip\z@skip
    \lineskip-.52ex
    \lineskiplimit\maxdimen
    \m@th
    \ialign{##\crcr\hidewidth\smash{$-$}\hidewidth\crcr$\pitchfork$\crcr}
  }%
}
\makeatother
\newcommand{\bd}{\partial}
\newcommand{\lang}{\begin{picture}(5,7)
\put(1.1,2.5){\rotatebox{45}{\line(1,0){6.0}}}
\put(1.1,2.5){\rotatebox{315}{\line(1,0){6.0}}}
\end{picture}}
\newcommand{\rang}{\begin{picture}(5,7)
\put(.1,2.5){\rotatebox{135}{\line(1,0){6.0}}}
\put(.1,2.5){\rotatebox{225}{\line(1,0){6.0}}}
\end{picture}}
\DeclareMathOperator{\id}{id}
\DeclareMathOperator{\im}{Im}
\DeclareMathOperator{\codim}{codim}
\DeclareMathOperator{\coker}{coker}
\DeclareMathOperator{\supp}{supp}
\DeclareMathOperator{\inter}{Int}
\DeclareMathOperator{\sign}{sign}
\DeclareMathOperator{\sgn}{sgn}
\DeclareMathOperator{\indx}{ind}
\DeclareMathOperator{\alt}{Alt}
\DeclareMathOperator{\Aut}{Aut}
\DeclareMathOperator{\trace}{trace}
\DeclareMathOperator{\ad}{ad}
\DeclareMathOperator{\End}{End}
\DeclareMathOperator{\Ad}{Ad}
\DeclareMathOperator{\Lie}{Lie}
\DeclareMathOperator{\spn}{span}
\DeclareMathOperator{\dv}{div}
\DeclareMathOperator{\grad}{grad}
\DeclareMathOperator{\Sym}{Sym}
\DeclareMathOperator{\tr}{tr}
\DeclareMathOperator{\sheafhom}{\mathscr{H}\text{\kern -3pt {\calligra\large om}}\,}
\newcommand*\myhrulefill{%
   \leavevmode\leaders\hrule depth-2pt height 2.4pt\hfill\kern0pt}
\newcommand\niceending[1]{%
  \begin{center}%
    \LARGE \myhrulefill \hspace{0.2cm} #1 \hspace{0.2cm} \myhrulefill%
  \end{center}}
\newcommand*\sectionend{\niceending{\decofourleft\decofourright}}
\newcommand*\subsectionend{\niceending{\decosix}}
\def\upint{\mathchoice%
    {\mkern13mu\overline{\vphantom{\intop}\mkern7mu}\mkern-20mu}%
    {\mkern7mu\overline{\vphantom{\intop}\mkern7mu}\mkern-14mu}%
    {\mkern7mu\overline{\vphantom{\intop}\mkern7mu}\mkern-14mu}%
    {\mkern7mu\overline{\vphantom{\intop}\mkern7mu}\mkern-14mu}%
  \int}
\def\lowint{\mkern3mu\underline{\vphantom{\intop}\mkern7mu}\mkern-10mu\int}
%
%--------Hypersetup--------
%
\hypersetup{
    colorlinks,
    citecolor=black,
    filecolor=black,
    linkcolor=blue,
    urlcolor=blacksquare
}
%
%--------Solution--------
%
\newenvironment{solution}
  {\begin{proof}[Solution]}
  {\end{proof}}
%
%--------Graphics--------
%
%\graphicspath{ {images/} }
%
\begin{document}
%
\author{Jeffrey Jiang}
%
\title{Chern-Weil Theory}
%
\maketitle
%
\tableofcontents
%
\section{Lie Algebra Valued Differential Forms}
%
\begin{defn}
let $E \to M$ be a vector bundle. An \ib{$E$-valued differential $k$-form} is
a section of $\Lambda^kT^*M \otimes E$. The space of $E$-valued $k$-forms is denoted
$\Omega^k_M(E)$. In the case that $E$ is a trivial bundle $M \times V$, we
abbreviated this to $V$-valued $k$-forms, and denote this space by $\Omega^k_M(V)$.
In a local trivialization of $E$, an element $\omega \in \Omega^k_M(E)$ can
be thought of as a vector of $k$-forms.
\end{defn}
%
One thing to keep in mid is that, unless $E$ is a trivial bundle, the
vector of $k$-forms in a local trivialization does not transform tensorially --
the transformation law depends both on the coordinates on $M$ as well was the
transition functions of the bundle $E$. That being said, it is still extremely
useful for local computations. \\

The situation we will be interested in is when $E$ is a trivial bundle $P \times \g$,
where $\g$ is some Lie algebra. We will briefly discuss some of the common notation
and operations on Lie algebra valued forms. Many of these operations are analogous to
operations on real valued forms, but can differ in subtle ways. Fix a basis $\set{\xi_i}$
for $\g$. Then any Lie algebra valued $k$-form $\Theta \in \Omega^k_P(\g)$ can be written
uniquely as
\[
\Theta = \Theta^i \otimes \xi_i
\]
for real valued $k$-forms $\Theta^i \in \Omega^k_P$. For brevity, we usually omit the
tensor symbol, and abbreviate this to $\Theta^i\xi_i$. Given a $k$-form
$\Theta \in \Omega^k_P(\g)$ and an $\ell$-form $\Omega \in \Omega^\ell_P(\g)$, we
define their wedge product $\Theta\wedge\Omega \in \Omega^{k+\ell}(\g\otimes\g)$to be
\[
\Theta \wedge \Omega \defeq (\Theta^i\wedge\Omega^j)\xi_i \otimes  \xi_j
\]
Using the usual wedge product on differential forms. Note that this wedge product
does not have a normalizing factor, which will explain various factors of $2$
and $1/2$ when doing computations with Lie algebra valued forms. In addition, we note
that, unlike for real valued forms, we do not necessarily have $\Theta\wedge\Theta = 0$.
We also note that this definition for the wedge product works for general vector bundle
valued forms as well, but when we have a Lie algebra, we have a Lie bracket
$[\cdot,\cdot]: \g \otimes \g \to \g$, which allows us to define
\[
[\Theta\wedge\Omega] \defeq (\Theta^i\wedge\Omega^j)[\xi_i,\xi_j]
\]
Finally, we can extend the usual exterior derivative $d : \Omega^k_P \to \Omega^{k+1}_P$
to $\Omega^k_P(\g)$ in a natural way, namely
\[
d\Theta \defeq d\Theta^i\xi_i
\]
%
\section{Connections on Principal Bundles}
%
\begin{defn}
Let $M$ be a smooth manifold and $G$ a Lie group. A \ib{principal $G$-bundle} is a
smooth manifold $P$ equipped with a smooth right $G$ action and a map $\pi : P \to M$ such
that
\begin{enumerate}
  \item $\pi$ is a submersion
  \item The action of $G$ on $P$ preserves the fibers of $\pi$
  \item The action of $G$ on $P$ is free and transitive on each fiber
  \item For each $p \in M$, there exists a neighborhood $U$ containing $p$
  and a $G$-equivariant diffeomorphism $\varphi : \pi\inv(U) \to U \times G$
  such that the we get the commutative diagram
  \[\begin{tikzcd}
  \pi\inv(U) \ar[dr, "\pi"']\ar[rr] && U \times G \ar[dl] \\
  & U
  \end{tikzcd}\]
  where the map $U \times G$ is projection onto the first factor, and the
  $G$ action on $U \times G$ is right multiplication on the second factor.
\end{enumerate}
\end{defn}
%
\begin{exmp}
Let $E \to M$ be a rank $k$ vector bundle. From $E$ we can construct the \ib{frame bundle}
$\mathcal{B}(E)$, which as a set is
\[
\mathcal{B}(E) = \set{(p,b) ~:~ p \in M,~ b : \R^k \to E_p \text{ is an isomorphism.}}
\]
This comes with a natural projection $\mathcal{B}(E) \to M$ by forgetting the
second factor, and local coordinates for $M$ and local trivializations for $E$ endow
$\mathcal{B}(E)$ the structure of a smooth manifold such that the projection
map is a submersion. There is a natural right action of $\GL_k\R$ on $\mathcal{B}(E)$
given by precomposition, and it is clear that this action is both free and transitive
on any given fiber $\mathcal{B}(E)_p$. In addition, local trivializations of $E$
induce local trivializations for $\mathcal{B}(E)$, which are $\GL_k\R$-equivariant.
Therefore, we have that $\mathcal{B}(E)$ is a principal $\GL_k\R$-bundle over
$M$.
\end{exmp}
%
An important construction is the \ib{associated bundle construction}, where given
a principal $G$-bundle $P \to M$ and a space $F$ with a left $G$ action, we can
construct a fiber bundle $P\times_G F \to M$ with model fiber $F$.
%
\begin{defn}
Let $P \to M$ be a principal $G$-bundle, and $F$ a smooth manifold with a left $G$-action.
The \ib{associated fiber bundle} $P\times_G F \to M$ is the set
\[
P \times_G F \defeq P \times F / (p,f) \sim (p\cdot g, g\inv\cdot f)
\]
\end{defn}
%
Local trivializations for $P$ induce local trivializations on $P \times_G F$, which
allows us to equip the set $P \times_G F$ with a topology and smooth structure such
that the induced map $P\times_G F \to M$ is a fiber bundle over $M$ with model fiber $F$.
%
\begin{exmp}
For a Lie group $G$, we have the adjoint action $\Ad : G \to \GL(\g)$. Then given
a principal $G$-bundle $P \to M$, we can construct the \ib{adjoint bundle}
$\g_P \to M \defeq P \times_G \g$, which is a bundle with model fiber $\g$.
\end{exmp}
%
\begin{exmp}
Let $E \to M$ be a rank $k$ vector bundle, and $\pi : \mathcal{B}(E) \to M$ its associated
$\GL_k\R$ bundle of frames. There is a natural action of $\GL_k\R$ on $\R^k$, which
allows us to constructed the associated bundle $\mathcal{B}(E) \times_{\GL_k\R} \R^k$.
We claim that this bundle is isomorphic to the original bundle $E$. \\

To show this, we construct an explicit isomorphism. Let $[b,v]$ denote the equivalence
class of $(b,v)$ in $\mathcal{B}(E) \times_{\GL_k\R} \R^k$. Define the map
$\varphi : \mathcal{B}(E) \times_{\GL_k\R} \to E$ by $\varphi[b,v] = b(v)$.
We first need to verify that this map is well defined. Suppose we have another
representative $(a,w)$ for the equivalence class $[b,v]$. Then we know there
exists some $g \in \GL_k\R$ such that $(a,w) = (b \circ g, g\inv(v))$. Then we would
have $\varphi[a,w] = (b \circ g)(g\inv(v)) = b(v)$. Therefore $\varphi$ is well defined.
In addition, it is clear that this map has an inverse. For any $e \in E$, define
the map $\psi : E \to \mathcal{B}(E) \times_{\GL_k\R} \R^k$ by
$\psi(e) = [b, e']$ where $b$ is a basis for the fiber containing $e$, and $e'$ is
the coordinate representation of $e$ with respect to the basis $b$.
\end{exmp}
%
The example with frame bundles gives us a good idea as to how to think of
principal bundles and associated bundles. A principal bundle can be thought of as
coordinates systems for an associated bundle, and the relation we used to define
the associated bundle can be thought of as the transformation law of the associated
bundle. \\

We now move on to connections on principal bundles. Let $P \to M$ be a principal $G$-bundle.
Since $\pi$ is a submersion, it is constant rank, so $\ker d\pi$ defines a
subbundle $V$ called the \ib{vertical distribution}. This gives us the exact sequence
of vector bundles
\[\begin{tikzcd}
0 \ar[r] & V \ar[r] & TP \ar[r] & \pi^*TM \ar[r] & 0
\end{tikzcd}\]
%
\begin{defn}
A \ib{connection} on $P$ is a distribution $H \subset TP$ such that
\begin{enumerate}
  \item $V \oplus H = TP$
  \item $H_{p \cdot g} = d(R_g)_pH_p$, where $R_g : P \to P$ is the action of $g$
\end{enumerate}
\end{defn}
%
Equivalently, it is a choice of splitting of the exact sequence specified above. Another
fruitful way to view connections is in terms of differential forms.
For a Lie group $G$ with Lie algebra $\g$, we have an exponential map
$\exp : \g \to G$, where given a left invariant vector field $X \in \g$, we have
$\exp(X) = c_X(1)$, where $c_X : (-\varepsilon, \varepsilon) \to G$
is the integral curve of $X$ with $c(0) = e$. Then if we have a principal $G$-bundle
$P \to M$ and a point $p \in P$, we get a curve $\gamma_X$ defined by
\[
\gamma_X(t) = p\cdot \exp(tX)
\]
Since the action of $G$ preserves the fiber $P_p$, the tangent vector
\[
\dt{\gamma}_X = \frac{d}{dt}\bigg\vert_{t=0}\gamma_X(t)
\]
is contained in the vertical space $V_p$. In addition, since the action is free,
we have that $\dt{\gamma}_X = 0$ if and only if $X = 0$. Therefore, we get
an injective linear map given by $X \mapsto \dt{\gamma}_X$. By a dimension count,
this gives an isomorphism $\g \to V_p$. Doing this over all points $p \in P$, we get
a vertical vector field $\widehat{X}$ on $P$, and gives us an isomorphism
$\underline{\g} \to V$, where $\underline{\g}$ denotes the trivial bundle $P \times \g$.
Suppose we are given a connection $H \subset TP$. Then the decomposition
$T_pP = V_p \oplus H_p$ gives us a projection map $T_pP \to V_p$ with kernel $H_p$.
Using the identification $V_p \cong \g$, this gives us an element of $\g$. Doing
this over all points, we see that $H$ is equivalent to the data of a
$\g$-valued $1$-form $\Theta \in \Omega^1_P(\g)$, called the \ib{connection $1$-form}.
The first thing we want to observe is how $\Theta$ transforms with respect to
the $G$-action on $P$. To do this, we first observe how the vector fields
we defined above transform.
%
\begin{prop}
Let $X \in \g$ and let $\widehat{X} \in \mathfrak{X}(P)$ be the vector field
induced by $X$. For $g \in G$, let $R_g$ denote the diffeomorphism given by
the right action of $g$. Then
\[
(R_g)_*(\widehat{X}) = \widehat{\Ad_{g\inv}X}
\]
\end{prop}
%
\begin{proof}
We compute
\begin{align*}
(R_g)_*(\widehat{X})_p
&= (R_g)_* \left( \frac{d}{dt}\bigg\vert_{t=0}p\cdot\exp(tX) \right) \\
&= \frac{d}{dt}\bigg\vert_{t=0}p\cdot(\exp(tX)g) \\
&= \frac{d}{dt}\bigg\vert_{t=0}pg\cdot(g\inv\exp(tX)g) \\[5pt]
&= \widehat{\Ad_{g\inv} X}
\end{align*}
\end{proof}
%
This allows us to compute how a connection $1$-form transforms.
%
\begin{prop}
A connection $1$-form $\Theta \in \Omega^1_P(\g)$ satisfies
\[
R_g^*\Theta = \Ad_{g\inv}\Theta
\]
Where $\Ad_{g\inv}\Theta$ is $1$-form defined by
\[
(\Ad_{g\inv}\Theta)(v) = \Ad_{g\inv}(\Theta(v))
\]
\end{prop}
%
\begin{proof}
For any $p$, we get a decomposition $T_pP = V_p \oplus H_p$, so any tangent vector
$v \in T_pP$ can be uniquely decomposed as $v = \widehat{X} + h$ where $X \in \g$
and $h \in H_p$. We compute
\begin{align*}
(R_g^*\Theta)_p(v) &= (R_g^*\Theta)_p(\widehat{X} + h) \\
&= \Theta_{p \cdot g}((R_g)_*(\widehat{X}) + (R_g)_*(h)) \\
&= \Theta_{p\cdot g}(\widehat{\Ad_{g\inv}X}) \\
&= \widehat{\Ad_{g\inv} X} \\
&= \Ad_{g\inv}\Theta_p(\widehat{X} + h)
\end{align*}
where we use the transformation law for $\widehat{X}$, the fact that $H_p$ is
the kernel of $\Theta_p$, and the fact that $H$ is $G$-invariant.
\end{proof}
%
%TODO descent to the base
%
The identification of the vertical bundle with the trivial bundle $\underline{\g}$ gives
us a nice characterization of differential forms valued in an associated bundle of
$P$.
%
\begin{thm}
Let $P \to M$ be a principal $G$ bundle, and $E = P\times_G W$ an associated vector bundle
induced by a representation $\rho : G \to \GL(W)$. Then there is a bijective correspondence
\[
\Omega^k(E) \leftrightarrow\set{\alpha \in \Omega^k_P(W) ~:~
R_g^*\alpha = \rho(g)\inv\alpha, ~\forall\xi\in\g, \iota_\xi\alpha = 0}
\]
\end{thm}
%
This comes from a general principle that appropriately $G$-equivariant objects
on $P$ should descend to objects on $M$.
%
\begin{defn}
let $P \to M$ be a principal $G$ bundle with connection $\Theta$. The \ib{curvature form}
of the connection $\Theta$ is a $2$-form $\Omega \in \Omega^2_P(\g)$ defined by
\[
\Omega = d\Theta + \frac{1}{2}[\Theta \wedge \Theta]
\]
where in a local trivialization, $d\Theta$ is a matrix of $2$-forms given by
$(d\Theta)^i_j = d(\Theta)^i_j$, and
\[
[\Theta \wedge \Theta](X,Y) = [\Theta(X),\Theta(Y)] - [\Theta(Y), \Theta(X)] =
2[\Theta(X), \Theta(Y)]
\]
\end{defn}
%
As with the connection $1$-form, we first observe how the curvature form $\Omega$
transforms with respect to the $G$-action on $P$.
%
\begin{prop} \enumbreak
\begin{enumerate}
  \item $R_g^*\Omega = \Ad_{g\inv}\Omega$
  \item $\iota_\xi(\Omega) = 0$ for all $\xi \in \g$.
\end{enumerate}
\end{prop}
%
Before proving the proposition, we state an extremely helpful formula.
%
\begin{lem}[\ib{Cartan's magic formula}]
Let $\xi$ be a vector field, and let $\omega$ be a $k$-form. Let $\mathcal{L}_\xi$
denote the Lie derivative along $\xi$. Then
\[
\mathcal{L}_\xi\omega = d(\iota_\xi(\omega)) + \iota_\xi(d\omega)
\]
\end{lem}
%
\begin{proof}[Proof of proposition] \enumbreak
\begin{enumerate}
  \item We compute
  \begin{align*}
  R_g^*\Omega &= R_g^*d\Theta + \frac{1}{2}R_g^*[\Theta \wedge \Theta] \\
  &= d(R_g^*\Theta) + \frac{1}{2}[R_g^*\Theta\wedge R_g^*\Theta] \\
  &= \Ad_{g\inv}d\Theta + \frac{1}{2}[\Ad_{g\inv}\Theta \wedge \Ad_{g\inv}\Theta] \\
  & = \Ad_{g\inv}\Omega
  \end{align*}
  \item Let $\xi \in \g$. Then
  \[
  \iota_{\xi}(\Omega) = \iota_\xi(d\Theta) + \frac{1}{2}\iota_\xi[\Theta \wedge \Theta]
  \]
  Using Cartan's magic formula, we have that
  \[
  \mathcal{L}_\xi\Theta = d(\iota_\xi(\Theta)) + \iota_\xi(d\Theta) = \iota_\xi(\Theta)
  \]
  Since $\iota_\xi(\Theta) = \xi$, which is constant, which then implies that
  $d\iota_\xi(\Theta) = 0$. We then compute
  \begin{align*}
  \mathcal{L}_\xi\Theta &= \frac{d}{dt}\bigg\vert_{t=0}R^*_{\exp(t\xi)}\Theta \\
  &= \frac{d}{dt}\bigg\vert_{t=0}\Ad_{\exp(t\xi)\inv}\Theta \\
  &= [-\xi,\Theta]
  \end{align*}
  We also compute
  \[
  \frac{1}{2}\iota_\xi[\Theta\wedge\Theta] = \frac{1}{2}[\iota_\xi(\Theta) \wedge \Theta]
  = [\xi,\Theta]
  \]
  Therefore, $\iota_\xi(\Omega) = 0$.
\end{enumerate}
\end{proof}
%
As a result, the curvature form $\Omega$ descends to the base manifold, giving
us a $2$-form valued in the adjoint bundle $\g_P$.
%
\begin{thm}[\ib{The Bianchi Identity}]
\[
d\Omega + [\Theta\wedge\Omega] = 0
\]
\end{thm}
%
\begin{proof}
Fix a basis $\set{\xi_i}$ for $\g$, and let $\Theta = \Theta^i\xi_i$. We then compute
\begin{align*}
d\Omega &= d^2\Theta + \frac{1}{2}d[\Theta \wedge \Theta] \\
&= \frac{1}{2}d(\Theta^i \wedge \Theta^j)[\xi_i,\xi_j] \\
&= \frac{1}{2}(d\Theta^i \wedge \Theta^j - \Theta^i \wedge d\Theta^j)[\xi_i,\xi_j] \\
&= \frac{1}{2}(d\Theta^i \wedge \Theta^j - d\Theta^j \wedge \Theta^i)[\xi_i,\xi_j] \\
&= (d\Theta^i \wedge \Theta^j - d\Theta^j \wedge \Theta^i)[\xi_i,\xi_j]
\end{align*}
where the last equality comes from skew-symmetry of the Lie bracket. On the other hand,
we compute
\begin{align*}
[\Theta\wedge\Omega]
&= \left[\Theta\wedge\left(d\Theta + \frac{1}{2}[\Theta\wedge\Theta]\right)\right] \\
&=[\Theta\wedge d\Theta] + \frac{1}{2}[[\Theta\wedge\Theta]\wedge\Theta] \\
&= (\Theta^i\wedge d\Theta^j)[\xi_i,\xi_j] \\
&= (d\Theta^j \wedge \Theta^i)[\xi_i,\xi_j] \\
&= -(d\Theta^i \wedge \Theta^j - d\Theta^j\wedge\Theta^i)[\xi_i,\xi_j]
\end{align*}
Where we use skew symmetry of the bracket for the last equality, and use the
Jacobi identity to deduce the vanishing of $[[\Theta\wedge\Theta]\wedge\Theta]$.
Putting these two computations together yields the Bianchi identity.
\end{proof}
%
\section{The Chern-Weil Homomorphism}
%
The fact that a curvature form $\Omega$ descends to a $\g_P$-valued $2$-form on
the base suggests that the curvature might have something to say about the topology
of the base manifold $M$. As we'll see, this takes the form of characteristic
classes.
%
\begin{defn}
Let $G$ be a Lie group with Lie algebra $\g$. An \ib{invariant polynomial of degree $k$}
is a function $p \in \mathrm{Sym}^k\g^*$ such that
\[
p(X_1, \ldots, X_k) = p(\Ad_g(X_1), \ldots \Ad_g(X_k))
\]
We denote the space of degree $k$ invariant polynomials by $I^k(G)$, and
let $I(G) = \bigoplus_k I^k(G)$ denote the space of all invariant polynomials.
\end{defn}
%
We'll discuss specific examples in detail later. For now, we move on with the general
theory. But first, a short lemma
%
\begin{lem}
Let $p \in I^k(G)$. Then for any $\xi, X_1, \ldots X_k \in \g$, we have the identity
\[
\sum_{i=1}^k p(X_1, \ldots [\xi, X_i], \ldots X_n) = 0
\]
\end{lem}
%
\begin{proof}
Since $p$ is an invariant polynomial, we have the equation
\[
p(\Ad_{\exp(t\xi)}(X_1),\ldots,\Ad_{\exp(t\xi)}(X_n)) = p(X_1,\ldots,X_n)
\]
Differentiating with respect to $t$ at $t=0$, we get the desired identity, where
we use the product rule, as well as the fact that
\[
\frac{d}{dt}\bigg\vert_{t=0}\Ad_{\exp(t\xi)}(X) = [\xi,X]
\]
\end{proof}
%
The algebra of invariant polynomials can be naturally identified with homogeneous
polynomials on $\g$, where given an invariant polynomial $p \in I^k(G)$, we get a
function on $\g$ via the function $X \mapsto p(X, \ldots, X)$. Upon fixing a basis
for $\g$, we can recover a polynomial. We can recover the polynomial $p$ from this
function via a formula analogous to the polarization  identity for recovering a bilinear
form from its quadratic form. In the case that $G$ is a matrix group, the idea
is to think of $p$ as a polynomial in the entries of $X$.
%
\begin{thm}[\ib{The Chern-Weil Homomorphism}]
Let $P \to M$ be a principal $G$-bundle over $M$, and $\Omega \in \Omega^2_P(\g)$
the curvature form corresponding to some connection on $P$. Then we get a map
$I(G) \to \Omega^\bullet_M$, where for $p \in I^k(G)$, we map
\[
p \mapsto \omega_p \defeq
p(\underbrace{\Omega \wedge \cdots \wedge \Omega}_{k \text{ times}})
\]
\end{thm}
%
There's a lot to unpack here.
\begin{proof}
We first make sense of the formula. We have that $\Omega \in \Omega^2_P(\g)$, so
a priori, $\omega_p$ doesn't descend to $M$. Our hope is that invariance of $p$, along
with the transformation laws of $\Omega$ means that this form will descend to the base.
Punting that problem to later, we can feed $2k$ tangent vectors to
$\Omega \wedge\cdots\wedge \Omega$, giving us $k$ elements $X_1,\ldots, X_k \in \g$, which
we can then feed to $p$, giving us a real number. Therefore, the types check out. \\

We now check that the form $\omega_p \in \Omega^{2k}_P$ descends to $M$. This means we
need to check two things :
\begin{enumerate}
  \item $R_g^*\omega_p = \Ad_{g\inv}\omega_p$
  \item $\iota_\xi\omega_p = 0$ for any $\xi \in \g$.
\end{enumerate}
For the first property, we compute
\begin{align*}
R_g^*\omega_p &= R_g^*(p(\Omega \wedge\cdots\wedge\Omega)) \\
&= p(R_g^*\Omega \wedge \cdots \wedge R_g^*\Omega) \\
&= p(\Ad_{g\inv}\Omega \wedge\cdots\Ad_{g\inv}\Omega) \\
&= p(\Omega\wedge\cdots\wedge\Omega)
\end{align*}
For the second property, let $\xi\in \g$. Then
\begin{align*}
(\iota_\xi\omega_p)
&= \iota_\xi(p(\Omega\wedge\cdots\wedge\Omega)) \\
&= \sum_i p(\Omega\wedge\cdots\wedge\iota_\xi\Omega\wedge\cdots\wedge\Omega) \\
&= 0
\end{align*}
where we use the fact that $\iota_\xi\Omega = 0$. Therefore, we have that
$\omega_p$ descends to the base, giving us a form in $\Omega^{2k}_M$.
\end{proof}
%
\begin{thm}
The Chern-Weil homomorphism maps an invariant polynomial $p \in I(G)$ to a closed
form in $\Omega^{2k}_M$, so it descends to a map $I(G) \to H^{2k}(M,\R)$.
\end{thm}
%
\begin{proof}
Let $\Theta$ denote the connection $1$-form. We want to show that
$\omega_p$ is closed. We compute, using the Leibniz rule for exterior differentiation,
\begin{align*}
d\omega_p &= d(p(\Omega \wedge\cdots\wedge\Omega)) \\
&= \sum_i p(\Omega\wedge\cdots\wedge d\Omega\wedge\cdots\wedge\Omega) \\
&= -\sum_i p(\Omega\wedge\cdots\wedge [\Theta\wedge\Omega]\cdots\wedge\Omega) \\
&= 0
\end{align*}
where we use the Bianchi identity, along with the fact that $\Theta(\xi) = \xi$.
\end{proof}
%
\section{The Case of $\GL_n\C$ : Chern Classes}
%
When we restrict our attention to $\GL_n\C$, and the principal bundle
$P \to M$ is the $\GL_n\C$-bundle of frames for a complex vector bundle,
the fruit of our labor will be the Chern classes of the vector bundle. To do this,
we first need to understand the invariant polynomials on $\gl_n\C$. In this case,
the adjoint action of $\GL_n\C$ on $\gl_n\C$ is conjugation. We also have a basis
for $\gl_n\C$ given by the elementary matrices $E_{ij}$ that are zero everywhere except
for a $1$ in the $(i,j)$ entry. The invariant polynomials on $\gl_n\C$ will come
from a familiar friend : the determinant. We know that the $\det$ is multiplicative,
so we clearly have that $\det(AXA\inv) = \det(X)$ for any $X \in \gl_n\C$ and
any $A \in \GL_n\C$, so $\det$ is an invariant polynomial. The determinant will
be the only tool we will need to construct all the invariant polynomials.
%
\begin{defn}
The \ib{elementary invariant polynomials} $f_0, \ldots f_n \in I(\GL_n)$ are
the polynomials defined by the formula
\[
\det(tI + X) = \sum_{i=0}^n f_i(X)t^{n-i}
\]
Where $t$ is an indeterminate variable, and $X = X^i_j$ is an $n\times n$ matrix
of indeterminate variables.
\end{defn}
%
\begin{exmp}[$n=2$]
Let $X$ be the matrix
\[
X = \begin{pmatrix}
a & b \\
c & d
\end{pmatrix}
\]
where $a,b,c,d$ are indeterminate variables. Then we have that
\begin{align*}
\det(tI + X) &= \det \begin{pmatrix}
t+a & b \\
c & t+d
\end{pmatrix} \\[5pt]
&= (t+a)(t+d) - bc \\
&= t^2 + (a+d)t +ad - bc
\end{align*}
This give us
\begin{align*}
f_0(X) &= 1 \\
f_1(X) &= a+d \\
f_2(X) &= ad - bc
\end{align*}
All the polynomials are visibly invariant, since $f_0$ is vacuously invariant,
$f_1$ is the trace, and $f_2$ is the determinant
\end{exmp}
%
\begin{exmp}[$n=3$]
Let $X$ be the indeterminate matrix
\[
X = \begin{pmatrix}
a & b & c \\
p & q & r \\
x & y & z
\end{pmatrix}
\]
We then laboriously compute
\begin{align*}
  &\det \begin{pmatrix}
  a+t & b & c \\
  p & q+t & r \\
  x & y & z+t
  \end{pmatrix} \\[5pt]
  &=  t^3 + (a+q+z)t^2 + (aq+az+qz-ry-bp+cx)t + aqz - ary - pbz + bxr + cpy - cxq
\end{align*}
Which gives us
\begin{align*}
f_0(X) &= 1 \\
f_1(X) &= a+q+z \\
f_2(X) &= (aq+az+qz-ry-bp+cx) \\
f_3(X) &= aqz - ary - pbz + bxr + cpy - cxq
\end{align*}
While things look very complicated, these polynomials are just the usual culprits.
$f_0$ and $f_1$ are analogous to the case of $\gl_2$. The polynomial $f_2$ is slightly
tricky : it is the polynomial $1/2(\tr(X)^2 -\tr(X^2))$. The polynomial $f_3$ is just
$\det$.
\end{exmp}
%
We then make two key observations
\begin{enumerate}
  \item If $p \in I(\GL_n\C)$ and $X \in \gl_n\C$ is diagonalizable, conjugation
  invariance then implies that $p$ is just a polynomial in the eigenvalues of $X$.
  \item The polynomials $f_k$ suspiciously look like the elementary symmetric polynomials
  \[
  \sigma_k = \sum_{1 \leq i_1 < i_2 \ldots < i_k \leq n} t_{i_1}\cdots t_{i_k}
  \]
\end{enumerate}
%
To make use of the first observation, we need the following theorem.
%
\begin{thm}
The set of diagonalizable matrices in $\GL_n\C$ is dense.
\end{thm}
%
\begin{proof}
We note that any matrix with distinct eigenvalues is diagonalizable. Given any
non-diagonalizable matrix $A$, we can always find an arbitrarily small perturbation
$A + \varepsilon B$ that has distinct eigenvalues.
\end{proof}
%
\begin{thm}
let $T$ be the indeterminate diagonal matrix
\[
T = \begin{pmatrix}
t_1 & ~ & ~ \\
~ & \ddots & ~ \\
~ & ~ & t_n
\end{pmatrix}
\]
Then the mapping $I(\GL_n\C) \to \C[t_1,\ldots t_n]^{S_n}$
\[
P \mapsto \widetilde{P}(t_1,\ldots t_n) \defeq P(T)
\]
is an isomorphism.
\end{thm}
%
\begin{proof}
The map is clearly an algebra homomorphism, so to check surjectivity, it suffices
to check that the image contains the elementary symmetric polynomials, since any
symmetric polynomial is a polynomial $f(\sigma_1, \ldots \sigma_n)$ in the
elementary symmetric polynomials $\sigma_k$. For an indeterminate variable
$\lambda$, we have that
\[
\det(\lambda I + X) \mapsto \det(\lambda I + T) = \prod_{i=1}^n (\lambda+t_i)
= \sum_{k=0}^n\sigma_k(t_1,\ldots,t_k)\lambda^{n-k}
\]
We also know that
\[
\det(\lambda I + T) = \sum_{k=0}^n f_k(X)\lambda^{n-k}
\]
Then since the map is a homomorphism, we can compare the terms of degree $\lambda^j$
to conclude that  $f_k(X) \mapsto \sigma_k(t_1,\ldots t_n)$. Consequently, the map is
surjective, since given any symmetric polynomial $p(\sigma_1,\ldots \sigma_k)$,
we have that it is the image of the invariant polynomial $p(f_1,\ldots f_k)$. \\

For injectivity, we have that if an invariant polynomial $p$ is in the kernel, then
it must vanish on all diagonal matrices. Since $p$ is continuous as a function of
the entries of the matrix, and the set of diagonalizable matrices is dense,
we must have that $p = 0$.
\end{proof}
%
This gives a complete description of the invariant polynomials for $\GL_n\C$.
%
\begin{defn}
Let $P \to M$ be a principal $\GL_k\C$ bundle  with curvature form $\Omega$.
Then the \ib{$j^{th}$ Chern class} is the cohomology class
\[
\left[\frac{i}{2\pi}f_k(\Omega)\right]
\]
\end{defn}
%
The preceding discussion gives us something immediate
%
\begin{thm}
The Chern classes generate the image of the Chern-Weil homomorphism : every
cohomology class in the image can be written as a polynomial in Chern classes.
\end{thm}
%
\section{The Story for Vector Bundles}
%
Every vector bundle $E \to M$ gives rise to a frame bundle $\mathcal{B}(E) \to M$.
The discussion of Chern-Weil theory and characteristic classes for vector
bundles is then subsumed by the principal bundle theory. However, we should
make sure that what we have done agrees with the usual notions on vector bundles.
In addition, we study the relationship between principal bundles and their
associated vector bundles.
%
\end{document}