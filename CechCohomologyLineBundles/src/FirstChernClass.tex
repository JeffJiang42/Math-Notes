\section{The First Chern Class}
%
On any complex manifold $X$, we have the following exact sequence of sheaves:
\[\begin{tikzcd}
0 \ar[r] & \Z \ar[r] & \O_X \ar[r, "\mathrm{exp}"] & \O_X^\times \ar[r] & 0
\end{tikzcd}\]
where $\mathrm{exp}$ is the sheaf morphism where a function $f \in \O_X(U)$ is
mapped to the function $e^f$. This gives us the long exact sequence in sheaf
cohomology
\[\begin{tikzcd}
0 \ar[r] & H^0(X, \Z) \ar[r] & H^0(X, \O_X) \ar[r] & H^0(X, \O_X^\times) \ar[dll]\\
& H^1(X, \Z) \ar[r] & H^1(X, \O_X) \ar[r] & H^1(X, \O_X^\times) \ar[dll, "c_1"']\\
& H^2(X, \Z) \ar[r] & \cdots
\end{tikzcd}\]
%
We are particularly interested in the boundary map $H^1(X,\O_X^\times) \to H^2(X, \Z)$,
which we called $c_1$. Recall that we have an isomorphism
$H^1(X,\O_X^\times) \cong \mathrm{Pic}(X)$.
%
\begin{defn}
The \ib{first Chern class} of a line bundle $L$ is its image under the map
$c_1$, using the canonical identification $H^1(X,\O_X^\times) \cong \mathrm{Pic}(X)$.
\end{defn}
%
The first Chern class will prove to be a powerful invariant of line bundles, and
our goal will be to understand these cohomology classes in terms of the geometry
of the manifold $X$. We first note some properties of $c_1$ that can be immediately
deduced from the definition and our knowledge of the group structure on $\mathrm{Pic}(X)$.
%
\begin{prop} \enumbreak
\begin{enumerate}
  \item $c_1(L_1 \otimes L_2) = c_1(L_1) + c_1(L_2)$
  \item $c_1(L^*) = -c_1(L)$.
  \item For a map $F : X \to Y$, we have $c_1(F^*L) = F^*c_1(L)$.
\end{enumerate}
\end{prop}
The first two observations are immediate, and the last one follows from functoriality
of cohomology. \\

Using some Hodge theory, we can already learn something about $\CP^n$ using the
first Chern class.
%
\begin{thm}
\[
\mathrm{Pic}(\CP^n) \cong \Z
\]
\end{thm}
%
\begin{proof}
Hodge theory gives us the Hodge decomposition
\[
H^k(\CP^n, \C) = \bigoplus_{p+q=k} H^{p,q}(X, \C)
\]
as well as isomorphisms
\[
H^q(\CP^n, \Omega^{p,0}_{\CP^n}) \cong H^{p,q}(\CP^n, \C)
\]
We know the singular cohomology of $\CP^n$ is given by
\[
H^k(\CP^n, \C) \cong \begin{cases}
\C & k \text{ is even} \\
0 & \text{otherwise}
\end{cases}
\]
In particular, if $p+q$ is odd, then we know that $H^{p,q}(X,\C) = 0$.
A special case is
\[
0 = H^{1,0}(\CP^n, \C) \cong H^1(\CP^n, \Omega^{0,0}_{\CP^n}) = H^1(\CP^n, \O_{\CP^n})
\]
Therefore, we have that in the long exact sequence induced by the exponential exact
sequence, we have
\[\begin{tikzcd}
0 \ar[r] & \mathrm{Pic}(\CP^n) \ar[r, "c_1"] & \Z
\end{tikzcd}\]
which tells us that $c_1$ is injective. Since the only subgroups of $\Z$
are isomorphic to $\Z$, this gives $\mathrm{Pic}(\CP^n) \cong \Z$. In addition,
it tell us that $c_1$ is a complete isomorphism invariant. Two holomorphic line bundles
over $\CP^n$ are isomorphic if and only if they have the same first Chern class.
\end{proof}
%
Given that the first Chern class is defined using sheaves of holomorphic functions,
as well as our preliminary observation with $\CP^n$, one might first think that $c_1$ is
an invariant of holomorphic line bundles, but the truth is that it is a slightly coarser
invariant -- it cannot distinguish between holomorphic line bundles and smooth complex
line bundles. To see this, let $C^\infty_\C$ denote the sheaf of smooth $\C$-valued
functions on $X$. We note that we also have the exact sequence
\[\begin{tikzcd}
0 \ar[r] & \Z \ar[r] & C_\C^\infty \ar[r, "\mathrm{exp}"] & (C_\C^\infty)^\times \ar[r] & 0
\end{tikzcd}\]
which similarly induces a long exact sequence in cohomology, yielding a boundary
map $\delta : H^1(X, (C^\infty)^\times) \to H^2(X, \Z)$. Using functoriality
of sheaf cohomology, the inclusions $\O_X \hookrightarrow C^\infty$ and
$\O_X^\times \hookrightarrow (C^\infty)^\times$ induce maps on cohomology, giving
us the commutative diagram
\[\begin{tikzcd}
H^1(X,C^\infty_\C) \ar[r] & H^1(X, (C^\infty_\C)^\times) \ar[r, "\delta"] & H^2(X,\Z) \\
H^1(X, \O_X) \ar[r] \ar[u] & H^1(X, \O_X^\times)
\ar[u] \ar[r, "c_1"'] & H^(X, \Z) \ar[u, equal]
\end{tikzcd}\]
with exact rows. Given a holomorphic line bundle $L$, we can forget its complex
structure, regarding it as a complex line bundle, i.e. an element of
$H^1(X, (C^\infty_\C)^\times)$, and then take the image under $\delta$. Commutativity
of the diagram implies that this is the same thing as $c_1(L)$. Since
$C^\infty_\C$ is a fine sheaf, we have that this diagram is actually
\[\begin{tikzcd}
0 \ar[r] & H^1(X, (C_\C^\infty)^\times) \ar[r, "\delta"] & H^2(X,\Z) \\
H^1(X, \O_X) \ar[r] \ar[u] & H^1(X, \O_X^\times)
\ar[u] \ar[r, "c_1"'] & H^(X, \Z) \ar[u, equal]
\end{tikzcd}\]
which tells us that $\delta$ is injective. Therefore, on smooth complex line bundles,
the first Chern class is a perfect invariant -- two complex line bundles are smoothly
isomorphic if and only if they have the same first Chern class. However, this also
reveals to us that it cannot distinguish between different holomorphic structures
on the same underlying complex line bundle. However, we can use the complex geometry as
a tool  to compute certain quantities of a holomorphic bundle, which will turn out to be
topological invariants of the bundle, independent of the holomorphic structure. Our goal
will be to realize the first Chern class of a line bundle as the cohomology class
associated to the curvature of a connection. Eventually, we will add a third characterization
to the mix by introducing the Atiyah class. \\


We first discuss the relationship of $c_1$ with the curvature of a connection on
a complex line bundle. We will discuss connections on complex vector bundles in
general, and will then specialize to the case of line bundles.
%
\begin{defn}
Let $E \to X$ be a complex vector bundle, and let $\mathcal{A}^k(E)$ denote the
sheaf of smooth $E$-valued $k$-forms, i.e. the sheaf of sections of the bundle
$\Lambda^k(T^*X)_\C \otimes E$, which is a sheaf of $\C^\infty_\C$-modules. A
\ib{connection} $\nabla$ is a $\C$-linear map
$\nabla : \mathcal{A}^0 \to \mathcal{A}^1$
where for a smooth function $f$ and local section $s$, we have the \ib{Leibniz rule}
\[
\nabla(fs) = df \otimes s + f\nabla s
\]
\end{defn}
%
A connection can be naturally extended to an operator
$\nabla : \mathcal{A}^k \to \mathcal{A}^{k+1}$, where for a $k$-form $\alpha$
and a local section $s$ of $E$, we define
\[
\nabla(\alpha \otimes s) = d\alpha \otimes s + (-1)^k \alpha \wedge \nabla s
\]
In a similar fashion, the connection $\nabla$ on $E$ induces connection on associated
bundles (e.g. tensor powers, duals). In particular, it induces a connection on the bundle
$\End(E)$, which by abuse of notation we also denote $\nabla$. Given a section $F$ of
$\End(E)$, we have that the action of $\nabla F$ on a section $s$ is given by the formula
\[
(\nabla F)(s) = \nabla(F(s)) - F(\nabla s)
\]
\begin{prop}
Let $\nabla_1$ and $\nabla_2$ be two connections on a complex vector bundle
$E \to X$. Then $\nabla_1 - \nabla_2$ is an element of $\mathcal{A}^1(\End E)$.
\end{prop}
%
\begin{proof}
We want to show that the difference is $C^\infty_\C$-linear. We compute
\begin{align*}
(\nabla_1 - \nabla_2)(fs) &= \nabla_1(fs) - \nabla_2(fs) \\
&= df \otimes s + f\nabla_1s - (df \otimes s + f\nabla_2s) \\
&= f(\nabla_1 - \nabla_2)(s)
\end{align*}
\end{proof}
%
\begin{prop}
Let $a \in \mathcal{A}^1(\End E)$ and let $\nabla$ be a connection on $E$. Then
$\nabla + a$ is a connection on $E$, where given a local
section $s$, we define $\nabla + a$ by
\[
(\nabla + a)s = \nabla s + as
\]
where given a tangent vector $v$, we have that $a$ acts on $s$ by $a(v)s$.
\end{prop}
%
\begin{proof}
We clearly have that $\nabla + a$ is $\C$-linear. To check that it satisfies the
Leibniz rule, we compute
\begin{align*}
(\nabla + a)(fs) &= \nabla(fs) + a(fs) \\
&= df \otimes s + f\nabla s + fas \\
&= df \otimes s + f(\nabla + a)(s)
\end{align*}
\end{proof}
%
Therefore, given a complex vector bundle $E \to X$ equipped with connection, the
connection is locally of the form $d + A$, where $d$ is the usual complexified
de Rham differential and $A$ is a matrix of $1$-forms.
%
\begin{defn}
Let $E \to X$ be a complex vector bundle equipped with a connection $\nabla$.
The \ib{curvature} of $\nabla$ is defined to be the map
$F_{\nabla} \defeq \nabla^2 : \mathcal{A}^0(E) \to \mathcal{A}^2(E)$.
\end{defn}
%
One might expect the curvature to be a second order differential operator, but a
small miracle happens.
%
\begin{prop}
The curvature transformation $F_\nabla$ is $C^\infty_\C$-linear, i.e. it defines
a global section of $\mathcal{A}^2(\End E)$.
\end{prop}
%
\begin{proof}
We compute for a function $f$ and local section $s$
\begin{align*}
F_\nabla(fs) &= \nabla(\nabla(fs)) \\
&= \nabla(df \otimes s + f\nabla s) \\
&= d^2f \otimes s - df \wedge \nabla s + \nabla(f\nabla s) \\
&= -df \wedge \nabla s + df \wedge \nabla s + f\nabla^2 s \\
&= f\nabla^2 s \\
&= fF_\nabla(s)
\end{align*}
\end{proof}
%
\begin{thm}[\ib{The Bianchi Identity}]
Let $F_\nabla$ be the curvature form of a connection on a vector bundle $E \to X$.
Then
\[
\nabla(F_\nabla) = 0
\]
\end{thm}
%
\begin{proof}
Let $s$ be a section of $E$. By the definition of the induced connection on $\End(E)$, we
compute
\begin{align*}
(\nabla(F_\nabla))(s) &= \nabla(F_\nabla(s)) - F_\nabla(\nabla s) \\
&= \nabla^3s - \nabla^3s \\
&= 0
\end{align*}
\end{proof}
%
Finally we make one more observation
%
\begin{prop}
Let $\nabla$ be a connection on $E \to X$, and $\nabla' = \nabla + A$. Then
\[
F_{\nabla'} = F_\nabla + dA + A \wedge A
\]
\end{prop}
%
\begin{proof}
We compute locally for a section $s$,
\begin{align*}
F_{\nabla'} &= (d + A)((d+A)(s)) \\
&= (d+A)(ds + A(s)) \\
&= d^2s + A(ds) + d(A(s)) + (A \wedge A)(s) \\
&= dA(s) + (A \wedge A)(s)
\end{align*}
\end{proof}
%
In the case of a line bundle $L \to X$, much of this discussion simplifies. First of all
we have that the bundle $\End(L)$ is a trivial bundle, so we have that every
connection is locally of the form $d + A$ for a $1$-form $A$. This allows several
small miracles to occur.
%
\begin{prop}
The curvature $F_\nabla$ of a connection on a complex line bundle $L \to M$
is a globally defined $2$-form in $H^2(X, \C)$.
\end{prop}
%
\begin{proof}
We have that locally, $\nabla = d + A$, so we have locally
\[
F_\nabla =  dA + A \wedge A = dA
\]
We then need to show that these local definitions glue. In two trivializing
neighborhoods $U_i$ and $U_j$ with transition functions $\psi_{ij}$ and $\psi_{ji}$,
let $A_i$ and $A_j$ denote the connection $1-$forms over each trivialization.
Let $s$ be a local section over $U_i \cap U_j$ that is nonzero and constant when viewed as a
section of the trivial bundle $U_i \cap U_j \times \C$. Therefore, we have that
Then the local representations $s_i$ and $s_j$ are related by $\psi_{ij}s_j = s_i$.
In addition, we have that over $U_i$, $\nabla s = ds_i + A_is_i = A_is_i$, since
$s_i$ is a constant function, and we have a similar picture over $U_j$. We then
compute
\[
A_i\psi_{ij}s_j = \nabla(\psi_{ij}s_j) = d\psi_{ij}s_j + \psi_{ij}\nabla s_j
= d\psi_{ij}s_j + \psi_{ij}A_js_j
\]
Therefore, we have
\[
A_i\psi_{ij} = d\psi_{ij} + \psi_{ij}A_j
\]
Multiplying both sides by $\psi_{ij}\inv$, moving terms around, we find
\[
A_j = A_i - \psi_{ij}\inv d\psi_{ij}
\]
We then verify that the local descriptions of $F_\nabla$ agree on the intersection. We
compute
\begin{align*}
F_\nabla &= dA_i \\
&= \psi_{ij}\inv d\psi_{ij} \wedge \psi_{ij}\inv d\psi_{ij} + dA_i \\
&= \psi_{ij}^{-2}d\psi_{ij} \wedge d\psi_{ij} + dA_i \\
&= -d\psi_{ij}\inv \wedge d\psi_{ij} + dA_i \\
&= d(A_i - \psi_{ij}\inv d\psi_{ij}) \\
&= dA_j
\end{align*}
Where we use the fact that $d(\psi_{ij}\inv) = \psi_{ij}^{-2}d\psi_{ij}$. Therefore,
the local descriptions of $F_\nabla$ glue to a global $2$-form.
\end{proof}
%
In addition, since we have locally that $F_\nabla$ is given by $dA_i$, we have that
$F_\nabla$ is closed, so it defines a cohomology class in $H^2(X,\C)$. This gives us
two ways of obtaining a cohomology class from a line bundle. The first is by taking
the first Chern class from the exponential exact sequence, and the other is
by taking the curvature form of a connection.
%
\begin{thm}
Let $L \to X$ be a complex line bundle, and let $\nabla$ be any connection on $L$ with
curvature form $F_\nabla$. Then
\[
\left[\frac{i}{2\pi}F_\nabla\right] = c_1(L)
\]
\end{thm}
%
\begin{proof}
We first give an explicit formula for the \v{C}ech cocycle defining $c_1(L)$.
Fix a good cover $\mathcal{U} = \set{U_i}$ for $X$, and let the $\psi_{ij}$ be the
transition functions for $L$ with respect to this covering. The $\psi_{ij}$
determine a \v{C}ech cocycle in $H^1(X, (C^\infty_\C)^\times)$. By surjectivity of
$\exp$, by passing to a finer cover we may assume that we can find a branch of the
logarithm for each $\psi_{ij}$, so $\log\psi_{ij}$ is well defined, and defines a
\v{C}ech cocycle in $H^1(X, (C^\infty)^\times)$. Then since
the $\psi_{ij}$ satisfy the cocycle condition
\[
\psi_{ij}\psi_{jk}\psi_{ik}\inv = 1
\]
we have that
\[
\log\psi_{ij} + \log\psi{jk} - \log\psi_{ik} \in 2\pi i\Z
\]
So we have that
\[
\frac{1}{2\pi i}\log\psi_{ij} + \log\psi_{jk} - \log\psi_{ik} \in \Z
\]
is the cocycle representing $c_1(L)$. In the natural map
$H^2(X,\Z) \to H^2(X,\C)$, we have that the cocycle has the same formula (up to a sign
depending on convention). \\

Then let $\nabla$ be any connection on $L$, with local connection forms $A_i$.
We have that $F_\nabla$ is a closed $2$-form, with local description $F_\nabla = dA_i$.
Let $\mathcal{A}^k(X)$ denote the sheaf of smooth $k$-forms over $X$, and
let $\mathcal{Z}^k(X)$ denote the sheaf of closed smooth $k$-forms over $X$.
Then we have the exact sequences of sheaves
\[\begin{tikzcd}
0 \ar[r] & \mathcal{Z}^1(X) \ar[r] & \mathcal{A}^1(X) \ar[r, "d"] & \mathcal{Z}^2(X)
\ar[r] & 0
\end{tikzcd}\]
\[\begin{tikzcd}
0 \ar[r] & \C \ar[r] & \mathcal{A}^0(X) \ar[r, "d"] & \mathcal{Z}^1(X) \ar[r] & 0
\end{tikzcd}\]
which give rise to boundary homomorphisms
$\delta_1 : H^0(X, \mathcal{Z}^2(X)) \to H^1(X, \mathcal{Z}^2(X))$ and
$\delta_2 : H^1(X, \mathcal{Z}^1(X)) \to H^2(X, \C)$ respectively. The
fact that the local definitions $F_\nabla = dA_i$ glue to a global form
is exactly the statement that the $dA_i$ form a \v{C}ech cocycle in
$H^0(X, \mathcal{Z}^2(X))$, i.e. $dA_i - dA_j = 0$ for all $i,j$ with
$U_i \cap U_k \neq \emptyset$. Following the definition of the boundary homomorphism,
this implies that it is the image of the cocycle determined by the $A_i$ under
$d$, and then applying the \v{C}ech differential we get that
$\delta_1(F_\nabla)$ is the cocycle $\set{A_j - A_i}$. Then recall from our earlier
calculation that $A_j = A_i - \psi_{ij}\inv d\psi_{ij}$, we get that
\[
A_j - A_i = -\psi_{ij}\inv d\psi_{ij}
\]
which is exactly $-d\log\psi_{ij}$. Then tracing though the definition of the
boundary homomorphism $\delta_2$, we have that $\delta_2(A_j - A_i)$
is the preimage of the \v{C}ech differential applied to the cocycle
$\log\psi_{ij}$, which is exactly $-(\log\psi_{ij} + \log\psi_{jk} - \log\psi_{ik})$.
Putting everything together, we find that
\[
c_1(L) = \frac{1}{2\pi i}\log\psi_{ij} + \log\psi_{jk} - \log\psi_{ik}
= \left[ \frac{i}{2\pi}F_\nabla \right]
\]
\end{proof}
%