%
\section{Vanishing Theorems}
%
\iffalse
\emph{It was said about Lefschetz that he had never stated a false theorem or gave a
correct proof.} \\
\fi

For the following discussion, we will restrict our attention to a compact K\"ahler
manifold $X$, which will allow us to use some tools from Hodge theory, which we will
assume.
%
\begin{defn}
A \ib{positive $(1,1)$ form} on $X$ is a
differential form $\omega \in \mathcal{A}^{1,1}$ such that in local
holomorphic coordinates $\set{z^i}$, we have that
\[
\omega = \frac{i}{2} \sum_{i,j} h_{ij}(z) dz^i \wedge d\zbar^j
\]
where $h_{ij}$ is a positive definite Hermitian matrix for all $z$. Equivalently,
$\omega$ is a positive form if and only if it represents the K\"ahler form for some
Hermitian metric on $X$.
\end{defn}
%
\begin{defn}
A holomorphic line bundle $L \to X$ is \ib{positive} if there exists a Hermitian metric
$h$ on $X$ with Chern connection $\nabla$ such that $(i/2\pi)F_\nabla$ is a
positive $(1,1)$ form.
\end{defn}
%
Positivity of a line bundle turns out to only depend on the first Chern class of
the line bundle. From our previous discussion, this tells us that it only cares
about the topological isomorphism class of the line bundle, and is independent of
a holomorphic structure. To show this, we'll need two useful lemmas.
The first is a classic lemma from K\"ahler geometry.
%
\begin{lem}[\ib{The $\partial\dbar$-lemma}]
Let $\eta$ be a smooth complex valued form such that $\eta$ is both $\partial$ and
$\dbar$ closed. Then if $\eta$ is $d$, $\partial$, or $\dbar$-exact, then there
exists a form $\xi$ such that $\eta = \partial\dbar\xi$. Furthermore, if $\eta$ is
real (i.e. $\overline{\eta} = \eta$), we may take $i\xi$ to be real.
\end{lem}
%
The other lemma characterizes curvatures of metric compatible connection on line
bundles.
%
\begin{lem}
Let $L \to X$ be a holomorphic line bundle equipped with Hermitian metric $h$. Locally,
$h$ is given by a positive function $h(z)$ on $X$. Then the curvature $F_\nabla$ of
the metric compatible connection $\nabla$ is given by the formula
\[
F_\nabla = -\partial\dbar\log h(z)
\]
\end{lem}
%
\begin{proof}
This follows from our earlier computations for Chern connections. Locally, the Chern
connection $\nabla$ is given by $d + A$ for a complex $1$-form $A$, which
from our earlier computations, must be holomorphic and satisfy
\[
A = \overline{h}\inv \partial \overline{h}
\]
However, since $h$ is real, this is the same as $A = h\inv\partial h$.
The curvature is then given by $F_\nabla = dA = (\partial A + \dbar A)$. However,
we note that by a type check, $\partial A$ is of type $(2,0)$, and since the curvature
of the Chern connection is of type $(1,1)$, we have that
\[
F_\nabla = \dbar\left(\frac{\partial h}{h}\right) = \dbar\partial\log h
\]
the desired identity then follows from the fact that $\partial$ and $\dbar$ anticommute.
\end{proof}
%
\begin{prop}
Let $L \to X$ be a line bundle and let $\omega$ be a real closed $(1,1)$ form such
that the cohomology class of $\omega$ is equal to $c_1(L)$. Then there exists a
Hermitian metric on $L$ with metric compatible connection $\nabla$ such that
$\omega = (i/2\pi)F_\nabla$.
\end{prop}
%
\begin{proof}
Let $h$ be any Hermitian metric on $L$, and $\nabla$ the Chern connection for $h$.
We know that the curvature $F_\nabla$ satisfies
\[
\left[\frac{i}{2\pi}F_\nabla\right] = c_1(L)
\]
Then suppose we have another Hermitian metric $h'$ with Chern connection $\nabla'$.
We know that we can write $h'$ differs from $h$ by multiplication by a smooth positive
function, so we may write $h' = e^\rho h$ for some smooth function $\rho$. Then the
curvature of $\nabla'$ satisfies
\begin{align*}
F_{\nabla'} &= -\partial\dbar\log e^\rho h \\
&= -\partial\dbar(\log e^\rho + \log h) \\
&= -\partial\dbar\rho + F_\nabla
\end{align*}
Which tells us that $F_\nabla = \partial\dbar\rho + F_{\nabla'}$.
Then given a real closed $(1,1)$ form $\omega$ with $[(i/2\pi)\omega] = c_1(L)$,
if we solve the equation
\[
F_\nabla = \partial\dbar\rho + \omega
\]
Then the Chern connection $\nabla'$ of the Hermitian form $e^\rho h$ will satisfy
$F_{\nabla} = \partial\dbar\rho + F_{\nabla'}$, which then implies that
$F_{\nabla'} = \omega$. To solve the equation, we note that $\omega$ and
$F_\nabla$ are cohomologous by assumption, so their difference $F_\nabla - \omega$
is $d$-exact. Furthermore, from the local formula
\[
F_\nabla = -\partial\dbar\log h
\]
we see that $F_\nabla$ is both $\partial$ and $\dbar$-closed. Finally, the
fact that $\omega$ is type $(1,1)$ and $d$-closed implies that it must be both
$\partial$ and $\dbar$-closed by a simple type check, so we get that
$F_\nabla - \omega$ satisfies the conditions of the $\partial\dbar$-lemma,
giving us that $F_\nabla - \omega = \partial\dbar\rho$ for some real function
$\rho$.
\end{proof}
%
Recall that on a K\"ahler manifold $X$ with K\"ahler form $\omega$, we have the
Lefschetz operator $L : \mathcal{A}^{p,q} \to \mathcal{A}^{p+1,q+1}$ given by
wedging with $\omega$. The operator $L$, together with its adjoint
$\Lambda : \mathcal{A}^{p,q} \to \mathcal{A}^{p-1,q-1}$
satisfy the identities
\begin{enumerate}
  \item $[\Lambda, L] = (n-(p+q))\id_{\mathcal{A}^{p,q}(X)}$
  \item $[\Lambda,\dbar] = -i\partial^*$
\end{enumerate}
%
where $n$ is the complex dimension of $X$. The operators $\Lambda$ and $L$ can
naturally be extended to vector bundle valued forms, and since $\Lambda$ and
$L$ are both $0^{th}$ order operators, the first identity still holds for
their extended versions. The second identity generalizes in a slightly modified
form.
%
\begin{prop}[\ib{Nakano identity}]
Let $E$ be a holomorphic vector bundle equipped with a Hermitian metric with
Chern connection $\nabla$. Then
\[
[\Lambda, \dbar_E] = -i(\nabla^{1,0})^*
\]
where $(\nabla^{1,0})^*$ is the adjoint of $\nabla^{1,0}$ with respect to the
Hermitian inner product on $\mathcal{A}^{p,q}(E)$, which is explicitly given by the
formula
\[
(\nabla^{1,0})^* = \overline{\star}_E\nabla^{1,0}_{E^*}\overline{\star}_E
\]
where $\nabla_{E^*}$ is the Chern connection for the dual bundle equipped with
the dual Hermitian form.
\end{prop}
%
\begin{proof}
Since it suffices to check in a local trivialization, we may assume that we have
taken an orthonormal frame for $E$. However, this cannot be a holomorphic trivialization,
so there will be some complications. We take inventory of the various operators in
this trivialization. Since the trivialization was from an orthonormal frame, the
operator $\star_E$ agrees with the usual Hodge star, so
$\overline{\star}_E = \overline{\star}$. In addition, we can write the Chern connection
$\nabla$ as $d + A$ for some skew-Hermitian matrix of $1$-forms $A$. This then gives
us
\[
\nabla_{E^*} = d + A^\dagger = d - A
\]
The $\dbar_E$-operator is of the form  $\dbar + A^{0,1}$, which is one of the
added complications when the trivialization is not holomorphic. If the trivialization
were holomorphic, we would have had $\dbar_E = \dbar$. Putting things together, we find
\begin{align*}
(\nabla^{1,0})^* &= \overline{\star}_E\nabla_{E^*}^{1,0}\overline{\star}_E  \\
&= \overline{\star}(\partial - A^{1,0})\overline{\star} \\
&= \partial^* - (A^{1,0})^\dagger
\end{align*}
We then compute
\begin{align*}
[\Lambda,\dbar_E] + i(\nabla^{1,0})^* &=
[\Lambda, \dbar + A^{0,1}] + i(\partial^* - (A^{1,0})^\dagger) \\
&= [\Lambda,\dbar] + [\Lambda,A^{0,1}] + i\partial^* - i(A^{1,0})^\dagger
\end{align*}
The ordinary K\"ahler identity then gives us that this is equal to
$[\Lambda,A^{0,1}] - i(A^{1,0})^\dagger$. Since this is an order $0$ operator,
it suffices to verify that it vanishes when restricted to any fiber of $E$. However,
since we can always pick a smooth local trivialization of $E$ in which $A = 0$,
we have that $[\Lambda,A^{0,1}] - i(A^{1,0})^\dagger = 0$, which them
implies that
\[
[\Lambda,\dbar_E] = -i(\nabla^{1,0})^*
\]
\end{proof}
%
We then prove one final lemma.
%
\begin{lem}
Let $E$ be a holomorphic vector bundles with a Hermitian metric and Chern connection
$\nabla$. Let $(\cdot,\cdot)$ denote the Hermitian inner product on
$\mathcal{A}^{p,q}(E)$ induced by the Hermitian metric on $E$ and the Hermitian metric
on $X$. Then for a harmonic form $\alpha \in \mathcal{H}^{p,q}(E)$, we have the
following inequalities:
\begin{align*}
i(F_\nabla\Lambda\alpha,\alpha) &\leq 0 \\
i(\Lambda F_\nabla\alpha,\alpha) &\geq 0
\end{align*}
\end{lem}
%
\begin{proof}
Since the Chern connection satisfies $\nabla^{0,1} = \dbar_E$ and the curvature
$F_\nabla$ is of type $(1,1)$. we know that
\[
F_\nabla = \nabla^{1,0}\dbar_E + \dbar_E\nabla^{1,0}
\]
We then compute
\begin{align*}
i(F_\nabla\Lambda\alpha,\alpha) &= i(\nabla^{1,0}\dbar_E\Lambda\alpha,\alpha)
+ i(\dbar_E\nabla^{1,0}\Lambda\alpha,\alpha) \\
&= i(\dbar_E\Lambda\alpha,(\nabla^{1,0})^*\alpha)
+ i(\nabla^{1,0}\Lambda\alpha,\dbar_E^*\alpha) \\
&= i(\dbar_E\Lambda\alpha,(\nabla^{1,0})^*\alpha)
+ i(\nabla^{1,0}\Lambda\alpha,0) \\
&= -(\dbar_E\Lambda\alpha, i(\nabla^{1,0})^*\alpha) \\
&= (\dbar_E\Lambda\alpha, [\Lambda,\dbar_E]\alpha) \\
&= (\dbar_E\Lambda\alpha, \Lambda\dbar_E\alpha)
- (\dbar_E\Lambda\alpha,\dbar_E\Lambda\alpha) \\
&= -\norm{\dbar_E\Lambda\alpha}_{L^2}^2 \leq 0
\end{align*}
We perform a similar computation for the other inequalty.
\begin{align*}
i(\Lambda F_\nabla\alpha,\alpha) &= i(\Lambda\nabla^{1,0}\dbar_E\alpha,\alpha)
+ i(\Lambda\dbar_E\nabla^{1,0}\alpha,\alpha) \\
&= i(\Lambda\dbar_E\nabla^{1,0}\alpha,\alpha) \\
&= i([\Lambda,\dbar_E]\nabla^{1,0}\alpha,\alpha)
+i(\dbar_E\Lambda\nabla^{1,0}\alpha,\alpha) \\
&= i(-i(\nabla^{1,0})^*\nabla^{1,0}\alpha,\alpha)
+ i(\Lambda\nabla^{1,0}\alpha,\dbar_E^*\alpha) \\
&= (\nabla^{1,0}\alpha,\nabla^{1,0}\alpha) \\
&= \norm{\nabla^{1,0}\alpha}_{L^2}^2 \geq 0
\end{align*}
\end{proof}
%
With all that done, we are ready to prove the Kodaira-Nakano vanishing theorem.
%
\begin{thm}[\ib{Kodaira-Nakano Vanishing}]
Let $L \to X$ be a positive line bundle over a compact K\"ahler manifold $X$ of complex
dimension $n$. Then for $p+1 > n$, we have
\[
H^q(X,\Omega^{p,0}_X\otimes L) = 0
\]
\end{thm}
%
\begin{proof}
From Hodge theory we know that $H^q(X,\Omega_X^{p,0} \otimes L)$ is isomorphic
to the space $\mathcal{H}^{p,q}(L)$ of harmonic $L$-valued $(p,q)$-forms. Therefore,
it suffices to show that no harmonic $(p,q)$-forms exist when $p+q > n$. Since
$L$ is a positive line bundle, it admits a Hermitian metric with such that
$(i/2\pi)F_\nabla$ is equal to the K\"ahler form. Therefore, its action on
$\mathcal{A}^{p,q}(L)$ is the same as the Lefschetz operator $L$. Using the
commutation relation
\[
[\Lambda,L] = (n-(p+q))\id_{\mathcal{A}^{p,q}(L)}
\]
we get that
\[
\frac{i}{2\pi}([\Lambda,F_\nabla]\alpha,\alpha)
= ([\Lambda,L]\alpha,\alpha) = (n-(p+q))\norm{\alpha}^2_{L^2}
\]
However, we know that
\[
i([\Lambda,F_\nabla]\alpha,\alpha) = i(\Lambda F_\nabla\alpha,\alpha)
- i(F_\nabla\Lambda\alpha,\alpha) \geq 0
\]
So if $p+q > n$, we must have that $\norm{\alpha}^2_{L^2} = 0$, i.e. $\alpha = 0$.
\end{proof}
%
The lemmas we proved earlier also give rise to another vanshing theorem due to
Serre.
%
\begin{thm}[\ib{Serre Vanishing}]
Let $L \to X$ be a positive line bundle. Then for any holomorphic vector bundle
$E \to X$, there exists a positive integer $m_0$ such that
\[
H^q(X,E \otimes L^m) = 0
\]
For any $m \geq m_0$.
\end{thm}
%
\begin{proof}
We first fix Hermitian metrics on $E$ and $L$, where the Hermitian metric on $L$ is
chosen such that the Chern connection $\nabla_L$ satisfies
$(i/2\pi)F_{\nabla_L} = \omega$, where $\omega$ is the K\"ahler form of $X$. We
note that an easy calculation gives us that $\nabla_{L^m} = m\omega$.
Then let $\nabla_E$ denote the Chern connection for $E$, and let
\[
\nabla = \nabla_E \otimes \nabla_{L^m} = \nabla_E \otimes \id + \id \otimes \nabla_{L^m}
\]
The curvature of $\nabla$ is then given by
\[
F_\nabla = F_{\nabla_E} \otimes \id + m(\id \otimes \omega)
\]
Using our lemmas from before, we then compute for
$\alpha \in \mathcal{H}^{p,q}(E \otimes L^m)$
\begin{align*}
i([\Lambda,F_\nabla]\alpha,\alpha) &= ([\Lambda,F_{\nabla_E}]\alpha,\alpha)
+ i([\Lambda,mL]\alpha,\alpha) \\
&= i([\Lambda,F_{\nabla_E}]\alpha,\alpha) + (m(n-(p+q)))\norm{\alpha}^2_{L^2}
\end{align*}
The operator $[\Lambda,F_{\nabla_E}]$ is algebraic, so we can use a fiberwise
Cauchy-Schwarz inequality
\[
|\langle[\Lambda,F_{\nabla_E}]\alpha,\alpha\rangle|
\leq \norm{[\Lambda,F_{\nabla_E}]}\norm{\alpha}^2
\]
where we have chosen our favorite operator norm on $\End(E_x)$ for each fiber
and we are computing the inner products and norms fiberwise. Compactness of $X$
then ensures that we can obtain a global bound $C$ such that
$C \geq |\langle[\Lambda,F_{\nabla_E}]\alpha,\alpha\rangle|$
Then since
$i([\Lambda,F_\nabla]\alpha,\alpha) \geq 0$, we have that when $p+q > n$, a
sufficiently large choice of $m$ will make $m(n-(p+q))$ sufficiently negative
such that the only we we can have $\alpha = 0$.
\end{proof}
%
With the Kodaira-Nakano Vanishing Theorem and more help from Hodge theory, we have
anotherway to compute the sheaf cohomology of line bundles over $\CP^n$. To
do this we note two facts.
\begin{enumerate}
  \item $\O(1)$ is a positive line bundle.
  \item $\O(-n-1) \cong K_{\CP^n}$.
\end{enumerate}
%
The proof of the first fact follows from an easy computation of the curvature.
$\O(1)$ embdeds into the trivial bundle $\CP^n \times (\C^{n+1})^*$, which gives
it a Hermitian metric coming from the obvious one on $(\C^{n+1})^*$. Then it can be
shown that the Chern connection of this metric is just the Fubini-Study metric on
$\CP^n$ (in fact, the standard definition of the Fubini-Study is secretly
this construction). The proof of the second fact amounts to checking the
determinants of the Jacobians of the transition functions for the standard open
cover of $\CP^n$, and noting that they agree with the transition functions
for $\O(-n-1)$. We will also use the following result from Hodge theory.
%
\begin{thm}[\ib{Serre Duality}]
Let $X$ be a compact K\"ahler manifold of complex dimension $n$ and let $E \to X$
be a holomorphic vector bundle.
Then
\[
H^{n-q}(X, E^*\otimes K_X) \cong H^q(X,E)^*
\]
\end{thm}
%
We then apply the Kodaira-Nakano Vanishing Theorem to deduce which cohomology
groups are $0$. For $m > 0$, the vanishing theorem tells us that for $p+q > n$, we have
$H^q(\CP^n,\Omega^{p,0} \otimes \O(m)) = 0$. Then since $\Omega^{n,0} = K_X = \O(-n-1)$,
this tells us that $H^q(\CP^n,\O(m)) = 0$ for $q > 0$ and $m \geq -n$. Serre duality
then gives a similar vanishing result for $K_X \otimes \O(m) = \O(-m-n-1)$.
Putting everything together we get that
\[
H^q(\CP^n,\O(m)) = \begin{cases}
0 & 0 < q < n \\
0 & q = 0, m < 0 \\
0 & q = n , m > -n-1
\end{cases}
\]
Recall that the global sections of the line bundle $\O(m)$ are given by
\[
H^0(\CP^n,\O(m)) = \C[z_0,\ldots z_n]_m
\]
for $m \geq 0$, are are $0$ otherwise. Serre duality then tells us that
\[
H^n(X, \O(-m - n - 1)) \cong H^0(X,\O(m))^*
\]
Therefore, we get that if $m \geq 0$,
\[
H^q(\CP^n, \O(m)) = \begin{cases}
\C[z_0,\ldots z_n]_m & q = 0 \\
0 & \text{otherwise}
\end{cases}
\]
In the case that $m < 0$, we have
\[
H^q(\CP^n, \O(m)) = \begin{cases}
\C[z_0,\ldots z_n]_{n-m+1}^* & q = n \\
0 & \text{otherwise}
\end{cases}
\]
%