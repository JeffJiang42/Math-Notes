\section{\v{C}ech Cohomology}
%
For the most part, we will consider ``sufficiently nice" topological spaces.
For the most part, think of a space $X$ as a (smooth, complex) manifold, an analytic
space, or a quasicompact separated scheme if you're feeling adventurous.
%
\begin{defn}
Let $X$ be a space and $\mathcal{F}$ a sheaf of abelian groups over $X$. Let
$\mathcal{U} = \set{U_i}_{i \in \N}$ be a countable open cover of $X$ that is locally
finite, i.e. for any $x \in X$, only finitely many $U_i$ contain $x$. For
$I = \set{i_1, \ldots, i_k}$, let
\[
U_I \defeq \bigcap_{i \in I} U_i
\]
Then define the \ib{\v{C}ech cochain groups} of $\mathcal{F}$ for the cover
$\mathcal{U}$ by
\begin{align*}
C^k(\mathcal{U}, \mathcal{F}) \defeq \prod_{|I| = k+1} \mathcal{F}(U_I)
\end{align*}
an element of $C^k(\mathcal{U},\mathcal{F})$ is called a \ib{\v{C}ech cochain}.
For a $k$-cochain $\sigma$, and $I = \set{i_0,\ldots i_k}$, we denote the
component of $\sigma$ over $U_I$ as $\sigma_I$ or $\sigma_{i_0,\ldots i_k}$.
\end{defn}
%
The \v{C}ech cochain groups are equipped with a differential
$d : C^k(\mathcal{U}, \mathcal{F}) \to C^{k+1}(\mathcal{U}, \mathcal{F})$
where for $\sigma \in C^k(\mathcal{U}, \mathcal{F})$, the ${i_0, \ldots i_{k+1}}$
component of $d\sigma$ is given by
\[
(d\sigma)_{i_0, \ldots i_{k+1}}
= \sum_{j = 1}^{p+1}(-1)^j\sigma_{i_0, \ldots, \widehat{i_j}, \ldots, i_{k+1}}
\vert_{U_0 \cap \cdots \cap U_{k+1}}
\]
where $\widehat{i_j}$ denotes that $i_j$ is missing. We have that $d^2 = 0$ for a
similar reason that $d^2 = 0$ for singular cohomology, you get repeats of
terms with opposite signs due to the omitted index. We denote the kernel of
$d : C^i(\mathcal{U}, \mathcal{F}) \to C^{i+1}(\mathcal{U}, \mathcal{F})$ as
$Z^i(\mathcal{U}, \mathcal{F})$, and we say that the element are $i$-cocycles.
We denote the image of
$d : C^{i-1}(\mathcal{U}, \mathcal{F}) \to C^i(\mathcal{U}, \mathcal{F})$ as
$B^i(\mathcal{U}, \mathcal{F})$, and we call the elements $i$-coboundaries.
%
\begin{defn}
The \ib{\v{C}ech cohomology groups} of $\mathcal{F}$ with respect to the cover
$\mathcal{U}$, denoted $\check{H}^i(\mathcal{U}, \mathcal{F})$ is the cohomology of
the \v{C}ech complex
\[\begin{tikzcd}
0 \ar[r, "d"] & C^0(\mathcal{U}, \mathcal{F}) \ar[r, "d"] &
C^1(\mathcal{U}, \mathcal{F}) \ar[r, "d"]
& \cdots
\end{tikzcd}\]
i.e. we have
\[
\check{H}^i(\mathcal{U}, \mathcal{F}) \defeq
\frac{Z^i(\mathcal{U}, \mathcal{F})}{B^i(\mathcal{U}, \mathcal{F})}
\]
\end{defn}
%
\begin{defn}
Given an open cover $\mathcal{U} = \set{U_i}$, a \ib{refinement} of $\mathcal{U}$
is another open cover $\mathcal{V} = \set{V_j}$ such that every $V_j$ is contained
in some $U_i$. If $\mathcal{V}$ is a refinement of $\mathcal{U}$, we write
$\mathcal{V} < \mathcal{U}$.
\end{defn}
%
If $\mathcal{V} < \mathcal{U}$, then we know we can find a map
$\varphi : \N \to \N$ such that $V_i \subset U_\varphi(i)$. Consequently, we
can restrict sections over $U_i$ to sections over $V_{\varphi(i)}$, so this
induces a chain map
$\rho_\varphi : C^k(\mathcal{U}, \mathcal{F}) \to C^k(\mathcal{V},\mathcal{F})$
where
\[
(\rho_\varphi(\sigma))_{i_0,\ldots, i_k}
= \sigma_{\varphi(i_0),\ldots \varphi(i_k)}\vert_{U_{i_0} \cap \cdots \cap U_{i_k}}
\]
This map commutes with the differentials for $C^\bullet(\mathcal{U},\mathcal{F})$
and $C^\bullet(\mathcal{V},\mathcal{F})$, so it descends to homomorphisms
$\check{H}^i(\mathcal{U},\mathcal{F}) \to \check{H}^i(\mathcal{V}, \mathcal{F})$.
It can be shown that a different choice of chain map $\rho_\psi$ for
$\psi : \N \to \N$ is chain homotopic to $\rho_\varphi$, so the induced maps
on cohomology are independent of our choice of $\varphi$.
%
\begin{defn}
The \ib{\v{C}ech cohomology groups} of a sheaf $\mathcal{F}$ over $X$ is the limit over
refinements
\[
\check{H}^i(X, \mathcal{F})
\defeq \lim_{\mathcal{V} < \mathcal{U}} \check{H}^i(\mathcal{V}, \mathcal{F})
\]
i.e. the quotient disjoint union
$\amalg_\mathcal{U}\check{H}^i(\mathcal{U}, \mathcal{F}$
over all refinements, where we identify $\sigma \in \check{H}^i(\mathcal{U}, \mathcal{F}$
and $\tau \in \check{H}^i(\mathcal{V}, \mathcal{F})$ if
$\mathcal{V}$ refines $\mathcal{U}$ and $\rho(\sigma) = \tau$ under the map induced on
cohomology by the refinement.
\end{defn}
%
This definition of the \v{C}ech cohomology groups is essentially useless for
computation. It's true power comes from the following theorem, due to Leray.
%
\begin{thm}[\ib{Leray}]
Suppose $\mathcal{U} = \set{U_i}$ is an open cover of $X$ that is \ib{acyclic} with
respect to the sheaf $\mathcal{F}$, i.e. for $|I| > 1$ and any $i$,
\[
\check{H}^i(U_I, \mathcal{F}) = 0
\]
Then
\[
\check{H}^i(\mathcal{U}, \mathcal{F}) = \check{H}^i(X, \mathcal{F})
\]
\end{thm}
%
The intuition to keep in mind for an acyclic cover is the notion of a good cover
in differential geometry. On a smooth manifold $M$, there exists a covering of
$M$ by open sets $\set{U_i}$ such that any nonempty intersections are contractible,
which is done by taking geodesic balls around each point in $M$. Since the
homotopical information on each $U_i$ is trivial, the only nontrivial topological
information in the cohomology of $M$ comes from how the sets are glued together to
form $M$. As with exact sequences of chain complexes, short exact sequences
of sheaves give long exact sequences in sheaf cohomology.
%
\begin{thm}
Let
\[\begin{tikzcd}
0 \ar[r] & \mathcal{E} \ar[r, "\alpha"] & \mathcal{F} \ar[r, "\beta"] & \mathcal{G}
\ar[r] & 0
\end{tikzcd}\]
be an exact sequence of sheaves over $X$. Then this induces a long exact sequence in
cohomology:
\[\begin{tikzcd}
0 \ar[r] & \check{H}^0(X,\mathcal{E}) \ar[r, "\alpha^*"]
& \check{H}^0(X,\mathcal{F}) \ar[r, "\beta^*"] &
\check{H}^0(X, \mathcal{G}) \ar[dll, "\delta"'] \\
& \check{H}^1(X, \mathcal{E}) \ar[r, "\alpha^*"]
& \check{H}^1(X, \mathcal{F})\ar[r, "\beta^*"]
& \check{H}^1(X, \mathcal{G}) \ar[dll, "\delta"'] \\
& ~ & \cdots & \cdots
\end{tikzcd}\]
\end{thm}
%
\begin{proof}
We first define the maps
$\alpha^* : \check{H}^i(X,\mathcal{E}) \to \check{H}^i(X, \mathcal{F})$
and $\beta^* : H^i(X, \mathcal{F}) \to \check{H}^i(X, \mathcal{G})$, and will then define
the connecting homomorphism
$\delta : \check{H}^i(X,\mathcal{E}) \to \check{H}^{i+1}(X, \mathcal{G})$. Given
an open cover $\mathcal{U} = \set{U_i}$ of $X$, the sheaf morphism $\alpha$ gives for
each open set $U_i$ a homomorphism $\alpha(U_i) : \mathcal{E}(U_i) \to \mathcal{F}(U_i)$,
which induces a chain map
$C^\bullet(\mathcal{U},\mathcal{F}) \to C^\bullet(\mathcal{U}, \mathcal{G}))$. Since
the maps $\alpha(U_i)$ commute with restriction maps, this chain map commutes with
the differentials, so it descends to a map on cohomology
$\check{H}^\bullet(\mathcal{U},\mathcal{F})\to\check{H}^\bullet(\mathcal{U},\mathcal{F})$,
which, after taking the limit over refinements or choosing $\mathcal{U}$ to be
simultaneously acyclic for $\mathcal{E}$ and $\mathcal{F}$, gives us the induced
map $\alpha^* : \check{H}^\bullet(X,\mathcal{F})\to\check{H}^\bullet(X,\mathcal{F})$.
The map $\beta^*$ is defined similarly. \\

The construction of the connecting homomorphism
mirrors the construction for singular (or de Rham) cohomology. We represent
an element of $\check{H}^i(X,\mathcal{G})$ with a cocycle
$\sigma \in Z^i(\mathcal{U}, \mathcal{G})$ with respect to some open cover $\mathcal{U}$.
By surjectivity of $\beta$, by potentially passing to a refinement, we can
write $\sigma = \beta(\tau)$ for some $\tau \in C^i(\mathcal{U}, \mathcal{F})$. Then
since the induced map on chains commutes with the differentials, we have that
$d\beta(\tau) = \beta(d\tau) = 0$, since $\tau$ is a cocycle. Therefore,
$d\tau$ is in the kernel of $\beta^*$, so we can write $d\tau = \alpha(\eta)$
for some $\eta \in C^{i+1}(\mathcal{U}, \mathcal{F})$. We then define
$\delta(\sigma)$ to be the class of $\eta$ in the limit. We note that
this is independent of our choice of $\tau$, since any other choice of
preimage of $\sigma$ differs by an element of the form $\alpha(e)$ for some cocycle
$e$ by exactness. Then since $\alpha$ commutes with the differentials, we get
$d(\tau + \alpha(e)) = d\tau + d\alpha(e) = d\tau + \alpha(de) = d\tau$.
\end{proof}
%
We make one observation about \v{C}ech cohomology
%
\begin{prop}
The $0^{th}$ \v{C}ech cohomology group is isomorphic to the space of global sections,
i.e.
\[
\check{H}^0(\mathcal{U}, \mathcal{F}) \cong \Gamma(X, \mathcal{F})
\]
\end{prop}
%
\begin{proof}
The $0^{th}$ cohomology is just the kernel of the map
$d : C^0(\mathcal{U},\mathcal{F}) \to C^1(\mathcal{U}, \mathcal{F})$.
For a $0$-cochain $\sigma$, we have that
\[
(d\sigma)_{ij} = \sigma_j\vert_{U_i \cap U_j} - \sigma_i\vert_{U_i \cap U_j}
\]
We then claim that the map $\Gamma(X, \mathcal{F}) \to \ker d$ sending a section
$\sigma$ to the cocycle $\tilde{\sigma}$ defined by
\[
\tilde{\sigma}_i = \sigma\vert_{U_i}
\]
is bijective. It is surjective, since any $0$-cocycle contained in the kernel
is a collection of local sections that agrees on intersections, which is exactly
a global section of $\mathcal{F}$. In addition, it is injective, since if
a section restricts to $0$ on every open set, it is the zero section.
\end{proof}
%TODO Functoriality of Cech cohomology