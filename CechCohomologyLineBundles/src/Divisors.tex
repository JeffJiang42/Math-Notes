%
\section{Divisors}
%
Given a complex manifold $X$, a complex hypersurface $Y \subset X$ (also called an
analytic hypersurface) is locally cut out a single holomorphic function, i.e. there is
an open cover $\set{U_i}$ of $X$ such that $Y \cap U_i$ is the vanishing locus of a
single holomorphic function $g_i$. We say that the functions $g_i$ are
\ib{defining functions} for $Y$. The hypersurface $Y$ is said to be \ib{irreducible}
if at any $p \in Y$, the defining function $g$ for $Y$ in a neighborhood about
$p$ is irreducible in the local ring $\O_{X,p}$. If the defining function for a
hypersurfaces is not irreducible, then it can be written as a product of irreducible
functions $g_i$ in $\O_{X,p}$, in which case we can write $Y$ as the union of the
irreducible hypersurfaces determined by the $g_i$. \\

On the complex line $\C$, the irreducible hypersurfaces are just points $p \in \C$,
which have defining functions $f(z) = (z-p)$. Recall that all meromorphic functions
on $\C$ are of the form $p(z)/q(z)$ for polynomials $p(z),q(z) \in \C[z]$. A great
deal of the study of meromorphic functions on $\C$ revolves around studying their zeroes
and poles, which is something that divisors will encapsulate for higher dimensional
complex manifolds.
%
\begin{defn}
Let $Y \subset X$ be an irreducible hypersurface. Let $f \in \mathcal{K}(X)$ be a
meromorphic function, and fix a point $y \in Y$. Then $Y$ is locally cut out by an
irreducible holomorphic function $g \in \O_{X,y}$. The \ib{order} of $f$ at
$y$, denoted $\mathrm{ord}_{Y,y}(f)$ is the smallest integer $n$ such that there exists
an invertible holomorphic function $h \in \O_{X,y}$ such that $f= g^nh$.
\end{defn}
%
We note that this definition is independent of our choice of defining function, since
any two irreducible defining functions for $Y$ differ by a unit in $\O_{X,y}$. In
addition, if $Y$ is irreducible, it is also independent of our choice of point $y \in Y$,
since if a holomorphic function $g$ is irreducible in $\O_{X,y}$, then it is also
irreducible in $\O_{X,y'}$ for any point $y'$ sufficiently close. Therefore, we can
simply write $\mathrm{ord}_Y(f) \defeq \mathrm{ord}_{Y,y}(f)$ for any point $y \in Y$,
provided that $Y$ is an irreducible hypersurface. The intuition behind the definition
is that the order of $f$ at $y$ should be the degree to which the function vanishes
at the point $y$, or the degree of the pole at $y$
%
\begin{exmp}
Let $X = \C$, and $Y = \set{p}$, so $Y$ has locally (in fact globally) defining
function $g(z) = (z-p)$. Then let $f$ be the meromorphic function
\[
f(z) = \frac{(z-p)^k}{q(z)}
\]
where $q(z)$ is a polynomial that does not vanish at $p$.
Then the order of $f$ at $p$ is, as you would expect, $k$. The role of $h$
is played by the locally invertible function $1/q(z)$, which is a unit in the local
ring $\O_{\C,p}$. If instead we have
\[
f(z) = \frac{q(z)}{(z-p)^k}
\]
for a polynomial $q(z)$ that does not vanish at $p$, then the order of $f$
at $y$ is $-k$, where the role of $h$ is played by $q(z)$.
\end{exmp}
%
A simple, but important observation is that order is additive on products.
%
\begin{prop}
Let $Y \subset X$ be an irreducible hypersurface with locally defining function
$g$ at $y \in Y$. Let $f_1,f_2 \in \mathcal{K}(X)$ be meromorphic functions. Then
\[
\mathrm{ord}_Y(f_1f_2) = \mathrm{ord}_Y(f_1) + \mathrm{ord}_Y(f_2)
\]
\end{prop}
%
Some more key observations are that for a nonvanishing holomorphic function $f$, we have
that $\mathrm{ord}_Y(f) = 0$ for any irreducible hypersurface $Y \subset X$, and
that $\mathrm{ord}_Y(f) = 0 \mathrm{ord}_Y(1/f)$.
%
\begin{proof}
In the local ring $\O_{X,y}$, we have that
\[
f_1 = g^{\mathrm{ord}_Y(f_1)}h_1 \qquad f_2 = g^{\mathrm{ord}_Y(f_2)}h_2
\]
Therefore, we have that
\[
f_1f_2 = g^{\mathrm{ord}_Y(f_1) + \mathrm{ord}_Y(f_2)}h_1h_2
\]
The function $h_1h_2$ does not vanish at $y$, and we have that
$\mathrm{ord}_Y(f_1) + \mathrm{ord}_Y(f_2)$ must be the smallest integer for
$f_1f_2$, since $\mathrm{ord}_Y(f_1)$ and $\mathrm{ord}_Y(f_2)$ are minimal for
$f_1$ and $f_2$ respectively.
\end{proof}
%
\begin{defn}
A \ib{divisor} $D$ on $X$ is a formal integral combination
\[
D = \sum a_i[Y_i]
\]
where $Y_i$ is an irreducible hypersurface for all $i$. The group of divisors on
$X$ is denoted $\mathrm{Div}(X)$. The divisor $D$ is said to be an
\ib{effective divisor} if the $a_i$ are all nonnegative.
\end{defn}
%
The intuition for a divisor is that it should interpreted as a prescription for zeroes
and poles of a meromorphic function. For example, in $\C$, you should think of
the divisor
\[
D = p_1 + 4p_2 - 4p_3 - 6p_4
\]
as corresponding the the meromorphic function
\[
f(z) = \frac{(z-p_1)(z-p_2)^4}{(z-p_3)^4(z-p_4)^6}
\]
Of course, this isn't exactly true or precise. For example, we could multiply the
numerator or denominator by nonvanishing holomorphic functions and then obtain a function
with the same poles and zeroes as $f$. From this perspective, and effective
divisor is a divisor that only prescribes zeroes.
%
\begin{defn}
Let $f \in \mathcal{K}(X)$ be a meromorphic function. Then the \ib{zero divisor} of
$f$ is the divisor
\[
(f)_0 = \sum_{\mathrm{ord}_Y(f) > 0} \mathrm{ord}_Y(f)[Y]
\]
where the sum is taken over all irreducible hypersurfaces $Y > 0$. Likewise, the
\ib{pole divisor} of $f$ is defined to be the divisor
\[
(f)_\infty = \sum_{\mathrm{ord}_Y(f) < 0} \mathrm{ord}_Y(f)[Y]
\]
The \ib{associated divisor} of $f$ is the divisor
\[
(f) = (f)_0 - (f)_\infty = \sum_Y \mathrm{ord}_Y(f)
\]
\end{defn}
%
This makes precise what we mean by divisors being prescriptions for poles and
zeroes, but the correspondence between meromorphic functions and divisors is
far from bijective, as we noted above. Our observation was that if any two meromorphic
functions differed by a nonvanishing holomorphic function, then they will define the
same divisor. This turns out to be the only redundancy.
%
\begin{thm}
Let $X$ be a complex manifold. Then let $\mathcal{K}^\times_X$ denote the sheaf of
nonzero meromorphic functions, and $\O_X^\times \subset \mathcal{K}^\times_X$ the sheaf
of nonvanishing holomorphic functions, which are both sheaves of abelian groups
under multiplication. Then we have an isomorphism
\[
\Gamma(X, \mathcal{K}^\times_X/\O^\times_X) \to \mathrm{Div}(X)
\]
\end{thm}
%
\begin{proof}
An global section $\sigma$ of the quotient sheaf $\mathcal{K}_X^\times / \O_X^\times$
is a collection of pairs $(U_i, f_i)$ of meromorphic functions
$f_i \in \mathcal{K}_X^\times(U_i)$ such that on the intersections $U_i \cap U_j$, their
quotient  $f_i/f_j$ is an element of $\O_X^\times$, i.e. a nonvanishing holomorphic
function over $U_i \cap U_j$. Therefore, over the intersection, we have that
\[
\mathrm{ord}_Y(f_i) - \mathrm{ord}_Y(f_j) = \mathrm{ord}_Y(f_i/f_j) = 0
\]
for any irreducible hypersurface $Y$. Therefore, we have that
$\mathrm{ord}_Y(f_i) = \mathrm{ord}_Y(f_j)$. Therefore, the order
$\mathrm{ord}_Y(\sigma)$ of $\sigma$ is well-defined for any irreducible hypersurface
$Y \subset X$, so we get a well-defined map
\begin{align*}
\Gamma(X, \mathcal{K}^\times_X / \O_X^\times) &\to \mathrm{Div}(X) \\
\sigma &\mapsto (\sigma) \defeq \sum_Y \mathrm{ord}_Y(\sigma)[Y]
\end{align*}
To show that this mapping is bijective, we provide an inverse map. Let
$D = \sum_i a_i [Y_i] \in \mathrm{Div}(X)$. Then we can find an open cover $\set{U_i}$
of $X$ where in each open set $U_j$, any irreducible hypersurface $Y_i$ intersecting
$U_j$ nontrivially is the vanishing locus of a holomorphic function $g_{ij}$, which
is unique up to multiplication by an element in $\O_X^\times(U_j)$. Then we claim that
the functions $f_j \defeq \prod_i g_{ij}$ glue to a global section of
$\mathcal{K}_X^\times / \O_X^\times$, i.e. on any $U_j \cap U_k$, we have that
$f_j/f_k$ is a nonvanishing holomorphic function. However, on any intersection
$U_j \cap U_k$, the functions $g_{ij}$ and $g_{ik}$ define the same irreducible
hypersurface $Y_i$, so they differ by multiplication by a nonvanishing holomorphic
function. Therefore, the $f_j$ glue to a global section. \\

It is easily verified that these two constructions are inverses, so the maps
are bijective. Furthermore, the map is a group homomorphism since order is
additive on products of meromorphic functions, so it is a group isomorphism.
\end{proof}
%
\begin{rem*}
For complex manifolds, we have that global sections of the quotient sheaf and divisors
are isomorphic, but in the algebro-geometric setting, this is not true without some
smoothness assumptions. In algebraic geometry, elements of
$\Gamma(X, \mathcal{K}_X^\times/\mathcal{O}_X^\times)$ are called \ib{Cartier divisors},
while the elements of what we called $\mathrm{Div}(X)$ are called \ib{Weil divisors}.
\end{rem*}
%
One of the big punchlines for divisors is their correspondence with holomorphic
line bundles over $X$, which we can make explicit using the isomorphism with
$\Gamma(X, \mathcal{K}^\times_X / O^\times_X)$. In the proof we gave above, we noted
that a global section of $\mathcal{K}^\times_X / O^\times_X$ can be seen as a collection
of meromorphic functions $f_i$ on an open cover $\set{U_i}$, such that on intersections,
the function $f_i/f_j$ is a nonvanishing holomorphic function. In addition, it is easily
seen that the functions satisfy the cocycle condition, so they can be used
to define transition functions for a line bundle, which we call $\O(D)$. The
mapping $D \mapsto \O(D)$ is clearly a group homomorphism, so it defines a natural
map $\mathrm{Div}(X) \to \mathrm{Pic}(X)$. In particular, this shows that line bundles
are closely related to the codimension $1$ geometry of the complex manifold $X$. \\

Like line bundles, divisors also pull back, but there is some more subtlety
when working with divisors from the perspective of Weil divisors.
Given a holomorphic map $f : X \to Y$ and a hypersurface $D \subset Y$,
we would like to define the pullback $f^*[D]$ of the divisor $[D]$ to be
$[f\inv(D)]$, and then extend linearly to all divisors. In most cases, this is fine,
and $f\inv(D)$ is still codimension $1$, since locally it will be given by
the vanishing of $f \circ g$ where $g$ is some locally defining function for
$D$. However, if the image of $X$ is contained in $D$, then $f\inv(D) = X$, which
isn't codimension $1$! One way to fix this is to restrict our attention to maps
with dense image. The other is to just be careful when pulling back a divisor
along a map. The other issue involves irreducibility. In general,
given an irreducible hypersurface $D \subset Y$, the inverse image $f\inv(D)$
need not be irreducible. This can be solved by defining $f^*[D]$ to be
the divisor corresponding to the sum $\sum_{n_i}n_i[Y_i]$ where $n_i$ is the
exponent of the locally definining function for $[Y_i]$ in a factorization
of $f \circ g$, where $g$ is a locally defining function for $D$. Alternatively,
one can instead use the perspective of Cartier divisors.
Given a section of $\mathcal{K}_Y^\times/\O_Y^\times$, we can write it
in terms of an open cover by a collection of pairs $\set{(U_i, f_i)}$ for
open sets $U_i$ and $f_i \in \mathcal{K}_Y$ whose quotients are holomorphic
on $U_i \cap U_j$. Then we define the pullback of this divisor to be
the Cartier divisor $\set{(f\inv(U_i), f \circ f_i)}$. It is easy to verify
that this coincides with the definition in terms of Weil divisors when that
definition makes sense. Furthermore, unpacking the definition of the line
bundle $\O(D)$ corresponding to a Cartier divisor
$D \in \Gamma(Y,\mathcal{K}_Y^\times/\O_Y^\times)$, we find that
\[
\O(f^*D) \cong f^*\O(D)
\]
whenever the pullback divisor $f^*D$ makes sense. \\

Given an irreducible hypersurface $D$, we get a line bundle $\O(D)$ -- the
correspondence the other way comes in the form of sections. Given a holomorphic
line bundle $L \to X$ and a global section $s \in \Gamma(X,L)$, we define a divisor.
$Z(s)$ as follows : fix an open cover $\set{U_i}$ such that $L$ admits local
holomorphic trivializations $\varphi_i :L\vert_{U_i} \to U_i \times \C$ with
transition functions $\psi_{ij}$. This gives holomorphic functions
$\varphi_i \circ s : U_i \to \C$, which together define a Cartier divisor
$\set{(U_i, \varphi_i \circ s)}$ which we define to be $Z(s)$. It is easy to
verify that this definition is independent of our choice of trivialization.
%
\begin{prop} \enumbreak
\begin{enumerate}
  \item Let $s \in \Gamma(X,L)$ be a nonzero section. Then $\O(Z(s)) \cong L$.
  \item For an effective divisor $D$, there exists a nonzero section $s$ of $\O(D)$
  such that $Z(s) = D$.
\end{enumerate}
\end{prop}
%
\begin{proof} \enumbreak
\begin{enumerate}
  \item Fix an open cover $\set{U_i}$ with trivializations
  $\varphi_i : L\vert_{U_i} \to U_i \times \C$ and transition functions
  $\psi_{ij}$. We want to show that the divisor $Z(s) = \set{(U_i, f_i)}$
  (where $f_i \defeq \varphi_i \circ s$) defines a line bundle isomorphic to $L$.
  The line bundle defined by $Z(s)$ is the line bundle given by the cocycle
  $\set{(U_i, f_i/f_j)}$. We then note that since the $f_i$ came from a section $s$,
  they satisfy the compatibility condition $f_i = \psi_{ij}f_j$, so
  $f_i/f_j = \psi_{ij}$.
  \item For an effetive divisor $D$, the associated Cartier divisor
  $\set{(U_i,f_i)}$ has the property that the $f_i$ are all holomorphic,
  since they will be products of holomorphic functions with irreducible germs
  at some $p_i \in U_i$. The line bundle $\O(D)$ is given by the cocycle
  $\set{(U_i, f_i/f_j)}$. Then by construction, the $f_i$ glue
  together to a global section of $\O(D)$, since we have $f_i = f_i/f_j \cdot f_j$.
\end{enumerate}
\end{proof}
%
We note that the preceding discussion relies on the existence of a nonzero global
holomorphic section of $L$, which may or may not exist. We also note that the
correspondence only concerns effective divisors. We can generalize this to line
bundles with no global sections with the idea of a \ib{meromorphic section} of a line
bundle $L$, i.e. a global section of the sheaf $\mathcal{K}_X \otimes_{\O_X} L$. Much
like holomorphic sections, a meromorphic section can be written in a trivializing open
cover $\set{U_i}$ of $X$ as a collection of meromorphic functions
$(f_i/g_i) \in \mathcal{K}_X(U_i)$ satisfying the compatibility condition
$(f_i/g_i) = \psi_{ij}(f_j/g_j)$, where the $\psi_{ij}$ are the transition functions
of $L$. Then given a meromorphic section $s$, we can define a divisor in a similar
manner as $Z(s)$, except the Cartier divisor we obtain consists of meromorphic
functions over the $U_i$ rather than just holomorphic functions. This gives us the
following generalization of the previous proposition:
%
\begin{prop} \enumbreak
\begin{enumerate}
  \item Let $s \in \Gamma(X,\mathcal{K}_X\otimes_{\O_X}L)$ be a global meromorphic
  section of $L$, and let $D(s)$ be the divisor associated to $s$. Then
  $\O(D(s)) \cong L$.
  \item For any divisor $D$, there exists a nonzero meromorphic section $s$
  of $L$ such that $D(s) = D$.
\end{enumerate}
\end{prop}
%
The proof of this is near identical. One thing to note is that given
a meromorphic section $s$, we get that $D(1/s) = -D$.\\

Using these propositions, we can get a better understanding of the sheaf
of sections of the line bundle $\O(D)$ Let $D \subset X$ be a
divisor. Let $\mathcal{L}(D) \subset \mathcal{K}_X)$ denote the subsheaf of
meromorphic functions on $X$ such that $D + (f) \geq 0$. If we write
$D = \sum_i a_iV_i$ with $a_i \in \Z$ and $V_i$ irreducible hypersurfaces,
the functions in $\mathcal{L}(D)$ are exactly the meromorphic functions $f$
that are holomorphic on the complement of the $V_i$, and satisfy
$\mathrm{ord}_{V_i}(f) \geq -a_i$. Working locally, suppose $g_i$ is a
locally defining function for $V_i$. Then if $a_i$ is positive, then
this condition says that $fg_i^{a_i}$ is holomorphic. If $a_i$ is negative,
than this condition says that $f/g_i^{a_i}$ is holomorphic. Then let
$s_0 \in \Gamma(X,\mathcal{K}_X\otimes_{\O_X} \O(D))$ be a nonzero global meromorphic
section with $D(s_0) = D$. Then given a  holomorphic section $s$ of
$\O(D)$, we get a meromorphic function $f_s \in \Gamma(X,\mathcal{K}_X)$
defined locally by $f_s = s/s_0$. We then have that
$(f_s) = D(s) - D(s_0) \geq - D$, so $f_s \in \mathcal{L}(D)$. In the
other direction, we can obtain a holomorphic section of $\O(D)$ from
a meromorphic function $f \in \mathcal{L}(D)$ by taking $fs_0$. Furthermore,
these mappings are clearly inverses of each other, giving us an isomorphism
of sheaves $\mathcal{L}(D) \cong \O(D))$. Similarly, multiplication with $1/s_0$
gives an isomorphism $\mathcal{L}(-D) \cong \O(-D)$. Under this identification,
we note that multipying a section of $\O(-D)$ with $s_0$ yields a meromorphic
function which has no poles, and vanishes along the hypersurfaces appearing
in $D$ to the specified orders. In particular, if $D$ is a hypersurface,
then we get an exact sequence of sheaves
\[\begin{tikzcd}
0 \ar[r] & \O(-D) \ar[r] & \O_X \ar[r] & \O(D)\vert_D \ar[r] & 0
\end{tikzcd}\]
where the first map is multiplication by $s_0$, and the second map is
restricting sections from $X$ to $D$, and we are implicitly making the
identifications $\mathcal{L}(D) \cong \O(D)$ and $L(-D) \cong \O(-D)$.
In fact, we get an exact sequence for any holomorphic vector bundle
$E$. If we let $\mathcal{E}$ denote the sheaf of holomorphic sections of $E$,
then we get an exact sequence of sheaves
\[\begin{tikzcd}
0 \ar[r] & \mathcal{E} \otimes \O(-D) \ar[r] & \mathcal{E} \ar[r] & \mathcal{E}\vert_D
\ar[r] & 0
\end{tikzcd}\]
where again the first map is multiplication by $s_0$ and the second map is
restriction.
%
%TODO Divisors in terms of sheaf cohomology, line bundle of a divisor corresponding to
%the image under a boundary map
%