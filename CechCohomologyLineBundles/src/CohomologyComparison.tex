\section{Comparison of Cohomology Theories}
%
The fact that many of the sheaves we encounter naturally have trivial sheaf cohomology
might come as a surprise, since we know we can extract topological data from
these sheaves. The reason for this is that they provide good resolutions of
other sheaves with nontrivial sheaf cohomology. U
%
\begin{prop}[Poincar\'e Lemma]
Every closed smooth $k$-form $\omega$ is locally exact, i.e. for a sufficiently
small $U$, we have that $\omega\vert_U = d\eta$ for some $k-1$-form $\eta$.
\end{prop}
%
\begin{prop}[$\dbar$-Poincar\'e Lemma]
Ever closed smooth $(p,q)$-form $\omega$ is locally $\dbar$-exact
\end{prop}
%
\begin{cor}
The de Rham complex
\[\begin{tikzcd}
A^0(X) \ar[r, "d"] & A^1(X) \ar[r, "d"] & \cdots
\end{tikzcd}\]
is an exact sequence of sheaves.
\end{cor}
%
The constant sheaf $\underline{\R}$ of locally constant real-valued functions
naturally lives as a subsheaf of $A^0(X)$, and we know that this exactly
the kernel of $d : A^0(X) \to A^1(X)$. This tells us that the inclusion
$0 \to \underline{\R} \to A^\bullet(X)$ is a resolution of $\underline{\R}$,
called the \ib{de Rham resolution}. Furthermore,
since all the $A^i(X)$ are sheaves of $C^\infty$-modules, they are fine, so
the resolution is a resolution of $\underline{\R}$ by acyclic sheaves. Therefore,
we get the isomorphisms
\[
H^i(X, \underline{\R}) \cong H^i_{dR}(X)
\]
A similar story holds for the sheaf cohomology of the sheaf of sections of a holomorphic
vector bundle $E \to X$. The $\dbar$-Poincar\'e lemma implies that the Dolbeault
complex
\[\begin{tikzcd}
\mathcal{A}^0(E) \ar[r, "\dbar_E"] & \mathcal{A}^1(E) \ar[r, "\dbar_e"] & \cdots
\end{tikzcd}\]
of sheaves of smooth sections of $(\Lambda^i T^*X)_\C \otimes E$ is an exact sequence,
since $\dbar_E$ is defined locally in terms of the operator $\dbar$ on $X$.
Then since the kernel of $\dbar_E : \mathcal{A}^0(E) \to \mathcal{A}^1(E)$ is
exactly the sheaf $\mathcal{E}$ of holomorphic sections of $E$, we get that
$0 \to \mathcal{E} \to \mathcal{A}^\bullet(E)$ is an acyclic resolution of
$\mathcal{E}$, which gives us isomorphisms
\[
H^i(X,\mathcal{E}) \cong H^i_{\dbar}(X, E)
\]
%
%TODO Singular cohomology
%
%TODO Simplicial cohomology and another proof of de Rham?
%
%TODO Cech vs derived functor cohomology
%