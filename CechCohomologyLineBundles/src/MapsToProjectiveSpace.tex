%
\section{Maps to Projective Space}
%
Another application of the Kodaira-Nakano vanishing theorem is the Kodaira embedding
theorem. To discuss this, we will need to first discuss two things:
%
\begin{enumerate}
  \item The relationship between line bundles and maps to $\CP^n$.
  \item Blow-ups, which allow us to turn points into divisors, which
  allows us to pass through the divisor-line bundle correspondence.
\end{enumerate}
%
Suppose we are given a holomorphic line bundle $L \to X$ over a complex manifold
$X$. Then given a section $s : X \to L$ and an open cover $\set{U_i}$
such that $L\vert_{U_i}$ is trivial, the section can be identified with
a collection of holomorphic functions $s_i : U_i \to \C$ such that
$s_i = \psi_{ij}s_j$, where the $\psi_{ij}$ are the transition functions of
$L$ with respect to the cover $\set{U_i}$.
%
\begin{prop}
Suppose $s_0,\ldots s_N$ are sections of a holomorphic line bundle $L \to X$ such
that for any $x \in X$, we have $s_i(x) \neq 0$ for some $i$. Then the map
$\varphi : X \to \CP^N$ defined in local trivializations by the mapping
\[
x \mapsto [s_0(x): \ldots :s_N(x)]
\]
is well defined and satisfies $\varphi^*\O_{\CP^N}(1) \cong L$.
\end{prop}
%
\begin{proof}
We first verify that $\varphi$ is well-defined. let $\set{U_\alpha}$ be
an open cover trivializing $L$, with transition functions $\psi_{\alpha\beta}$.
Then let $s_{i,\alpha} : U_\alpha \to \C$ the functions corresponding to the $s_i$
with respect to these local trivializations, so
$s_{i,\alpha} = \psi_{\alpha\beta}s_{i,\beta}$. We note that over any
$U_\alpha$, the map we specified is well defined, since at least one
$s_{i,\alpha}$ is nonzero by assumption. We then need to verify that
the maps specified on the $U_\alpha$ agree on the intersections $U_{\alpha\beta}$,
which amounts to the fact that for $x \in U_{\alpha\beta}$ we have
\[
[s_{i,\alpha}(x):\ldots :s_{N,\alpha}(x)] =
[\psi_{\alpha\beta}(x)s_{i,\beta}(x): \ldots : \psi_{\alpha\beta}s_{N,\beta}(x)]
\]
since homogeneous coordinates are only defined up to scaling. This also shows
that the map is independent of our choice of trivializations. \\

To show that $\varphi^*\O_{\CP^N} \cong L$, it suffices to show that
$\varphi^*\O_{\CP^N} \cong \O(Z(s_0))$. Conside the hyperplane divisor
$H$ on $\CP^N$, corresponding to the hyperplane $\CP^{N-1}$ defined by
the vanishing of the section $z_0$ of $\O(1)$. By intersecting the hyperplane
with the image of $\varphi_L$, we get a divisor $D$ on the image of $\varphi_L$
corresponding to restriction of $\O(1)$. Furthermore, we observe that
$\varphi^*D = Z(s_0)$ by construction, since $s_0$ is the first component
of the map $\varphi_L$. Therefore, $\varphi^*\O_{\CP^n} \cong \O(Z(s_0)) \cong L$.
\end{proof}
%
If the $s_i$ simultaneously vanish, we do not get a map
$X \to \CP^N$, but we do get a map $X / \mathrm{Bs}(s_0,\ldots s_N) \to \CP^N$, where
$\mathrm{Bs}(s_0,\ldots s_N)$ is called the \ib{base locus} of the sections
$s_i$, and is defined to be the subset of $X$ where the $s_i$ simultaneously vanish. \\

We collect some terminology concerning this discussion regarding line bundles
and maps to projective space.
%
\begin{defn}
Let $L \to X$ be a holomorphic line bundle.
\begin{enumerate}
  \item A \ib{linear system} of $L$ is a subspace of $\Gamma(X,L)$.
  The \ib{complete linear system} is the entire space $\Gamma(X,L)$.
  \item If there exist sections $s_0,\ldots s_N$ of $L$ with
  $\mathrm{Bs}(s_0,\ldots s_N) = \emptyset$, then $L$ is \ib{globally generated}.
\end{enumerate}
\end{defn}
%
Given a linear system $S \subset \Gamma(X,L)$ of $L$, upon fixing a basis
$s_0,\ldots s_N$ of sections for $S$, we get a map
$X-\mathrm{Bs}(s_0,\ldots s_N) \to \CP^N$. While the map depends on this
choice of basis, we see from the formula for the associated map that the
map associated to any other basis is related by the projective linear automorphism
of $\CP^N$ corresponding to the change of basis matrix.
%
\begin{defn}
A line bundle $L \to X$ is \ib{very ample} if the complete linear system
$\Gamma(X,L)$ defines a map to $\CP^N$ for some $N$. The bundle $L$ is said to be
\ib{ample} if some tensor power $L^k$ for $k > 0$ of $L$ is very ample.
\end{defn}
%