\section{The Chern Connection}
%
\begin{defn}
Let $E \to X$ be a complex vector bundle. A \ib{Hermitian structure} on $E$
is the data of a smoothly varying hermitian form $H(x)$ on each fiber $E_x$.
If $E$ is a complex vector bundle equipped with a Hermitian structure, we call it
a \ib{Hermitian vector bundle}.
\end{defn}
%
Hermitian structures always exist due to convexity of the space of Hermitian forms.
Therefore, we can define Hermitian forms locally in neighborhoods where the bundle
$E$ is trivial, and then use a partition of unity to glue them together.
%
Like in the case of Riemannian geometry, we can ask for a connection to be compatible
with a Hermitian structure.
%
\begin{defn}
Let $E \to X$ be a Hermitian vector bundle. A connection $\nabla$ on $E$ is a
\ib{Hermitian connection} if for sections $s_1, s_2$, we have
\[
dh(s_1,s_2) = h(\nabla s_1, s_2) + h(s_1,\nabla s_2)
\]
\end{defn}
%
In the complex world, we can also impose a holomorphic structure on vector bundles,
and we can also ask for a compatibility condition with the holomorphic structure.
%
\begin{defn}
Let $E \to X$ be a holomorphic structure. A connection $\nabla$ on $E$ is said
to be \ib{compatible with the holomorphic structure} if the $(0,1)$ component
$\nabla^{0,1}$ is equal to $\dbar_E$.
\end{defn}
%
This leads us to a natural question : given a holomorphic vector bundle $E \to X$
equipped with a Hermitian structure $h$, can we find a connection $\nabla$ that
is compatible both with the Hermitian structure and the holomorphic structure?
%
\begin{thm}[\ib{The Chern connection}]
Let $E \to X$ be a Hermitian holomorphic vector bundle. There exists a unique
Hermitian connection $\nabla$, called the \ib{Chern connection} that is compatible
with the holomorphic structure.
\end{thm}
%
\begin{proof}
The proof has a similar flavor to other proofs of existence and uniqueness of global
objects. We use the compatibility conditions to determine local conditions that
the connection must satisfy, and show that it uniquely specifies the connection.
Therefore, the local definitions glue to a global one. In a local trivialization
of $E$, we have that any connection $\nabla$ is of the form $d + A$ for a matrix
valued $1$-form $A$. In addition, we have that the Hermitian structure gives a
Hermitian form $H(x)$ at each point $x$. Let $\set{E_i}$ the local
frame for $E$, where each $E_i$ is just the constant section $x \mapsto e_i$. Then we
get a coordinate representation of the Hermitian form $H$.
$H^i_j(x) = h(E_i,E_j)$. We clearly have that $dE_i = 0$, since all the sections
are constant, so $\nabla E_i = AE_i$. The the condition that $\nabla$ is Hermitian gives
us
\[
dH^i_j  = dH(E_i,E_j) = h(AE_i, E_j) + h(E_i, AE_j)
\]
Writing the matrix $A$ in coordinates $A^i_j$, this gives us
\[
dH^i_j = \overline{A^i_k}h(E_k, E_j) + h(E_i, E_k)A^k_j
= \overline{A^i_k}H^k_j + H^i_kA^k_j
\]
which tells us that we must have that
\[
dH = A^\dagger H + HA
\]
We then consider the restrictions imposed on $\nabla$ by the condition that
$\nabla$ is compatible with the holomorphic structure. Decomposing
$\nabla = \nabla^{1,0} + \nabla^{0,1}$, we want $\nabla^{0,1} = \dbar_E$. In our local
picture, we have that $\nabla = \partial + \dbar + A$, and $\dbar_E$ is just $\dbar$.
Therefore, we must have that $A^{0,1} = 0$, so $A = A^{1,0}$. Decomposing
$d = \partial + \dbar$ and applying it to our Hermitian condition, we find that
\[
\partial H + \dbar H = (A^{1,0})^\dagger H + HA^{1,0}
\]
comparing the bidegrees then gives
\begin{align*}
\partial H &= HA^{1,0}\\
\dbar H &= (A^{1,0})^\dagger H
\end{align*}
Moving $H$ to the other side of the equation and conjugating the equation gives
\[
\partial\overline{H} (\overline{H\inv}) = (A^{1,0})^T = A^T
\]
which gives a unique characterization of the matrix $A$, as well as a local definition,
showing existence and uniqueness of the Chern connection.
\end{proof}
%
Unsurprisingly, if we have a connection compatible with a Hermitian or holomorphic
structure, then the corresponding curvature forms are also compatible.
%
\begin{prop} \enumbreak
\begin{enumerate}
  \item Let $E \to X$ be a Hermitian vector bundle with Hermitian connection $\nabla$.
  Then $F_\nabla$ is locally given by a skew-Hermitian matrix $C$ of $2$-forms.
  Globally, we have
  \[
  h(F_\nabla s_1, s_2) + h(s_1, F_\nabla s_2) = 0
  \]
  \item Let $E \to X$ be a holomorphic vector bundle. Then $F_\nabla$ has
  no $(0,2)$ part, i.e. $F_\nabla = F^{2,0}_\nabla + F^{1,1}_\nabla$.
\end{enumerate}
\end{prop}
%
\begin{proof} \enumbreak
\begin{enumerate}
  \item Locally $E$ is isomorphic to the trivial bundle with the standard Hermitian
  structure, so $\nabla = d + A$ for a skew-Hermitian matrix $A$. Then using
  the local formula
  \[
  F_\nabla = dA + A \wedge A
  \]
  we want to show that $F_\nabla^\dagger = -F_\nabla$. We compute
  \begin{align*}
  F_\nabla^\dagger &= d(A^\dagger) + (A \wedge A)^\dagger \\
  &= d(A^\dagger) - A^\dagger \wedge A^\dagger
  \end{align*}
  The fact that $(A\wedge A)^\dagger = -A^\dagger \wedge A^\dagger$ is slightly tricky,
  and comes from the skew symmetry of the wedge product. Explicitly in components,
  we have
  \begin{align*}
  ((A \wedge A)^\dagger)^i_j &= (\overline{A} \wedge \overline{A})^j_i \\
  &= \overline{A}^j_k \wedge \overline{A}^k_i \\
  &= -\overline{A}^i_k \wedge \overline{A}^k_j \\
  &= - (A^\dagger \wedge A^\dagger)^i_j
  \end{align*}
  We the complete the computation, using the fact that $A$ is skew-Hermitian
  \begin{align*}
  F_\nabla^\dagger &= d(A^\dagger) - A^\dagger \wedge A^\dagger \\
  &= -dA - A \wedge A \\
  &= -F_\nabla
  \end{align*}
  \item Locally, we may write $\nabla = d + A$ Since $\nabla$ is compatible with the
  holomorphic structure, we also have that $A = A^{1,0}$. We have that locally
  \[
  F_\nabla = dA + A \wedge A
  \]
  Splitting $d = \partial + \dbar$, we get
  \[
  F_\nabla = (\partial + \dbar)(A) + A \wedge A = \dbar A + \partial A + A \wedge A
  \]
  and since $A$ is type $(1,0)$, we get that $\dbar A$ is type $(1,1)$ and
  $\partial A + A \wedge A$ is type $(2,0)$.
\end{enumerate}
\end{proof}
%
Putting these together, we get a result on the type of the curvature of a Chern connection.
%
\begin{prop}
Let $\nabla$ be the Chern connection for a Hermitian holomorphic vector bundle
$E \to X$. Then $F_\nabla$ is type $(1,1)$.
\end{prop}
%
\begin{proof}
Since $\nabla$ is Hermitian, we have that $F_\nabla^\dagger = - F_\nabla$.
Since $\nabla$ is compatible with the holomorphic structure, we have that
$F_\nabla = F^{2,0}_\nabla + F^{1,1}_\nabla$. Then consider
\[
F_\nabla^\dagger = (F^{2,0}_\nabla)^\dagger + (F^{1,1}_\nabla)^\dagger
\]
We have that $(F^{2,0}_\nabla)^\dagger$ is type $(0,2)$ and $(F^{1,1}_\nabla)^\dagger$
is type $(1,1)$. Therefore, in order to have $F_\nabla^\dagger = -F_\nabla$, we
must have that $(F^{1,1}_\nabla)^\dagger = -F^{1,1}_\nabla$ and $F^{2,0}_\nabla = 0$
by a simple type check.
\end{proof}
%
This yields an important result regarding the curvature of of Chern connection.
%
\begin{thm}
The curvature form $F_\nabla$ of a Chern connection on a holomorphic vector
bundle $E \to X$ is $\dbar_E$-closed, so it defines a Dolbeault cohomology class
$[F_\nabla] \in H^1(X, \Omega_X \otimes \End(E))$, where $\Omega_X$ denotes the
bundle of holomorphic $1$-forms.
\end{thm}
%
\begin{proof}
Applying the Bianchi Identity to the curvature form $F_\nabla = F_\nabla^{1,1}$,
we find
\begin{align*}
0 = \nabla(F_\nabla)^{1,2} = \dbar_E F_\nabla
\end{align*}
where we use the fact that the induced connection on $\End(E)$ is also
compatible with the holomorphic structure.
\end{proof}
%
This gives some intuition behind the Chern connection. For an arbitrary connection on
a holomorphic vector bundle $E$, we need not have that the curvature is
$\dbar_E$-closed. This shows that asking for a connection to be a Chern connection
is a sort of integrability condition. In addition, it gives a generalization of the
first Chern class for line bundles to holomorphic bundles of arbitrary rank. In
the line bundle case, we did not have to worry about curvature being closed,
but in the higher dimensional case, we have to restrict our attention to Chern
connections.

Given a holomorphic vector bundle $E \to X$, we now get a recipe for producing Dolbeault
cohomology classes in $H^1(X, \Omega_X \otimes \End(E))$. In addition, We can
place a Hermitian structure on $E$, and then compute the curvature of the Chern
connection. Naturally, the next question is whether this cohomology class
depends on the choice of Hermitian structure.
%
\begin{thm}
The cohomology class $[F_\nabla]$ of the curvature of a Chern connection on
a holomorphic vector bundle $E$ is independent on the choice of Hermitian structure.
\end{thm}
%
\begin{proof}
Let $\nabla$ be the Chern connection with respect to some Hermitian structure
on $E$. Then any other connection is of the form $\nabla + A$ with curvature
$F_{\nabla + A} = F_\nabla + \nabla A + A \wedge A$. Then if
$\nabla' = \nabla + A$ is the Chern connection with respect to a different Hermitian
structure, then $A$ must be type $(2,0)$. In addition, we know that both
$F_\nabla$ and $F_{\nabla'} = F_\nabla + \nabla A + A \wedge A$ are both type $(1,1)$.
In order for this to be true, we must have that $\nabla A + A \wedge A$ is
of type $(1,1)$. Since $A \wedge A$ is of type $(2,0)$, we get
\[
(\nabla A + A \wedge A) = (\nabla A \wedge A)^{1,1} = \nabla A
\]
Then since $A$ is type $(1,0)$, the $(1,1)$ part of $A$ is $\nabla^{0,1}A = \dbar_E A$
since $\nabla$ is compatible with the holomorphic structure. Therefore,
we have that $\nabla A + A \wedge A = \dbar_E A$, giving us
\[
F_{\nabla'} = F_\nabla + \dbar_E A
\]
so $[F_{\nabla'}] = [F_\nabla]$.
\end{proof}
%
\begin{defn}
Let $E \to X$ be a holomorphic vector bundle of rank $k$, and
$\mathcal{U} = \set{U_i}$ a good cover of $X$ with local trivializations
$\varphi_i : E\vert_{U_i} \to U_i \times \C^k$ and transition functions
$\psi_{ij} : U_i \cap U_j \to \GL_k\C$. The \ib{Atiyah class} of $E$, denoted
$A(E)$ is the element $A(E) \in H^1(X, \Omega_X \otimes \End E)$ determined by
the cocycle $\sigma_{ij} = \varphi_j\inv \circ (\psi_{ij}\inv d\psi_{ij}) \circ \varphi_j$,
where $\Omega_X$ denotes the bundle of holomorphic $1$-forms.
\end{defn}
%
We first make sense of the formula for the cocycle. A component $\sigma_{ij}$ of
the \v{C}ech cocycle $\sigma$ should be a local section
$\sigma_{ij} : U_i \cap U_j \to \Omega_X \otimes \End E$. What this means
is that we should be able to feed $\sigma_{ij}$ the following data:
\begin{enumerate}
  \item A point $p \in U_i \cap U_j$
  \item A tangent vector $v \in T_pX$
  \item A vector $w \in E_p$ in the fiber
\end{enumerate}
%
and we should expect another vector in $E_p$ to be the output. We have that
$\varphi(p,w)$ should be a pair $(p, w') \in U_i \cap U_j \times \C^k$. Then
since $\psi_{ij} : U_i \cap U_j \to \GL_k\C$, we have that
$d(\psi_{ij})_p : T_pX \to M_n\C$. Therefore, we have that
$d(\psi_{ij})_p(v)$ is a matrix, which we can then apply to $w'$.
Then we can apply the matrix inverse $(\psi_{ij}(p))\inv$ to the result, leaving
us with some other pair $(p,w'') \in U_i \cap U_j \times \C^k$, and then applying the
inverse of the local trivialization $\varphi_j$ to this gives us the desired output.
The fact that this defines a cocycle essentially boils down to the fact that the
$\psi_{ij}$ satisfy the cocycle condition, along with using the chain rule for
the differentials $d\psi_{ij}$.
%
%TODO Chern classes and Chern-Weil
%
%TODO Line bundles and maps to projective space