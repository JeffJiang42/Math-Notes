%
\section{The Adjunction Formula}
%
The relationship between line bundles and the codimension $1$ geometry of $X$ can also
be seen through the Adjunction Formula, which relates the normal bundle of a
hypersurface $S \subset X$ (a line bundle!) with the divisor $[S]$ it defines.
%
\begin{defn}
Let $S \subset X$ be a complex submanifold. Then the \ib{normal bundle} of $S$, denoted
$N_S$, is the quotient vector bundle
\[
N_S \defeq T^{1,0}X\vert_S/T^{1,0}S
\]
Where $T^{1,0}X$ and $T^{1,0}S$ denote the holomorphic tangent bundles of $X$
and $S$ respectively. The \ib{conormal bundle} of $S$ is the dual bundle $N_S^*$.
\end{defn}
%
From standard linear algebra, we can regard elements of $N_S^*$ as tangent covectors
that vanish on the tangent space of $S$. \\

In the case that $S$ is a hypersurface (i.e complex codimension $1$), then we have that
$S$ defines a divisor $[S] \in \mathrm{Div}(X)$. In addition, $N_S$ is a line
bundle. One would hope that the divisor $[S]$ and $N_S$ are closely related by the
correspondence between divisors and line bundles. This turns out to be the case!
%
\begin{prop}
\[
N^*_S \cong \O(-S)\vert_{S}
\]
\end{prop}
%
\begin{proof}
Let $\set{U_\alpha}$ be an open cover of $X$ such that $S$ is locally defined
by holomorphic functions $f_\alpha \in \O_X(U_\alpha)$. Then the line bundle
$\O(S)$ is given by the \v{C}ech cocycle $\set{g_{\alpha\beta}}$ where
\[
g_{\alpha\beta} = \frac{f_\alpha}{f_\beta}
\]
Since $f_\alpha$ vanishes along $V \cap U_\alpha$, the differential $df_\alpha$
vanishes on the tangent vectors of $V \cap U_\alpha$ tangent to $V$, which is to say
that it defines a local section of the conormal bundle $N^*_S$ over the open set
$V \cap U_\alpha$. Furthermore, since $S$ is nonsingular, $df_\alpha$ is nonzero
over every point of $V \cap U_\alpha$. Over $V \cap U_\alpha \cap U_\beta$, we have
\begin{align*}
df_\alpha &= d(g_{\alpha\beta}f_\beta) \\
&= f_\beta dg_{\alpha\beta} + g_{\alpha\beta} df_\beta \\
&= g_{\alpha\beta}df_\beta
\end{align*}
where we use the fact that $f_\beta$ vanishes on $V \cap U_\beta$. Therefore, we have
that we can interpret the $\set{df_\alpha}$ as local sections of
$N_S^* \otimes \O(S)\vert_S$ which glue together to form a global nonvanishing sections
of $N_S^* \otimes \O(S)\vert_S$, which tells us that $N_S^* \otimes \O(S)\vert_S$ is
a trivializable line bundle. Therefore, we have that $N_S^* \cong \O(-S)\vert_S$.
\end{proof}
%
This gives us a particularly nice description of the canonical bundle
$K_S \defeq (T^{n,0}S)^*$ of a hypersurface $S$ in terms of the canonical bundle of $X$.
%
\begin{prop}
Let $S \subset X$ be a hypersurface. Then
\[
K_S \cong (K_X \otimes \O(S))\vert_S
\]
\end{prop}
%
\begin{proof}
By the definition of the normal bundle, we have the exact sequence of vector bundles
over $S$
\[\begin{tikzcd}
0 \ar[r] & T^{1,0}S \ar[r] & T^{1,0}X\vert_S \ar[r] & N_S \ar[r] & 0
\end{tikzcd}\]
Dualizing, this becomes
\[\begin{tikzcd}
0 \ar[r] & N_S^* \ar[r] & (T^{1,0}X\vert_S)^* \ar[r] & (T^{1,0}S)^* \ar[r] & 0
\end{tikzcd}\]
From this, a linear algebra fact gives us that the determinant line of
$(T^{1,0}X\vert_S)^*$ is isomorphic to the tensor product of the determinant
line bundle of the other two, i.e.
\[
K_X\vert_S \cong K_S \otimes N_S^*
\]
where we use the fact that $N_S^*$ is a line bundle to conclude that it is its own
determinant line bundle. Then since $N_S^* \cong \O(-S)$, we get that
\[
K_X\vert_S \otimes \O(S)\vert_S \cong K_S
\]
\end{proof}
%
This is particularly useful for hypersurfaces in $\CP^n$, since we know that
the canonical bundle of $\CP^n$ is $\O(-n-1)$ and we also can work with the
conormal bundle in the form of an ideal sheaf. Indeed, the discussion above
can be done on the levels of sheaves of sections, where we identify
the sheaf of sections of $N^*_S$ with the sheaf $\mathcal{I}/\mathcal{I}^2$,
where $\mathcal{I}$ denotes the ideal sheaf of holomorphic functions vanishing along $S$.
%