\section{\v{C}ech Cohomology and Line Bundles}
%
Let $\mathcal{U} = \set{U_i}$ be a covering of $X$, and $\mathcal{F}$ a sheaf
of abelian groups over $X$. Then with respect to this cover, a \v{C}ech 2-cocycle
$\sigma \in Z^2(\mathcal{U}, \mathcal{F})$ is defined by the equation
\[
0 = (d\sigma)_{ijk} = \sigma_{jk}\vert_{U_i \cap U_j \cap U_k}
- \sigma_{ik}\vert_{U_i \cap U_j \cap U_k} + \sigma_{ij}\vert_{U_i \cap U_j \cap U_k}
\]
Written multiplicatively (and omitting the restriction), this becomes
\[
1 = \sigma_{jk}\sigma_{ik}\inv\sigma_{ij}
\]
which, since the group is abelian, is equivalent to
\[
\sigma_{ik} = \sigma_{ij}\sigma_{jk}
\]
which looks exactly like a cocycle condition for transition functions of a line
bundle. Recall that given a holomorphic line bundle $\pi : L \to X$, we have local
trivializations -- we can find a cover $\mathcal{U} = \set{U_i}$ with maps
$\varphi_i : \pi\inv(U_i) \to U_i \times \C$ such that
\[\begin{tikzcd}
\pi\inv(U_i) \ar[rr, "\varphi_i"] \ar[dr] & & U_i \times \C \ar[dl] \\
& U_i
\end{tikzcd}\]
commutes, where the maps to $U_i$ are the projections. Therefore, if we consider
the map
$\varphi_i \circ \varphi_k\inv : U_i \cap U_j \times \C \to U_i \cap U_j \times \C$,
we have that $\varphi(x, \lambda) = (x, \psi_{ij}(x)(\lambda))$, where the functions
$\psi_{ij} : U_i\cap U_j \to \GL_1\C$ are holomorphic. The $\psi_{ij}$
are called the \ib{transition functions} of the line bundle $L$.
%
\begin{prop}
The transition functions $\psi_{ij}$ satisfy the following conditions
\begin{enumerate}
  \item $\psi_{ij}\psi_{ji} = 1$ (i.e. the constant function $x \mapsto 1$)
  \item $\psi_{ij}\psi_{jk} = \psi_{ik}$.
\end{enumerate}
The second condition is often called the \ib{cocycle condition}, in reference to the
identity we derived above for the defining property of a \v{C}ech cocycle.
\end{prop}
%
\begin{proof}
Consider the map $\varphi_i\circ\varphi_j\inv\circ\varphi_j\circ\varphi_i\inv = \id$.
We compute under the action on a general element $(x,\lambda)$
\[\begin{tikzcd}
(x,\lambda) \ar[r, "\varphi_j\circ\varphi_i\inv"] & (x,\psi_{ji}(x)(\lambda))
\ar[r, "\varphi_i\circ\varphi_j\inv"] & (x,\psi_{ij}(x)\psi_{ji}(x)(\lambda))
\end{tikzcd}\]
Therefore, we have that $\psi_{ij}(x)\psi_{ji}(x) = 1$ for all $x$, showing the
first property. For the second property, we do the same thing. Consider
the function
$\varphi_i\circ\varphi_j\inv\circ\varphi_j\circ\varphi_k = \varphi_i\circ\varphi_k\inv$.
Then fore $(x,\lambda)$, we compute the action of this function to be
\[\begin{tikzcd}
(x,\lambda) \ar[r, "\varphi_j\circ\varphi_k\inv"] & (x,\psi_{jk}(x)(\lambda))
\ar[r, "\varphi_i\circ\varphi_j\inv"] & (x,\psi_{ij}(x)\psi_{jk}(x)(\lambda))
\end{tikzcd}\]
So we get that $\psi_{ij}(x)\psi_{jk}(x) = \psi_{ik}(x)$ for all $x$.
\end{proof}
%
Under the \v{C}ech differential, the image of a \v{C}ech $0$-cochain $\sigma$ is given
by
\[
(d\sigma)_{ij} = \sigma_j - \sigma_i
\]
written multiplicatively, this becomes
\[
(d\sigma)_{ij} = \sigma_j\sigma_i\inv
\]
In the same spirit, we translate this to a statement regarding transition functions
of a line bundle.
%
\begin{prop}
Let $\pi L \to X$ be a holomorphic line bundle where the transition functions
$\psi_{ij}$ with respect to a cover $\set{U_i}$ satisfy the \ib{coboundary condition},
i.e. there exist holomorphic functions $\sigma_i : U_i \to \GL_1\C$ such that
\[
\psi_{ij} = \sigma_j\sigma_i\inv
\]
Then $L$ is a trivial line bundle.
\end{prop}
%
\begin{proof}
It suffices to provide a nonvanishing section $X \to L$. A section $s : X \to L$ is
equivalent to functions $s_i : U_i \to \C$ with the compatibility condition
\[
s_i = \psi_{ij}s_j
\]
define the $s_i$ by $s_i = \sigma_i\inv$. Then they satisfy the compatibility condition,
since
\[
\psi_{ij}\sigma_j = \sigma_j\sigma_i\inv\sigma_j\inv = \sigma_i\inv = s_i
\]
then since the $\sigma_i$ are functions to $\GL_1\C = \C^\times$, they glue to
a global nonvanishing section, so $L$ is isomorphic to the trivial line bundle
$X \times \C$.
\end{proof}
%
Recall that isomorphism classes of line bundles over $X$ form a group
under tensor product, where the inverse of a line bundle $L$ is the dual bundle
$L^*$. Given line bundles $L,L' \to X$ and an open cover $\mathcal{U} = \set{U_i}$ of $X$
in which both $L$ and $L'$ are trivialized over the $U_i$ (for instance, a good cover
of $X$), let $\psi_{ij}$ be the transition functions for $L$ and let $\varphi_{ij}$
be the transition functions for $L'$. Then the transition functions for $L\otimes L'$
are $\psi_{ij}\varphi_{ij}$, and the transition functions for $L^*$ are given
by $\varphi_{ij}\inv$.
%
\begin{thm}
Let $X$ be a complex manifold, and $\O_X$ its sheaf of holomorphic functions.
Then let $\O_X^\times$ be the sheaf of invertible functions, which is a
sheaf of abelian groups under multiplication. Then we have a group isomorphism
\[
\check{H}^1(X,\O_X^\times) \cong \mathrm{Pic}(X)
\]
\end{thm}
%
\begin{proof}
Fix a good cover $\mathcal{U} = \set{U_i}$ for $X$. Since all the sets and their
nonempty intersections are contractible, we have that
$\check{H}^i(U_i, \mathcal{F}) = 0$ for all $i > 0$ where $\mathcal{F}$ is the sheaf
of sections of any line bundle. Since all the $U_i$ are contractible, we also
have that any line bundle over $U_i$ is trivial, so it admits transition
functions $\psi_{ij}$ with respect to this cover. As shown above, the
functions $\psi_{ij}$ exactly define a \v{C}ech $1$-cocycle, and any
\v{C}ech coboundary defines a trivial bundle. In addition, we have that
the the transition functions of a tensor product are exactly the products
of the transition functions. Putting everything together, this tells us that
the mapping $L \mapsto \set{\psi_{ij}}$ sending a line bundle to the cocycles
determined by its transition functions is a bijective group homomorphism.
\end{proof}
%