%
\section{Sheaf Cohomology}
%
Over a complex manifold $X$, we have many different cohomology theories at our disposal:
\begin{enumerate}
  \item The singular cohomology groups $H^i_{\text{sing}}(X,\Z)$.
  \item The de Rham cohomology groups $H^i_{dR}(X)$.
  \item The Dolbeault cohomology groups of a holomorphic vector bundle
  $H^i_{\dbar}(X, E)$.
  \item The \v{C}ech cohomology groups of a sheaf $\check{H}^i(X,\mathcal{F})$.
\end{enumerate}
%
We want to compare these various cohomology theories. To do so, we show that many of
these cohomology theories are computing the same thing : sheaf cohomology.
%
\begin{rem*}
While we might explicitly work with sheaves of abelian groups, the following discussion
is applicable to sheaves of $\mathcal{O}_X$ modules, $C^\infty$ modules, etc.
\end{rem*}
%
\begin{defn}
The \ib{global sections functor} $\Gamma(X,\cdot)$ is a functor
$\mathsf{Ab}(X) \to \mathsf{Ab}$ of the category of sheaves of abelian groups over $X$ to
the category of abelian groups, where given a sheaf of abelian groups $\mathcal{F}$,
\[
\Gamma(X,\mathcal{F}) \defeq \mathcal{F}(X)
\]
\end{defn}
%
The functor is left-exact, i.e. given an exact sequence of sheaves
\[\begin{tikzcd}
0 \ar[r] & \mathcal{E} \ar[r] & \mathcal{F} \ar[r] & \mathcal{G}
\end{tikzcd}\]
we get an exact sequence
\[\begin{tikzcd}
0 \ar[r] & \Gamma(X,\mathcal{E}) \ar[r]
& \Gamma(X,\mathcal{F}) \ar[r] & \Gamma(X,\mathcal{G})
\end{tikzcd}\]
%
However, the functor is not right exact, which is due to the local definitions of
injectivity and surjectivity. The sheaf axiom guarantees that being injective on
stalks implies that a sheaf morphism is injective on sections, but it does not
imply the same thing for surjectivity. As an example, let $\mathcal{Z}^k$ be
the sheaf of closed smooth $k$-forms, and let $\mathcal{B}^k$ be the sheaf of exact
$k$-forms. Then the inclusion $\mathcal{B}^k \hookrightarrow \mathcal{Z}^k$ is surjective,
since every closed $k$-form is exact in a sufficiently small neighborhood. However,
if $H^k_{dR}(X) \neq 0$, then $\Gamma(X,\mathcal{Z}^k) \to \Gamma(X, \mathcal{B}^k)$
is not surjective.
%
\begin{defn}
The \ib{sheaf cohomology groups} $H^i(X, \mathcal{F})$ of a sheaf $\mathcal{F}$ over $X$
are the right derived functors of the global sections functor applied to $\mathcal{F}$
\[
H^i(X,\mathcal{F}) \defeq R^i\Gamma(X,\mathcal{F})
\]
i.e, we take an injective resolution $\mathcal{I}^\bullet = \set{\mathcal{I}^j}$ of
$\mathcal{F}$
\[\begin{tikzcd}
0 \ar[r] & \mathcal{F} \ar[r] & \mathcal{I}^0 \ar[r] & \mathcal{I}^1 \ar[r] & \cdots
\end{tikzcd}\]
and apply $\Gamma(X,\cdot)$ term-wise to the sequence to get
\[\begin{tikzcd}
0 \ar[r] & \Gamma(X,\mathcal{F}) \ar[r] & \Gamma(X,\mathcal{I}^0) \ar[r]
& \Gamma(X,\mathcal{I}^1) \ar[r] & \cdots
\end{tikzcd}\]
and then compute the cohomology of this sequence.
\end{defn}
%
We note that in order for these right derived functors to be defined, there need to
be \emph{enough injectives}. We say that an abelian category $\mathcal{A}$ has enough
injectives if every object $A \in \mathrm{Ob}(\mathcal{A})$ admits an injective
map $A \hookrightarrow I$ into an injective object $I$, where an injective object
is defined to be an object $I$ where given any map $X \to I$ and an injection
$X \hookrightarrow I$, there exists a map $Y \to Q$ such that the following diagram
commutes
\[\begin{tikzcd}
X \ar[dr]\ar[rr,hookrightarrow] & & Y \ar[dl, dashed] \\
& Q
\end{tikzcd}\]
Alternatively, the pullback map $\hom(Y,Q) \to \hom(X,Q)$ is surjective. We will take
the following results on faith:
%
\begin{thm}
The categories $\mathsf{Ab}$, $\mathsf{Ab}(X)$, $\mathsf{Mod}_{\O_X}$,
and $\mathsf{Mod}_{C^\infty}$ have
enough injectives.
\end{thm}
%
Like with the definition of the \v{C}ech cohomology groups as limits, the
definition in terms of injective resolutions is practically useless computationally,
since injective sheaves are hard to write down and difficult to find in the wild.
The name of the game here is to find a nicer class of resolutions we can take.
%
\begin{defn}
Let $F : \mathcal{A} \to \mathcal{B}$ be a left-exact additive functor. An object
$A \in \mathrm{Ob}(\mathcal{A})$ is \ib{acyclic} for the functor $F$ if
$R^iF(A) = 0$ for all $i > 0$.
\end{defn}
%
\begin{prop}
Let $F : \mathcal{A} \to \mathcal{B}$ be an additive functor, and
$A \in \mathrm{Ob}(\mathcal{A})$. Then let $A \to M^\bullet$ be a resolution of $A$
by $F$-acyclic objects. Then $R^iF(A)$ is isomorphic to the $i^{th}$ cohomology
of the complex $F(M^\bullet)$.
\end{prop}
%
\begin{proof}
Let $d^0 : M^0 \to M^1$. Then let $B = \coker (A \to M^0)$. By exactness of
\[\begin{tikzcd}
0 \ar[r] & A \ar[r] & M^0 \ar[r, "d^0"] & M^1
\end{tikzcd}\]
we get a short exact sequence
\[\begin{tikzcd}
0 \ar[r] & A \ar[r] & M^0 \ar[r] & B \ar[r] & 0
\end{tikzcd}\]
where the map $M^0 \to B$ is the map $d^0$, which descends to $B$ by exactness.
We then take injective resolutions $A \to I^\bullet$, $M^0 \to J^\bullet$, and
$B \to K^\bullet$. The maps $A \to M^0$ and $M^0 \to B$, which gives a short exact
sequence of chain maps between resolutions by using the defining property
of injective objects, giving us
\[\begin{tikzcd}
0 \ar[r] & A \ar[r] \ar[d] & M^0 \ar[r] \ar[d] & B \ar[r] \ar[d] & 0 \\
0 \ar[r] & I^0 \ar[r] \ar[d] & J^0 \ar[r] \ar[d] & K^0 \ar[r] \ar[d] & 0 \\
0 \ar[r] & I^1 \ar[r] \ar[d] & J^1 \ar[r] \ar[d] & K^1 \ar[r] \ar[d] & 0 \\
& \vdots  & \vdots  & \vdots &
\end{tikzcd}\]
which gives us a long exact sequence in cohomology. Noting that the cohomology
of the respective sequences are just the right derived functors of $A$, $M^0$,
and $B$, we get the long exact sequence
\[\begin{tikzcd}
0 \ar[r] & R^0F(A) \ar[r] & R^0F(M^0) \ar[r] & R^0F(B) \ar[dll] \\
& R^1F(A) \ar[r] & R^1F(M^0) \ar[r] & R^1F(B) \ar[dll] \\
& R^2F(A) \ar[r] & R^2F(M^0) \ar[r] & R^2F(B) \ar[dll] \\
& ~ & \cdots & \cdots
\end{tikzcd}\]
Then using the fact that $M^0$ is $F$-acyclic, along with the fact that $R^0F = F$, we
get that this long exact sequence is actually
\[\begin{tikzcd}
0 \ar[r] & F(A) \ar[r] & F(M^0) \ar[r] & F(B) \ar[dll] \\
& R^1F(A) \ar[r] & 0 \ar[r] & R^1F(B) \ar[dll] \\
& R^2F(A) \ar[r] & 0 \ar[r] & R^2F(B) \ar[dll] \\
& ~ & \cdots & \cdots
\end{tikzcd}\]
which gives us isomorphisms $R^iF(B) \to R^{i+1}F(A)$ for $i > 0$, as well as
an isomorphism $R^1F(A) \cong \coker F(M^0) \to F(B)$. To compute the
cokernel of that map, we note that $B$ admits the resolution
\[\begin{tikzcd}
0 \ar[r] & B \ar[r] & M^1 \ar[r] & \cdots
\end{tikzcd}\]
where the map $B \to M^1$ is the map induced by $d^0$, using exactness of the acyclic
resolution. Then since $F$ is left-exact, we get that
\[\begin{tikzcd}
0 \ar[r] & F(B) \ar[r] & F(M^1) \ar[r] & \cdots
\end{tikzcd}\]
is exact, giving us that $F(B)$ is isomorphic to the kernel of $F(M^1) \to F(M^2)$. We
then note that the image of $F(M^0) \to F(M^1)$ is contained in $F(B)$, regarded
as the kernel of $F(M^1) \to F(M^2)$. We then get
\[
R^1F(A) \cong \frac{\ker(F(M^1) \to F(M^2))}{\im(F(M^1) \to F(M^2))}
= H^1(F(M^\bullet))
\]
Then to get the isomorphism for $R^2F(A)$, we note that since
$R^1F(B) \cong R^2F(A)$, we can play the same game using the resolution of $B$
to compute $R^1F(B)$ to be $H^2(F(M^\bullet))$, and then inductively repeat
the process with the cokernel of $M^1 \to M^2$ to get $R^3F(A)$ and so on.
\end{proof}
%
Something that is silly to observe, but useful.
%
\begin{prop}
Injective objects are $F$-acyclic.
\end{prop}
%
\begin{proof}
Let $I$ be injective. Then take the injective resolution $0 \to I \to I \to 0$
where the map is the identity map.
\end{proof}
%
Therefore, to compute right derived functors, it suffices to find acyclic resolutions.
This gets us one step closer to finding nice resolutions for computing sheaf cohomology.
%
\begin{defn}
A sheaf $\mathcal{F}$ over $X$ is \ib{flasque} (also called \ib{flabby}) if
the restriction maps are surjective.
\end{defn}
%
Most sheaves in nature aren't flasque. However, flasque sheaves are useful for giving
us acyclic resolutions. To do this, we'll need some lemmas.
%
\begin{lem}
Let
\[\begin{tikzcd}
0 \ar[r] & \mathcal{E} \ar[r, "\alpha"] & \mathcal{F} \ar[r,"\beta"]
& \mathcal{G} \ar[r] & 0
\end{tikzcd}\]
be a short exact sequence where $\mathcal{E}$ is flasque. Then for any open set $U$,
the sequence
\[\begin{tikzcd}
0 \ar[r] & \mathcal{E}(U) \ar[r, "\alpha(U)"] & \mathcal{F}(U) \ar[r, "\beta(U)"]
& \mathcal{G}(U) \ar[r] & 0
\end{tikzcd}\]
is exact.
\end{lem}
%
\begin{proof}
By left-exactness of taking sections, it suffices to show that
$\beta(U) : \mathcal{F}(U) \to \mathcal{G}(U)$ is surjective. Let
$\sigma \in \mathcal{G}(U)$. Since $\beta$ is a surjective sheaf morphism, for any
$x \in U$, the induced map on stalks $\beta_x : \mathcal{F}_x \to \mathcal{G}_x$ is
surjective, which implies that there exists a sufficiently small neighborhood
$V_x \subset U$ of $x$ where
$\beta(V_x) : \mathcal{F}(V_x) \to \mathcal{G}(V_x)$ is surjective,
so we can find $\tau_x \in \mathcal{F}(V_x)$ such that
$\beta(V_x)(\tau) = \sigma\vert_{V_x}$. Then let $y \in U$ such that
$V_x \cap V_y \neq \emptyset$, and let $\tau_y \in \mathcal{F}(V_y)$ such that
$\beta(V_y)(\tau_y) = \sigma\vert_{V_y}$. Then we know that
$\tau_x\vert_{V_x \cap V_y} - \tau_y\vert_{V_x \cap V_y} \in \ker\beta(V_x \cap V_y)$,
so it is the image of an element $k \in \mathcal{E}(V_x \cap V_y)$. Since $\mathcal{E}$
is flasque, we know that we can lift $k$ to an section $\chi_{x,y} \in \mathcal{E}(V_x)$.
Then the element $\tau_x' = \tau_x - \alpha(V_x)(\chi_{x,y})$ still maps to
$\sigma\vert_{V_x}$  under $\beta(V_x)$ by exactness, and restricts to $\tau_y$ on
$V_x \cap V_y$. Therefore, $\tau_x'$ and $\tau_y$ glue to a section over $V_x \cup V_y$
that maps to $\sigma\vert_{V_x \cup V_y}$.  We then find a maximal pair $(W, \tau)$ such
that $\tau \in \mathcal{F}(W)$ and $\beta(W)(\tau) = \sigma\vert_W$. Then we must
necessarily have $W = U$, since otherwise, we can find another open subset $V$ of $U$
and a section $\varphi$ over $V$ mapping to $\sigma\vert_V$, and extend $\tau$ to a
section over $W \cup V$, since if $W \cap V \neq \emptyset$, we can use our above
argument, and otherwise, no work needs to be done. Either way, finding such a $U$
and $\varphi$ contradicts maximality of $(W,\tau)$.
\end{proof}
%
\begin{lem}
Let
\[\begin{tikzcd}
0 \ar[r] & \mathcal{E} \ar[r,"\alpha"] & \mathcal{F} \ar[r,"\beta"]
& \mathcal{G} \ar[r] & 0
\end{tikzcd}\]
be a short exact sequence of sheaves where $\mathcal{E}$ and $\mathcal{F}$ are flasque.
Then $\mathcal{G}$ is flasque.
\end{lem}
%
\begin{proof}
Let $V \subset U$. Then given $\sigma \in \mathcal{G}(V)$, we want to show that
there exits $\widetilde{\sigma} \in \mathcal{G}(U)$ that restricts to $\sigma$. Since
$\mathcal{E}$ is flasque, we have that
\[\begin{tikzcd}
0 \ar[r] & \mathcal{E}(V) \ar[r, "\alpha(V)"] & \mathcal{F}(V) \ar[r,"\beta(V)"]
& \mathcal{G}(V) \ar[r] & 0
\end{tikzcd}\]
is exact, so we can find a section $\tau \in \mathcal{F}(V)$ with
$\beta{V}(\tau) = \sigma$. Then since $\mathcal{F}$ is flasque, this lifts to
an element $\widetilde{\tau} \in \mathcal{F}(U)$. Then taking
$\widetilde{\sigma} = \beta(U)(\widetilde{\tau})$ gives us the desired section, since
the properties of a sheaf morphism implies that the following diagram commutes:
\[\begin{tikzcd}
\mathcal{F}(U) \ar[r, "\beta(U)"] \ar[d] & \mathcal{G}(U) \ar[d] \\
\mathcal{F}(V) \ar[r, "\beta(V)"'] & \mathcal{G}(V)
\end{tikzcd}\]
\end{proof}
%
\begin{prop}
Flasque sheaves are acyclic for the global sections functor $\Gamma(X, \cdot)$.
\end{prop}
%
\begin{proof}
Let $\mathcal{F}$ be a flasque sheaf. We first embed $\mathcal{F}$ into an injective
flasque sheaf $\mathcal{I}$. Since $\mathsf{Ab}$ has enough injectives, we can embed
each stalk $\mathcal{F}_x$ into an injective group $I_x$. Then define the sheaf
$\mathcal{I}$ by
\[
\mathcal{I}(U) = \prod_{x \in U} I_x
\]
and the restriction maps are the projection maps
$\prod_{x \in U} I_x \to \prod_{x \in V} I_x$. Since these maps are surjective,
$\mathcal{I}$ is flasque. In addition, it is injective by construction. Then
$\mathcal{F}$ embeds into $\mathcal{I}$ by composing the inclusions
$\mathcal{F}(U)\hookrightarrow\prod_{x\in U}\mathcal{F}_x\hookrightarrow\prod_{x\in V}I_x$
Then let $\mathcal{G}$ be the cokernel of $\mathcal{F} \hookrightarrow \mathcal{I}$,
giving us the exact sequence of sheaves
\[\begin{tikzcd}
0 \ar[r] & \mathcal{F} \ar[r] & \mathcal{I} \ar[r] & \mathcal{G} \ar[r] & 0
\end{tikzcd}\]
taking resolutions of $\mathcal{F}$, $\mathcal{I}$ and $\mathcal{G}$ gives
us a long exact sequence in cohomology
\[\begin{tikzcd}
0 \ar[r] & H^0(X,\mathcal{F}) \ar[r] & H^0(X,\mathcal{I}) \ar[r]
& H^0(X,\mathcal{G}) \ar[dll] \\
& H^1(X,\mathcal{F}) \ar[r] & H^1(X, \mathcal{I}) \ar[r] & H^1(X,\mathcal{G})\ar[dll] \\
& H^2(X,\mathcal{F}) \ar[r] & H^2(X, \mathcal{I}) \ar[r] & H^2(X,\mathcal{G})\ar[dll] \\
& ~ & \cdots & \cdots
\end{tikzcd}\]
The since $H^0$ is just $\Gamma(X,\cdot)$ and $\mathcal{I}$ is injective, this becomes
\[\begin{tikzcd}
0 \ar[r] & \mathcal{F}(X) \ar[r] & \mathcal{I}(X) \ar[r] & \mathcal{G}(X) \ar[dll] \\
& H^1(X,\mathcal{F}) \ar[r] & 0 \ar[r] & H^1(X,\mathcal{G}) \ar[dll] \\
& H^2(X,\mathcal{F}) \ar[r] & 0 \ar[r] & H^2(X,\mathcal{G}) \ar[dll] \\
& ~ & \cdots & \cdots
\end{tikzcd}\]
Since $\mathcal{F}$ is flasque, we have that
\[\begin{tikzcd}
0 \ar[r] & \mathcal{F}(X) \ar[r] & \mathcal{I}(X) \ar[r] & \mathcal{G}(X) \ar[r] & 0
\end{tikzcd}\]
is exact, so this implies that $H^1(X,\mathcal{F}) = 0$. In addition, for $i > 0$,
we get isomorphisms $H^i(X,\mathcal{G}) \to H^{i+1}(X, \mathcal{F})$. To get that
$H^2(X,\mathcal{F}) = 0$, we note that $\mathcal{G}$ is flasque, so we can
repeat the argument to get that $H^1(X,\mathcal{G}) = H^2(X, \mathcal{F})$, and
then continue for the other cohomology groups.
\end{proof}
%
As a consequence, it suffices to find resolutions by flasque sheaves to compute sheaf
cohomology. The good news here is that every sheaf admits a canonical
acyclic resolution, the \ib{Godement resolution}. For a sheaf $\mathcal{F}$,
let $\mathcal{F}_{\mathrm{God}}$ be the sheaf $U \mapsto \prod_{x \in U}\mathcal{F}_x$,
which is clearly flasque. Then we construct a flasque resolution for
$\mathcal{F}$ as follows : embed
$\mathcal{F} \hookrightarrow \mathcal{F}_{\mathrm{God}}$, and let $\mathcal{G}^1$ be
the cokernel. Then let the next sheaf in the sequence be $\mathcal{G}^1_{\mathrm{God}}$,
where the map $\mathcal{F}_{\mathrm{God}} \to \mathcal{G}^1_{\mathrm{God}}$ is the
quotient map $\mathcal{F}_{\mathrm{God}} \to \mathcal{G}^1$ composed in the the inclusion
$\mathcal{G}^1 \hookrightarrow \mathcal{G}^1_{\mathrm{God}}$. Then take $\mathcal{G}^2$
to be the cokernel of $\mathcal{F}_{\mathrm{God}} \to \mathcal{G}^1_{\mathrm{God}}$,
and continue with $\mathcal{G}^2_{\mathrm{God}}$ and so on. Pictorially, the
Godement construction is constructed as follows:
\[\begin{tikzcd}
0 \ar[r] & \mathcal{F} \ar[r] & \mathcal{F}_{\mathrm{God}} \ar[dr] \ar[r]
& \mathcal{G}^1 \ar[r] \ar[d] & 0 \\
& & & \mathcal{G}^1_{\mathrm{God}} \ar[r] \ar[dr] & \mathcal{G}^2 \ar[r] \ar[d] & 0\\
& & & & \mathcal{G}^2_{\mathrm{God}} \ar[r] \ar[dr] & \mathcal{G}^3 \ar[r] \ar[d] & 0 \\
& & & & & \ddots
\end{tikzcd}\]
%
Once more, this resolution is computationally useless, but it serves the purpose of
showing that every sheaf admits a resolution by flasque sheaves. With this in hand,
we can finally show that we can compute sheaf cohomology with a nice class of sheaves.
%
\begin{defn}
Let $\mathcal{A}$ be a sheaf of rings over $X$ such that $\mathcal{A}$ admits
\ib{partitions of unity}, i.e. for any open cover $\set{U_i}$ of $X$, there exist global
sections $f_i \in \mathcal{A}(X)$ such that $\sum_i f_i = 1$ and $f_i$ is supported
in $U_i$, and where over any particular open set, all but finitely many of the $f_i$
are $0$. Then a sheaf of $\mathcal{A}$-modules is a \ib{fine sheaf}.
\end{defn}
%
These are sheaves we care about, and appear in nature.
%
\begin{exmp}
Let $M$ be a smooth manifold. Then the sheaf $C^\infty$ of smooth functions
admits partitions of unity. Therefore, $\mathsf{Mod}_{C^\infty}$ consists of fine
sheaves.
\end{exmp}
%
The punchline is that fine sheaves are acyclic, which gives us a source of reasonable
sheaves with which to build resolutions.
%
\begin{thm}
Fine sheaves are acyclic with respect to $\Gamma(X,\cdot)$.
\end{thm}
%
\begin{proof}
Let $\mathcal{F}$ be a fine sheaf -- a sheaf of modules over a sheaf of rings
$\mathcal{A}$ that admits partitions of unity. Then by taking the
Godement resolution of $\mathcal{F}$, we get an injective resolution of
flasque $\mathcal{A}$-modules.
\[\begin{tikzcd}
0 \ar[r] & \mathcal{F} \ar[r] & \mathcal{I}^0 \ar[r, "d^0"] & \mathcal{I}^1 \ar[r, "d^1"]
& \cdots
\end{tikzcd}\]
Then since flasque sheaves are acyclic, we get that we can compute the sheaf
cohomology of $\mathcal{F}$ as
\[
H^i(X, \mathcal{F})
= \frac{\ker d^{i+1}(X) : \mathcal{I}^i(X) \to \mathcal{I}^{i+1}(X)}
{\im d^i(X) : \mathcal{I}^{i-1}(X) \to \mathcal{I}^i(X)}
\]
Then let $\alpha \in \ker d^{i+1}(X)$, by exactness, we know that locally, in a
sufficiently small open cover $\set{U_j}$, the map $d^{i}(U_j)$ is surjective,
so we can find $\beta_j \in \mathcal{I}^{i-1}(U_j)$ such that
$d^i(V_j)(\beta_j) = \alpha\vert_{U_j}$. Then $f_i\beta_i$ determines a global
section that is equal to $f_j\beta_j$ on $U_i$ and $0$ elsewhere, and we get that
$\sum_jf_j\beta_j$ maps to $\sum_j f_j\alpha\vert_{U_j} = \alpha$ under $d^i(X)$.
Therefore, the sequence is exact on global sections for $i > 0$, so $\mathcal{F}$
is acyclic.
\end{proof}
%
This tells us that many of the sheaves we know about, like sheaves of smooth sections
of a vector bundle are trivial from the perspective of sheaf cohomology.
%
\section{Extensions and $H^1$}
%
For a sheaf of $\O_X$-modules (where $X$ is a scheme or complex manifold) $\mathcal{F}$,
the first sheaf cohomology group $H^1(X,\mathcal{F})$ has a fairly explicit geometric
interpretation.
%
\begin{defn}
Let $\mathcal{E},\mathcal{G}$ be sheaves of $\O_X$-modules. An \ib{extension} of
$\mathcal{E}$ by $\mathcal{G}$ is a short exact sequence of sheaves
\[\begin{tikzcd}
0 \ar[r] & \mathcal{E} \ar[r] & \mathcal{F} \ar[r] & \mathcal{G} \ar[r] & 0
\end{tikzcd}\]
In other words, it is the data of a sheaf $\mathcal{F}$ containing $\mathcal{E}$
as a subsheaf such that the quotient is $\mathcal{G}$.
\end{defn}
%
\begin{prop}
Let $\mathcal{E}$ be a sheaf of $\mathcal{O}_X$-modules. Then there is a
bijective correspondence
\[
\set{\text{Extensions of }\mathcal{E} \text{ by } \O_X}
\longleftrightarrow H^1(X,\mathcal{E})
\]
\end{prop}
%
\begin{proof}
We provide maps in both directions. In one direction, let
\[\begin{tikzcd}
0 \ar[r] & \mathcal{E} \ar[r] & \mathcal{F} \ar[r] & \O_X \ar[r] & 0
\end{tikzcd}\]
be an extension of $\mathcal{E}$ by $\O_X$. The long exact sequence in sheaf
cohomology gives a boundary map $\delta : H^0(X,\O_X) \to H^1(X,\mathcal{E})$.
So we can associate an extension to the element $\delta(1) \in H^1(X,\mathcal{E})$. \\

In the other direction, let $\beta \in H^1(X,\mathcal{E})$. Fix an injective sheaf
$\mathcal{I}$ with an injection $\mathcal{E} \hookrightarrow \mathcal{I}$, and
let $\mathcal{Q}$ denote the quotient sheaf $\mathcal{I}/\mathcal{E}$. This gives us
us the short exact sequence
\[\begin{tikzcd}
0 \ar[r] & \mathcal{E} \ar[r] & \mathcal{I} \ar[r] & \mathcal{Q} \ar[r] & 0
\end{tikzcd}\]
The long exact sequence in sheaf cohomology gives us
\[\begin{tikzcd}
H^0(X,\mathcal{Q}) \ar[r] & H^1(X,\mathcal{E}) \ar[r] & 0
\end{tikzcd}\]
where we use the fact that $\mathcal{I}$ is injective to deduce that
$H^1(X,\mathcal{I}) = 0$. Therefore, we know that $\beta$ is the image of
some $\widetilde{\beta} \in H^0(X,\mathcal{Q}) = \hom(\O_X,\mathcal{Q})$. Taking
the fiber product, we get the diagram
\[\begin{tikzcd}
0 \ar[r] & \mathcal{E} \ar[d, equal] \ar[r] & \mathcal{I} \times_{\mathcal{Q}} \O_X
\ar[r] \ar[d] & \O_X \ar[r]\ar[d,"\widetilde{\beta}"] & 0 \\
0 \ar[r] & \mathcal{E} \ar[r] & \mathcal{I} \ar[r] & \mathcal{Q} \ar[r] & 0
\end{tikzcd}\]
We can then take the extension to be the upper row.
\end{proof}
%
In the case where $\mathcal{E}$ is locally free (e.g. the sheaf of sections
of a holomorphic vector bundle), then any extension of $\mathcal{E}$ by
$\mathcal{O}_X$ is locally free. For instance, we can fix a good (or affine) cover,
in which case the short exact sequence defined by the extension splits. 
%