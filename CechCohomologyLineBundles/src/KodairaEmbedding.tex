%
\section{The Kodaira Embedding Theorem}
%
Our goal is to prove the following theorem:
%
\begin{thm}[\ib{Kodaira Embedding}]
A holomorphic line bundle $L \to X$ over a compact complex manifold $X$ is ample if and
only if it is positive.
\end{thm}
%
Recall that $L$ is positive if there exists a Hermitian metric on $H$ such
that the curvature of the Chern connection is a K\"ahler form for
some Hermitian metric on $X$. \\

One direction of the implication is easy -- given an ample line bundle $L \to X$,
the very ample line bundle $L^k \to X$ defines an embedding $\varphi_L : X \to \CP^N$
such that $L^k$ is the restriction of $\O(1)$ to $X$. This gives us a Hermitian
metric on $L^k$ such that the curvature of the Chern connection is the
restriction of the Fubini-Study metric to $X$, so $L^k$ is positive. Then since
$c_1(L^k) = kc_1(L)$ and any representative $c_1(L)$ can be represented by the
curvature of some Hermitian metric, we get that $L$ is positive as well. \\

Before we prove the Kodaira embedding theorem, we will identify some criteria
to determine when a basis of sections $s_0,\ldots s_N$ for $\Gamma(X,L)$ determine
an embedding. The first is that the base locus $\mathrm{Bs}(s_0,\ldots s_N)$ of points
where the $s_i$ simultaneously vanish should be empty. One way to phrase
this is to say that the map $\Gamma(X,L) \to L_x$ given by evaluating a section
at $x$ is surjective. If we let $\mathcal{I}_{\set{x}}$ denote the ideal sheaf
of the point $\set{x}$ (i.e. the sheaf of holomorphic functions vanishing at $x$),
the evaluation map comes from the long exact sequence in sheaf cohomology from the
short exact sequence of sheaves
\[\begin{tikzcd}
0 \ar[r] & L \otimes \mathcal{I}_{\set{x}} \ar[r] & L \ar[r] & L_x \ar[r] & 0
\end{tikzcd}\]
where we abuse notation and let $L_x$ denote the skyscraper sheaf with
stalk $L_x$ at $x$. \\

Another requirement for the map to be embedding is that it is injective.
We claim that this means for any distinct points $x_1,x_2 \in X$, there must exist
a section $s$ of $L$ such that $s(x_1) = 0$ and $s(x_2) \neq 0$. To see this,
suppose that there existed so such section. Then for any section $s$ of $L$,
we would either have that $s(x_1) = s(x_2) = 0$ or that both $s(x_1),s_(x_2) \neq 0$,
which implies that in any local trivializations about $x_1$ and $x_2,$ their images
under $\varphi_L$ differ by some nonzero constant, i.e. define the same point in $\CP^N$.
Therefore, the map cannot be injective. This is equivalent
to the map $H^0(X,L) \to L_{x_1} \oplus L_{x_2}$ given by evaluating a section
at $x_1$ and $x_2$ being surjective, since we can rescale sections by a complex
number $\lambda$. This similarly comes from the long exact sequence induced
by the short exact sequence of sheaves
\[\begin{tikzcd}
0 \ar[r] & L \otimes \mathcal{I}_{\set{x_1,x_2}} \ar[r] & L \ar[r]
& L_{x_1} \oplus L_{x_2} \ar[r] & 0
\end{tikzcd}\]
Finally, for the map to be embedding, it must be an immmersion, since $X$ is
compact, it is automatically proper. Therefore, it suffices to verify the the
derivative of the induced map is injective. We want to determine a criterion
for the derivative of the map $\varphi_L : X \to \CP^N$ to be injective at $x \in X$.
Fix a section $s_0 \in \Gamma(X,L)$ such that $s_0(x) \neq 0$, which must exist
since $L$ is basepoint free. We can then complete $s_0$ to a basis
$s_0, \ldots s_N$ of $\Gamma(X,L)$ such that the $s_i(x) = 0$ for each $i$.
Then the map $\varphi_L$ (up to the action of some projective linear automorphism
of $\CP^N$) is given in a local trivialization by $y \mapsto [s_0(y):\ldots : s_N(y)]$
where we abuse notation by letting $s_i$ denote the function $U \to \C$ that $s_i$ is
identified with. Since $s_0(x) \neq 0$, some neighborhood of $x$ is mapped into the
subset of $\CP^N$ where the first coordinate is nonzero, so using the standard
chart on $\CP^N$, the map $\varphi_L$ is represented by the map $U \to \C^N$
given by
\[
y \mapsto \left(\frac{s_1(y)}{s_0(y)},\ldots , \frac{s_N(y)}{s_0(y)}\right)
\]
If we let $t_i$ denote the function $s_i/s_0$, the transpose of the
derivative $d(\varphi_L)_x$ is given by the pullback map sending
$dz_i$ to $dt_i$. Then since the derivative is injective if and only if its
transpose is surjective, our criterion for $\varphi_L$ to be an immersion at
$x$ is for the $dt_i$ to span the holomorphic cotangent space $(T^{1,0}_xX)^*$.
As with the other criteria, we want to relate this to some short exact sequence of
sheaves. Consider the map
$\Gamma(X, L\otimes \mathcal{I}_{\set{x}}) \to L_x \otimes (T^{1,0}_xX)^*$
given in a local trivialization by differentiating a section
$s \in \Gamma(X, L\otimes \mathcal{I}_{\set{x}})$ at $x$.
This is well defined since the term coming from the Leibniz rule vanishes
due to the fact that the section $s$ vanishes at $x$. This
again comes from the long exact sequence coming from the short
exact sequence of sheaves
\[\begin{tikzcd}
0 \ar[r] & L \otimes \mathcal{I}^2_{\set{x}} \ar[r] & L \otimes \mathcal{I}_{\set{x}}
\ar[r] & L_x \otimes (T^{1,0}_xX)^* \ar[r] & 0
\end{tikzcd}\]
where we identify the holomorphic cotangent space with
$\mathcal{I}_{\set{x}}/\mathcal{I}_{\set{x}}^2$ by taking power series expansions in a
neighborhood of $x$. Furthermore, the functions $t_i$ can be interpreted
as sections of $L \otimes \mathcal{I}_{\set{x}}$, and are mapped to
the $dt_i$, interpreting them as sections of $L_x \otimes (T^{1,0}_xX)^*$.
Therefore, injectivity of $d(\varphi_L)_x$ is equivalent to surjectivity of
the map $\Gamma(X, L\otimes \mathcal{I}_{\set{x}}) \to L_x \otimes (T^{1,0}_xX)^*$ \\

Having identified and reformulated the criteria we need to check, we
prove a lemma. Let $\pi : \widehat{X} \to X$ denote the blow up of $X$
at a finite number of distinct points $x_1,\ldots, x_\ell$, with corresponding
exceptional divisors $E_i$.
%
\begin{lem}
Let $L \to X$ be a positive line bundle, and $M \to X$ an arbitrary line bundle.
Then for any positive integers $n_1,\ldots n_\ell$, the line bundle
$\pi^*(L^k \otimes M) \otimes \O(-\sum_i n_iE_i)$ on $\widehat{X}$ is
positive for $k$ sufficiently large.
\end{lem}
%
\begin{proof}
In neighborhoods $U_i$ of the points $x_i$, the blow up map
$\pi : \widehat{X} \to U$ restricted to $\widehat{U_i} \defeq \pi\inv(U_i)$
is given by the blow up of $U_i$ at the point $x_i$, where we can identify
$U_i$ with a subset of $\C^n$, so the $\widehat{U_i}$ is given by the restriction of
the projection $\O(-1) \to \C^n$ to $U_i$. This identifies $\O(E_i)\vert_{U_i}$
with the pullback of $\O(-1)$ along the map $\widehat{U_i} \to \CP^{n-1}$.
pulling back the standard Hermitian form on $\O(1)$, we get a Hermitian metric
on $\O(-E_i)\vert_{U_i}$, and consequently on $\O(-n_iE_i)$ as well. By taking an
arbitrary Hermitian metric on $\O(-\sum_i n_iE_i)$ over the complement of the
$U_i$ and gluing with a partition of unity, we get a Hermitian metric on all
of $\O(-\sum_i n_iE_i)$ where the curvature $F_\nabla$ of the Chern connection
is positive in a neighborhood of the $x_i$. This gives us that
$\pi*(k\alpha + \beta) + (i/2\pi)F_\nabla$ is positive for any real $(1,1)$
forms $\alpha$ and $\beta$ with $\alpha$ positive and $k$ sufficiently large.
Taking $\alpha$ and $\beta$ to be representatives for $c_1(L)$ and $c_1(M)$
respectively, we get the desired result, using the fact that $L$ is positive.
\end{proof}
%
Using the lemma, we are then ready to prove the Kodaira embedding theorem.
%
\begin{proof}[Proof of Kodaira embedding]
As mentioned before, one direction is already proven, so all that is left
to show is that positivity implies ampleness, i.e. given a positive line
bundle $L \to X$, some positive tensor power $L^k$ defines an embedding
$\varphi_{L^k} : X \to \CP^N$. \\

We first show that for $k$ sufficiently large, the line bundle $L^k$ is
basepoint-free, i.e. the map $\Gamma(X,L^k) \to L^k_x$ is surjective for every
$x \in X$. Fix $x \in X$ and let $\pi : \widehat{X} \to X$ denote the blow up
of $X$ along the point $x$, with exceptional divisor $E$. Then consider
the map $\Gamma(X,L^k) \to \Gamma(\widehat{X}, \pi^*L^k)$ given by pulling
back sections along $\pi$. Since $\pi$ is surjective, any two sections whose
pullbacks agree must be the same, so the map is injective. If $\dim_\C X = 1$,
the point $x$ is a divisor, so $\pi : \widehat{X} \to X$ is an isomorphism,
and the map $\Gamma(X,L^k) \to \Gamma(\widehat{X}, \pi^*L^k)$ is bijective.
Otherwise, the point $x$ is of codimension $\geq 2$, so given a section
of $\pi^*L^k$, restricting to the complement of $X$ yields a
section of $\pi^*L^k$ over $\widehat{X} - E$, which is the same as a section of
$L^k$ over $X - \set{x}$, and then Hartog's theorem allows us to extend this
to a section of $L^k$ over all of $X$ that pulls back to our original section of
$\pi^*L^k$, so the map $\Gamma(X,L^k) \to \Gamma(\widehat{X}, \pi^*L^k)$ is
bijective in this case as well. We then note that $\pi^*L^k\vert_E$ is a
trivial bundle, and is isomorphic to the constant sheaf with fiber $L^k_x$,
i.e. $\pi^*L^k\vert_E \cong \O_E \otimes L^k_x$. We want to show that
$x$ is not contained in the base locus of $L^k$, which amounts to showing that
the map $\Gamma(X,L^k) \to L^k_x$ is surjective. From the above observation,
it suffices to show that the map
$\Gamma(\widehat{X},\pi^*L^k) \to \Gamma(\widehat{X},\pi^*L^k\vert_E)$ is surjective.
The restriction map
$\Gamma(\widehat{X},L^k) \to \Gamma(\pi^*L^k\vert_E) \cong \Gamma(E,\O_E) \otimes L^k_x$
is induced by the short exact sequence of sheaves
\[\begin{tikzcd}
0 \ar[r] & \pi^*L^k \otimes \O(-E) \ar[r] & \pi^*L^k \ar[r] & \pi^*L^k\vert_E
\ar[r] & 0
\end{tikzcd}\]
we want to show that the cokernel of
$\Gamma(\widehat{X},L^k) \to \Gamma(\pi^*L^k\vert_E)$ is trivial. By exactness
of the long exact sequence in sheaf cohomology, we have that the cokernel embeds
in $H^1(\widehat{X},\pi^*L^k\otimes\O(-E))$, so it suffices to show that
$H^1(\widehat{X},\pi^*L^k\otimes\O(-E))$ vanishes. We then use the fact that the
canonical bundle of $\widehat{X}$ is given by
$K_{\widehat{X}} \cong \pi^*K_X \otimes \O((n-1)E)$ where $n = \dim_\C X$,
which gives us that
\[
\pi^*L^k\otimes K^*_{\widehat{X}}\otimes\O(-E)\cong\pi^*(L^k\otimes K^*_X)\otimes\O(-nE)
\]
We then apply the previous lemma with $M = K^*_X$ to conclude that
the bundle $\pi^*(L^k\otimes K^*_X)\otimes\O(-nE)$ is positive for $k$ sufficiently
large. Therefore by Kodaira vanishing, taking $k$ sufficently large also
guarantees that
$H^1(\widehat{X},K_{\widehat{X}} \otimes \pi^*(L^k\otimes K^*_X)\otimes\O(-nE)) = 0$.
We then observe that
$K_{\widehat{X}}\otimes\pi^*(L^k\otimes K^*_X)\otimes\O(-nE)\cong\pi^*L^k\otimes\O(-E)$,
which gives us the desired result that $H^1(\widehat{X},\pi^*L^k\otimes\O(-E)) = 0$.
Therefore, $x$ is not contained in the base locus of $L^k$ for $k$ sufficiently large.
We note that the choice of $k$ depends on $x$. To conclude that there exists
a $k$ such that the base locus is empty, we note that we have a map
$\Gamma(X,L^k) \to \Gamma(X,L^{2k})$ given by $s \mapsto s \otimes s$.
This gives us that the base locus of $L^k$ contains the base locus of $L^{2k}$,
since the section $s$ vanishes if and only if $s \otimes s$ vanishes.
This gives us a decreasing chain of closed subsets, which has empty intersection
since for any $x \in X$, we have show that there exists some $k$ such that
$x$ is not a basepoint of $L^k$. Then using compactness of $X$, we have
that we can obtain a global bound $k$ such that no $x$ is a basepoint of $L^k$. \\

To show that the map is injective, we want to show that the map
$\Gamma(X,L) \to \Gamma(L_{\set{x_1}} \oplus L_{\set{x_2}})$ is surjective.
This follows from a near identical argument as above, where we replace
$\widehat{X}$ with the blow up of $X$ at two points. \\

Having shown that the map $\varphi_{L^k}$ (for $k$ sufficiently large)
is well-defined and injective, all that is left to show is that the derivative
at $x$ is injective for all $x \in X$. Again, let $\pi : \widehat{X} \to X$ denote the
blow up of $X$ at $x$, and let $E = \pi\inv(x)$ denote the
To this use the fact that the holomorphic
sections of $\O(-nE)$ can be identified with the holomorphic functions on
$\widehat{X}$ that vanish to $n^{th}$ order along $E$. Then since $E = \pi\inv(x)$,
we get an identification of $\O(-nE)$ with the pullback sheaf
$\pi^*\mathcal{I}_{\set{x}}^n$. Recall we have the exact sequence
\[\begin{tikzcd}
0 \ar[r] & \mathcal{I}^2_{\set{x}} \ar[r] & \mathcal{I}_{\set{x}}
\ar[r] & (T^{1,0}_xX)^* \ar[r] & 0
\end{tikzcd}\]
which, upon twisting with $L^k$, yields the short exact sequence
\[\begin{tikzcd}
0 \ar[r] & L^k \otimes \mathcal{I}^2_{\set{x}} \ar[r] & L^k \otimes \mathcal{I}_{\set{x}}
\ar[r] & L^k_x \otimes (T^{1,0}_xX)^* \ar[r] & 0
\end{tikzcd}\]
where use the fact that $L^k \otimes (T^{1,0}_xX)^* \cong L^k_x \otimes (T^{1,0}_xX)^*$.
Taking global sections yields a map
$\Gamma(X, L^k\otimes\mathcal{I}_{\set{x}}) \to L^k_x \otimes (T^{1,0}_xX)^*$, and
from the preceding discussion, we want to show that this map is surjective.
Like before, we do this by showing the cokernel is trivial. To do this, we first note
that we have identification
$\Gamma(X,L^k\otimes\mathcal{I}_{\set{x}})\cong\Gamma(\widehat{X},\pi^*L^k\otimes\O(-E))$.
Next, we note that the exception divisor $E$ is just the projectivization
of $T_x^{1,0}X$, and $\O_E(-E) \cong \O(1)$. Finally, we note that the global sections
of $\O(1)$ over $\mathbb{P}(T_x^{1,0}X)$ is just $(T^{1,0}_xX)^*$. This identification
yields an isomorphism
$L^k_x \otimes (T^{1,0}_xX)^* \cong L^k_x \otimes \Gamma(E,\O_E(-E))$,
and the taking global sections of the restriction map
$\pi^*L^k\otimes\O(-E) \to L^k_x\O_E(-E)$ yields the following commutative diagram:
\[\begin{tikzcd}
\Gamma(X, L^k \otimes \mathcal{I}_{\set{x}}) \ar[d]\ar[r] &
L^k_x \otimes (T^{1,0}_xX)^* \ar[d]\\
\Gamma(\widehat{X},\pi^*L^k\otimes\O(-E)) \ar[r] & L^k_x \otimes \Gamma(E,\O_E(-E))
\end{tikzcd}\]
where the vertical maps are isomorphisms. Therefore, to conclude that the
cokernel of $\Gamma(X, L^k\otimes\mathcal{I}_{\set{x}}) \to L^k_x \otimes (T^{1,0}_xX)^*$
is trivial it suffices to show that $H^1(\widehat{X}, \pi^*L^k\otimes\O(-2E)) = 0$,
which again follows from Kodaira vanishing when $k$ is sufficient large. Furthermore,
compactness of $X$ again lets us find a global uniform bound $k$ such that
$\varphi_{L^k}$ is injective for every $x \in X$, which concludes the proof,
as we have shown that for $k$ sufficiently large, $\varphi_{L^k}$ is defined,
injective, and an embedding.
\end{proof}
%