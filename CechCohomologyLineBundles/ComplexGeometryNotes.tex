\documentclass[psamsfonts, 12pt]{amsart}
%
%-------Packages---------
%
\usepackage[h margin=1 in, v margin=1 in]{geometry}
\usepackage{amssymb,amsfonts}
\usepackage[all,arc]{xy}
\usepackage{tikz-cd}
\usepackage{enumerate}
\usepackage{mathrsfs}
\usepackage{amsthm}
\usepackage{mathpazo}
\usepackage{float}
\usepackage[backend=biber]{biblatex}
\addbibresource{bibliography.bib}
%\usepackage{charter} %another font
%\usepackage{eulervm} %Vakil font
\usepackage{yfonts}
\usepackage{mathtools}
\usepackage{enumitem}
\usepackage{mathrsfs}
\usepackage{fourier-orns}
\usepackage[all]{xy}
\usepackage{hyperref}
\usepackage{url}
\usepackage{mathtools}
\usepackage{graphicx}
\usepackage{pdfsync}
\usepackage{mathdots}
\usepackage{calligra}
\usepackage{import}
\usepackage{xifthen}
\usepackage{pdfpages}
\usepackage{transparent}

\usepackage{tgpagella}
\usepackage[T1]{fontenc}
%
\usepackage{listings}
\usepackage{color}

\definecolor{dkgreen}{rgb}{0,0.6,0}
\definecolor{gray}{rgb}{0.5,0.5,0.5}
\definecolor{mauve}{rgb}{0.58,0,0.82}

\lstset{frame=tb,
  language=Matlab,
  aboveskip=3mm,
  belowskip=3mm,
  showstringspaces=false,
  columns=flexible,
  basicstyle={\small\ttfamily},
  numbers=none,
  numberstyle=\tiny\color{gray},
  keywordstyle=\color{blue},
  commentstyle=\color{dkgreen},
  stringstyle=\color{mauve},
  breaklines=true,
  breakatwhitespace=true,
  tabsize=3
  }
%
%--------Theorem Environments--------
%
\newtheorem{thm}{Theorem}[section]
\newtheorem*{thm*}{Theorem}
\newtheorem{cor}[thm]{Corollary}
\newtheorem{prop}[thm]{Proposition}
\newtheorem{lem}[thm]{Lemma}
\newtheorem*{lem*}{Lemma}
\newtheorem{conj}[thm]{Conjecture}
\newtheorem{quest}[thm]{Question}
%
\theoremstyle{definition}
\newtheorem{defn}[thm]{Definition}
\newtheorem*{defn*}{Definition}
\newtheorem{defns}[thm]{Definitions}
\newtheorem{con}[thm]{Construction}
\newtheorem{exmp}[thm]{Example}
\newtheorem{exmps}[thm]{Examples}
\newtheorem{notn}[thm]{Notation}
\newtheorem{notns}[thm]{Notations}
\newtheorem{addm}[thm]{Addendum}
\newtheorem{exer}[thm]{Exercise}
%
\theoremstyle{remark}
\newtheorem{rem}[thm]{Remark}
\newtheorem*{claim}{Claim}
\newtheorem*{aside*}{Aside}
\newtheorem*{rem*}{Remark}
\newtheorem*{hint*}{Hint}
\newtheorem*{note}{Note}
\newtheorem{rems}[thm]{Remarks}
\newtheorem{warn}[thm]{Warning}
\newtheorem{sch}[thm]{Scholium}
%
%--------Macros--------
\renewcommand{\qedsymbol}{$\blacksquare$}
\renewcommand{\sl}{\mathfrak{sl}}
\newcommand{\Bord}{\mathsf{Bord}}
\renewcommand{\hom}{\mathsf{Hom}}
\renewcommand{\emptyset}{\varnothing}
\renewcommand{\O}{\mathcal{O}}
\newcommand{\R}{\mathbb{R}}
\newcommand{\ib}[1]{\textbf{\textit{#1}}}
\newcommand{\Q}{\mathbb{Q}}
\newcommand{\Z}{\mathbb{Z}}
\newcommand{\N}{\mathbb{N}}
\newcommand{\C}{\mathbb{C}}
\newcommand{\A}{\mathbb{A}}
\newcommand{\F}{\mathbb{F}}
\newcommand{\M}{\mathcal{M}}
\newcommand{\dbar}{\overline{\partial}}
\newcommand{\zbar}{\overline{z}}
\renewcommand{\S}{\mathbb{S}}
\newcommand{\V}{\vec{v}}
\newcommand{\RP}{\mathbb{RP}}
\newcommand{\CP}{\mathbb{CP}}
\newcommand{\B}{\mathcal{B}}
\newcommand{\GL}{\mathrm{GL}}
\newcommand{\SL}{\mathrm{SL}}
\newcommand{\SP}{\mathrm{SP}}
\newcommand{\SO}{\mathrm{SO}}
\newcommand{\SU}{\mathrm{SU}}
\newcommand{\gl}{\mathfrak{gl}}
\newcommand{\g}{\mathfrak{g}}
\newcommand{\Bun}{\mathsf{Bun}}
\newcommand{\inv}{^{-1}}
\newcommand{\bra}[2]{ \left[ #1, #2 \right] }
\newcommand{\set}[1]{\left\lbrace #1 \right\rbrace}
\newcommand{\abs}[1]{\left\lvert#1\right\rvert}
\newcommand{\norm}[1]{\left\lVert#1\right\rVert}
\newcommand{\transv}{\mathrel{\text{\tpitchfork}}}
\newcommand{\defeq}{\vcentcolon=}
\newcommand{\enumbreak}{\ \\ \vspace{-\baselineskip}}
\let\oldexists\exists
\renewcommand\exists{\oldexists~}
\let\oldL\L
\renewcommand\L{\mathfrak{L}}
\makeatletter
\newcommand{\incfig}[2]{%
    \fontsize{48pt}{50pt}\selectfont
    \def\svgwidth{\columnwidth}
    \scalebox{#2}{\input{#1.pdf_tex}}
}
%
\newcommand{\tpitchfork}{%
  \vbox{
    \baselineskip\z@skip
    \lineskip-.52ex
    \lineskiplimit\maxdimen
    \m@th
    \ialign{##\crcr\hidewidth\smash{$-$}\hidewidth\crcr$\pitchfork$\crcr}
  }%
}
\makeatother
\newcommand{\bd}{\partial}
\newcommand{\lang}{\begin{picture}(5,7)
\put(1.1,2.5){\rotatebox{45}{\line(1,0){6.0}}}
\put(1.1,2.5){\rotatebox{315}{\line(1,0){6.0}}}
\end{picture}}
\newcommand{\rang}{\begin{picture}(5,7)
\put(.1,2.5){\rotatebox{135}{\line(1,0){6.0}}}
\put(.1,2.5){\rotatebox{225}{\line(1,0){6.0}}}
\end{picture}}
\DeclareMathOperator{\id}{id}
\DeclareMathOperator{\im}{Im}
\DeclareMathOperator{\codim}{codim}
\DeclareMathOperator{\coker}{coker}
\DeclareMathOperator{\supp}{supp}
\DeclareMathOperator{\inter}{Int}
\DeclareMathOperator{\sign}{sign}
\DeclareMathOperator{\sgn}{sgn}
\DeclareMathOperator{\indx}{ind}
\DeclareMathOperator{\alt}{Alt}
\DeclareMathOperator{\Aut}{Aut}
\DeclareMathOperator{\trace}{trace}
\DeclareMathOperator{\ad}{ad}
\DeclareMathOperator{\End}{End}
\DeclareMathOperator{\Ad}{Ad}
\DeclareMathOperator{\Lie}{Lie}
\DeclareMathOperator{\spn}{span}
\DeclareMathOperator{\dv}{div}
\DeclareMathOperator{\grad}{grad}
\DeclareMathOperator{\Sym}{Sym}
\DeclareMathOperator{\sheafhom}{\mathscr{H}\text{\kern -3pt {\calligra\large om}}\,}
\newcommand*\myhrulefill{%
   \leavevmode\leaders\hrule depth-2pt height 2.4pt\hfill\kern0pt}
\newcommand\niceending[1]{%
  \begin{center}%
    \LARGE \myhrulefill \hspace{0.2cm} #1 \hspace{0.2cm} \myhrulefill%
  \end{center}}
\newcommand*\sectionend{\niceending{\decofourleft\decofourright}}
\newcommand*\subsectionend{\niceending{\decosix}}
\def\upint{\mathchoice%
    {\mkern13mu\overline{\vphantom{\intop}\mkern7mu}\mkern-20mu}%
    {\mkern7mu\overline{\vphantom{\intop}\mkern7mu}\mkern-14mu}%
    {\mkern7mu\overline{\vphantom{\intop}\mkern7mu}\mkern-14mu}%
    {\mkern7mu\overline{\vphantom{\intop}\mkern7mu}\mkern-14mu}%
  \int}
\def\lowint{\mkern3mu\underline{\vphantom{\intop}\mkern7mu}\mkern-10mu\int}
%
%--------Hypersetup--------
%
\hypersetup{
    colorlinks,
    citecolor=black,
    filecolor=black,
    linkcolor=blue,
    urlcolor=blacksquare
}
%
%--------Solution--------
%
\newenvironment{solution}
  {\begin{proof}[Solution]}
  {\end{proof}}
%
%--------Graphics--------
%
%\graphicspath{ {images/} }
%
\begin{document}
%
\author{Jeffrey Jiang}
%
\title{Sheaf Cohomology, Line Bundles, and Divisors}
%
\maketitle
%
\tableofcontents
%
\section{\v{C}ech Cohomology}
%
For the most part, we will consider ``sufficiently nice" topological spaces.
For the most part, think of a space $X$ as a (smooth, complex) manifold, an analytic
space, or a quasicompact separated scheme if you're feeling adventurous.
%
\begin{defn}
Let $X$ be a space and $\mathcal{F}$ a sheaf of abelian groups over $X$. Let
$\mathcal{U} = \set{U_i}_{i \in \N}$ be a countable open cover of $X$ that is locally
finite, i.e. for any $x \in X$, only finitely many $U_i$ contain $x$. For
$I = \set{i_1, \ldots, i_k}$, let
\[
U_I \defeq \bigcap_{i \in I} U_i
\]
Then define the \ib{\v{C}ech cochain groups} of $\mathcal{F}$ for the cover
$\mathcal{U}$ by
\begin{align*}
C^k(\mathcal{U}, \mathcal{F}) \defeq \prod_{|I| = k+1} \mathcal{F}(U_I)
\end{align*}
an element of $C^k(\mathcal{U},\mathcal{F})$ is called a \ib{\v{C}ech cochain}.
For a $k$-cochain $\sigma$, and $I = \set{i_0,\ldots i_k}$, we denote the
component of $\sigma$ over $U_I$ as $\sigma_I$ or $\sigma_{i_0,\ldots i_k}$.
\end{defn}
%
The \v{C}ech cochain groups are equipped with a differential
$d : C^k(\mathcal{U}, \mathcal{F}) \to C^{k+1}(\mathcal{U}, \mathcal{F})$
where for $\sigma \in C^k(\mathcal{U}, \mathcal{F})$, the ${i_0, \ldots i_{k+1}}$
component of $d\sigma$ is given by
\[
(d\sigma)_{i_0, \ldots i_{k+1}}
= \sum_{j = 1}^{p+1}(-1)^j\sigma_{i_0, \ldots, \widehat{i_j}, \ldots, i_{k+1}}
\vert_{U_0 \cap \cdots \cap U_{k+1}}
\]
where $\widehat{i_j}$ denotes that $i_j$ is missing. We have that $d^2 = 0$ for a
similar reason that $d^2 = 0$ for singular cohomology, you get repeats of
terms with opposite signs due to the omitted index. We denote the kernel of
$d : C^i(\mathcal{U}, \mathcal{F}) \to C^{i+1}(\mathcal{U}, \mathcal{F})$ as
$Z^i(\mathcal{U}, \mathcal{F})$, and we say that the element are $i$-cocycles.
We denote the image of
$d : C^{i-1}(\mathcal{U}, \mathcal{F}) \to C^i(\mathcal{U}, \mathcal{F})$ as
$B^i(\mathcal{U}, \mathcal{F})$, and we call the elements $i$-coboundaries.
%
\begin{defn}
The \ib{\v{C}ech cohomology groups} of $\mathcal{F}$ with respect to the cover
$\mathcal{U}$, denoted $\check{H}^i(\mathcal{U}, \mathcal{F})$ is the cohomology of
the \v{C}ech complex
\[\begin{tikzcd}
0 \ar[r, "d"] & C^0(\mathcal{U}, \mathcal{F}) \ar[r, "d"] &
C^1(\mathcal{U}, \mathcal{F}) \ar[r, "d"]
& \cdots
\end{tikzcd}\]
i.e. we have
\[
\check{H}^i(\mathcal{U}, \mathcal{F}) \defeq
\frac{Z^i(\mathcal{U}, \mathcal{F})}{B^i(\mathcal{U}, \mathcal{F})}
\]
\end{defn}
%
\begin{defn}
Given an open cover $\mathcal{U} = \set{U_i}$, a \ib{refinement} of $\mathcal{U}$
is another open cover $\mathcal{V} = \set{V_j}$ such that every $V_j$ is contained
in some $U_i$. If $\mathcal{V}$ is a refinement of $\mathcal{U}$, we write
$\mathcal{V} < \mathcal{U}$.
\end{defn}
%
If $\mathcal{V} < \mathcal{U}$, then we know we can find a map
$\varphi : \N \to \N$ such that $V_i \subset U_\varphi(i)$. Consequently, we
can restrict sections over $U_i$ to sections over $V_{\varphi(i)}$, so this
induces a chain map
$\rho_\varphi : C^k(\mathcal{U}, \mathcal{F}) \to C^k(\mathcal{V},\mathcal{F})$
where
\[
(\rho_\varphi(\sigma))_{i_0,\ldots, i_k}
= \sigma_{\varphi(i_0),\ldots \varphi(i_k)}\vert_{U_{i_0} \cap \cdots \cap U_{i_k}}
\]
This map commutes with the differentials for $C^\bullet(\mathcal{U},\mathcal{F})$
and $C^\bullet(\mathcal{V},\mathcal{F})$, so it descends to homomorphisms
$\check{H}^i(\mathcal{U},\mathcal{F}) \to \check{H}^i(\mathcal{V}, \mathcal{F})$.
It can be shown that a different choice of chain map $\rho_\psi$ for
$\psi : \N \to \N$ is chain homotopic to $\rho_\varphi$, so the induced maps
on cohomology are independent of our choice of $\varphi$.
%
\begin{defn}
The \ib{\v{C}ech cohomology groups} of a sheaf $\mathcal{F}$ over $X$ is the limit over
refinements
\[
\check{H}^i(X, \mathcal{F})
\defeq \lim_{\mathcal{V} < \mathcal{U}} \check{H}^i(\mathcal{V}, \mathcal{F})
\]
i.e. the quotient disjoint union
$\amalg_\mathcal{U}\check{H}^i(\mathcal{U}, \mathcal{F}$
over all refinements, where we identify $\sigma \in \check{H}^i(\mathcal{U}, \mathcal{F}$
and $\tau \in \check{H}^i(\mathcal{V}, \mathcal{F})$ if
$\mathcal{V}$ refines $\mathcal{U}$ and $\rho(\sigma) = \tau$ under the map induced on
cohomology by the refinement.
\end{defn}
%
This definition of the \v{C}ech cohomology groups is essentially useless for
computation. It's true power comes from the following theorem, due to Leray.
%
\begin{thm}[\ib{Leray}]
Suppose $\mathcal{U} = \set{U_i}$ is an open cover of $X$ that is \ib{acyclic} with
respect to the sheaf $\mathcal{F}$, i.e. for $|I| > 1$ and any $i$,
\[
\check{H}^i(U_I, \mathcal{F}) = 0
\]
Then
\[
\check{H}^i(\mathcal{U}, \mathcal{F}) = \check{H}^i(X, \mathcal{F})
\]
\end{thm}
%
The intuition to keep in mind for an acyclic cover is the notion of a good cover
in differential geometry. On a smooth manifold $M$, there exists a covering of
$M$ by open sets $\set{U_i}$ such that any nonempty intersections are contractible,
which is done by taking geodesic balls around each point in $M$. Since the
homotopical information on each $U_i$ is trivial, the only nontrivial topological
information in the cohomology of $M$ comes from how the sets are glued together to
form $M$. As with exact sequences of chain complexes, short exact sequences
of sheaves give long exact sequences in sheaf cohomology.
%
\begin{thm}
Let
\[\begin{tikzcd}
0 \ar[r] & \mathcal{E} \ar[r, "\alpha"] & \mathcal{F} \ar[r, "\beta"] & \mathcal{G}
\ar[r] & 0
\end{tikzcd}\]
be an exact sequence of sheaves over $X$. Then this induces a long exact sequence in
cohomology:
\[\begin{tikzcd}
0 \ar[r] & \check{H}^0(X,\mathcal{E}) \ar[r, "\alpha^*"]
& \check{H}^0(X,\mathcal{F}) \ar[r, "\beta^*"] &
\check{H}^0(X, \mathcal{G}) \ar[dll, "\delta"'] \\
& \check{H}^1(X, \mathcal{E}) \ar[r, "\alpha^*"]
& \check{H}^1(X, \mathcal{F})\ar[r, "\beta^*"]
& \check{H}^1(X, \mathcal{G}) \ar[dll, "\delta"'] \\
& ~ & \cdots & \cdots
\end{tikzcd}\]
\end{thm}
%
\begin{proof}
We first define the maps
$\alpha^* : \check{H}^i(X,\mathcal{E}) \to \check{H}^i(X, \mathcal{F})$
and $\beta^* : H^i(X, \mathcal{F}) \to \check{H}^i(X, \mathcal{G})$, and will then define
the connecting homomorphism
$\delta : \check{H}^i(X,\mathcal{E}) \to \check{H}^{i+1}(X, \mathcal{G})$. Given
an open cover $\mathcal{U} = \set{U_i}$ of $X$, the sheaf morphism $\alpha$ gives for
each open set $U_i$ a homomorphism $\alpha(U_i) : \mathcal{E}(U_i) \to \mathcal{F}(U_i)$,
which induces a chain map
$C^\bullet(\mathcal{U},\mathcal{F}) \to C^\bullet(\mathcal{U}, \mathcal{G}))$. Since
the maps $\alpha(U_i)$ commute with restriction maps, this chain map commutes with
the differentials, so it descends to a map on cohomology
$\check{H}^\bullet(\mathcal{U},\mathcal{F})\to\check{H}^\bullet(\mathcal{U},\mathcal{F})$,
which, after taking the limit over refinements or choosing $\mathcal{U}$ to be
simultaneously acyclic for $\mathcal{E}$ and $\mathcal{F}$, gives us the induced
map $\alpha^* : \check{H}^\bullet(X,\mathcal{F})\to\check{H}^\bullet(X,\mathcal{F})$.
The map $\beta^*$ is defined similarly. \\

The construction of the connecting homomorphism
mirrors the construction for singular (or de Rham) cohomology. We represent
an element of $\check{H}^i(X,\mathcal{G})$ with a cocycle
$\sigma \in Z^i(\mathcal{U}, \mathcal{G})$ with respect to some open cover $\mathcal{U}$.
By surjectivity of $\beta$, by potentially passing to a refinement, we can
write $\sigma = \beta(\tau)$ for some $\tau \in C^i(\mathcal{U}, \mathcal{F})$. Then
since the induced map on chains commutes with the differentials, we have that
$d\beta(\tau) = \beta(d\tau) = 0$, since $\tau$ is a cocycle. Therefore,
$d\tau$ is in the kernel of $\beta^*$, so we can write $d\tau = \alpha(\eta)$
for some $\eta \in C^{i+1}(\mathcal{U}, \mathcal{F})$. We then define
$\delta(\sigma)$ to be the class of $\eta$ in the limit. We note that
this is independent of our choice of $\tau$, since any other choice of
preimage of $\sigma$ differs by an element of the form $\alpha(e)$ for some cocycle
$e$ by exactness. Then since $\alpha$ commutes with the differentials, we get
$d(\tau + \alpha(e)) = d\tau + d\alpha(e) = d\tau + \alpha(de) = d\tau$.
\end{proof}
%
We make one observation about \v{C}ech cohomology
%
\begin{prop}
The $0^{th}$ \v{C}ech cohomology group is isomorphic to the space of global sections,
i.e.
\[
\check{H}^0(\mathcal{U}, \mathcal{F}) \cong \Gamma(X, \mathcal{F})
\]
\end{prop}
%
\begin{proof}
The $0^{th}$ cohomology is just the kernel of the map
$d : C^0(\mathcal{U},\mathcal{F}) \to C^1(\mathcal{U}, \mathcal{F})$.
For a $0$-cochain $\sigma$, we have that
\[
(d\sigma)_{ij} = \sigma_j\vert_{U_i \cap U_j} - \sigma_i\vert_{U_i \cap U_j}
\]
We then claim that the map $\Gamma(X, \mathcal{F}) \to \ker d$ sending a section
$\sigma$ to the cocycle $\tilde{\sigma}$ defined by
\[
\tilde{\sigma}_i = \sigma\vert_{U_i}
\]
is bijective. It is surjective, since any $0$-cocycle contained in the kernel
is a collection of local sections that agrees on intersections, which is exactly
a global section of $\mathcal{F}$. In addition, it is injective, since if
a section restricts to $0$ on every open set, it is the zero section.
\end{proof}
%TODO Functoriality of Cech cohomology
%
%
\section{\v{C}ech Cohomology on $\CP^n$}
%
We now compute \v{C}ech cohomology for various sheaves over $\CP^n$. The main objects
of interest are the line bundles $\O(k)$. We have that $\CP^n$ admits a nice cover
$\mathcal{U} = \set{U_i}$ where $U_i$ is the open set where the coordinate $z_i$ does not
vanish. We will soon show that this covering is acyclic for the structure sheaf
$\O_{\CP^n}$, and since all line bundles are trivial on the open sets in this cover, this
means that the covering will acyclic for any line bundle $\O(k)$. So it suffices to
compute \v{C}ech cohomology with respect to this cover. In particular, we note that for
any line bundle $\O(k) \to \CP^n$, the transition functions $\psi_{ij}$ for $\O(k)$
are $\psi_{ij}(\ell) = (z_j/z_i)^k$. We already know one cohomology group:
%
\begin{thm}
The $0^{th}$ cohomology group $\check{H}^0(\CP^n, \O(k))$ is isomorphic to the space
$\C[x_0, \ldots x_n]_k$ of homogeneous degree $k$ polynomials in the variables
$x_0, \ldots , x_n$.
\end{thm}
%
We compute the rest of the cohomology now, which we do in stages.
%
\begin{thm}
For any $i > n$, we have $\check{H}^n(\mathcal{U},\O(k)) = 0$.
\end{thm}
%
\begin{proof}
The cover of $\CP^i$ by the $U_i$ has cardinality $n+1$. Therefore,
$C^{n+1}(\mathcal{U}, \O(k)) = 0$.
\end{proof}
%
To compute the rest of the cohomology groups, we first prove a lemma characterizing
local sections of $\O(k)$.
%
\begin{lem}
Let $\pi : \C^{n-1} - \set{0} \to \CP^n$ be the usual projection sending
$z \in C^{n-1}$ to $\mathrm{span}\set{z}$. Then the space of sections
$\O(k)(U)$ is isomorphic to the space of homogeneous of degree $k$  holomorphic functions
$f : \pi\inv(U) \to \C$ i.e.
\[
f(tz_0, \ldots, tz_n) = t^kf(z_0,\ldots, z_n)
\]
\end{lem}
%
\begin{proof}
Let $\sigma \in \O(k)(U)$ be a section. Then the set $\set{U \cap U_i}$ is an
open cover of $U$, so $\sigma$ is determined by its restrictions
$\sigma_i \defeq \sigma_i\vert_{U \cap U_i}$. Since the bundle $\O(d)$ is trivial
over the $U_i$, the local sections $\sigma_i$ can be identified with holomorphic
functions $U \cap U_i \to \C$ with the compatibility condition
\[
\sigma_i([z_0 :\ldots : z_n])
= \left(\frac{z_j}{z_i}\right)^k\sigma_j([z_0: \ldots : z_n])
\]
We then give maps in both directions. Given a section
$\sigma \in \O(k)(U)$, define the function $f_\sigma$ by
\[
f_\sigma(z_0,\ldots z_n) = z_i^k\sigma_i(\pi(z_0,\ldots z_n))
\]
We must verify that this is well-defined, i.e. it is independent of our choice of $i$.
We compute
\begin{align*}
f_\sigma(z_0,\ldots z_n)
&=  z_i^k\sigma_i(\pi(z_0,\ldots z_n)) \\
&=  \left(\frac{z_j}{z_i}\right)^kz_i^k\sigma_j(\pi(z_0,\ldots z_n)) \\
&= z_j^k\sigma_k(\pi(z_0, \ldots z_n)) \\
\end{align*}
So this determines a well defined function on $\pi\inv(U)$. In addition, it is
visibly homogeneous of degree $k$, since the $\sigma_i$ are constant on lines and
$z_i^k$ is homogeneous of degree $k$. To show this is an isomorphism, we provide
an inverse. Given a homogeneous function $f$ of degree $k$ on $\pi\inv(U)$, define
the section $\sigma_f$ locally by
\[
(\sigma_f)_i([z_0: \ldots :z_n]) = \frac{f(z_0,\ldots z_n)}{z_i^k}
\]
then to show that this defines a section, we must show that they agree on intersections
using the transition functions. We compute
\[
\left(\frac{z_i}{z_j}\right)^k(\sigma_f)_i\vert_{U \cap U_i \cap U_j}([z_0: \ldots : z_n])
= \left(\frac{z_i}{z_j}\right)^k\frac{f(z_0,\ldots, z_n)}{z_i^k}
= \frac{f(z_0, \ldots z_n)}{z_j^k}
= (\sigma_f)_j
\]
The two mappings provided are visibly inverses, since one is essentially multiplication
by $z_j^k$ and the other is essentially division by $z_j^k$.
\end{proof}
%
Over intersections of the distinguished open sets $U_i$, the sections have
a particularly nice form. Under the projection $\pi : \C^{n+1}-\set{0} \to \CP^n$,
the preimage of $U_I$ for $I = \set{i_0, \ldots i_d}$ is just $\C^{n+1}$ minus
the coordinate axes $z_{i_j} = 0$. By taking power series, a holomorphic
function on $\pi\inv(U_I)$ is given by Laurent series where the $z_{i_j}$ can appear
in negative degree. Being homogeneous of degree $d$ implies that all the
terms in the series expansion must be homogeneous of degree $k$, where the
degree of $(z_k)^a/(z_{i_j})^b$ is $a - b$. Consequently, all such holomorphic
functions must be polynomials in $\C[z_0, \ldots z_n, z_{i_0}\inv, \ldots z_{i_d}\inv]$
of degree $k$.
% TODO Compute cohomology
\iffalse
We now compute the $n^{th}$ cohomology groups.
%
\begin{thm}
\[
\check{H}^i(\mathcal{U}, \O(k)) = \begin{cases}
\C[z_0, \ldots z_n]_{-k-n-1} & -k-n-1 \geq 0 \\
0 & \text{otherwise}
\end{cases}
\]
\end{thm}
%
\begin{proof}
Since $C^{n+1}(\mathcal{U}, \O(k)) = 0$, we have that
$\check{H}^n(\mathcal{U},\O(k))$ is just the cokernel of the differential
$d : C^{n-1}(\mathcal{U}, \O(k)) \to C^n(\mathcal{U}, \O(k))$.
\end{proof}
\fi
%
%TODO Prove that the covering is acyclic
%
%
\section{\v{C}ech Cohomology and Line Bundles}
%
Let $\mathcal{U} = \set{U_i}$ be a covering of $X$, and $\mathcal{F}$ a sheaf
of abelian groups over $X$. Then with respect to this cover, a \v{C}ech 2-cocycle
$\sigma \in Z^2(\mathcal{U}, \mathcal{F})$ is defined by the equation
\[
0 = (d\sigma)_{ijk} = \sigma_{jk}\vert_{U_i \cap U_j \cap U_k}
- \sigma_{ik}\vert_{U_i \cap U_j \cap U_k} + \sigma_{ij}\vert_{U_i \cap U_j \cap U_k}
\]
Written multiplicatively (and omitting the restriction), this becomes
\[
1 = \sigma_{jk}\sigma_{ik}\inv\sigma_{ij}
\]
which, since the group is abelian, is equivalent to
\[
\sigma_{ik} = \sigma_{ij}\sigma_{jk}
\]
which looks exactly like a cocycle condition for transition functions of a line
bundle. Recall that given a holomorphic line bundle $\pi : L \to X$, we have local
trivializations -- we can find a cover $\mathcal{U} = \set{U_i}$ with maps
$\varphi_i : \pi\inv(U_i) \to U_i \times \C$ such that
\[\begin{tikzcd}
\pi\inv(U_i) \ar[rr, "\varphi_i"] \ar[dr] & & U_i \times \C \ar[dl] \\
& U_i
\end{tikzcd}\]
commutes, where the maps to $U_i$ are the projections. Therefore, if we consider
the map
$\varphi_i \circ \varphi_k\inv : U_i \cap U_j \times \C \to U_i \cap U_j \times \C$,
we have that $\varphi(x, \lambda) = (x, \psi_{ij}(x)(\lambda))$, where the functions
$\psi_{ij} : U_i\cap U_j \to \GL_1\C$ are holomorphic. The $\psi_{ij}$
are called the \ib{transition functions} of the line bundle $L$.
%
\begin{prop}
The transition functions $\psi_{ij}$ satisfy the following conditions
\begin{enumerate}
  \item $\psi_{ij}\psi_{ji} = 1$ (i.e. the constant function $x \mapsto 1$)
  \item $\psi_{ij}\psi_{jk} = \psi_{ik}$.
\end{enumerate}
The second condition is often called the \ib{cocycle condition}, in reference to the
identity we derived above for the defining property of a \v{C}ech cocycle.
\end{prop}
%
\begin{proof}
Consider the map $\varphi_i\circ\varphi_j\inv\circ\varphi_j\circ\varphi_i\inv = \id$.
We compute under the action on a general element $(x,\lambda)$
\[\begin{tikzcd}
(x,\lambda) \ar[r, "\varphi_j\circ\varphi_i\inv"] & (x,\psi_{ji}(x)(\lambda))
\ar[r, "\varphi_i\circ\varphi_j\inv"] & (x,\psi_{ij}(x)\psi_{ji}(x)(\lambda))
\end{tikzcd}\]
Therefore, we have that $\psi_{ij}(x)\psi_{ji}(x) = 1$ for all $x$, showing the
first property. For the second property, we do the same thing. Consider
the function
$\varphi_i\circ\varphi_j\inv\circ\varphi_j\circ\varphi_k = \varphi_i\circ\varphi_k\inv$.
Then fore $(x,\lambda)$, we compute the action of this function to be
\[\begin{tikzcd}
(x,\lambda) \ar[r, "\varphi_j\circ\varphi_k\inv"] & (x,\psi_{jk}(x)(\lambda))
\ar[r, "\varphi_i\circ\varphi_j\inv"] & (x,\psi_{ij}(x)\psi_{jk}(x)(\lambda))
\end{tikzcd}\]
So we get that $\psi_{ij}(x)\psi_{jk}(x) = \psi_{ik}(x)$ for all $x$.
\end{proof}
%
Under the \v{C}ech differential, the image of a \v{C}ech $0$-cochain $\sigma$ is given
by
\[
(d\sigma)_{ij} = \sigma_j - \sigma_i
\]
written multiplicatively, this becomes
\[
(d\sigma)_{ij} = \sigma_j\sigma_i\inv
\]
In the same spirit, we translate this to a statement regarding transition functions
of a line bundle.
%
\begin{prop}
Let $\pi L \to X$ be a holomorphic line bundle where the transition functions
$\psi_{ij}$ with respect to a cover $\set{U_i}$ satisfy the \ib{coboundary condition},
i.e. there exist holomorphic functions $\sigma_i : U_i \to \GL_1\C$ such that
\[
\psi_{ij} = \sigma_j\sigma_i\inv
\]
Then $L$ is a trivial line bundle.
\end{prop}
%
\begin{proof}
It suffices to provide a nonvanishing section $X \to L$. A section $s : X \to L$ is
equivalent to functions $s_i : U_i \to \C$ with the compatibility condition
\[
s_i = \psi_{ij}s_j
\]
define the $s_i$ by $s_i = \sigma_i\inv$. Then they satisfy the compatibility condition,
since
\[
\psi_{ij}\sigma_j = \sigma_j\sigma_i\inv\sigma_j\inv = \sigma_i\inv = s_i
\]
then since the $\sigma_i$ are functions to $\GL_1\C = \C^\times$, they glue to
a global nonvanishing section, so $L$ is isomorphic to the trivial line bundle
$X \times \C$.
\end{proof}
%
Recall that isomorphism classes of line bundles over $X$ form a group
under tensor product, where the inverse of a line bundle $L$ is the dual bundle
$L^*$. Given line bundles $L,L' \to X$ and an open cover $\mathcal{U} = \set{U_i}$ of $X$
in which both $L$ and $L'$ are trivialized over the $U_i$ (for instance, a good cover
of $X$), let $\psi_{ij}$ be the transition functions for $L$ and let $\varphi_{ij}$
be the transition functions for $L'$. Then the transition functions for $L\otimes L'$
are $\psi_{ij}\varphi_{ij}$, and the transition functions for $L^*$ are given
by $\varphi_{ij}\inv$.
%
\begin{thm}
Let $X$ be a complex manifold, and $\O_X$ its sheaf of holomorphic functions.
Then let $\O_X^\times$ be the sheaf of invertible functions, which is a
sheaf of abelian groups under multiplication. Then we have a group isomorphism
\[
\check{H}^1(X,\O_X^\times) \cong \mathrm{Pic}(X)
\]
\end{thm}
%
\begin{proof}
Fix a good cover $\mathcal{U} = \set{U_i}$ for $X$. Since all the sets and their
nonempty intersections are contractible, we have that
$\check{H}^i(U_i, \mathcal{F}) = 0$ for all $i > 0$ where $\mathcal{F}$ is the sheaf
of sections of any line bundle. Since all the $U_i$ are contractible, we also
have that any line bundle over $U_i$ is trivial, so it admits transition
functions $\psi_{ij}$ with respect to this cover. As shown above, the
functions $\psi_{ij}$ exactly define a \v{C}ech $1$-cocycle, and any
\v{C}ech coboundary defines a trivial bundle. In addition, we have that
the the transition functions of a tensor product are exactly the products
of the transition functions. Putting everything together, this tells us that
the mapping $L \mapsto \set{\psi_{ij}}$ sending a line bundle to the cocycles
determined by its transition functions is a bijective group homomorphism.
\end{proof}
%
%
%
\section{Divisors}
%
Given a complex manifold $X$, a complex hypersurface $Y \subset X$ (also called an
analytic hypersurface) is locally cut out a single holomorphic function, i.e. there is
an open cover $\set{U_i}$ of $X$ such that $Y \cap U_i$ is the vanishing locus of a
single holomorphic function $g_i$. We say that the functions $g_i$ are
\ib{defining functions} for $Y$. The hypersurface $Y$ is said to be \ib{irreducible}
if at any $p \in Y$, the defining function $g$ for $Y$ in a neighborhood about
$p$ is irreducible in the local ring $\O_{X,p}$. If the defining function for a
hypersurfaces is not irreducible, then it can be written as a product of irreducible
functions $g_i$ in $\O_{X,p}$, in which case we can write $Y$ as the union of the
irreducible hypersurfaces determined by the $g_i$. \\

On the complex line $\C$, the irreducible hypersurfaces are just points $p \in \C$,
which have defining functions $f(z) = (z-p)$. Recall that all meromorphic functions
on $\C$ are of the form $p(z)/q(z)$ for polynomials $p(z),q(z) \in \C[z]$. A great
deal of the study of meromorphic functions on $\C$ revolves around studying their zeroes
and poles, which is something that divisors will encapsulate for higher dimensional
complex manifolds.
%
\begin{defn}
Let $Y \subset X$ be an irreducible hypersurface. Let $f \in \mathcal{K}(X)$ be a
meromorphic function, and fix a point $y \in Y$. Then $Y$ is locally cut out by an
irreducible holomorphic function $g \in \O_{X,y}$. The \ib{order} of $f$ at
$y$, denoted $\mathrm{ord}_{Y,y}(f)$ is the smallest integer $n$ such that there exists
an invertible holomorphic function $h \in \O_{X,y}$ such that $f= g^nh$.
\end{defn}
%
We note that this definition is independent of our choice of defining function, since
any two irreducible defining functions for $Y$ differ by a unit in $\O_{X,y}$. In
addition, if $Y$ is irreducible, it is also independent of our choice of point $y \in Y$,
since if a holomorphic function $g$ is irreducible in $\O_{X,y}$, then it is also
irreducible in $\O_{X,y'}$ for any point $y'$ sufficiently close. Therefore, we can
simply write $\mathrm{ord}_Y(f) \defeq \mathrm{ord}_{Y,y}(f)$ for any point $y \in Y$,
provided that $Y$ is an irreducible hypersurface. The intuition behind the definition
is that the order of $f$ at $y$ should be the degree to which the function vanishes
at the point $y$, or the degree of the pole at $y$
%
\begin{exmp}
Let $X = \C$, and $Y = \set{p}$, so $Y$ has locally (in fact globally) defining
function $g(z) = (z-p)$. Then let $f$ be the meromorphic function
\[
f(z) = \frac{(z-p)^k}{q(z)}
\]
where $q(z)$ is a polynomial that does not vanish at $p$.
Then the order of $f$ at $p$ is, as you would expect, $k$. The role of $h$
is played by the locally invertible function $1/q(z)$, which is a unit in the local
ring $\O_{\C,p}$. If instead we have
\[
f(z) = \frac{q(z)}{(z-p)^k}
\]
for a polynomial $q(z)$ that does not vanish at $p$, then the order of $f$
at $y$ is $-k$, where the role of $h$ is played by $q(z)$.
\end{exmp}
%
A simple, but important observation is that order is additive on products.
%
\begin{prop}
Let $Y \subset X$ be an irreducible hypersurface with locally defining function
$g$ at $y \in Y$. Let $f_1,f_2 \in \mathcal{K}(X)$ be meromorphic functions. Then
\[
\mathrm{ord}_Y(f_1f_2) = \mathrm{ord}_Y(f_1) + \mathrm{ord}_Y(f_2)
\]
\end{prop}
%
Some more key observations are that for a nonvanishing holomorphic function $f$, we have
that $\mathrm{ord}_Y(f) = 0$ for any irreducible hypersurface $Y \subset X$, and
that $\mathrm{ord}_Y(f) = 0 \mathrm{ord}_Y(1/f)$.
%
\begin{proof}
In the local ring $\O_{X,y}$, we have that
\[
f_1 = g^{\mathrm{ord}_Y(f_1)}h_1 \qquad f_2 = g^{\mathrm{ord}_Y(f_2)}h_2
\]
Therefore, we have that
\[
f_1f_2 = g^{\mathrm{ord}_Y(f_1) + \mathrm{ord}_Y(f_2)}h_1h_2
\]
The function $h_1h_2$ does not vanish at $y$, and we have that
$\mathrm{ord}_Y(f_1) + \mathrm{ord}_Y(f_2)$ must be the smallest integer for
$f_1f_2$, since $\mathrm{ord}_Y(f_1)$ and $\mathrm{ord}_Y(f_2)$ are minimal for
$f_1$ and $f_2$ respectively.
\end{proof}
%
\begin{defn}
A \ib{divisor} $D$ on $X$ is a formal integral combination
\[
D = \sum a_i[Y_i]
\]
where $Y_i$ is an irreducible hypersurface for all $i$. The group of divisors on
$X$ is denoted $\mathrm{Div}(X)$.
\end{defn}
%
The intuition for a divisor is that it should interpreted as a prescription for zeroes
and poles of a meromorphic function. For example, in $\C$, you should think of
the divisor
\[
D = p_1 + 4p_2 - 4p_3 - 6p_4
\]
as corresponding the the meromorphic function
\[
f(z) = \frac{(z-p_1)(z-p_2)^4}{(z-p_3)^4(z-p_4)^6}
\]
Of course, this isn't exactly true or precise. For example, we could multiply the
numerator or denominator by nonvanishing holomorphic functions and then obtain a function
with the same poles and zeroes as $f$.
%
\begin{defn}
Let $f \in \mathcal{K}(X)$ be a meromorphic function. Then the \ib{zero divisor} of
$f$ is the divisor
\[
(f)_0 = \sum_{\mathrm{ord}_Y(f) > 0} \mathrm{ord}_Y(f)[Y]
\]
where the sum is taken over all irreducible hypersurfaces $Y > 0$. Likewise, the
\ib{pole divisor} of $f$ is defined to be the divisor
\[
(f)_\infty = \sum_{\mathrm{ord}_Y(f) < 0} \mathrm{ord}_Y(f)[Y]
\]
The \ib{associated divisor} of $f$ is the divisor
\[
(f) = (f)_0 - (f)_\infty = \sum_Y \mathrm{ord}_Y(f)
\]
\end{defn}
%
This makes precise what we mean by divisors being prescriptions for poles and
zeroes, but the correspondence between meromorphic functions and divisors is
far from bijective, as we noted above. Our observation was that if any two meromorphic
functions differed by a nonvanishing holomorphic function, then they will define the
same divisor. This turns out to be the only redundancy.
%
\begin{thm}
Let $X$ be a complex manifold. Then let $\mathcal{K}^\times_X$ denote the sheaf of
nonzero meromorphic functions, and $\O_X^\times \subset \mathcal{K}^\times_X$ the sheaf
of nonvanishing holomorphic functions, which are both sheaves of abelian groups
under multiplication. Then we have an isomorphism
\[
\Gamma(X, \mathcal{K}^\times_X/\O^\times_X) \to \mathrm{Div}(X)
\]
\end{thm}
%
\begin{proof}
An global section $\sigma$ of the quotient sheaf $\mathcal{K}_X^\times / \O_X^\times$ is a
collection of pairs $(U_i, f_i)$ of meromorphic functions
$f_i \in \mathcal{K}_X^\times(U_i)$ such that on the intersections $U_i \cap U_j$, their
quotient  $f_i/f_j$ is an element of $\O_X^\times$, i.e. a nonvanishing holomorphic
function over $U_i \cap U_j$. Therefore, over the intersection, we have that
\[
\mathrm{ord}_Y(f_i) - \mathrm{ord}_Y(f_j) = \mathrm{ord}_Y(f_i/f_j) = 0
\]
For any irreducible hypersurface $Y$. Therefore, we have that
$\mathrm{ord}_Y(f_i) = \mathrm{ord}_Y(f_j)$. Therefore, the order
$\mathrm{ord}_Y(\sigma)$ of $\sigma$ is well-defined for any irreducible hypersurface
$Y \subset X$, so we get a well-defined map
\begin{align*}
\Gamma(X, \mathcal{K}^\times_X / \O_X^\times) &\to \mathrm{Div}(X) \\
\sigma &\mapsto (\sigma) \defeq \sum_Y \mathrm{ord}_Y(\sigma)[Y]
\end{align*}
To show that this mapping is bijective, we provide an inverse map. Let
$D = \sum_i a_i [Y_i] \in \mathrm{Div}(X)$. Then we can find an open cover $\set{U_i}$
of $X$ where in each open set $U_j$, any irreducible hypersurface $Y_i$ intersecting
$U_j$ nontrivially is the vanishing locus of a holomorphic function $g_{ij}$, which
is unique up to multiplication by an element in $\O_X^\times(U_j)$. Then we claim that
the functions $f_j \defeq \prod_i g_{ij}$ glue to a global section of
$\mathcal{K}_X^\times / \O_X^\times$, i.e. on any $U_j \cap U_k$, we have that
$f_j/f_k$ is a nonvanishing holomorphic function. However, on any intersection
$U_j \cap U_k$, the functions $g_{ij}$ and $g_{ik}$ define the same irreducible
hypersurface $Y_i$, so they differ by multiplication by a nonvanishing holomorphic
function. Therefore, the $f_j$ glue to a global section. \\

It is easily verified that these two constructions are inverses, so the maps
are bijective. Furthermore, the map is a group homomorphism since order is
additive on products of meromorphic functions, so it is a group isomorphism.
\end{proof}
%
\begin{rem*}
For complex manifolds, we have that global sections of the quotient sheaf and divisors
are isomorphic, but in the algebro-geometric setting, this is not true without some
smoothness assumptions. In algebraic geometry, elements of
$\Gamma(X, \mathcal{K}_X^\times/\mathcal{O}_X^\times)$ are called \ib{Weil divisors},
while the elements of what we called $\mathrm{Div}(X)$ are called \ib{Cartier divisors}.
\end{rem*}
%
One of the big punchlines for divisors is their correspondence with holomorphic
line bundles over $X$, which we can make explicit using the isomorphism with
$\Gamma(X, \mathcal{K}^\times_X / O^\times_X)$. In the proof we gave above, we noted
that a global section of $\mathcal{K}^\times_X / O^\times_X$ can be seen as a collection
of meromorphic functions $f_i$ on an open cover $\set{U_i}$, such that on intersections,
the function $f_i/f_j$ is a nonvanishing holomorphic function. In addition, it is easily
seen that the functions satisfy the cocycle condition, so they can be used
to define transition functions for a line bundle, which we call $\O(D)$. The
mapping $D \mapsto \O(D)$ is clearly a group homomorphism, so it defines a natural
map $\mathrm{Div}(X) \to \mathrm{Pic}(X)$. In particular, this shows that line bundles
are closely related to the codimension $1$ geometry of the complex manifold $X$.
%
%TODO Divisors in terms of sheaf cohomology, line bundle of a divisor corresponding to
%the image under a boundary map
%
%
%
\section{Sheaf Cohomology}
%
Over a complex manifold $X$, we have many different cohomology theories at our disposal:
\begin{enumerate}
  \item The singular cohomology groups $H^i_{\text{sing}}(X,\Z)$.
  \item The de Rham cohomology groups $H^i_{dR}(X)$.
  \item The Dolbeault cohomology groups of a holomorphic vector bundle
  $H^i_{\dbar}(X, E)$.
  \item The \v{C}ech cohomology groups of a sheaf $\check{H}^i(X,\mathcal{F})$.
\end{enumerate}
%
We want to compare these various cohomology theories. To do so, we show that many of
these cohomology theories are computing the same thing : sheaf cohomology.
%
\begin{rem*}
While we might explicitly work with sheaves of abelian groups, the following discussion
is applicable to sheaves of $\mathcal{O}_X$ modules, $C^\infty$ modules, etc.
\end{rem*}
%
\begin{defn}
The \ib{global sections functor} $\Gamma(X,\cdot)$ is a functor
$\mathsf{Ab}(X) \to \mathsf{Ab}$ of the category of sheaves of abelian groups over $X$ to
the category of abelian groups, where given a sheaf of abelian groups $\mathcal{F}$,
\[
\Gamma(X,\mathcal{F}) \defeq \mathcal{F}(X)
\]
\end{defn}
%
The functor is left-exact, i.e. given an exact sequence of sheaves
\[\begin{tikzcd}
0 \ar[r] & \mathcal{E} \ar[r] & \mathcal{F} \ar[r] & \mathcal{G}
\end{tikzcd}\]
we get an exact sequence
\[\begin{tikzcd}
0 \ar[r] & \Gamma(X,\mathcal{E}) \ar[r]
& \Gamma(X,\mathcal{F}) \ar[r] & \Gamma(X,\mathcal{G})
\end{tikzcd}\]
%
However, the functor is not right exact, which is due to the local definitions of
injectivity and surjectivity. The sheaf axiom guarantees that being injective on
stalks implies that a sheaf morphism is injective on sections, but it does not
imply the same thing for surjectivity. As an example, let $\mathcal{Z}^k$ be
the sheaf of closed smooth $k$-forms, and let $\mathcal{B}^k$ be the sheaf of exact
$k$-forms. Then the inclusion $\mathcal{B}^k \hookrightarrow \mathcal{Z}^k$ is surjective,
since every closed $k$-form is exact in a sufficiently small neighborhood. However,
if $H^k_{dR}(X) \neq 0$, then $\Gamma(X,\mathcal{Z}^k) \to \Gamma(X, \mathcal{B}^k)$
is not surjective.
%
\begin{defn}
The \ib{sheaf cohomology groups} $H^i(X, \mathcal{F})$ of a sheaf $\mathcal{F}$ over $X$
are the right derived functors of the global sections functor applied to $\mathcal{F}$
\[
H^i(X,\mathcal{F}) \defeq R^i\Gamma(X,\mathcal{F})
\]
i.e, we take an injective resolution $\mathcal{I}^\bullet = \set{\mathcal{I}^j}$ of
$\mathcal{F}$
\[\begin{tikzcd}
0 \ar[r] & \mathcal{F} \ar[r] & \mathcal{I}^0 \ar[r] & \mathcal{I}^1 \ar[r] & \cdots
\end{tikzcd}\]
and apply $\Gamma(X,\cdot)$ term-wise to the sequence to get
\[\begin{tikzcd}
0 \ar[r] & \Gamma(X,\mathcal{F}) \ar[r] & \Gamma(X,\mathcal{I}^0) \ar[r]
& \Gamma(X,\mathcal{I}^1) \ar[r] & \cdots
\end{tikzcd}\]
and then compute the cohomology of this sequence.
\end{defn}
%
We note that in order for these right derived functors to be defined, there need to
be \emph{enough injectives}. We say that an abelian category $\mathcal{A}$ has enough
injectives if every object $A \in \mathrm{Ob}(\mathcal{A})$ admits an injective
map $A \hookrightarrow I$ into an injective object $I$, where an injective object
is defined to be an object $I$ where given any map $X \to I$ and an injection
$X \hookrightarrow I$, there exists a map $Y \to Q$ such that the following diagram
commutes
\[\begin{tikzcd}
X \ar[dr]\ar[rr,hookrightarrow] & & Y \ar[dl, dashed] \\
& Q
\end{tikzcd}\]
Alternatively, the pullback map $\hom(Y,Q) \to \hom(X,Q)$ is surjective. We will take
the following results on faith:
%
\begin{thm}
The categories $\mathsf{Ab}$, $\mathsf{Ab}(X)$, $\mathsf{Mod}_{\O_X}$,
and $\mathsf{Mod}_{C^\infty}$ have
enough injectives.
\end{thm}
%
Like with the definition of the \v{C}ech cohomology groups as limits, the
definition in terms of injective resolutions is practically useless computationally,
since injective sheaves are hard to write down and difficult to find in the wild.
The name of the game here is to find a nicer class of resolutions we can take.
%
\begin{defn}
Let $F : \mathcal{A} \to \mathcal{B}$ be a left-exact additive functor. An object
$A \in \mathrm{Ob}(\mathcal{A})$ is \ib{acyclic} for the functor $F$ if
$R^iF(A) = 0$ for all $i > 0$.
\end{defn}
%
\begin{prop}
Let $F : \mathcal{A} \to \mathcal{B}$ be an additive functor, and
$A \in \mathrm{Ob}(\mathcal{A})$. Then let $A \to M^\bullet$ be a resolution of $A$
by $F$-acyclic objects. Then $R^iF(A)$ is isomorphic to the $i^{th}$ cohomology
of the complex $F(M^\bullet)$.
\end{prop}
%
\begin{proof}
Let $d^0 : M^0 \to M^1$. Then let $B = \coker (A \to M^0)$. By exactness of
\[\begin{tikzcd}
0 \ar[r] & A \ar[r] & M^0 \ar[r, "d^0"] & M^1
\end{tikzcd}\]
we get a short exact sequence
\[\begin{tikzcd}
0 \ar[r] & A \ar[r] & M^0 \ar[r] & B \ar[r] & 0
\end{tikzcd}\]
where the map $M^0 \to B$ is the map $d^0$, which descends to $B$ by exactness.
We then take injective resolutions $A \to I^\bullet$, $M^0 \to J^\bullet$, and
$B \to K^\bullet$. The maps $A \to M^0$ and $M^0 \to B$, which gives a short exact
sequence of chain maps between resolutions by using the defining property
of injective objects, giving us
\[\begin{tikzcd}
0 \ar[r] & A \ar[r] \ar[d] & M^0 \ar[r] \ar[d] & B \ar[r] \ar[d] & 0 \\
0 \ar[r] & I^0 \ar[r] \ar[d] & J^0 \ar[r] \ar[d] & K^0 \ar[r] \ar[d] & 0 \\
0 \ar[r] & I^1 \ar[r] \ar[d] & J^1 \ar[r] \ar[d] & K^1 \ar[r] \ar[d] & 0 \\
& \vdots  & \vdots  & \vdots &
\end{tikzcd}\]
which gives us a long exact sequence in cohomology. Noting that the cohomology
of the respective sequences are just the right derived functors of $A$, $M^0$,
and $B$, we get the long exact sequence
\[\begin{tikzcd}
0 \ar[r] & R^0F(A) \ar[r] & R^0F(M^0) \ar[r] & R^0F(B) \ar[dll] \\
& R^1F(A) \ar[r] & R^1F(M^0) \ar[r] & R^1F(B) \ar[dll] \\
& R^2F(A) \ar[r] & R^2F(M^0) \ar[r] & R^2F(B) \ar[dll] \\
& ~ & \cdots & \cdots
\end{tikzcd}\]
Then using the fact that $M^0$ is $F$-acyclic, along with the fact that $R^0F = F$, we
get that this long exact sequence is actually
\[\begin{tikzcd}
0 \ar[r] & F(A) \ar[r] & F(M^0) \ar[r] & F(B) \ar[dll] \\
& R^1F(A) \ar[r] & 0 \ar[r] & R^1F(B) \ar[dll] \\
& R^2F(A) \ar[r] & 0 \ar[r] & R^2F(B) \ar[dll] \\
& ~ & \cdots & \cdots
\end{tikzcd}\]
which gives us isomorphisms $R^iF(B) \to R^{i+1}F(A)$ for $i > 0$, as well as
an isomorphism $R^1F(A) \cong \coker F(M^0) \to F(B)$. To compute the
cokernel of that map, we note that $B$ admits the resolution
\[\begin{tikzcd}
0 \ar[r] & B \ar[r] & M^1 \ar[r] & \cdots
\end{tikzcd}\]
where the map $B \to M^1$ is the map induced by $d^0$, using exactness of the acyclic
resolution. Then since $F$ is left-exact, we get that
\[\begin{tikzcd}
0 \ar[r] & F(B) \ar[r] & F(M^1) \ar[r] & \cdots
\end{tikzcd}\]
is exact, giving us that $F(B)$ is isomorphic to the kernel of $F(M^1) \to F(M^2)$. We
then note that the image of $F(M^0) \to F(M^1)$ is contained in $F(B)$, regarded
as the kernel of $F(M^1) \to F(M^2)$. We then get
\[
R^1F(A) \cong \frac{\ker(F(M^1) \to F(M^2))}{\im(F(M^1) \to F(M^2))}
= H^1(F(M^\bullet))
\]
Then to get the isomorphism for $R^2F(A)$, we note that since
$R^1F(B) \cong R^2F(A)$, we can play the same game using the resolution of $B$
to compute $R^1F(B)$ to be $H^2(F(M^\bullet))$, and then inductively repeat
the process with the cokernel of $M^1 \to M^2$ to get $R^3F(A)$ and so on.
\end{proof}
%
Something that is silly to observe, but useful.
%
\begin{prop}
Injective objects are $F$-acyclic.
\end{prop}
%
\begin{proof}
Let $I$ be injective. Then take the injective resolution $0 \to I \to I \to 0$
where the map is the identity map.
\end{proof}
%
Therefore, to compute right derived functors, it suffices to find acyclic resolutions.
This gets us one step closer to finding nice resolutions for computing sheaf cohomology.
%
\begin{defn}
A sheaf $\mathcal{F}$ over $X$ is \ib{flasque} (also called \ib{flabby}) if
the restriction maps are surjective.
\end{defn}
%
Most sheaves in nature aren't flasque. However, flasque sheaves are useful for giving
us acyclic resolutions. To do this, we'll need some lemmas.
%
\begin{lem}
Let
\[\begin{tikzcd}
0 \ar[r] & \mathcal{E} \ar[r, "\alpha"] & \mathcal{F} \ar[r,"\beta"]
& \mathcal{G} \ar[r] & 0
\end{tikzcd}\]
be a short exact sequence where $\mathcal{E}$ is flasque. Then for any open set $U$,
the sequence
\[\begin{tikzcd}
0 \ar[r] & \mathcal{E}(U) \ar[r, "\alpha(U)"] & \mathcal{F}(U) \ar[r, "\beta(U)"]
& \mathcal{G}(U) \ar[r] & 0
\end{tikzcd}\]
is exact.
\end{lem}
%
\begin{proof}
By left-exactness of taking sections, it suffices to show that
$\beta(U) : \mathcal{F}(U) \to \mathcal{G}(U)$ is surjective. Let
$\sigma \in \mathcal{G}(U)$. Since $\beta$ is a surjective sheaf morphism, for any
$x \in U$, the induced map on stalks $\beta_x : \mathcal{F}_x \to \mathcal{G}_x$ is
surjective, which implies that there exists a sufficiently small neighborhood
$V_x \subset U$ of $x$ where
$\beta(V_x) : \mathcal{F}(V_x) \to \mathcal{G}(V_x)$ is surjective,
so we can find $\tau_x \in \mathcal{F}(V_x)$ such that
$\beta(V_x)(\tau) = \sigma\vert_{V_x}$. Then let $y \in U$ such that
$V_x \cap V_y \neq \emptyset$, and let $\tau_y \in \mathcal{F}(V_y)$ such that
$\beta(V_y)(\tau_y) = \sigma\vert_{V_y}$. Then we know that
$\tau_x\vert_{V_x \cap V_y} - \tau_y\vert_{V_x \cap V_y} \in \ker\beta(V_x \cap V_y)$,
so it is the image of an element $k \in \mathcal{E}(V_x \cap V_y)$. Since $\mathcal{E}$
is flasque, we know that we can lift $k$ to an section $\chi_{x,y} \in \mathcal{E}(V_x)$.
Then the element $\tau_x' = \tau_x - \alpha(V_x)(\chi_{x,y})$ still maps to
$\sigma\vert_{V_x}$  under $\beta(V_x)$ by exactness, and restricts to $\tau_y$ on
$V_x \cap V_y$. Therefore, $\tau_x'$ and $\tau_y$ glue to a section over $V_x \cup V_y$
that maps to $\sigma\vert_{V_x \cup V_y}$.  We then find a maximal pair $(W, \tau)$ such
that $\tau \in \mathcal{F}(W)$ and $\beta(W)(\tau) = \sigma\vert_W$. Then we must
necessarily have $W = U$, since otherwise, we can find another open subset $V$ of $U$
and a section $\varphi$ over $V$ mapping to $\sigma\vert_V$, and extend $\tau$ to a
section over $W \cup V$, since if $W \cap V \neq \emptyset$, we can use our above
argument, and otherwise, no work needs to be done. Either way, finding such a $U$
and $\varphi$ contradicts maximality of $(W,\tau)$.
\end{proof}
%
\begin{lem}
Let
\[\begin{tikzcd}
0 \ar[r] & \mathcal{E} \ar[r,"\alpha"] & \mathcal{F} \ar[r,"\beta"]
& \mathcal{G} \ar[r] & 0
\end{tikzcd}\]
be a short exact sequence of sheaves where $\mathcal{E}$ and $\mathcal{F}$ are flasque.
Then $\mathcal{G}$ is flasque.
\end{lem}
%
\begin{proof}
Let $V \subset U$. Then given $\sigma \in \mathcal{G}(V)$, we want to show that
there exits $\widetilde{\sigma} \in \mathcal{G}(U)$ that restricts to $\sigma$. Since
$\mathcal{E}$ is flasque, we have that
\[\begin{tikzcd}
0 \ar[r] & \mathcal{E}(V) \ar[r, "\alpha(V)"] & \mathcal{F}(V) \ar[r,"\beta(V)"]
& \mathcal{G}(V) \ar[r] & 0
\end{tikzcd}\]
is exact, so we can find a section $\tau \in \mathcal{F}(V)$ with
$\beta{V}(\tau) = \sigma$. Then since $\mathcal{F}$ is flasque, this lifts to
an element $\widetilde{\tau} \in \mathcal{F}(U)$. Then taking
$\widetilde{\sigma} = \beta(U)(\widetilde{\tau})$ gives us the desired section, since
the properties of a sheaf morphism implies that the following diagram commutes:
\[\begin{tikzcd}
\mathcal{F}(U) \ar[r, "\beta(U)"] \ar[d] & \mathcal{G}(U) \ar[d] \\
\mathcal{F}(V) \ar[r, "\beta(V)"'] & \mathcal{G}(V)
\end{tikzcd}\]
\end{proof}
%
\begin{prop}
Flasque sheaves are acyclic for the global sections functor $\Gamma(X, \cdot)$.
\end{prop}
%
\begin{proof}
Let $\mathcal{F}$ be a flasque sheaf. We first embed $\mathcal{F}$ into an injective
flasque sheaf $\mathcal{I}$. Since $\mathsf{Ab}$ has enough injectives, we can embed
each stalk $\mathcal{F}_x$ into an injective group $I_x$. Then define the sheaf
$\mathcal{I}$ by
\[
\mathcal{I}(U) = \prod_{x \in U} I_x
\]
and the restriction maps are the projection maps
$\prod_{x \in U} I_x \to \prod_{x \in V} I_x$. Since these maps are surjective,
$\mathcal{I}$ is flasque. In addition, it is injective by construction. Then
$\mathcal{F}$ embeds into $\mathcal{I}$ by composing the inclusions
$\mathcal{F}(U)\hookrightarrow\prod_{x\in U}\mathcal{F}_x\hookrightarrow\prod_{x\in V}I_x$
Then let $\mathcal{G}$ be the cokernel of $\mathcal{F} \hookrightarrow \mathcal{I}$,
giving us the exact sequence of sheaves
\[\begin{tikzcd}
0 \ar[r] & \mathcal{F} \ar[r] & \mathcal{I} \ar[r] & \mathcal{G} \ar[r] & 0
\end{tikzcd}\]
taking resolutions of $\mathcal{F}$, $\mathcal{I}$ and $\mathcal{G}$ gives
us a long exact sequence in cohomology
\[\begin{tikzcd}
0 \ar[r] & H^0(X,\mathcal{F}) \ar[r] & H^0(X,\mathcal{I}) \ar[r]
& H^0(X,\mathcal{G}) \ar[dll] \\
& H^1(X,\mathcal{F}) \ar[r] & H^1(X, \mathcal{I}) \ar[r] & H^1(X,\mathcal{G})\ar[dll] \\
& H^2(X,\mathcal{F}) \ar[r] & H^2(X, \mathcal{I}) \ar[r] & H^2(X,\mathcal{G})\ar[dll] \\
& ~ & \cdots & \cdots
\end{tikzcd}\]
The since $H^0$ is just $\Gamma(X,\cdot)$ and $\mathcal{I}$ is injective, this becomes
\[\begin{tikzcd}
0 \ar[r] & \mathcal{F}(X) \ar[r] & \mathcal{I}(X) \ar[r] & \mathcal{G}(X) \ar[dll] \\
& H^1(X,\mathcal{F}) \ar[r] & 0 \ar[r] & H^1(X,\mathcal{G}) \ar[dll] \\
& H^2(X,\mathcal{F}) \ar[r] & 0 \ar[r] & H^2(X,\mathcal{G}) \ar[dll] \\
& ~ & \cdots & \cdots
\end{tikzcd}\]
Since $\mathcal{F}$ is flasque, we have that
\[\begin{tikzcd}
0 \ar[r] & \mathcal{F}(X) \ar[r] & \mathcal{I}(X) \ar[r] & \mathcal{G}(X) \ar[r] & 0
\end{tikzcd}\]
is exact, so this implies that $H^1(X,\mathcal{F}) = 0$. In addition, for $i > 0$,
we get isomorphisms $H^i(X,\mathcal{G}) \to H^{i+1}(X, \mathcal{F})$. To get that
$H^2(X,\mathcal{F}) = 0$, we note that $\mathcal{G}$ is flasque, so we can
repeat the argument to get that $H^1(X,\mathcal{G}) = H^2(X, \mathcal{F})$, and
then continue for the other cohomology groups.
\end{proof}
%
As a consequence, it suffices to find resolutions by flasque sheaves to compute sheaf
cohomology. The good news here is that every sheaf admits a canonical
acyclic resolution, the \ib{Godement resolution}. For a sheaf $\mathcal{F}$,
let $\mathcal{F}_{\mathrm{God}}$ be the sheaf $U \mapsto \prod_{x \in U}\mathcal{F}_x$,
which is clearly flasque. Then we construct a flasque resolution for
$\mathcal{F}$ as follows : embed
$\mathcal{F} \hookrightarrow \mathcal{F}_{\mathrm{God}}$, and let $\mathcal{G}^1$ be
the cokernel. Then let the next sheaf in the sequence be $\mathcal{G}^1_{\mathrm{God}}$,
where the map $\mathcal{F}_{\mathrm{God}} \to \mathcal{G}^1_{\mathrm{God}}$ is the
quotient map $\mathcal{F}_{\mathrm{God}} \to \mathcal{G}^1$ composed in the the inclusion
$\mathcal{G}^1 \hookrightarrow \mathcal{G}^1_{\mathrm{God}}$. Then take $\mathcal{G}^2$
to be the cokernel of $\mathcal{F}_{\mathrm{God}} \to \mathcal{G}^1_{\mathrm{God}}$,
and continue with $\mathcal{G}^2_{\mathrm{God}}$ and so on. Pictorially, the
Godement construction is constructed as follows:
\[\begin{tikzcd}
0 \ar[r] & \mathcal{F} \ar[r] & \mathcal{F}_{\mathrm{God}} \ar[dr] \ar[r]
& \mathcal{G}^1 \ar[r] \ar[d] & 0 \\
& & & \mathcal{G}^1_{\mathrm{God}} \ar[r] \ar[dr] & \mathcal{G}^2 \ar[r] \ar[d] & 0\\
& & & & \mathcal{G}^2_{\mathrm{God}} \ar[r] \ar[dr] & \mathcal{G}^3 \ar[r] \ar[d] & 0 \\
& & & & & \ddots
\end{tikzcd}\]
%
Once more, this resolution is computationally useless, but it serves the purpose of
showing that every sheaf admits a resolution by flasque sheaves. With this in hand,
we can finally show that we can compute sheaf cohomology with a nice class of sheaves.
%
\begin{defn}
Let $\mathcal{A}$ be a sheaf of rings over $X$ such that $\mathcal{A}$ admits
\ib{partitions of unity}, i.e. for any open cover $\set{U_i}$ of $X$, there exist global
sections $f_i \in \mathcal{A}(X)$ such that $\sum_i f_i = 1$ and $f_i$ is supported
in $U_i$, and where over any particular open set, all but finitely many of the $f_i$
are $0$. Then a sheaf of $\mathcal{A}$-modules is a \ib{fine sheaf}.
\end{defn}
%
These are sheaves we care about, and appear in nature.
%
\begin{exmp}
Let $M$ be a smooth manifold. Then the sheaf $C^\infty$ of smooth functions
admits partitions of unity. Therefore, $\mathsf{Mod}_{C^\infty}$ consists of fine
sheaves.
\end{exmp}
%
The punchline is that fine sheaves are acyclic, which gives us a source of reasonable
sheaves with which to build resolutions.
%
\begin{thm}
Fine sheaves are acyclic with respect to $\Gamma(X,\cdot)$.
\end{thm}
%
\begin{proof}
Let $\mathcal{F}$ be a fine sheaf -- a sheaf of modules over a sheaf of rings
$\mathcal{A}$ that admits partitions of unity. Then by taking the
Godement resolution of $\mathcal{F}$, we get an injective resolution of
flasque $\mathcal{A}$-modules.
\[\begin{tikzcd}
0 \ar[r] & \mathcal{F} \ar[r] & \mathcal{I}^0 \ar[r, "d^0"] & \mathcal{I}^1 \ar[r, "d^1"]
& \cdots
\end{tikzcd}\]
Then since flasque sheaves are acyclic, we get that we can compute the sheaf
cohomology of $\mathcal{F}$ as
\[
H^i(X, \mathcal{F})
= \frac{\ker d^{i+1}(X) : \mathcal{I}^i(X) \to \mathcal{I}^{i+1}(X)}
{\im d^i(X) : \mathcal{I}^{i-1}(X) \to \mathcal{I}^i(X)}
\]
Then let $\alpha \in \ker d^{i+1}(X)$, by exactness, we know that locally, in a
sufficiently small open cover $\set{U_j}$, the map $d^{i}(U_j)$ is surjective,
so we can find $\beta_j \in \mathcal{I}^{i-1}(U_j)$ such that
$d^i(V_j)(\beta_j) = \alpha\vert_{U_j}$. Then $f_i\beta_i$ determines a global
section that is equal to $f_j\beta_j$ on $U_i$ and $0$ elsewhere, and we get that
$\sum_jf_j\beta_j$ maps to $\sum_j f_j\alpha\vert_{U_j} = \alpha$ under $d^i(X)$.
Therefore, the sequence is exact on global sections for $i > 0$, so $\mathcal{F}$
is acyclic.
\end{proof}
%
This tells us that many of the sheaves we know about, like sheaves of smooth sections
of a vector bundle are trivial from the perspective of sheaf cohomology.
%
%
\section{Comparison of Cohomology Theories}
%
The fact that many of the sheaves we encounter naturally have trivial sheaf cohomology
might come as a surprise, since we know we can extract topological data from
these sheaves. The reason for this is that they provide good resolutions of
other sheaves with nontrivial sheaf cohomology. 
%
\begin{prop}[Poincar\'e Lemma]
Every closed smooth $k$-form $\omega$ is locally exact, i.e. for a sufficiently
small $U$, we have that $\omega\vert_U = d\eta$ for some $k-1$-form $\eta$.
\end{prop}
%
\begin{prop}[$\dbar$-Poincar\'e Lemma]
Ever closed smooth $(p,q)$-form $\omega$ is locally $\dbar$-exact
\end{prop}
%
\begin{cor}
The de Rham complex
\[\begin{tikzcd}
A^0(X) \ar[r, "d"] & A^1(X) \ar[r, "d"] & \cdots
\end{tikzcd}\]
is an exact sequence of sheaves.
\end{cor}
%
The constant sheaf $\underline{\R}$ of locally constant real-valued functions
naturally lives as a subsheaf of $A^0(X)$, and we know that this exactly
the kernel of $d : A^0(X) \to A^1(X)$. This tells us that the inclusion
$0 \to \underline{\R} \to A^\bullet(X)$ is a resolution of $\underline{\R}$,
called the \ib{de Rham resolution}. Furthermore,
since all the $A^i(X)$ are sheaves of $C^\infty$-modules, they are fine, so
the resolution is a resolution of $\underline{\R}$ by acyclic sheaves. Therefore,
we get the isomorphisms
\[
H^i(X, \underline{\R}) \cong H^i_{dR}(X)
\]
A similar story holds for the sheaf cohomology of the sheaf of sections of a holomorphic
vector bundle $E \to X$. The $\dbar$-Poincar\'e lemma implies that the Dolbeault
complex
\[\begin{tikzcd}
\mathcal{A}^0(E) \ar[r, "\dbar_E"] & \mathcal{A}^1(E) \ar[r, "\dbar_e"] & \cdots
\end{tikzcd}\]
of sheaves of smooth sections of $(\Lambda^i T^*X)_\C \otimes E$ is an exact sequence,
since $\dbar_E$ is defined locally in terms of the operator $\dbar$ on $X$.
Then since the kernel of $\dbar_E : \mathcal{A}^0(E) \to \mathcal{A}^1(E)$ is
exactly the sheaf $\mathcal{E}$ of holomorphic sections of $E$, we get that
$0 \to \mathcal{E} \to \mathcal{A}^\bullet(E)$ is an acyclic resolution of
$\mathcal{E}$, which gives us isomorphisms
\[
H^i(X,\mathcal{E}) \cong H^i_{\dbar}(X, E)
\]
%
%TODO Singular cohomology
%
%TODO Simplicial cohomology and another proof of de Rham?
%
%TODO Cech vs derived functor cohomology
%
%
\section{The First Chern Class}
%
On any complex manifold $X$, we have the following exact sequence of sheaves:
\[\begin{tikzcd}
0 \ar[r] & \Z \ar[r] & \O_X \ar[r, "\mathrm{exp}"] & \O_X^\times \ar[r] & 0
\end{tikzcd}\]
where $\mathrm{exp}$ is the sheaf morphism where a function $f \in \O_X(U)$ is
mapped to the function $e^f$. This gives us the long exact sequence in sheaf
cohomology
\[\begin{tikzcd}
0 \ar[r] & H^0(X, \Z) \ar[r] & H^0(X, \O_X) \ar[r] & H^0(X, \O_X^\times) \ar[dll]\\
& H^1(X, \Z) \ar[r] & H^1(X, \O_X) \ar[r] & H^1(X, \O_X^\times) \ar[dll, "c_1"']\\
& H^2(X, \Z) \ar[r] & \cdots
\end{tikzcd}\]
%
We are particularly interested in the boundary map $H^1(X,\O_X^\times) \to H^2(X, \Z)$,
which we called $c_1$. Recall that we have an isomorphism
$H^1(X,\O_X^\times) \cong \mathrm{Pic}(X)$.
%
\begin{defn}
The \ib{first Chern class} of a line bundle $L$ is its image under the map
$c_1$, using the canonical identification $H^1(X,\O_X^\times) \cong \mathrm{Pic}(X)$.
\end{defn}
%
The first Chern class will prove to be a powerful invariant of line bundles, and
our goal will be to understand these cohomology classes in terms of the geometry
of the manifold $X$. We first note some properties of $c_1$ that can be immediately
deduced from the definition and our knowledge of the group structure on $\mathrm{Pic}(X)$.
%
\begin{prop} \enumbreak
\begin{enumerate}
  \item $c_1(L_1 \otimes L_2) = c_1(L_1) + c_1(L_2)$
  \item $c_1(L^*) = -c_1(L)$.
  \item For a map $F : X \to Y$, we have $c_1(F^*L) = F^*c_1(L)$.
\end{enumerate}
\end{prop}
The first two observations are immediate, and the last one follows from functoriality
of cohomology. \\

Using some Hodge theory, we can already learn something about $\CP^n$ using the
first Chern class.
%
\begin{thm}
\[
\mathrm{Pic}(\CP^n) \cong \Z
\]
\end{thm}
%
\begin{proof}
Hodge theory gives us the Hodge decomposition
\[
H^k(\CP^n, \C) = \bigoplus_{p+q=k} H^{p,q}(X, \C)
\]
as well as isomorphisms
\[
H^q(\CP^n, \Omega^{p,0}_{\CP^n}) \cong H^{p,q}(\CP^n, \C)
\]
We know the singular cohomology of $\CP^n$ is given by
\[
H^k(\CP^n, \C) \cong \begin{cases}
\C & k \text{ is even} \\
0 & \text{otherwise}
\end{cases}
\]
In particular, if $p+q$ is odd, then we know that $H^{p,q}(X,\C) = 0$.
A special case is
\[
0 = H^{1,0}(\CP^n, \C) \cong H^1(\CP^n, \Omega^{0,0}_{\CP^n}) = H^1(\CP^n, \O_{\CP^n})
\]
Therefore, we have that in the long exact sequence induced by the exponential exact
sequence, we have
\[\begin{tikzcd}
0 \ar[r] & \mathrm{Pic}(\CP^n) \ar[r, "c_1"] & \Z
\end{tikzcd}\]
which tells us that $c_1$ is injective. Since the only subgroups of $\Z$
are isomorphic to $\Z$, this gives $\mathrm{Pic}(\CP^n) \cong \Z$. In addition,
it tell us that $c_1$ is a complete isomorphism invariant. Two holomorphic line bundles
over $\CP^n$ are isomorphic if and only if they have the same first Chern class.
\end{proof}
%
Given that the first Chern class is defined using sheaves of holomorphic functions,
as well as our preliminary observation with $\CP^n$, one might first think that $c_1$ is
an invariant of holomorphic line bundles, but the truth is that it is a slightly coarser
invariant -- it cannot distinguish between holomorphic line bundles and smooth complex
line bundles. To see this, let $C^\infty_\C$ denote the sheaf of smooth $\C$-valued
functions on $X$. We note that we also have the exact sequence
\[\begin{tikzcd}
0 \ar[r] & \Z \ar[r] & C_\C^\infty \ar[r, "\mathrm{exp}"] & (C_\C^\infty)^\times \ar[r] & 0
\end{tikzcd}\]
which similarly induces a long exact sequence in cohomology, yielding a boundary
map $\delta : H^1(X, (C^\infty)^\times) \to H^2(X, \Z)$. Using functoriality
of sheaf cohomology, the inclusions $\O_X \hookrightarrow C^\infty$ and
$\O_X^\times \hookrightarrow (C^\infty)^\times$ induce maps on cohomology, giving
us the commutative diagram
\[\begin{tikzcd}
H^1(X,C^\infty_\C) \ar[r] & H^1(X, (C^\infty_\C)^\times) \ar[r, "\delta"] & H^2(X,\Z) \\
H^1(X, \O_X) \ar[r] \ar[u] & H^1(X, \O_X^\times)
\ar[u] \ar[r, "c_1"'] & H^(X, \Z) \ar[u, equal]
\end{tikzcd}\]
with exact rows. Given a holomorphic line bundle $L$, we can forget its complex
structure, regarding it as a complex line bundle, i.e. an element of
$H^1(X, (C^\infty_\C)^\times)$, and then take the image under $\delta$. Commutativity
of the diagram implies that this is the same thing as $c_1(L)$. Since
$C^\infty_\C$ is a fine sheaf, we have that this diagram is actually
\[\begin{tikzcd}
0 \ar[r] & H^1(X, (C_\C^\infty)^\times) \ar[r, "\delta"] & H^2(X,\Z) \\
H^1(X, \O_X) \ar[r] \ar[u] & H^1(X, \O_X^\times)
\ar[u] \ar[r, "c_1"'] & H^(X, \Z) \ar[u, equal]
\end{tikzcd}\]
which tells us that $\delta$ is injective. Therefore, on smooth complex line bundles,
the first Chern class is a perfect invariant -- two complex line bundles are smoothly
isomorphic if and only if they have the same first Chern class. However, this also
reveals to us that it cannot distinguish between different holomorphic structures
on the same underlying complex line bundle. However, we can use the complex geometry as
a tool  to compute certain quantities of a holomorphic bundle, which will turn out to be
topological invariants of the bundle, independent of the holomorphic structure. Our goal
will be to realize the first Chern class of a line bundle as the cohomology class
associated to the curvature of a connection. Eventually, we will add a third characterization
to the mix by introducing the Atiyah class. \\


We first discuss the relationship of $c_1$ with the curvature of a connection on
a complex line bundle. We will discuss connections on complex vector bundles in
general, and will then specialize to the case of line bundles.
%
\begin{defn}
Let $E \to X$ be a complex vector bundle, and let $\mathcal{A}^k(E)$ denote the
sheaf of smooth $E$-valued $k$-forms, i.e. the sheaf of sections of the bundle
$\Lambda^k(T^*X)_\C \otimes E$, which is a sheaf of $\C^\infty_\C$-modules. A
\ib{connection} $\nabla$ is a $\C$-linear map
$\nabla : \mathcal{A}^0 \to \mathcal{A}^1$
where for a smooth function $f$ and local section $s$, we have the \ib{Leibniz rule}
\[
\nabla(fs) = df \otimes s + f\nabla s
\]
\end{defn}
%
A connection can be naturally extended to an operator
$\nabla : \mathcal{A}^k \to \mathcal{A}^{k+1}$, where for a $k$-form $\alpha$
and a local section $s$ of $E$, we define
\[
\nabla(\alpha \otimes s) = d\alpha \otimes s + (-1)^k \alpha \wedge \nabla s
\]
In a similar fashion, the connection $\nabla$ on $E$ induces connection on associated
bundles (e.g. tensor powers, duals). In particular, it induces a connection on the bundle
$\End(E)$, which by abuse of notation we also denote $\nabla$. Given a section $F$ of
$\End(E)$, we have that the action of $\nabla F$ on a section $s$ is given by the formula
\[
(\nabla F)(s) = \nabla(F(s)) - F(\nabla s)
\]
\begin{prop}
Let $\nabla_1$ and $\nabla_2$ be two connections on a complex vector bundle
$E \to X$. Then $\nabla_1 - \nabla_2$ is an element of $\mathcal{A}^1(\End E)$.
\end{prop}
%
\begin{proof}
We want to show that the difference is $C^\infty_\C$-linear. We compute
\begin{align*}
(\nabla_1 - \nabla_2)(fs) &= \nabla_1(fs) - \nabla_2(fs) \\
&= df \otimes s + f\nabla_1s - (df \otimes s + f\nabla_2s) \\
&= f(\nabla_1 - \nabla_2)(s)
\end{align*}
\end{proof}
%
\begin{prop}
Let $a \in \mathcal{A}^1(\End E)$ and let $\nabla$ be a connection on $E$. Then
$\nabla + a$ is a connection on $E$, where given a local
section $s$, we define $\nabla + a$ by
\[
(\nabla + a)s = \nabla s + as
\]
where given a tangent vector $v$, we have that $a$ acts on $s$ by $a(v)s$.
\end{prop}
%
\begin{proof}
We clearly have that $\nabla + a$ is $\C$-linear. To check that it satisfies the
Leibniz rule, we compute
\begin{align*}
(\nabla + a)(fs) &= \nabla(fs) + a(fs) \\
&= df \otimes s + f\nabla s + fas \\
&= df \otimes s + f(\nabla + a)(s)
\end{align*}
\end{proof}
%
Therefore, given a complex vector bundle $E \to X$ equipped with connection, the
connection is locally of the form $d + A$, where $d$ is the usual complexified
de Rham differential and $A$ is a matrix of $1$-forms.
%
\begin{defn}
Let $E \to X$ be a complex vector bundle equipped with a connection $\nabla$.
The \ib{curvature} of $\nabla$ is defined to be the map
$F_{\nabla} \defeq \nabla^2 : \mathcal{A}^0(E) \to \mathcal{A}^2(E)$.
\end{defn}
%
One might expect the curvature to be a second order differential operator, but a
small miracle happens.
%
\begin{prop}
The curvature transformation $F_\nabla$ is $C^\infty_\C$-linear, i.e. it defines
a global section of $\mathcal{A}^2(\End E)$.
\end{prop}
%
\begin{proof}
We compute for a function $f$ and local section $s$
\begin{align*}
F_\nabla(fs) &= \nabla(\nabla(fs)) \\
&= \nabla(df \otimes s + f\nabla s) \\
&= d^2f \otimes s - df \wedge \nabla s + \nabla(f\nabla s) \\
&= -df \wedge \nabla s + df \wedge \nabla s + f\nabla^2 s \\
&= f\nabla^2 s \\
&= fF_\nabla(s)
\end{align*}
\end{proof}
%
\begin{thm}[\ib{The Bianchi Identity}]
Let $F_\nabla$ be the curvature form of a connection on a vector bundle $E \to X$.
Then
\[
\nabla(F_\nabla) = 0
\]
\end{thm}
%
\begin{proof}
Let $s$ be a section of $E$. By the definition of the induced connection on $\End(E)$, we
compute
\begin{align*}
(\nabla(F_\nabla))(s) &= \nabla(F_\nabla(s)) - F_\nabla(\nabla s) \\
&= \nabla^3s - \nabla^3s \\
&= 0
\end{align*}
\end{proof}
%
Finally we make one more observation
%
\begin{prop}
Let $\nabla$ be a connection on $E \to X$, and $\nabla' = \nabla + A$. Then
\[
F_{\nabla'} = F_\nabla + dA + A \wedge A
\]
\end{prop}
%
\begin{proof}
We compute locally for a section $s$,
\begin{align*}
F_{\nabla'} &= (d + A)((d+A)(s)) \\
&= (d+A)(ds + A(s)) \\
&= d^2s + A(ds) + d(A(s)) + (A \wedge A)(s) \\
&= dA(s) + (A \wedge A)(s)
\end{align*}
\end{proof}
%
In the case of a line bundle $L \to X$, much of this discussion simplifies. First of all
we have that the bundle $\End(L)$ is a trivial bundle, so we have that every
connection is locally of the form $d + A$ for a $1$-form $A$. This allows several
small miracles to occur.
%
\begin{prop}
The curvature $F_\nabla$ of a connection on a complex line bundle $L \to M$
is a globally defined $2$-form in $H^2(X, \C)$.
\end{prop}
%
\begin{proof}
We have that locally, $\nabla = d + A$, so we have locally
\[
F_\nabla =  dA + A \wedge A = dA
\]
We then need to show that these local definitions glue. In two trivializing
neighborhoods $U_i$ and $U_j$ with transition functions $\psi_{ij}$ and $\psi_{ji}$,
let $A_i$ and $A_j$ denote the connection $1-$forms over each trivialization.
Let $s$ be a local section over $U_i \cap U_j$ that is nonzero and constant when viewed as a
section of the trivial bundle $U_i \cap U_j \times \C$. Therefore, we have that
Then the local representations $s_i$ and $s_j$ are related by $\psi_{ij}s_j = s_i$.
In addition, we have that over $U_i$, $\nabla s = ds_i + A_is_i = A_is_i$, since
$s_i$ is a constant function, and we have a similar picture over $U_j$. We then
compute
\[
A_i\psi_{ij}s_j = \nabla(\psi_{ij}s_j) = d\psi_{ij}s_j + \psi_{ij}\nabla s_j
= d\psi_{ij}s_j + \psi_{ij}A_js_j
\]
Therefore, we have
\[
A_i\psi_{ij} = d\psi_{ij} + \psi_{ij}A_j
\]
Multiplying both sides by $\psi_{ij}\inv$, moving terms around, we find
\[
A_j = A_i - \psi_{ij}\inv d\psi_{ij}
\]
We then verify that the local descriptions of $F_\nabla$ agree on the intersection. We
compute
\begin{align*}
F_\nabla &= dA_i \\
&= \psi_{ij}\inv d\psi_{ij} \wedge \psi_{ij}\inv d\psi_{ij} + dA_i \\
&= \psi_{ij}^{-2}d\psi_{ij} \wedge d\psi_{ij} + dA_i \\
&= -d\psi_{ij}\inv \wedge d\psi_{ij} + dA_i \\
&= d(A_i - \psi_{ij}\inv d\psi_{ij}) \\
&= dA_j
\end{align*}
Where we use the fact that $d(\psi_{ij}\inv) = \psi_{ij}^{-2}d\psi_{ij}$. Therefore,
the local descriptions of $F_\nabla$ glue to a global $2$-form.
\end{proof}
%
In addition, since we have locally that $F_\nabla$ is given by $dA_i$, we have that
$F_\nabla$ is closed, so it defines a cohomology class in $H^2(X,\C)$. This gives us
two ways of obtaining a cohomology class from a line bundle. The first is by taking
the first Chern class from the exponential exact sequence, and the other is
by taking the curvature form of a connection.
%
\begin{thm}
Let $L \to X$ be a complex line bundle, and let $\nabla$ be any connection on $L$ with
curvature form $F_\nabla$. Then
\[
\left[\frac{i}{2\pi}F_\nabla\right] = c_1(L)
\]
\end{thm}
%
\begin{proof}
We first give an explicit formula for the \v{C}ech cocycle defining $c_1(L)$.
Fix a good cover $\mathcal{U} = \set{U_i}$ for $X$, and let the $\psi_{ij}$ be the
transition functions for $L$ with respect to this covering. The $\psi_{ij}$
determine a \v{C}ech cocycle in $H^1(X, (C^\infty_\C)^\times)$. By surjectivity of
$\exp$, by passing to a finer cover we may assume that we can find a branch of the
logarithm for each $\psi_{ij}$, so $\log\psi_{ij}$ is well defined, and defines a
\v{C}ech cocycle in $H^1(X, (C^\infty)^\times)$. Then since
the $\psi_{ij}$ satisfy the cocycle condition
\[
\psi_{ij}\psi_{jk}\psi_{ik}\inv = 1
\]
we have that
\[
\log\psi_{ij} + \log\psi{jk} - \log\psi_{ik} \in 2\pi i\Z
\]
So we have that
\[
\frac{1}{2\pi i}\log\psi_{ij} + \log\psi_{jk} - \log\psi_{ik} \in \Z
\]
is the cocycle representing $c_1(L)$. In the natural map
$H^2(X,\Z) \to H^2(X,\C)$, we have that the cocycle has the same formula (up to a sign
depending on convention). \\

Then let $\nabla$ be any connection on $L$, with local connection forms $A_i$.
We have that $F_\nabla$ is a closed $2$-form, with local description $F_\nabla = dA_i$.
Let $\mathcal{A}^k(X)$ denote the sheaf of smooth $k$-forms over $X$, and
let $\mathcal{Z}^k(X)$ denote the sheaf of closed smooth $k$-forms over $X$.
Then we have the exact sequences of sheaves
\[\begin{tikzcd}
0 \ar[r] & \mathcal{Z}^1(X) \ar[r] & \mathcal{A}^1(X) \ar[r, "d"] & \mathcal{Z}^2(X)
\ar[r] & 0
\end{tikzcd}\]
\[\begin{tikzcd}
0 \ar[r] & \C \ar[r] & \mathcal{A}^0(X) \ar[r, "d"] & \mathcal{Z}^1(X) \ar[r] & 0
\end{tikzcd}\]
which give rise to boundary homomorphisms
$\delta_1 : H^0(X, \mathcal{Z}^2(X)) \to H^1(X, \mathcal{Z}^2(X))$ and
$\delta_2 : H^1(X, \mathcal{Z}^1(X)) \to H^2(X, \C)$ respectively. The
fact that the local definitions $F_\nabla = dA_i$ glue to a global form
is exactly the statement that the $dA_i$ form a \v{C}ech cocycle in
$H^0(X, \mathcal{Z}^2(X))$, i.e. $dA_i - dA_j = 0$ for all $i,j$ with
$U_i \cap U_k \neq \emptyset$. Following the definition of the boundary homomorphism,
this implies that it is the image of the cocycle determined by the $A_i$ under
$d$, and then applying the \v{C}ech differential we get that
$\delta_1(F_\nabla)$ is the cocycle $\set{A_j - A_i}$. Then recall from our earlier
calculation that $A_j = A_i - \psi_{ij}\inv d\psi_{ij}$, we get that
\[
A_j - A_i = -\psi_{ij}\inv d\psi_{ij}
\]
which is exactly $-d\log\psi_{ij}$. Then tracing though the definition of the
boundary homomorphism $\delta_2$, we have that $\delta_2(A_j - A_i)$
is the preimage of the \v{C}ech differential applied to the cocycle
$\log\psi_{ij}$, which is exactly $-(\log\psi_{ij} + \log\psi_{jk} - \log\psi_{ik})$.
Putting everything together, we find that
\[
c_1(L) = \frac{1}{2\pi i}\log\psi_{ij} + \log\psi_{jk} - \log\psi_{ik}
= \left[ \frac{i}{2\pi}F_\nabla \right]
\]
\end{proof}
%
%
\section{The Chern Connection}
%
\begin{defn}
Let $E \to X$ be a complex vector bundle. A \ib{Hermitian structure} on $E$
is the data of a smoothly varying hermitian form $H(x)$ on each fiber $E_x$.
If $E$ is a complex vector bundle equipped with a Hermitian structure, we call it
a \ib{Hermitian vector bundle}.
\end{defn}
%
Hermitian structures always exist due to convexity of the space of Hermitian forms.
Therefore, we can define Hermitian forms locally in neighborhoods where the bundle
$E$ is trivial, and then use a partition of unity to glue them together.
%
Like in the case of Riemannian geometry, we can ask for a connection to be compatible
with a Hermitian structure.
%
\begin{defn}
Let $E \to X$ be a Hermitian vector bundle. A connection $\nabla$ on $E$ is a
\ib{Hermitian connection} if for sections $s_1, s_2$, we have
\[
dh(s_1,s_2) = h(\nabla s_1, s_2) + h(s_1,\nabla s_2)
\]
\end{defn}
%
In the complex world, we can also impose a holomorphic structure on vector bundles,
and we can also ask for a compatibility condition with the holomorphic structure.
%
\begin{defn}
Let $E \to X$ be a holomorphic structure. A connection $\nabla$ on $E$ is said
to be \ib{compatible with the holomorphic structure} if the $(0,1)$ component
$\nabla^{0,1}$ is equal to $\dbar_E$.
\end{defn}
%
This leads us to a natural question : given a holomorphic vector bundle $E \to X$
equipped with a Hermitian structure $h$, can we find a connection $\nabla$ that
is compatible both with the Hermitian structure and the holomorphic structure?
%
\begin{thm}[\ib{The Chern connection}]
Let $E \to X$ be a Hermitian holomorphic vector bundle. There exists a unique
Hermitian connection $\nabla$, called the \ib{Chern connection} that is compatible
with the holomorphic structure.
\end{thm}
%
\begin{proof}
The proof has a similar flavor to other proofs of existence and uniqueness of global
objects. We use the compatibility conditions to determine local conditions that
the connection must satisfy, and show that it uniquely specifies the connection.
Therefore, the local definitions glue to a global one. In a local trivialization
of $E$, we have that any connection $\nabla$ is of the form $d + A$ for a matrix
valued $1$-form $A$. In addition, we have that the Hermitian structure gives a
Hermitian form $H(x)$ at each point $x$. Let $\set{E_i}$ the local
frame for $E$, where each $E_i$ is just the constant section $x \mapsto e_i$. Then we
get a coordinate representation of the Hermitian form $H$.
$H^i_j(x) = h(E_i,E_j)$. We clearly have that $dE_i = 0$, since all the sections
are constant, so $\nabla E_i = AE_i$. The the condition that $\nabla$ is Hermitian gives
us
\[
dH^i_j  = dH(E_i,E_j) = h(AE_i, E_j) + h(E_i, AE_j)
\]
Writing the matrix $A$ in coordinates $A^i_j$, this gives us
\[
dH^i_j = \overline{A^i_k}h(E_k, E_j) + h(E_i, E_k)A^k_j
= \overline{A^i_k}H^k_j + H^i_kA^k_j
\]
which tells us that we must have that
\[
dH = A^\dagger H + HA
\]
We then consider the restrictions imposed on $\nabla$ by the condition that
$\nabla$ is compatible with the holomorphic structure. Decomposing
$\nabla = \nabla^{1,0} + \nabla^{0,1}$, we want $\nabla^{0,1} = \dbar_E$. In our local
picture, we have that $\nabla = \partial + \dbar + A$, and $\dbar_E$ is just $\dbar$.
Therefore, we must have that $A^{0,1} = 0$, so $A = A^{1,0}$. Decomposing
$d = \partial + \dbar$ and applying it to our Hermitian condition, we find that
\[
\partial H + \dbar H = (A^{1,0})^\dagger H + HA^{1,0}
\]
comparing the bidegrees then gives
\begin{align*}
\partial H &= HA^{1,0}\\
\dbar H &= (A^{1,0})^\dagger H
\end{align*}
Moving $H$ to the other side of the equation and conjugating the equation gives
\[
\partial\overline{H} (\overline{H\inv}) = (A^{1,0})^T = A^T
\]
which gives a unique characterization of the matrix $A$, as well as a local definition,
showing existence and uniqueness of the Chern connection.
\end{proof}
%
Unsurprisingly, if we have a connection compatible with a Hermitian or holomorphic
structure, then the corresponding curvature forms are also compatible.
%
\begin{prop} \enumbreak
\begin{enumerate}
  \item Let $E \to X$ be a Hermitian vector bundle with Hermitian connection $\nabla$.
  Then $F_\nabla$ is locally given by a skew-Hermitian matrix $C$ of $2$-forms.
  Globally, we have
  \[
  h(F_\nabla s_1, s_2) + h(s_1, F_\nabla s_2) = 0
  \]
  \item Let $E \to X$ be a holomorphic vector bundle. Then $F_\nabla$ has
  no $(0,2)$ part, i.e. $F_\nabla = F^{2,0}_\nabla + F^{1,1}_\nabla$.
\end{enumerate}
\end{prop}
%
\begin{proof} \enumbreak
\begin{enumerate}
  \item Locally $E$ is isomorphic to the trivial bundle with the standard Hermitian
  structure, so $\nabla = d + A$ for a skew-Hermitian matrix $A$. Then using
  the local formula
  \[
  F_\nabla = dA + A \wedge A
  \]
  we want to show that $F_\nabla^\dagger = -F_\nabla$. We compute
  \begin{align*}
  F_\nabla^\dagger &= d(A^\dagger) + (A \wedge A)^\dagger \\
  &= d(A^\dagger) - A^\dagger \wedge A^\dagger
  \end{align*}
  The fact that $(A\wedge A)^\dagger = -A^\dagger \wedge A^\dagger$ is slightly tricky,
  and comes from the skew symmetry of the wedge product. Explicitly in components,
  we have
  \begin{align*}
  ((A \wedge A)^\dagger)^i_j &= (\overline{A} \wedge \overline{A})^j_i \\
  &= \overline{A}^j_k \wedge \overline{A}^k_i \\
  &= -\overline{A}^i_k \wedge \overline{A}^k_j \\
  &= - (A^\dagger \wedge A^\dagger)^i_j
  \end{align*}
  We the complete the computation, using the fact that $A$ is skew-Hermitian
  \begin{align*}
  F_\nabla^\dagger &= d(A^\dagger) - A^\dagger \wedge A^\dagger \\
  &= -dA - A \wedge A \\
  &= -F_\nabla
  \end{align*}
  \item Locally, we may write $\nabla = d + A$ Since $\nabla$ is compatible with the
  holomorphic structure, we also have that $A = A^{1,0}$. We have that locally
  \[
  F_\nabla = dA + A \wedge A
  \]
  Splitting $d = \partial + \dbar$, we get
  \[
  F_\nabla = (\partial + \dbar)(A) + A \wedge A = \dbar A + \partial A + A \wedge A
  \]
  and since $A$ is type $(1,0)$, we get that $\dbar A$ is type $(1,1)$ and
  $\partial A + A \wedge A$ is type $(2,0)$.
\end{enumerate}
\end{proof}
%
Putting these together, we get a result on the type of the curvature of a Chern connection.
%
\begin{prop}
Let $\nabla$ be the Chern connection for a Hermitian holomorphic vector bundle
$E \to X$. Then $F_\nabla$ is type $(1,1)$.
\end{prop}
%
\begin{proof}
Since $\nabla$ is Hermitian, we have that $F_\nabla^\dagger = - F_\nabla$.
Since $\nabla$ is compatible with the holomorphic structure, we have that
$F_\nabla = F^{2,0}_\nabla + F^{1,1}_\nabla$. Then consider
\[
F_\nabla^\dagger = (F^{2,0}_\nabla)^\dagger + (F^{1,1}_\nabla)^\dagger
\]
We have that $(F^{2,0}_\nabla)^\dagger$ is type $(0,2)$ and $(F^{1,1}_\nabla)^\dagger$
is type $(1,1)$. Therefore, in order to have $F_\nabla^\dagger = -F_\nabla$, we
must have that $(F^{1,1}_\nabla)^\dagger = -F^{1,1}_\nabla$ and $F^{2,0}_\nabla = 0$
by a simple type check.
\end{proof}
%
This yields an important result regarding the curvature of of Chern connection.
%
\begin{thm}
The curvature form $F_\nabla$ of a Chern connection on a holomorphic vector
bundle $E \to X$ is $\dbar_E$-closed, so it defines a Dolbeault cohomology class
$[F_\nabla] \in H^1(X, \Omega_X \otimes \End(E))$, where $\Omega_X$ denotes the
bundle of holomorphic $1$-forms.
\end{thm}
%
\begin{proof}
Applying the Bianchi Identity to the curvature form $F_\nabla = F_\nabla^{1,1}$,
we find
\begin{align*}
0 = \nabla(F_\nabla)^{1,2} = \dbar_E F_\nabla
\end{align*}
where we use the fact that the induced connection on $\End(E)$ is also
compatible with the holomorphic structure.
\end{proof}
%
This gives some intuition behind the Chern connection. For an arbitrary connection on
a holomorphic vector bundle $E$, we need not have that the curvature is
$\dbar_E$-closed. This shows that asking for a connection to be a Chern connection
is a sort of integrability condition. In addition, it gives a generalization of the
first Chern class for line bundles to holomorphic bundles of arbitrary rank. In
the line bundle case, we did not have to worry about curvature being closed,
but in the higher dimensional case, we have to restrict our attention to Chern
connections.

Given a holomorphic vector bundle $E \to X$, we now get a recipe for producing Dolbeault
cohomology classes in $H^1(X, \Omega_X \otimes \End(E))$. In addition, We can
place a Hermitian structure on $E$, and then compute the curvature of the Chern
connection. Naturally, the next question is whether this cohomology class
depends on the choice of Hermitian structure.
%
\begin{thm}
The cohomology class $[F_\nabla]$ of the curvature of a Chern connection on
a holomorphic vector bundle $E$ is independent on the choice of Hermitian structure.
\end{thm}
%
\begin{proof}
Let $\nabla$ be the Chern connection with respect to some Hermitian structure
on $E$. Then any other connection is of the form $\nabla + A$ with curvature
$F_{\nabla + A} = F_\nabla + \nabla A + A \wedge A$. Then if
$\nabla' = \nabla + A$ is the Chern connection with respect to a different Hermitian
structure, then $A$ must be type $(2,0)$. In addition, we know that both
$F_\nabla$ and $F_{\nabla'} = F_\nabla + \nabla A + A \wedge A$ are both type $(1,1)$.
In order for this to be true, we must have that $\nabla A + A \wedge A$ is
of type $(1,1)$. Since $A \wedge A$ is of type $(2,0)$, we get
\[
(\nabla A + A \wedge A) = (\nabla A \wedge A)^{1,1} = \nabla A
\]
Then since $A$ is type $(1,0)$, the $(1,1)$ part of $A$ is $\nabla^{0,1}A = \dbar_E A$
since $\nabla$ is compatible with the holomorphic structure. Therefore,
we have that $\nabla A + A \wedge A = \dbar_E A$, giving us
\[
F_{\nabla'} = F_\nabla + \dbar_E A
\]
so $[F_{\nabla'}] = [F_\nabla]$.
\end{proof}
%
\begin{defn}
Let $E \to X$ be a holomorphic vector bundle of rank $k$, and
$\mathcal{U} = \set{U_i}$ a good cover of $X$ with local trivializations
$\varphi_i : E\vert_{U_i} \to U_i \times \C^k$ and transition functions
$\psi_{ij} : U_i \cap U_j \to \GL_k\C$. The \ib{Atiyah class} of $E$, denoted
$A(E)$ is the element $A(E) \in H^1(X, \Omega_X \otimes \End E)$ determined by
the cocycle $\sigma_{ij} = \varphi_j\inv \circ (\psi_{ij}\inv d\psi_{ij}) \circ \varphi_j$,
where $\Omega_X$ denotes the bundle of holomorphic $1$-forms.
\end{defn}
%
We first make sense of the formula for the cocycle. A component $\sigma_{ij}$ of
the \v{C}ech cocycle $\sigma$ should be a local section
$\sigma_{ij} : U_i \cap U_j \to \Omega_X \otimes \End E$. What this means
is that we should be able to feed $\sigma_{ij}$ the following data:
\begin{enumerate}
  \item A point $p \in U_i \cap U_j$
  \item A tangent vector $v \in T_pX$
  \item A vector $w \in E_p$ in the fiber
\end{enumerate}
%
and we should expect another vector in $E_p$ to be the output. We have that
$\varphi(p,w)$ should be a pair $(p, w') \in U_i \cap U_j \times \C^k$. Then
since $\psi_{ij} : U_i \cap U_j \to \GL_k\C$, we have that
$d(\psi_{ij})_p : T_pX \to M_n\C$. Therefore, we have that
$d(\psi_{ij})_p(v)$ is a matrix, which we can then apply to $w'$.
Then we can apply the matrix inverse $(\psi_{ij}(p))\inv$ to the result, leaving
us with some other pair $(p,w'') \in U_i \cap U_j \times \C^k$, and then applying the
inverse of the local trivialization $\varphi_j$ to this gives us the desired output.
The fact that this defines a cocycle essentially boils down to the fact that the
$\psi_{ij}$ satisfy the cocycle condition, along with using the chain rule for
the differentials $d\psi_{ij}$.
%
%TODO Chern classes and Chern-Weil
%
%TODO Line bundles and maps to projective space
%
%
\section{Vanishing Theorems}
%
\iffalse
\emph{It was said about Lefschetz that he had never stated a false theorem or gave a
correct proof.} \\
\fi

For the following discussion, we will restrict our attention to a compact K\"ahler
manifold $X$, which will allow us to use some tools from Hodge theory, which we will
assume.
%
\begin{defn}
A \ib{positive $(1,1)$ form} on $X$ is a
differential form $\omega \in \mathcal{A}^{1,1}$ such that in local
holomorphic coordinates $\set{z^i}$, we have that
\[
\omega = \frac{i}{2} \sum_{i,j} h_{ij}(z) dz^i \wedge d\zbar^j
\]
where $h_{ij}$ is a positive definite Hermitian matrix for all $z$. Equivalently,
$\omega$ is a positive form if and only if it represents the K\"ahler form for some
Hermitian metric on $X$.
\end{defn}
%
\begin{defn}
A holomorphic line bundle $L \to X$ is \ib{positive} if there exists a Hermitian metric
$h$ on $X$ with Chern connection $\nabla$ such that $(i/2\pi)F_\nabla$ is a
positive $(1,1)$ form.
\end{defn}
%
Positivity of a line bundle turns out to only depend on the first Chern class of
the line bundle. From our previous discussion, this tells us that it only cares
about the topological isomorphism class of the line bundle, and is independent of
a holomorphic structure. To show this, we'll need two useful lemmas.
The first is a classic lemma from K\"ahler geometry.
%
\begin{lem}[\ib{The $\partial\dbar$-lemma}]
Let $\eta$ be a smooth complex valued form such that $\eta$ is both $\partial$ and
$\dbar$ closed. Then if $\eta$ is $d$, $\partial$, or $\dbar$-exact, then there
exists a form $\xi$ such that $\eta = \partial\dbar\xi$. Furthermore, if $\eta$ is
real (i.e. $\overline{\eta} = \eta$), we may take $i\xi$ to be real.
\end{lem}
%
The other lemma characterizes curvatures of metric compatible connection on line
bundles.
%
\begin{lem}
Let $L \to X$ be a holomorphic line bundle equipped with Hermitian metric $h$. Locally,
$h$ is given by a positive function $h(z)$ on $X$. Then the curvature $F_\nabla$ of
the metric compatible connection $\nabla$ is given by the formula
\[
F_\nabla = -\partial\dbar\log h(z)
\]
\end{lem}
%
\begin{proof}
This follows from our earlier computations for Chern connections. Locally, the Chern
connection $\nabla$ is given by $d + A$ for a complex $1$-form $A$, which
from our earlier computations, must be holomorphic and satisfy
\[
A = \overline{h}\inv \partial \overline{h}
\]
However, since $h$ is real, this is the same as $A = h\inv\partial h$.
The curvature is then given by $F_\nabla = dA = (\partial A + \dbar A)$. However,
we note that by a type check, $\partial A$ is of type $(2,0)$, and since the curvature
of the Chern connection is of type $(1,1)$, we have that
\[
F_\nabla = \dbar\left(\frac{\partial h}{h}\right) = \dbar\partial\log h
\]
the desired identity then follows from the fact that $\partial$ and $\dbar$ anticommute.
\end{proof}
%
\begin{prop}
Let $L \to X$ be a line bundle and let $\omega$ be a real closed $(1,1)$ form such
that the cohomology class of $\omega$ is equal to $c_1(L)$. Then there exists a
Hermitian metric on $L$ with metric compatible connection $\nabla$ such that
$\omega = (i/2\pi)F_\nabla$.
\end{prop}
%
\begin{proof}
Let $h$ be any Hermitian metric on $L$, and $\nabla$ the Chern connection for $h$.
We know that the curvature $F_\nabla$ satisfies
\[
\left[\frac{i}{2\pi}F_\nabla\right] = c_1(L)
\]
Then suppose we have another Hermitian metric $h'$ with Chern connection $\nabla'$.
We know that we can write $h'$ differs from $h$ by multiplication by a smooth positive
function, so we may write $h' = e^\rho h$ for some smooth function $\rho$. Then the
curvature of $\nabla'$ satisfies
\begin{align*}
F_{\nabla'} &= -\partial\dbar\log e^\rho h \\
&= -\partial\dbar(\log e^\rho + \log h) \\
&= -\partial\dbar\rho + F_\nabla
\end{align*}
Which tells us that $F_\nabla = \partial\dbar\rho + F_{\nabla'}$.
Then given a real closed $(1,1)$ form $\omega$ with $[(i/2\pi)\omega] = c_1(L)$,
if we solve the equation
\[
F_\nabla = \partial\dbar\rho + \omega
\]
Then the Chern connection $\nabla'$ of the Hermitian form $e^\rho h$ will satisfy
$F_{\nabla} = \partial\dbar\rho + F_{\nabla'}$, which then implies that
$F_{\nabla'} = \omega$. To solve the equation, we note that $\omega$ and
$F_\nabla$ are cohomologous by assumption, so their difference $F_\nabla - \omega$
is $d$-exact. Furthermore, from the local formula
\[
F_\nabla = -\partial\dbar\log h
\]
we see that $F_\nabla$ is both $\partial$ and $\dbar$-closed. Finally, the
fact that $\omega$ is type $(1,1)$ and $d$-closed implies that it must be both
$\partial$ and $\dbar$-closed by a simple type check, so we get that
$F_\nabla - \omega$ satisfies the conditions of the $\partial\dbar$-lemma,
giving us that $F_\nabla - \omega = \partial\dbar\rho$ for some real function
$\rho$.
\end{proof}
%
Recall that on a K\"ahler manifold $X$ with K\"ahler form $\omega$, we have the
Lefschetz operator $L : \mathcal{A}^{p,q} \to \mathcal{A}^{p+1,q+1}$ given by
wedging with $\omega$. The operator $L$, together with its adjoint
$\Lambda : \mathcal{A}^{p,q} \to \mathcal{A}^{p-1,q-1}$
satisfy the identities
\begin{enumerate}
  \item $[\Lambda, L] = (n-(p+q))\id_{\mathcal{A}^{p,q}(X)}$
  \item $[\Lambda,\dbar] = -i\partial^*$
\end{enumerate}
%
where $n$ is the complex dimension of $X$. The operators $\Lambda$ and $L$ can
naturally be extended to vector bundle valued forms, and since $\Lambda$ and
$L$ are both $0^{th}$ order operators, the first identity still holds for
their extended versions. The second identity generalizes in a slightly modified
form.
%
\begin{prop}[\ib{Nakano identity}]
Let $E$ be a holomorphic vector bundle equipped with a Hermitian metric with
Chern connection $\nabla$. Then
\[
[\Lambda, \dbar_E] = -i(\nabla^{1,0})^*
\]
where $(\nabla^{1,0})^*$ is the adjoint of $\nabla^{1,0}$ with respect to the
Hermitian inner product on $\mathcal{A}^{p,q}(E)$, which is explicitly given by the
formula
\[
(\nabla^{1,0})^* = \overline{\star}_E\nabla^{1,0}_{E^*}\overline{\star}_E
\]
where $\nabla_{E^*}$ is the Chern connection for the dual bundle equipped with
the dual Hermitian form.
\end{prop}
%
\begin{proof}
Since it suffices to check in a local trivialization, we may assume that we have
taken an orthonormal frame for $E$. However, this cannot be a holomorphic trivialization,
so there will be some complications. We take inventory of the various operators in
this trivialization. Since the trivialization was from an orthonormal frame, the
operator $\star_E$ agrees with the usual Hodge star, so
$\overline{\star}_E = \overline{\star}$. In addition, we can write the Chern connection
$\nabla$ as $d + A$ for some skew-Hermitian matrix of $1$-forms $A$. This then gives
us
\[
\nabla_{E^*} = d + A^\dagger = d - A
\]
The $\dbar_E$-operator is of the form  $\dbar + A^{0,1}$, which is one of the
added complications when the trivialization is not holomorphic. If the trivialization
were holomorphic, we would have had $\dbar_E = \dbar$. Putting things together, we find
\begin{align*}
(\nabla^{1,0})^* &= \overline{\star}_E\nabla_{E^*}^{1,0}\overline{\star}_E  \\
&= \overline{\star}(\partial - A^{1,0})\overline{\star} \\
&= \partial^* - (A^{1,0})^\dagger
\end{align*}
We then compute
\begin{align*}
[\Lambda,\dbar_E] + i(\nabla^{1,0})^* &=
[\Lambda, \dbar + A^{0,1}] + i(\partial^* - (A^{1,0})^\dagger) \\
&= [\Lambda,\dbar] + [\Lambda,A^{0,1}] + i\partial^* - i(A^{1,0})^\dagger
\end{align*}
The ordinary K\"ahler identity then gives us that this is equal to
$[\Lambda,A^{0,1}] - i(A^{1,0})^\dagger$. Since this is an order $0$ operator,
it suffices to verify that it vanishes when restricted to any fiber of $E$. However,
since we can always pick a smooth local trivialization of $E$ in which $A = 0$,
we have that $[\Lambda,A^{0,1}] - i(A^{1,0})^\dagger = 0$, which them
implies that
\[
[\Lambda,\dbar_E] = -i(\nabla^{1,0})^*
\]
\end{proof}
%
We then prove one final lemma.
%
\begin{lem}
Let $E$ be a holomorphic vector bundles with a Hermitian metric and Chern connection
$\nabla$. Let $(\cdot,\cdot)$ denote the Hermitian inner product on
$\mathcal{A}^{p,q}(E)$ induced by the Hermitian metric on $E$ and the Hermitian metric
on $X$. Then for a harmonic form $\alpha \in \mathcal{H}^{p,q}(E)$, we have the
following inequalities:
\begin{align*}
i(F_\nabla\Lambda\alpha,\alpha) &\leq 0 \\
i(\Lambda F_\nabla\alpha,\alpha) &\geq 0
\end{align*}
\end{lem}
%
\begin{proof}
Since the Chern connection satisfies $\nabla^{0,1} = \dbar_E$ and the curvature
$F_\nabla$ is of type $(1,1)$. we know that
\[
F_\nabla = \nabla^{1,0}\dbar_E + \dbar_E\nabla^{1,0}
\]
We then compute
\begin{align*}
i(F_\nabla\Lambda\alpha,\alpha) &= i(\nabla^{1,0}\dbar_E\Lambda\alpha,\alpha)
+ i(\dbar_E\nabla^{1,0}\Lambda\alpha,\alpha) \\
&= i(\dbar_E\Lambda\alpha,(\nabla^{1,0})^*\alpha)
+ i(\nabla^{1,0}\Lambda\alpha,\dbar_E^*\alpha) \\
&= i(\dbar_E\Lambda\alpha,(\nabla^{1,0})^*\alpha)
+ i(\nabla^{1,0}\Lambda\alpha,0) \\
&= -(\dbar_E\Lambda\alpha, i(\nabla^{1,0})^*\alpha) \\
&= (\dbar_E\Lambda\alpha, [\Lambda,\dbar_E]\alpha) \\
&= (\dbar_E\Lambda\alpha, \Lambda\dbar_E\alpha)
- (\dbar_E\Lambda\alpha,\dbar_E\Lambda\alpha) \\
&= -\norm{\dbar_E\Lambda\alpha}_{L^2}^2 \leq 0
\end{align*}
We perform a similar computation for the other inequalty.
\begin{align*}
i(\Lambda F_\nabla\alpha,\alpha) &= i(\Lambda\nabla^{1,0}\dbar_E\alpha,\alpha)
+ i(\Lambda\dbar_E\nabla^{1,0}\alpha,\alpha) \\
&= i(\Lambda\dbar_E\nabla^{1,0}\alpha,\alpha) \\
&= i([\Lambda,\dbar_E]\nabla^{1,0}\alpha,\alpha)
+i(\dbar_E\Lambda\nabla^{1,0}\alpha,\alpha) \\
&= i(-i(\nabla^{1,0})^*\nabla^{1,0}\alpha,\alpha)
+ i(\Lambda\nabla^{1,0}\alpha,\dbar_E^*\alpha) \\
&= (\nabla^{1,0}\alpha,\nabla^{1,0}\alpha) \\
&= \norm{\nabla^{1,0}\alpha}_{L^2}^2 \geq 0
\end{align*}
\end{proof}
%
With all that done, we are ready to prove the Kodaira-Nakano vanishing theorem.
%
\begin{thm}[\ib{Kodaira-Nakano Vanishing}]
Let $L \to X$ be a positive line bundle over a compact K\"ahler manifold $X$ of complex
dimension $n$. Then for $p+1 > n$, we have
\[
H^q(X,\Omega^{p,0}_X\otimes L) = 0
\]
\end{thm}
%
\begin{proof}
From Hodge theory we know that $H^q(X,\Omega_X^{p,0} \otimes L)$ is isomorphic
to the space $\mathcal{H}^{p,q}(L)$ of harmonic $L$-valued $(p,q)$-forms. Therefore,
it suffices to show that no harmonic $(p,q)$-forms exist when $p+q > n$. Since
$L$ is a positive line bundle, it admits a Hermitian metric with such that
$(i/2\pi)F_\nabla$ is equal to the K\"ahler form. Therefore, its action on
$\mathcal{A}^{p,q}(L)$ is the same as the Lefschetz operator $L$. Using the
commutation relation
\[
[\Lambda,L] = (n-(p+q))\id_{\mathcal{A}^{p,q}(L)}
\]
we get that
\[
\frac{i}{2\pi}([\Lambda,F_\nabla]\alpha,\alpha)
= ([\Lambda,L]\alpha,\alpha) = (n-(p+q))\norm{\alpha}^2_{L^2}
\]
However, we know that
\[
i([\Lambda,F_\nabla]\alpha,\alpha) = i(\Lambda F_\nabla\alpha,\alpha)
- i(F_\nabla\Lambda\alpha,\alpha) \geq 0
\]
So if $p+q > n$, we must have that $\norm{\alpha}^2_{L^2} = 0$, i.e. $\alpha = 0$.
\end{proof}
%
The lemmas we proved earlier also give rise to another vanshing theorem due to
Serre.
%
\begin{thm}[\ib{Serre Vanishing}]
Let $L \to X$ be a positive line bundle. Then for any holomorphic vector bundle
$E \to X$, there exists a positive integer $m_0$ such that
\[
H^q(X,E \otimes L^m) = 0
\]
For any $m \geq m_0$.
\end{thm}
%
\begin{proof}
We first fix Hermitian metrics on $E$ and $L$, where the Hermitian metric on $L$ is
chosen such that the Chern connection $\nabla_L$ satisfies
$(i/2\pi)F_{\nabla_L} = \omega$, where $\omega$ is the K\"ahler form of $X$. We
note that an easy calculation gives us that $\nabla_{L^m} = m\omega$.
Then let $\nabla_E$ denote the Chern connection for $E$, and let
\[
\nabla = \nabla_E \otimes \nabla_{L^m} = \nabla_E \otimes \id + \id \otimes \nabla_{L^m}
\]
The curvature of $\nabla$ is then given by
\[
F_\nabla = F_{\nabla_E} \otimes \id + m(\id \otimes \omega)
\]
Using our lemmas from before, we then compute for
$\alpha \in \mathcal{H}^{p,q}(E \otimes L^m)$
\begin{align*}
i([\Lambda,F_\nabla]\alpha,\alpha) &= ([\Lambda,F_{\nabla_E}]\alpha,\alpha)
+ i([\Lambda,mL]\alpha,\alpha) \\
&= i([\Lambda,F_{\nabla_E}]\alpha,\alpha) + (m(n-(p+q)))\norm{\alpha}^2_{L^2}
\end{align*}
The operator $[\Lambda,F_{\nabla_E}]$ is algebraic, so we can use a fiberwise
Cauchy-Schwarz inequality
\[
|\langle[\Lambda,F_{\nabla_E}]\alpha,\alpha\rangle|
\leq \norm{[\Lambda,F_{\nabla_E}]}\norm{\alpha}^2
\]
where we have chosen our favorite operator norm on $\End(E_x)$ for each fiber
and we are computing the inner products and norms fiberwise. Compactness of $X$
then ensures that we can obtain a global bound $C$ such that
$C \geq |\langle[\Lambda,F_{\nabla_E}]\alpha,\alpha\rangle|$
Then since
$i([\Lambda,F_\nabla]\alpha,\alpha) \geq 0$, we have that when $p+q > n$, a
sufficiently large choice of $m$ will make $m(n-(p+q))$ sufficiently negative
such that the only we we can have $\alpha = 0$.
\end{proof}
%
With the Kodaira-Nakano Vanishing Theorem and more help from Hodge theory, we have
anotherway to compute the sheaf cohomology of line bundles over $\CP^n$. To
do this we note two facts.
\begin{enumerate}
  \item $\O(1)$ is a positive line bundle.
  \item $\O(-n-1) \cong K_{\CP^n}$.
\end{enumerate}
%
The proof of the first fact follows from an easy computation of the curvature.
$\O(1)$ embdeds into the trivial bundle $\CP^n \times (\C^{n+1})^*$, which gives
it a Hermitian metric coming from the obvious one on $(\C^{n+1})^*$. Then it can be
shown that the Chern connection of this metric is just the Fubini-Study metric on
$\CP^n$ (in fact, the standard definition of the Fubini-Study is secretly
this construction). The proof of the second fact amounts to checking the
determinants of the Jacobians of the transition functions for the standard open
cover of $\CP^n$, and noting that they agree with the transition functions
for $\O(-n-1)$. We will also use the following result from Hodge theory.
%
\begin{thm}[\ib{Serre Duality}]
Let $X$ be a compact K\"ahler manifold of complex dimension $n$ and let $E \to X$
be a holomorphic vector bundle.
Then
\[
H^{n-q}(X, E^*\otimes K_X) \cong H^q(X,E)^*
\]
\end{thm}
%
We then apply the Kodaira-Nakano Vanishing Theorem to deduce which cohomology
groups are $0$. For $m > 0$, the vanishing theorem tells us that for $p+q > n$, we have
$H^q(\CP^n,\Omega^{p,0} \otimes \O(m)) = 0$. Then since $\Omega^{n,0} = K_X = \O(-n-1)$,
this tells us that $H^q(\CP^n,\O(m)) = 0$ for $q > 0$ and $m \geq -n$. Serre duality
then gives a similar vanishing result for $K_X \otimes \O(m) = \O(-m-n-1)$.
Putting everything together we get that
\[
H^q(\CP^n,\O(m)) = \begin{cases}
0 & 0 < q < n \\
0 & q = 0, m < 0 \\
0 & q = n , m > -n-1
\end{cases}
\]
Recall that the global sections of the line bundle $\O(m)$ are given by
\[
H^0(\CP^n,\O(m)) = \C[z_0,\ldots z_n]_m
\]
for $m \geq 0$, are are $0$ otherwise. Serre duality then tells us that
\[
H^n(X, \O(-m - n - 1)) \cong H^0(X,\O(m))^*
\]
Therefore, we get that if $m \geq 0$,
\[
H^q(\CP^n, \O(m)) = \begin{cases}
\C[z_0,\ldots z_n]_m & q = 0 \\
0 & \text{otherwise}
\end{cases}
\]
In the case that $m < 0$, we have
\[
H^q(\CP^n, \O(m)) = \begin{cases}
\C[z_0,\ldots z_n]_{n-m+1}^* & q = n \\
0 & \text{otherwise}
\end{cases}
\]
%
%
\newpage
%
\nocite{*}
%
\printbibliography
%
\end{document}