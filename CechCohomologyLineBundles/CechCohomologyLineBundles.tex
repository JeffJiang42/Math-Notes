\documentclass[psamsfonts, 12pt]{amsart}
%
%-------Packages---------
%
\usepackage[h margin=1 in, v margin=1 in]{geometry}
\usepackage{amssymb,amsfonts}
\usepackage[all,arc]{xy}
\usepackage{tikz-cd}
\usepackage{enumerate}
\usepackage{mathrsfs}
\usepackage{amsthm}
\usepackage{mathpazo}
\usepackage{float}
\usepackage{spectralsequences}\usepackage[backend=biber]{biblatex}
\addbibresource{bibliography.bib}
%\usepackage{charter} %another font
%\usepackage{eulervm} %Vakil font
\usepackage{yfonts}
\usepackage{mathtools}
\usepackage{enumitem}
\usepackage{mathrsfs}
\usepackage{fourier-orns}
\usepackage[all]{xy}
\usepackage[unicode]{hyperref}
\usepackage{url}
\usepackage{mathtools}
\usepackage{graphicx}
\usepackage{pdfsync}
\usepackage{mathdots}
\usepackage{calligra}
\usepackage{import}
\usepackage{xifthen}
\usepackage{pdfpages}
\usepackage{transparent}

\newcommand{\incfig}[2]{%
    \fontsize{48pt}{50pt}\selectfont
    \def\svgwidth{\columnwidth}
    \scalebox{#2}{\input{#1.pdf_tex}}
}
%
\usepackage{tgpagella}
\usepackage[T1]{fontenc}
%
\usepackage{listings}
\usepackage{color}

\definecolor{dkgreen}{rgb}{0,0.6,0}
\definecolor{gray}{rgb}{0.5,0.5,0.5}
\definecolor{mauve}{rgb}{0.58,0,0.82}

\lstset{frame=tb,
  language=Matlab,
  aboveskip=3mm,
  belowskip=3mm,
  showstringspaces=false,
  columns=flexible,
  basicstyle={\small\ttfamily},
  numbers=none,
  numberstyle=\tiny\color{gray},
  keywordstyle=\color{blue},
  commentstyle=\color{dkgreen},
  stringstyle=\color{mauve},
  breaklines=true,
  breakatwhitespace=true,
  tabsize=3
  }
%
%--------Theorem Environments--------
%
\newtheorem{thm}{Theorem}[section]
\newtheorem*{thm*}{Theorem}
\newtheorem{cor}[thm]{Corollary}
\newtheorem{prop}[thm]{Proposition}
\newtheorem{lem}[thm]{Lemma}
\newtheorem*{lem*}{Lemma}
\newtheorem{conj}[thm]{Conjecture}
\newtheorem{quest}[thm]{Question}
%
\theoremstyle{definition}
\newtheorem{defn}[thm]{Definition}
\newtheorem*{defn*}{Definition}
\newtheorem{defns}[thm]{Definitions}
\newtheorem{con}[thm]{Construction}
\newtheorem{exmp}[thm]{Example}
\newtheorem{exmps}[thm]{Examples}
\newtheorem{notn}[thm]{Notation}
\newtheorem{notns}[thm]{Notations}
\newtheorem{addm}[thm]{Addendum}
\newtheorem{exer}[thm]{Exercise}
%
\theoremstyle{remark}
\newtheorem{rem}[thm]{Remark}
\newtheorem*{claim}{Claim}
\newtheorem*{aside*}{Aside}
\newtheorem*{rem*}{Remark}
\newtheorem*{hint*}{Hint}
\newtheorem*{note}{Note}
\newtheorem{rems}[thm]{Remarks}
\newtheorem{warn}[thm]{Warning}
\newtheorem{sch}[thm]{Scholium}
%
%--------Macros--------
\renewcommand{\qedsymbol}{$\blacksquare$}
\renewcommand{\sl}{\mathfrak{sl}}
\newcommand{\Bord}{\mathsf{Bord}}
\renewcommand{\hom}{\mathrm{Hom}}
\renewcommand{\emptyset}{\varnothing}
\renewcommand{\O}{\mathcal{O}}
\newcommand{\R}{\mathbb{R}}
\newcommand{\ib}[1]{\textbf{\textit{#1}}}
\newcommand{\Q}{\mathbb{Q}}
\newcommand{\Z}{\mathbb{Z}}
\newcommand{\N}{\mathbb{N}}
\renewcommand{\C}{\mathbb{C}}
\newcommand{\A}{\mathbb{A}}
\newcommand{\F}{\mathbb{F}}
\newcommand{\M}{\mathcal{M}}
\newcommand{\dbar}{\overline{\partial}}
\newcommand{\zbar}{\overline{z}}
\renewcommand{\S}{\mathbb{S}}
\renewcommand{\P}{\mathbb{P}}
\newcommand{\V}{\vec{v}}
\newcommand{\RP}{\mathbb{RP}}
\newcommand{\CP}{\mathbb{CP}}
\newcommand{\B}{\mathcal{B}}
\newcommand{\GL}{\mathrm{GL}}
\newcommand{\PGL}{\mathrm{PGL}}
\newcommand{\SL}{\mathrm{SL}}
\newcommand{\PSL}{\mathrm{PSL}}
\newcommand{\SP}{\mathrm{SP}}
\newcommand{\SO}{\mathrm{SO}}
\newcommand{\SU}{\mathrm{SU}}
\newcommand{\gl}{\mathfrak{gl}}
\newcommand{\g}{\mathfrak{g}}
\newcommand{\Bun}{\mathsf{Bun}}
\newcommand{\inv}{^{-1}}
\newcommand{\bra}[2]{ \left[ #1, #2 \right] }
\newcommand{\set}[1]{\left\lbrace #1 \right\rbrace}
\newcommand{\abs}[1]{\left\lvert#1\right\rvert}
\newcommand{\norm}[1]{\left\lVert#1\right\rVert}
\newcommand{\transv}{\mathrel{\text{\tpitchfork}}}
\newcommand{\defeq}{\vcentcolon=}
\newcommand{\enumbreak}{\ \\ \vspace{-\baselineskip}}
\let\oldexists\exists
\renewcommand\exists{\oldexists~}
\let\oldL\L
\renewcommand\L{\mathfrak{L}}
\makeatletter
\newcommand{\tpitchfork}{%
  \vbox{
    \baselineskip\z@skip
    \lineskip-.52ex
    \lineskiplimit\maxdimen
    \m@th
    \ialign{##\crcr\hidewidth\smash{$-$}\hidewidth\crcr$\pitchfork$\crcr}
  }%
}
\makeatother
\newcommand{\bd}{\partial}
\newcommand{\lang}{\begin{picture}(5,7)
\put(1.1,2.5){\rotatebox{45}{\line(1,0){6.0}}}
\put(1.1,2.5){\rotatebox{315}{\line(1,0){6.0}}}
\end{picture}}
\newcommand{\rang}{\begin{picture}(5,7)
\put(.1,2.5){\rotatebox{135}{\line(1,0){6.0}}}
\put(.1,2.5){\rotatebox{225}{\line(1,0){6.0}}}
\end{picture}}
\DeclareMathOperator{\id}{id}
\DeclareMathOperator{\im}{Im}
\DeclareMathOperator{\codim}{codim}
\DeclareMathOperator{\coker}{coker}
\DeclareMathOperator{\supp}{supp}
\DeclareMathOperator{\inter}{Int}
\DeclareMathOperator{\sign}{sign}
\DeclareMathOperator{\Stab}{Stab}
\DeclareMathOperator{\sgn}{sgn}
\DeclareMathOperator{\indx}{ind}
\DeclareMathOperator{\alt}{Alt}
\DeclareMathOperator{\Aut}{Aut}
\DeclareMathOperator{\trace}{trace}
\DeclareMathOperator{\ad}{ad}
\DeclareMathOperator{\End}{End}
\DeclareMathOperator{\Ad}{Ad}
\DeclareMathOperator{\Lie}{Lie}
\DeclareMathOperator{\spn}{span}
\DeclareMathOperator{\dv}{div}
\DeclareMathOperator{\grad}{grad}
\DeclareMathOperator{\Sym}{Sym}
\DeclareMathOperator{\sheafhom}{\mathscr{H}\text{\kern -3pt {\calligra\large om}}\,}
\newcommand*\myhrulefill{%
   \leavevmode\leaders\hrule depth-2pt height 2.4pt\hfill\kern0pt}
\newcommand\niceending[1]{%
  \begin{center}%
    \LARGE \myhrulefill \hspace{0.2cm} #1 \hspace{0.2cm} \myhrulefill%
  \end{center}}
\newcommand*\sectionend{\niceending{\decofourleft\decofourright}}
\newcommand*\subsectionend{\niceending{\decosix}}
\def\upint{\mathchoice%
    {\mkern13mu\overline{\vphantom{\intop}\mkern7mu}\mkern-20mu}%
    {\mkern7mu\overline{\vphantom{\intop}\mkern7mu}\mkern-14mu}%
    {\mkern7mu\overline{\vphantom{\intop}\mkern7mu}\mkern-14mu}%
    {\mkern7mu\overline{\vphantom{\intop}\mkern7mu}\mkern-14mu}%
  \int}
\def\lowint{\mkern3mu\underline{\vphantom{\intop}\mkern7mu}\mkern-10mu\int}
%
%--------Hypersetup--------
%
\hypersetup{
    colorlinks,
    citecolor=black,
    filecolor=black,
    linkcolor=blue,
    urlcolor=blacksquare
}
%
%--------Solution--------
%
\newenvironment{solution}
  {\begin{proof}[Solution]}
  {\end{proof}}
%
%--------Graphics--------
%
%\graphicspath{ {images/} }
\begin{document}
%
\author{Jeffrey Jiang}
%
\title{\v{C}ech Cohomology, Sheaf cohomology, and Line Bundles}
%
\maketitle
%
\section{\v{C}ech Cohomology}
%
For the most part, we will consider ``sufficiently nice" topological spaces.
For the most part, think of a space $X$ as a (smooth, complex) manifold, an analytic
space, or a quasicompact separated scheme if you're feeling adventurous.
%
\begin{defn}
Let $X$ be a space and $\mathcal{F}$ a sheaf of abelian groups over $X$. Let
$\mathcal{U} = \set{U_i}_{i \in \N}$ be a countable open cover of $X$ that is locally
finite, i.e. for any $x \in X$, only finitely many $U_i$ contain $x$. For
$I = \set{i_1, \ldots, i_k}$, let
\[
U_I \defeq \bigcap_{i \in I} U_i
\]
Then define the \ib{\v{C}ech cochain groups} of $\mathcal{F}$ for the cover
$\mathcal{U}$ by
\begin{align*}
C^k(\mathcal{U}, \mathcal{F}) \defeq \prod_{|I| = k+1} \mathcal{F}(U_I)
\end{align*}
an element of $C^k(\mathcal{U},\mathcal{F})$ is called a \ib{\v{C}ech cochain}.
For a $k$-cochain $\sigma$, and $I = \set{i_0,\ldots i_k}$, we denote the
component of $\sigma$ over $U_I$ as $\sigma_I$ or $\sigma_{i_0,\ldots i_k}$.
\end{defn}
%
The \v{C}ech cochain groups are equipped with a differential
$d : C^k(\mathcal{U}, \mathcal{F}) \to C^{k+1}(\mathcal{U}, \mathcal{F})$
where for $\sigma \in C^k(\mathcal{U}, \mathcal{F})$, the ${i_0, \ldots i_{k+1}}$
component of $d\sigma$ is given by
\[
(d\sigma)_{i_0, \ldots i_{k+1}}
= \sum_{j = 1}^{p+1}(-1)^j\sigma_{i_0, \ldots, \widehat{i_j}, \ldots, i_{k+1}}
\vert_{U_0 \cap \cdots \cap U_{k+1}}
\]
where $\widehat{i_j}$ denotes that $i_j$ is missing. We have that $d^2 = 0$ for a
similar reason that $d^2 = 0$ for singular cohomology, you get repeats of
terms with opposite signs due to the omitted index. We denote the kernel of
$d : C^i(\mathcal{U}, \mathcal{F}) \to C^{i+1}(\mathcal{U}, \mathcal{F})$ as
$Z^i(\mathcal{U}, \mathcal{F})$, and we say that the element are $i$-cocycles.
We denote the image of
$d : C^{i-1}(\mathcal{U}, \mathcal{F}) \to C^i(\mathcal{U}, \mathcal{F})$ as
$B^i(\mathcal{U}, \mathcal{F})$, and we call the elements $i$-coboundaries.
%
\begin{defn}
The \ib{\v{C}ech cohomology groups} of $\mathcal{F}$ with respect to the cover
$\mathcal{U}$, denoted $\check{H}^i(\mathcal{U}, \mathcal{F})$ is the cohomology of
the \v{C}ech complex
\[\begin{tikzcd}
0 \ar[r, "d"] & C^0(\mathcal{U}, \mathcal{F}) \ar[r, "d"] &
C^1(\mathcal{U}, \mathcal{F}) \ar[r, "d"]
& \cdots
\end{tikzcd}\]
i.e. we have
\[
\check{H}^i(\mathcal{U}, \mathcal{F}) \defeq
\frac{Z^i(\mathcal{U}, \mathcal{F})}{B^i(\mathcal{U}, \mathcal{F})}
\]
\end{defn}
%
\begin{defn}
Given an open cover $\mathcal{U} = \set{U_i}$, a \ib{refinement} of $\mathcal{U}$
is another open cover $\mathcal{V} = \set{V_j}$ such that every $V_j$ is contained
in some $U_i$. If $\mathcal{V}$ is a refinement of $\mathcal{U}$, we write
$\mathcal{V} < \mathcal{U}$.
\end{defn}
%
If $\mathcal{V} < \mathcal{U}$, then we know we can find a map
$\varphi : \N \to \N$ such that $V_i \subset U_\varphi(i)$. Consequently, we
can restrict sections over $U_i$ to sections over $V_{\varphi(i)}$, so this
induces a chain map
$\rho_\varphi : C^k(\mathcal{U}, \mathcal{F}) \to C^k(\mathcal{V},\mathcal{F})$
where
\[
(\rho_\varphi(\sigma))_{i_0,\ldots, i_k}
= \sigma_{\varphi(i_0),\ldots \varphi(i_k)}\vert_{U_{i_0} \cap \cdots \cap U_{i_k}}
\]
This map commutes with the differentials for $C^\bullet(\mathcal{U},\mathcal{F})$
and $C^\bullet(\mathcal{V},\mathcal{F})$, so it descends to homomorphisms
$\check{H}^i(\mathcal{U},\mathcal{F}) \to \check{H}^i(\mathcal{V}, \mathcal{F})$.
It can be shown that a different choice of chain map $\rho_\psi$ for
$\psi : \N \to \N$ is chain homotopic to $\rho_\varphi$, so the induced maps
on cohomology are independent of our choice of $\varphi$.
%
\begin{defn}
The \ib{\v{C}ech cohomology groups} of a sheaf $\mathcal{F}$ over $X$ is the limit over
refinements
\[
\check{H}^i(X, \mathcal{F})
\defeq \lim_{\mathcal{V} < \mathcal{U}} \check{H}^i(\mathcal{V}, \mathcal{F})
\]
i.e. the quotient disjoint union
$\amalg_\mathcal{U}\check{H}^i(\mathcal{U}, \mathcal{F}$
over all refinements, where we identify $\sigma \in \check{H}^i(\mathcal{U}, \mathcal{F}$
and $\tau \in \check{H}^i(\mathcal{V}, \mathcal{F})$ if
$\mathcal{V}$ refines $\mathcal{U}$ and $\rho(\sigma) = \tau$ under the map induced on
cohomology by the refinement.
\end{defn}
%
This definition of the \v{C}ech cohomology groups is essentially useless for
computation. It's true power comes from the following theorem, due to Leray.
%
\begin{thm}[\ib{Leray}]
Suppose $\mathcal{U} = \set{U_i}$ is an open cover of $X$ that is \ib{acyclic} with
respect to the sheaf $\mathcal{F}$, i.e. for $|I| > 1$ and any $i$,
\[
\check{H}^i(U_I, \mathcal{F}) = 0
\]
Then
\[
\check{H}^i(\mathcal{U}, \mathcal{F}) = \check{H}^i(X, \mathcal{F})
\]
\end{thm}
%
The intuition to keep in mind for an acyclic cover is the notion of a good cover
in differential geometry. On a smooth manifold $M$, there exists a covering of
$M$ by open sets $\set{U_i}$ such that any nonempty intersections are contractible,
which is done by taking geodesic balls around each point in $M$. Since the
homotopical information on each $U_i$ is trivial, the only nontrivial topological
information in the cohomology of $M$ comes from how the sets are glued together to
form $M$. As with exact sequences of chain complexes, short exact sequences
of sheaves give long exact sequences in sheaf cohomology.
%
\begin{thm}
Let
\[\begin{tikzcd}
0 \ar[r] & \mathcal{E} \ar[r, "\alpha"] & \mathcal{F} \ar[r, "\beta"] & \mathcal{G}
\ar[r] & 0
\end{tikzcd}\]
be an exact sequence of sheaves over $X$. Then this induces a long exact sequence in
cohomology:
\[\begin{tikzcd}
0 \ar[r] & \check{H}^0(X,\mathcal{E}) \ar[r, "\alpha^*"]
& \check{H}^0(X,\mathcal{F}) \ar[r, "\beta^*"] &
\check{H}^0(X, \mathcal{G}) \ar[dll, "\delta"'] \\
& \check{H}^1(X, \mathcal{E}) \ar[r, "\alpha^*"]
& \check{H}^1(X, \mathcal{F})\ar[r, "\beta^*"]
& \check{H}^1(X, \mathcal{G}) \ar[dll, "\delta"'] \\
& ~ & \cdots & \cdots
\end{tikzcd}\]
\end{thm}
%
\begin{proof}
We first define the maps
$\alpha^* : \check{H}^i(X,\mathcal{E}) \to \check{H}^i(X, \mathcal{F})$
and $\beta^* : H^i(X, \mathcal{F}) \to \check{H}^i(X, \mathcal{G})$, and will then define
the connecting homomorphism
$\delta : \check{H}^i(X,\mathcal{E}) \to \check{H}^{i+1}(X, \mathcal{G})$. Given
an open cover $\mathcal{U} = \set{U_i}$ of $X$, the sheaf morphism $\alpha$ gives for
each open set $U_i$ a homomorphism $\alpha(U_i) : \mathcal{E}(U_i) \to \mathcal{F}(U_i)$,
which induces a chain map
$C^\bullet(\mathcal{U},\mathcal{F}) \to C^\bullet(\mathcal{U}, \mathcal{G}))$. Since
the maps $\alpha(U_i)$ commute with restriction maps, this chain map commutes with
the differentials, so it descends to a map on cohomology
$\check{H}^\bullet(\mathcal{U},\mathcal{F})\to\check{H}^\bullet(\mathcal{U},\mathcal{F})$,
which, after taking the limit over refinements or choosing $\mathcal{U}$ to be
simultaneously acyclic for $\mathcal{E}$ and $\mathcal{F}$, gives us the induced
map $\alpha^* : \check{H}^\bullet(X,\mathcal{F})\to\check{H}^\bullet(X,\mathcal{F})$.
The map $\beta^*$ is defined similarly. \\

The construction of the connecting homomorphism
mirrors the construction for singular (or de Rham) cohomology. We represent
an element of $\check{H}^i(X,\mathcal{G})$ with a cocycle
$\sigma \in Z^i(\mathcal{U}, \mathcal{G})$ with respect to some open cover $\mathcal{U}$.
By surjectivity of $\beta$, by potentially passing to a refinement, we can
write $\sigma = \beta(\tau)$ for some $\tau \in C^i(\mathcal{U}, \mathcal{F})$. Then
since the induced map on chains commutes with the differentials, we have that
$d\beta(\tau) = \beta(d\tau) = 0$, since $\tau$ is a cocycle. Therefore,
$d\tau$ is in the kernel of $\beta^*$, so we can write $d\tau = \alpha(\eta)$
for some $\eta \in C^{i+1}(\mathcal{U}, \mathcal{F})$. We then define
$\delta(\sigma)$ to be the class of $\eta$ in the limit. We note that
this is independent of our choice of $\tau$, since any other choice of
preimage of $\sigma$ differs by an element of the form $\alpha(e)$ for some cocycle
$e$ by exactness. Then since $\alpha$ commutes with the differentials, we get
$d(\tau + \alpha(e)) = d\tau + d\alpha(e) = d\tau + \alpha(de) = d\tau$.
\end{proof}
%
We make one observation about \v{C}ech cohomology
%
\begin{prop}
The $0^{th}$ \v{C}ech cohomology group is isomorphic to the space of global sections, i.e.
\[
\check{H}^0(\mathcal{U}, \mathcal{F}) \cong \Gamma(X, \mathcal{F})
\]
\end{prop}
%
\begin{proof}
The $0^{th}$ cohomology is just the kernel of the map
$d : C^0(\mathcal{U},\mathcal{F}) \to C^1(\mathcal{U}, \mathcal{F})$.
For a $0$-cochain $\sigma$, we have that
\[
(d\sigma)_{ij} = \sigma_j\vert_{U_i \cap U_j} - \sigma_i\vert_{U_i \cap U_j}
\]
We then claim that the map $\Gamma(X, \mathcal{F}) \to \ker d$ sending a section
$\sigma$ to the cocycle $\tilde{\sigma}$ defined by
\[
\tilde{\sigma}_i = \sigma\vert_{U_i}
\]
is bijective. It is surjective, since any $0$-cocycle mapping in the kernel
is a collection of local sections that agrees on intersections, which is exactly
a global section of $\mathcal{F}$. In addition, it is injective, since if
a section restricts to $0$ on every open set, it is the zero section.
\end{proof}
%
\section{\v{C}ech Cohomology on $\CP^n$}
%
We now compute \v{C}ech cohomology for various sheaves over $\CP^n$. The main objects
of interest are the line bundles $\O(k)$. We have that $\CP^n$ admits a nice cover
$\mathcal{U} = \set{U_i}$ where $U_i$ is the open set where the coordinate $z_i$ does not
vanish. We will soon show that this covering is acyclic for the structure sheaf
$\O_{\CP^n}$, and since all line bundles are trivial on the open sets in this cover, this
means that the covering will acyclic for any line bundle $\O(k)$. So it suffices to
compute \v{C}ech cohomology with respect to this cover. In particular, we note that for
any line bundle $\O(k) \to \CP^n$, the transition functions $\psi_{ij}$ for $\O(k)$
are $\psi_{ij}(\ell) = (z_j/z_i)^k$. We already know one cohomology group:
%
\begin{thm}
The $0^{th}$ cohomology group $\check{H}^0(\CP^n, \O(k))$ is isomorphic to the space
$\C[x_0, \ldots x_n]_k$ of homogeneous degree $k$ polynomials in the variables
$x_0, \ldots , x_n$.
\end{thm}
%
We compute the rest of the cohomology now, which we do in stages.
%
\begin{thm}
For any $i > n$, we have $\check{H}^n(\mathcal{U},\O(k)) = 0$.
\end{thm}
%
\begin{proof}
The cover of $\CP^i$ by the $U_i$ has cardinality $n+1$. Therefore,
$C^{n+1}(\mathcal{U}, \O(k)) = 0$.
\end{proof}
%
To compute the rest of the cohomology groups, we first prove a lemma characterizing
local sections of $\O(k)$.
%
\begin{lem}
Let $\pi : \C^{n-1} - \set{0} \to \CP^n$ be the usual projection sending
$z \in C^{n-1}$ to $\mathrm{span}\set{z}$. Then the space of sections
$\O(k)(U)$ is isomorphic to the space of homogeneous of degree $k$  holomorphic functions
$f : \pi\inv(U) \to \C$ i.e.
\[
f(tz_0, \ldots, tz_n) = t^kf(z_0,\ldots, z_n)
\]
\end{lem}
%
\begin{proof}
Let $\sigma \in \O(k)(U)$ be a section. Then the set $\set{U \cap U_i}$ is an
open cover of $U$, so $\sigma$ is determined by its restrictions
$\sigma_i \defeq \sigma_i\vert_{U \cap U_i}$. Since the bundle $\O(d)$ is trivial
over the $U_i$, the local sections $\sigma_i$ can be identified with holomorphic
functions $U \cap U_i \to \C$ with the compatibility condition
\[
\sigma_i([z_0 :\ldots : z_n])
= \left(\frac{z_j}{z_i}\right)^k\sigma_j([z_0: \ldots : z_n])
\]
We then give maps in both directions. Given a section
$\sigma \in \O(k)(U)$, define the function $f_\sigma$ by
\[
f_\sigma(z_0,\ldots z_n) = z_i^k\sigma_i(\pi(z_0,\ldots z_n))
\]
We must verify that this is well-defined, i.e. it is independent of our choice of $i$.
We compute
\begin{align*}
f_\sigma(z_0,\ldots z_n)
&=  z_i^k\sigma_i(\pi(z_0,\ldots z_n)) \\
&=  \left(\frac{z_j}{z_i}\right)^kz_i^k\sigma_j(\pi(z_0,\ldots z_n)) \\
&= z_j^k\sigma_k(\pi(z_0, \ldots z_n)) \\
\end{align*}
So this determines a well defined function on $\pi\inv(U)$. In addition, it is
visibly homogeneous of degree $k$, since the $\sigma_i$ are constant on lines and
$z_i^k$ is homogeneous of degree $k$. To show this is an isomorphism, we provide
an inverse. Given a homogeneous function $f$ of degree $k$ on $\pi\inv(U)$, define
the section $\sigma_f$ locally by
\[
(\sigma_f)_i([z_0: \ldots :z_n]) = \frac{f(z_0,\ldots z_n)}{z_i^k}
\]
then to show that this defines a section, we must show that they agree on intersections
using the transition functions. We compute
\[
\left(\frac{z_i}{z_j}\right)^k(\sigma_f)_i\vert_{U \cap U_i \cap U_j}([z_0: \ldots : z_n])
= \left(\frac{z_i}{z_j}\right)^k\frac{f(z_0,\ldots, z_n)}{z_i^k}
= \frac{f(z_0, \ldots z_n)}{z_j^k}
= (\sigma_f)_j
\]
The two mappings provided are visibly inverses, since one is essentially multiplication
by $z_j^k$ and the other is essentially division by $z_j^k$.
\end{proof}
%
Over intersections of the distinguished open sets $U_i$, the sections have
a particularly nice form. Under the projection $\pi : \C^{n+1}-\set{0} \to \CP^n$,
the preimage of $U_I$ for $I = \set{i_0, \ldots i_d}$ is just $\C^{n+1}$ minus
the coordinate axes $z_{i_j} = 0$. By taking power series, a holomorphic
function on $\pi\inv(U_I)$ is given by Laurent series where the $z_{i_j}$ can appear
in negative degree. Being homogeneous of degree $d$ implies that all the
terms in the series expansion must be homogeneous of degree $k$, where the
degree of $(z_k)^a/(z_{i_j})^b$ is $a - b$. Consequently, all such holomorphic
functions must be polynomials in $\C[z_0, \ldots z_n, z_{i_0}\inv, \ldots z_{i_d}\inv]$
of degree $k$.
% TODO Compute cohomology
\iffalse
We now compute the $n^{th}$ cohomology groups.
%
\begin{thm}
\[
\check{H}^i(\mathcal{U}, \O(k)) = \begin{cases}
\C[z_0, \ldots z_n]_{-k-n-1} & -k-n-1 \geq 0 \\
0 & \text{otherwise}
\end{cases}
\]
\end{thm}
%
\begin{proof}
Since $C^{n+1}(\mathcal{U}, \O(k)) = 0$, we have that
$\check{H}^n(\mathcal{U},\O(k))$ is just the cokernel of the differential
$d : C^{n-1}(\mathcal{U}, \O(k)) \to C^n(\mathcal{U}, \O(k))$.
\end{proof}
\fi
%
%TODO Prove that the covering is acyclic
%
\section{\v{C}ech Cohomology and Line Bundles}
%
Let $\mathcal{U} = \set{U_i}$ be a covering of $X$, and $\mathcal{F}$ a sheaf
of abelian groups over $X$. Then with respect to this cover, a \v{C}ech 2-cocycle
$\sigma \in Z^2(\mathcal{U}, \mathcal{F})$ is defined by the equation
\[
0 = (d\sigma)_{ijk} = \sigma_{jk}\vert_{U_i \cap U_j \cap U_k}
- \sigma_{ik}\vert_{U_i \cap U_j \cap U_k} + \sigma_{ij}\vert_{U_i \cap U_j \cap U_k}
\]
Written multiplicatively (and omitting the restriction), this becomes
\[
1 = \sigma_{jk}\sigma_{ik}\inv\sigma_{ij}
\]
which, since the group is abelian, is equivalent to
\[
\sigma_{ik} = \sigma_{ij}\sigma_{jk}
\]
which looks exactly like a cocycle condition for transition functions of a line
bundle. Recall that given a holomorphic line bundle $\pi : L \to X$, we have local
trivializations -- we can find a cover $\mathcal{U} = \set{U_i}$ with maps
$\varphi_i : \pi\inv(U_i) \to U_i \times \C$ such that
\[\begin{tikzcd}
\pi\inv(U_i) \ar[rr, "\varphi_i"] \ar[dr] & & U_i \times \C \ar[dl] \\
& U_i
\end{tikzcd}\]
commutes, where the maps to $U_i$ are the projections. Therefore, if we consider
the map
$\varphi_i \circ \varphi_k\inv : U_i \cap U_j \times \C \to U_i \cap U_j \times \C$,
we have that $\varphi(x, \lambda) = (x, \psi_{ij}(x)(\lambda))$, where the functions
$\psi_{ij} : U_i\cap U_j \to \GL_1\C$ are holomorphic. The $\psi_{ij}$
are called the \ib{transition functions} of the line bundle $L$.
%
\begin{prop}
The transition functions $\psi_{ij}$ satisfy the following conditions
\begin{enumerate}
  \item $\psi_{ij}\psi_{ji} = 1$ (i.e. the constant function $x \mapsto 1$)
  \item $\psi_{ij}\psi_{jk} = \psi_{ik}$.
\end{enumerate}
The second condition is often called the \ib{cocycle condition}, in reference to the
identity we derived above for the defining property of a \v{C}ech cocycle.
\end{prop}
%
\begin{proof}
Consider the map $\varphi_i\circ\varphi_j\inv\circ\varphi_j\circ\varphi_i\inv = \id$.
We compute under the action on a general element $(x,\lambda)$
\[\begin{tikzcd}
(x,\lambda) \ar[r, "\varphi_j\circ\varphi_i\inv"] & (x,\psi_{ji}(x)(\lambda))
\ar[r, "\varphi_i\circ\varphi_j\inv"] & (x,\psi_{ij}(x)\psi_{ji}(x)(\lambda))
\end{tikzcd}\]
Therefore, we have that $\psi_{ij}(x)\psi_{ji}(x) = 1$ for all $x$, showing the
first property. For the second property, we do the same thing. Consider
the function
$\varphi_i\circ\varphi_j\inv\circ\varphi_j\circ\varphi_k = \varphi_i\circ\varphi_k\inv$.
Then fore $(x,\lambda)$, we compute the action of this function to be
\[\begin{tikzcd}
(x,\lambda) \ar[r, "\varphi_j\circ\varphi_k\inv"] & (x,\psi_{jk}(x)(\lambda))
\ar[r, "\varphi_i\circ\varphi_j\inv"] & (x,\psi_{ij}(x)\psi_{jk}(x)(\lambda))
\end{tikzcd}\]
So we get that $\psi_{ij}(x)\psi_{jk}(x) = \psi_{ik}(x)$ for all $x$.
\end{proof}
%
Under the \v{C}ech differential, the image of a \v{C}ech $0$-cochain $\sigma$ is given
by
\[
(d\sigma)_{ij} = \sigma_j - \sigma_i
\]
written multiplicatively, this becomes
\[
(d\sigma)_{ij} = \sigma_j\sigma_i\inv
\]
In the same spirit, we translate this to a statement regarding transition functions
of a line bundle.
%
\begin{prop}
Let $\pi L \to X$ be a holomorphic line bundle where the transition functions
$\psi_{ij}$ with respect to a cover $\set{U_i}$ satisfy the \ib{coboundary condition},
i.e. there exist holomorphic functions $\sigma_i : U_i \to \GL_1\C$ such that
\[
\psi_{ij} = \sigma_j\sigma_i\inv
\]
Then $L$ is a trivial line bundle.
\end{prop}
%
\begin{proof}
It suffices to provide a nonvanishing section $X \to L$. A section $s : X \to L$ is
equivalent to functions $s_i : U_i \to \C$ with the compatibility condition
\[
s_i = \psi_{ij}s_j
\]
define the $s_i$ by $s_i = \sigma_i\inv$. Then they satisfy the compatibility condition,
since
\[
\psi_{ij}\sigma_j = \sigma_j\sigma_i\inv\sigma_j\inv = \sigma_i\inv = s_i
\]
then since the $\sigma_i$ are functions to $\GL_1\C = \C^\times$, they glue to
a global nonvanishing section, so $L$ is isomorphic to the trivial line bundle
$X \times \C$.
\end{proof}
%
Recall that isomorphism classes of line bundles over $X$ form a group
under tensor product, where the inverse of a line bundle $L$ is the dual bundle
$L^*$. Given line bundles $L,L' \to X$ and an open cover $\mathcal{U_i}$ of $X$
in which both $L$ and $L'$ are trivialized over the $U_i$ (for instance, a good cover
of $X$), let $\psi_{ij}$ be the transition functions for $L$ and let $\varphi_{ij}$
be the transition functions for $L'$. Then the transition functions for $L\otimes L'$
are $\psi_{ij}\varphi_{ij}$, and the transition functions for $L^*$ are given
by $\varphi_{ij}\inv$.
%
\begin{thm}
Let $X$ be a complex manifold, and $\O_X$ its sheaf of holomorphic functions.
Then let $\O_X^\times$ be the sheaf of invertible functions, which is a
sheaf of abelian groups under multiplication. Then we have a group isomorphism
\[
\check{H}^1(X,\O_X^\times) \cong \mathrm{Pic}(X)
\]
\end{thm}
%
\begin{proof}
Fix a good cover $\mathcal{U} = \set{U_i}$ for $X$. Since all the sets and their
nonempty intersections are contractible, we have that
$\check{H}^i(U_i, \mathcal{F}) = 0$ for all $i > 0$ where $\mathcal{F}$ is the sheaf
of sections of any line bundle. Since all the $U_i$ are contractible, we also
have that any line bundle over $U_i$ is trivial, so it admits transition
functions $\psi_{ij}$ with respect to this cover. As shown above, the
functions $\psi_{ij}$ exactly define a \v{C}ech $1$-cocyle, and any
\v{C}ech coboundary defines a trivial bundle. In addition, we have that
the the transition functions of a tensor product are exactly the products
of the transition functions. Putting everything together, this tells us that
the mapping $L \mapsto \set{\psi_{ij}}$ is a bijective group homomorphism.
\end{proof}
%
\section{Divisors}
%
%
\section{Sheaf Cohomology}
%
Over a complex manifold $X$, we have many different cohomology theories at our disposal:
\begin{enumerate}
  \item The singular cohomology groups $H^i_{\text{sing}}(X,\Z)$.
  \item The de Rham cohomology groups $H^i_{dR}(X)$.
  \item The Dolbeault cohomology groups of a holomorphic vector bundle
  $H^i_{\dbar}(X, E)$.
  \item The \v{C}ech cohomology groups of a sheaf $\check{H}^i(X,\mathcal{F})$.
\end{enumerate}
%
We want to compare these various cohomology theories. To do so, we show that many of
these cohomology theories are computing the same thing : sheaf cohomology.
%
\begin{rem*}
While we might explicitly work with sheaves of abelian groups, the following discussion
is applicable to sheaves of $\mathcal{O}_X$ modules, $C^\infty$ modules, etc.
\end{rem*}
%
\begin{defn}
The \ib{global sections functor} $\Gamma(X,\cdot)$ is a functor
$\mathsf{Ab}(X) \to \mathsf{Ab}$ of the category of sheaves of abelian groups over $X$ to
the category of abelian groups, where given a sheaf of abelian groups $\mathcal{F}$,
\[
\Gamma(X,\mathcal{F}) \defeq \mathcal{F}(X)
\]
\end{defn}
%
The functor is left-exact, i.e. given an exact sequence of sheaves
\[\begin{tikzcd}
0 \ar[r] & \mathcal{E} \ar[r] & \mathcal{F} \ar[r] & \mathcal{G}
\end{tikzcd}\]
we get an exact sequence
\[\begin{tikzcd}
0 \ar[r] & \Gamma(X,\mathcal{E}) \ar[r]
& \Gamma(X,\mathcal{F}) \ar[r] & \Gamma(X,\mathcal{G})
\end{tikzcd}\]
%
However, the functor is not right exact, which is due to the local definitions of
injectivity and surjectivity. The sheaf axiom guarantees that being injective on
stalks implies that a sheaf morphism is injective on sections, but it does not
imply the same thing for surjectivity. As an example, let $\mathcal{Z}^k$ be
the sheaf of closed smooth $k$-forms, and let $B^k$ be the sheaf of exact $k$-forms.
Then the inclusion $\mathcal{B}^k \hookrightarrow \mathcal{Z}^k$ is surjective,
since every closed $k$-form is exact in a sufficiently small neighborhood. However,
if $H^k_{dR}(X) \neq 0$, then $\Gamma(X,\mathcal{Z}^k) \to \Gamma(X, \mathcal{B}^k)$
is not surjective.
%
\begin{defn}
The \ib{sheaf cohomology groups} $H^i(X, \mathcal{F})$ of a sheaf $\mathcal{F}$ over $X$
are the right derived functors of the global sections functor applied to $\mathcal{F}$
\[
H^i(X,\mathcal{F}) \defeq R^i\Gamma(X,\mathcal{F})
\]
i.e, we take an injective resolution $\mathcal{I}^\bullet = \set{\mathcal{I}^j}$ of
$\mathcal{F}$
\[\begin{tikzcd}
0 \ar[r] & \mathcal{F} \ar[r] & \mathcal{I}^0 \ar[r] & \mathcal{I}^1 \ar[r] & \cdots
\end{tikzcd}\]
and apply $\Gamma(X,\cdot)$ term-wise to the sequence to get
\[\begin{tikzcd}
0 \ar[r] & \Gamma(X,\mathcal{F}) \ar[r] & \Gamma(X,\mathcal{I}^0) \ar[r]
& \Gamma(X,\mathcal{I}^1) \ar[r] & \cdots
\end{tikzcd}\]
and then compute the cohomology of this sequence.
\end{defn}
%
We note that in order for these right derived functors to be defined, there need to
be \emph{enough injectives}. We say that an abelian category $\mathcal{A}$ has enough
injectives if every object $A \in \mathrm{Ob}(\mathcal{A})$ admits an injective
map $A \hookrightarrow I$ into an injective object $I$, where an injective object
is defined to be an object $I$ where given any map $X \to I$ and an injection
$X \hookrightarrow I$, there exists a map $Y \to Q$ such that the following diagram
commutes
\[\begin{tikzcd}
X \ar[dr]\ar[rr,hookrightarrow] & & Y \ar[dl, dashed] \\
& Q
\end{tikzcd}\]
Alternatively, the pullback map $\hom(Y,Q) \to \hom(X,Q)$ is surjective. We will take
the following results on faith:
%
\begin{thm}
The categories $\mathsf{Ab}$, $\mathsf{Ab}(X)$, $\mathsf{Mod}_{\O_X}$,
and $\mathsf{Mod}_{C^\infty}$ have
enough injectives.
\end{thm}
%
Like with the definition of the \v{C}ech cohomology groups as limits, the
definition in terms of injective resolutions is practically useless computationally,
since injective sheaves are hard to write down and difficult to find in the wild.
The name of the game here is to find a nicer class of resolutions we can take.
%
\begin{defn}
Let $F : \mathcal{A} \to \mathcal{B}$ be a left-exact additive functor. An object
$A \in \mathrm{Ob}(\mathcal{A})$ is \ib{acyclic} for the functor $F$ if
$R^iF(A) = 0$ for all $i > 0$.
\end{defn}
%
\begin{prop}
Let $F : \mathcal{A} \to \mathcal{B}$ be an additive functor, and
$A \in \mathrm{Ob}(\mathcal{A})$. Then let $A \to M^\bullet$ be a resolution of $A$
by $F$-acyclic objects. Then $R^iF(A)$ is isomorphic to the $i^{th}$ cohomology
of the complex $F(M^\bullet)$.
\end{prop}
%
\begin{proof}
Let $d^0 : M^0 \to M^1$. Then let $B = \coker d^0$ by exactness of
\[\begin{tikzcd}
0 \ar[r] & A \ar[r] & M^0 \ar[r, "d^0"] & M^1
\end{tikzcd}\]
we get a short exact sequence
\[\begin{tikzcd}
0 \ar[r] & A \ar[r] & M^0 \ar[r] & B \ar[r] & 0
\end{tikzcd}\]
where the map $M^0 \to B$ is the composition of $d^0$ with the natural map
$M^1 \to B$. We then take injective resolutions $A \to I^\bullet$,
$M^0 \to J^\bullet$, and $B \to K^\bullet$. The maps $A \to M^0$ and $M^0 \to B$ induce
a short exact sequence of chain maps between resolutions by using the defining property
of injective objects, giving us
\[\begin{tikzcd}
0 \ar[r] & A \ar[r] \ar[d] & M^0 \ar[r] \ar[d] & B \ar[r] \ar[d] & 0 \\
0 \ar[r] & I^0 \ar[r] \ar[d] & J^0 \ar[r] \ar[d] & K^0 \ar[r] \ar[d] & 0 \\
0 \ar[r] & I^1 \ar[r] \ar[d] & J^1 \ar[r] \ar[d] & K^1 \ar[r] \ar[d] & 0 \\
& \vdots  & \vdots  & \vdots &
\end{tikzcd}\]
which gives us a long exact sequence in cohomology. Noting that the cohomology
of the respective sequences are just the right derived functors of $A$, $M^0$,
and $B$, we get the long exact sequence
\[\begin{tikzcd}
0 \ar[r] & R^0F(A) \ar[r] & R^0F(M^0) \ar[r] & R^0F(B) \ar[dll] \\
& R^1F(A) \ar[r] & R^1F(M^0) \ar[r] & R^1F(B) \ar[dll] \\
& R^2F(A) \ar[r] & R^2F(M^0) \ar[r] & R^2F(B) \ar[dll] \\
& ~ & \cdots & \cdots
\end{tikzcd}\]
Then using the fact that $M^0$ is $F$-acyclic, along with the fact that $R^0F = F$, we
get that this long exact sequence is actually
\[\begin{tikzcd}
0 \ar[r] & F(A) \ar[r] & F(M^0) \ar[r] & F(B) \ar[dll] \\
& R^1F(A) \ar[r] & 0 \ar[r] & R^1F(B) \ar[dll] \\
& R^2F(A) \ar[r] & 0 \ar[r] & R^2F(B) \ar[dll] \\
& ~ & \cdots & \cdots
\end{tikzcd}\]
which gives us isomorphisms $R^iF(B) \to R^{i+1}F(A)$ for $i > 0$, as well as
an isomorphism $R^1F(A) \cong \coker F(M^0) \to F(B)$. To compute the
cokernel of that map, we note that $B$ admits the resolution
\[\begin{tikzcd}
0 \ar[r] & B \ar[r] & M^1 \ar[r] & \cdots
\end{tikzcd}\]
where the map $B \to M^1$ is the map induced by $d^0$, using exactness of acyclic
resolution. Then since $F$ is left-exact, we get that
\[\begin{tikzcd}
0 \ar[r] & F(B) \ar[r] & F(M^1) \ar[r] & \cdots
\end{tikzcd}\]
is exact, giving us that $F(B)$ is isomorphic to the kernel of $F(M^1) \to F(M^2)$.
Therefore, we get that
\[
R^1F(A) \cong \frac{\ker(F(M^1) \to F(M^2))}{\im(F(M^1) \to F(M^2))}
= H^1(F(M^\bullet))
\]
Then to get the isomorphism for $R^2F(A)$, we note that since
$R^1F(B) \cong R^2F(A)$, we can play the same game using the resolution of $B$
to compute $R^1F(B)$ to be $H^2(F(M^\bullet))$, and then inductively repeat
the process with the cokernel of $M^1 \to M^2$ to get $R^3F(A)$ and so on.
\end{proof}
%
Something that is silly to observe, but useful.
%
\begin{prop}
Injective objects are $F$-acyclic.
\end{prop}
%
\begin{proof}
Let $I$ be injective. Then take the injective resolution $0 \to I \to I \to 0$
where the map is the identity map.
\end{proof}
%
Therefore, to compute right derived functors, it suffices to find acyclic resolutions.
This gets us one step closer to finding nice resolutions for computing sheaf cohomology.
%
\begin{defn}
A sheaf $\mathcal{F}$ over $X$ is \ib{flasque} (also called \ib{flabby}) if
the restriction maps are surjective.
\end{defn}
%
Most sheaves in nature aren't flasque. However, flasque sheaves are useful for giving
us acyclic resolutions. To do this, we'll need some lemmas.
%
\begin{lem}
Let
\[\begin{tikzcd}
0 \ar[r] & \mathcal{E} \ar[r, "\alpha"] & \mathcal{F} \ar[r,"\beta"]
& \mathcal{G} \ar[r] & 0
\end{tikzcd}\]
be a short exact sequence where $\mathcal{E}$ is flasque. Then for any open set $U$,
the sequence
\[\begin{tikzcd}
0 \ar[r] & \mathcal{E}(U) \ar[r, "\alpha(U)"] & \mathcal{F}(U) \ar[r, "\beta(U)"]
& \mathcal{G}(U) \ar[r] & 0
\end{tikzcd}\]
is exact.
\end{lem}
%
\begin{proof}
By left-exactness of taking sections, it suffices to show that
$\beta(U) : \mathcal{F}(U) \to \mathcal{G}(U)$ is surjective. Let
$\sigma \in \mathcal{G}(U)$. Since $\beta$ is a surjective sheaf morphism, for any
$x \in U$, the induced map on stalks $\beta_x : \mathcal{F}_x \to \mathcal{G}_x$ is
surjective, which implies that there exists a sufficiently small neighborhood
$V_x \subset U$ of $x$ where
$\beta(V_x) : \mathcal{F}(V_x) \to \mathcal{G}(V_x)$ is surjective,
so we can find $\tau_x \in \mathcal{F}(V_x)$ such that
$\beta(V_x)(\tau) = \sigma\vert_{V_x}$. Then let $y \in U$ such that
$V_x \cap V_y \neq \emptyset$, and let $\tau_y \in \mathcal{F}(V_y)$ such that
$\beta(V_y)(\tau_y) = \sigma\vert_{V_y}$. Then we know that
$\tau_x\vert_{V_x \cap V_y} - \tau_y\vert_{V_x \cap V_y} \in \ker\beta(V_x \cap V_y)$,
so it is the image of an element $k \in \mathcal{E}(V_x \cap V_y)$. Since $\mathcal{E}$
is flasque, we know that we can lift $k$ to an section $\chi_{x,y} \in \mathcal{E}(V_x)$.
Then the element $\tau_x' = \tau_x - \alpha(V_x)(\chi_{x,y})$ still maps to
$\sigma\vert_{V_x}$  under $\beta(V_x)$ by exactness, and restricts to $\tau_y$ on
$V_x \cap V_y$. Therefore, $\tau_x'$ and $\tau_y$ glue to a section over $V_x \cup V_y$
that maps to $\sigma\vert_{V_x \cup V_y}$.  We then find a maximal pair $(W, \tau)$ such
that $\tau \in \mathcal{F}(W)$ and $\beta(W)(\tau) = \sigma\vert_W$. Then we must
necessarily have $W = U$, since otherwise, we can find another open subset $V$ of $U$
and a section $\varphi$ over $V$ mapping to $\sigma\vert_V$, and extend $\tau$ to a
section over $W \cup V$, since if $W \cap V \neq \emptyset$, we can use our above
argument, and otherwise, no work needs to be done. Either way, finding such a $U$
and $\varphi$ contradicts maximality of $(W,\tau)$.
\end{proof}
%
\begin{lem}
Let
\[\begin{tikzcd}
0 \ar[r] & \mathcal{E} \ar[r,"\alpha"] & \mathcal{F} \ar[r,"\beta"]
& \mathcal{G} \ar[r] & 0
\end{tikzcd}\]
be a short exact sequence of sheaves where $\mathcal{E}$ and $\mathcal{F}$ are flasque.
Then $\mathcal{G}$ is flasque.
\end{lem}
%
\begin{proof}
Let $V \subset U$. Then given $\sigma \in \mathcal{G}(V)$, we want to show that
there exits $\widetilde{\sigma} \in \mathcal{G}(U)$ that restricts to $\sigma$. Since
$\mathcal{E}$ is flasque, we have that
\[\begin{tikzcd}
0 \ar[r] & \mathcal{E}(V) \ar[r, "\alpha(V)"] & \mathcal{F}(V) \ar[r,"\beta(V)"]
& \mathcal{G}(V) \ar[r] & 0
\end{tikzcd}\]
is exact, so we can find a section $\tau \in \mathcal{F}(V)$ with
$\beta{V}(\tau) = \sigma$. Then since $\mathcal{F}$ is flasque, this lifts to
an element $\widetilde{\tau} \in \mathcal{F}(U)$. Then taking
$\widetilde{\sigma} = \beta(U)(\widetilde{\tau})$ gives us the desired section, since
the properties of a sheaf morphism implies that the following diagram commutes:
\[\begin{tikzcd}
\mathcal{F}(U) \ar[r, "\beta(U)"] \ar[d] & \mathcal{G}(U) \ar[d] \\
\mathcal{F}(V) \ar[r, "\beta(V)"'] & \mathcal{G}(V)
\end{tikzcd}\]
\end{proof}
%
\begin{prop}
Flasque sheaves are acyclic for the global sections functor $\Gamma(X, \cdot)$.
\end{prop}
%
\begin{proof}
Let $\mathcal{F}$ be a flasque sheaf. We first embed $\mathcal{F}$ into an injective
flasque sheaf $\mathcal{I}$. Since $\mathsf{Ab}$ has enough injectives, we can embed
each stalk $\mathcal{F}_x$ into an injective group $I_x$. Then define the sheaf
$\mathcal{I}$ by
\[
\mathcal{I}(U) = \prod_{x \in U} I_x
\]
and the restriction maps are the projection maps
$\prod_{x \in U} I_x \to \prod_{x \in V} I_x$. Since these maps are surjective,
$\mathcal{I}$ is flasque. In addition, it is injective by construction. Then
$\mathcal{F}$ embeds into $\mathcal{I}$ by composing the inclusions
$\mathcal{F}(U)\hookrightarrow\prod_{x\in U}\mathcal{F}_x\hookrightarrow\prod_{x\in V}I_x$
Then let $\mathcal{G}$ be the cokernel of $\mathcal{F} \hookrightarrow \mathcal{I}$,
giving us the exact sequence of sheaves
\[\begin{tikzcd}
0 \ar[r] & \mathcal{F} \ar[r] & \mathcal{I} \ar[r] & \mathcal{G} \ar[r] & 0
\end{tikzcd}\]
taking resolutions of $\mathcal{F}$, $\mathcal{I}$ and $\mathcal{G}$ gives
us a long exact sequence in cohomology
\[\begin{tikzcd}
0 \ar[r] & H^0(X,\mathcal{F}) \ar[r] & H^0(X,\mathcal{I}) \ar[r]
& H^0(X,\mathcal{G}) \ar[dll] \\
& H^1(X,\mathcal{F}) \ar[r] & H^1(X, \mathcal{I}) \ar[r] & H^1(X,\mathcal{G})\ar[dll] \\
& H^2(X,\mathcal{F}) \ar[r] & H^2(X, \mathcal{I}) \ar[r] & H^2(X,\mathcal{G})\ar[dll] \\
& ~ & \cdots & \cdots
\end{tikzcd}\]
The since $H^0$ is just $\Gamma(X,\cdot)$ and $\mathcal{I}$ is injective, this becomes
\[\begin{tikzcd}
0 \ar[r] & \mathcal{F}(X) \ar[r] & \mathcal{I}(X) \ar[r] & \mathcal{G}(X) \ar[dll] \\
& H^1(X,\mathcal{F}) \ar[r] & 0 \ar[r] & H^1(X,\mathcal{G}) \ar[dll] \\
& H^2(X,\mathcal{F}) \ar[r] & 0 \ar[r] & H^2(X,\mathcal{G}) \ar[dll] \\
& ~ & \cdots & \cdots
\end{tikzcd}\]
Since $\mathcal{F}$ is flasque, we have that
\[\begin{tikzcd}
0 \ar[r] & \mathcal{F}(X) \ar[r] & \mathcal{I}(X) \ar[r] & \mathcal{G}(X) \ar[r] & 0
\end{tikzcd}\]
is exact, so this implies that $H^1(X,\mathcal{F}) = 0$. In addition, for $i > 0$,
we get isomorphisms $H^i(X,\mathcal{G}) \to H^{i+1}(X, \mathcal{F})$. To get that
$H^2(X,\mathcal{F}) = 0$, we note that $\mathcal{G}$ is flasque, so we can
repeat the argument to get that $H^1(X,\mathcal{G}) = H^2(X, \mathcal{F})$, and
then continue for the other cohomology groups.
\end{proof}
%
As a consequence, it suffices to find resolutions by flasque sheaves to compute sheaf
cohomology. The good news here is that every sheaf admits a canonical
acyclic resolution, the \ib{Godement resolution}. For a sheaf $\mathcal{F}$,
let $\mathcal{F}_{\mathrm{God}}$ be the sheaf $U \mapsto \prod_{x \in U}\mathcal{F}_x$,
which is clearly flasque. Then we construct a flasque resolution for
$\mathcal{F}$ as follows : embed
$\mathcal{F} \hookrightarrow \mathcal{F}_{\mathrm{God}}$, and let $\mathcal{G}^1$ be
the cokernel. Then let the next sheaf in the sequence be $\mathcal{G}^1_{\mathrm{God}}$,
where the map $\mathcal{F}_{\mathrm{God}} \to \mathcal{G}^1_{\mathrm{God}}$ is the
quotient map $\mathcal{F}_{\mathrm{God}} \to \mathcal{G}^1$ composed in the the inclusion
$\mathcal{G}^1 \hookrightarrow \mathcal{G}^1_{\mathrm{God}}$. Then take $\mathcal{G}^2$
to be the cokernel of $\mathcal{F}_{\mathrm{God}} \to \mathcal{G}^1_{\mathrm{God}}$,
and continue with $\mathcal{G}^2_{\mathrm{God}}$ and so on. Pictorially, the
Godement construction is constructed as follows:
\[\begin{tikzcd}
0 \ar[r] & \mathcal{F} \ar[r] & \mathcal{F}_{\mathrm{God}} \ar[dr] \ar[r]
& \mathcal{G}^1 \ar[r] \ar[d] & 0 \\
& & & \mathcal{G}^1_{\mathrm{God}} \ar[r] \ar[dr] & \mathcal{G}^2 \ar[r] \ar[d] & 0\\
& & & & \mathcal{G}^2_{\mathrm{God}} \ar[r] \ar[dr] & \mathcal{G}^3 \ar[r] \ar[d] & 0 \\
& & & & & \ddots
\end{tikzcd}\]
%
Once more, this resolution is computationally useless, but it serves the purpose of
showing that every sheaf admits a resolution by flasque sheaves. With this in hand,
we can finally show that we can compute sheaf cohomology with a nice class of sheaves.
%
\begin{defn}
Let $\mathcal{A}$ be a sheaf of rings over $X$ such that $\mathcal{A}$ admits
\ib{partitions of unity}, i.e. for any open cover $\set{U_i}$ of $X$, there exist global
sections $f_i \in \mathcal{A}(X)$ such that $\sum_i f_i = 1$ and $f_i$ is supported
in $U_i$, and where over any particular open set, all but finitely many of the $f_i$
are $0$. Then a sheaf of $\mathcal{A}$-modules is a \ib{fine sheaf}.
\end{defn}
%
These are sheaves we care about, and appear in nature.
%
\begin{exmp}
Let $M$ be a smooth manifold. Then the sheaf $C^\infty$ of smooth functions
admits partitions of unity. Therefore, $\mathsf{Mod}_{C^\infty}$ consists of fine
sheaves.
\end{exmp}
%
The punchline is that fine sheaves are acyclic, which gives us a source of reasonable
sheaves with which to build resolutions.
%
\begin{thm}
Fine sheaves are acyclic with respect to $\Gamma(X,\cdot)$.
\end{thm}
%
\begin{proof}
Let $\mathcal{F}$ be a fine sheaf -- a sheaf of modules over a sheaf of rings
$\mathcal{A}$ that admits partitions of unity. Then by taking the
Godement resolution of $\mathcal{F}$, we get an injective resolution of
flasque $\mathcal{A}$-modules.
\[\begin{tikzcd}
0 \ar[r] & \mathcal{F} \ar[r] & \mathcal{I}^0 \ar[r, "d^0"] & \mathcal{I}^1 \ar[r, "d^1"]
& \cdots
\end{tikzcd}\]
Then since flasque sheaves are acyclic, we get that we can compute the sheaf
cohomology of $\mathcal{F}$ as
\[
H^i(X, \mathcal{F})
= \frac{\ker d^{i+1}(X) : \mathcal{I}^i(X) \to \mathcal{I}^{i+1}(X)}
{\im d^i(X) : \mathcal{I}^{i-1}(X) \to \mathcal{I}^i(X)}
\]
Then let $\alpha \in \ker d^{i+1}(X)$, by exactness, we know that locally, in a
sufficiently small open cover $\set{U_j}$, the map $d^{i}(U_j)$ is surjective,
so we can find $\beta_j \in \mathcal{I}^{i-1}(U_j)$ such that
$d^i(V_j)(\beta_j) = \alpha\vert_{U_j}$. Then $f_i\beta_i$ determines a global
section that is equal to $f_j\beta_j$ on $U_i$ and $0$ elsewhere, and we get that
$\sum_jf_j\beta_j$ maps to $\sum_j f_j\alpha\vert_{U_j} = \alpha$ under $d^i(X)$.
Therefore, the sequence is exact on global sections for $i > 0$, so $\mathcal{F}$
is acyclic.
\end{proof}
%
This tells us that many of the sheaves we know about, like sheaves of smooth sections
of a vector bundle are trivial from the perspective of sheaf cohomology.
%
\section{Comparison of Cohomology Theories}
%
The fact that many of the sheaves we encounter naturally have trivial sheaf cohomology
might come as a surprise, since we know we can extract topological data from
these sheaves. The reason for this is that they provide good resolutions of
other sheaves with nontrivial sheaf cohomology. U
%
\begin{prop}[Poincar\'e Lemma]
Every closed smooth $k$-form $\omega$ is locally exact, i.e. for a sufficiently
small $U$, we have that $\omega\vert_U = d\eta$ for some $k-1$-form $\eta$.
\end{prop}
%
\begin{prop}[$\dbar$-Poincar\'e Lemma]
Ever closed smooth $(p,q)$-form $\omega$ is locally $\dbar$-exact
\end{prop}
%
\begin{cor}
The de Rham complex
\[\begin{tikzcd}
A^0(X) \ar[r, "d"] & A^1(X) \ar[r, "d"] & \cdots
\end{tikzcd}\]
is an exact sequence of sheaves.
\end{cor}
%
The constant sheaf $\underline{\R}$ of locally constant real-valued functions
naturally lives as a subsheaf of $A^0(X)$, and we know that this exactly
the kernel of $d : A^0(X) \to A^1(X)$. This tells us that the inclusion
$0 \to \underline{\R} \to A^\bullet(X)$ is a resolution of $\underline{\R}$,
called the \ib{de Rham resolution}. Furthermore,
since all the $A^i(X)$ are sheaves of $C^\infty$-modules, they are fine, so
the resolution is a resolution of $\underline{\R}$ by acyclic sheaves. Therefore,
we get the isomorphisms
\[
H^i(X, \underline{\R}) \cong H^i_{dR}(X)
\]
A similar story holds for the sheaf cohomology of the sheaf of sections of a holomorphic
vector bundle $E \to X$. The $\dbar$-Poincar\'e lemma implies that the Dolbeault
complex
\[\begin{tikzcd}
\mathcal{A}^0(E) \ar[r, "\dbar_E"] & \mathcal{A}^1(E) \ar[r, "\dbar_e"] & \cdots
\end{tikzcd}\]
of sheaves of smooth sections of $(\Lambda^i T^*X)_\C \otimes E$ is an exact sequence,
since $\dbar_E$ is defined locally in terms of the operator $\dbar$ on $X$.
Then since the kernel of $\dbar_E : \mathcal{A}^0(E) \to \mathcal{A}^1(E)$ is
exactly the sheaf $\mathcal{E}$ of holomorphic sections of $E$, we get that
$0 \to \mathcal{E} \to \mathcal{A}^\bullet(E)$ is an acyclic resolution of
$\mathcal{E}$, which gives us isomorphisms
\[
H^i(X,\mathcal{E}) \cong H^i_{\dbar}(X, E)
\]
%
%TODO Cech vs derived functor cohomology
%
\newpage
%
\nocite{*}
%
\printbibliography
%
\end{document}