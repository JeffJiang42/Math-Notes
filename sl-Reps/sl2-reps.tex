\documentclass[psamsfonts]{amsart}
%
%-------Packages---------
%
\usepackage[h margin=1 in, v margin=1 in]{geometry}
\usepackage{amssymb,amsfonts}
\usepackage[all,arc]{xy}
\usepackage{tikz-cd}
\usepackage{enumerate}
\usepackage{mathrsfs}
\usepackage{amsthm}
\usepackage{mathpazo}
\usepackage{yfonts}
\usepackage{enumitem}
\usepackage{mathrsfs}
\usepackage{fourier-orns}
\usepackage[all]{xy}
\usepackage{hyperref}
\usepackage{cite}
\usepackage{url}
\usepackage{mathtools}
\usepackage{graphicx}
\usepackage{pdfsync}
\usepackage{mathdots}
\usepackage{calligra}
%
\usepackage{tgpagella}
\usepackage[T1]{fontenc}
%
\usepackage{listings}
\usepackage{color}

\definecolor{dkgreen}{rgb}{0,0.6,0}
\definecolor{gray}{rgb}{0.5,0.5,0.5}
\definecolor{mauve}{rgb}{0.58,0,0.82}

\lstset{frame=tb,
  language=Matlab,
  aboveskip=3mm,
  belowskip=3mm,
  showstringspaces=false,
  columns=flexible,
  basicstyle={\small\ttfamily},
  numbers=none,
  numberstyle=\tiny\color{gray},
  keywordstyle=\color{blue},
  commentstyle=\color{dkgreen},
  stringstyle=\color{mauve},
  breaklines=true,
  breakatwhitespace=true,
  tabsize=3
  }
%
%--------Theorem Environments--------
%
\newtheorem{thm}{Theorem}[section]
\newtheorem*{thm*}{Theorem}
\newtheorem{cor}[thm]{Corollary}
\newtheorem{prop}[thm]{Proposition}
\newtheorem{lem}[thm]{Lemma}
\newtheorem*{lem*}{Lemma}
\newtheorem{conj}[thm]{Conjecture}
\newtheorem{quest}[thm]{Question}
%
\theoremstyle{definition}
\newtheorem{defn}[thm]{Definition}
\newtheorem*{defn*}{Definition}
\newtheorem{defns}[thm]{Definitions}
\newtheorem{con}[thm]{Construction}
\newtheorem{exmp}[thm]{Example}
\newtheorem{exmps}[thm]{Examples}
\newtheorem{notn}[thm]{Notation}
\newtheorem{notns}[thm]{Notations}
\newtheorem{addm}[thm]{Addendum}
\newtheorem{exer}[thm]{Exercise}
%
\theoremstyle{remark}
\newtheorem{rem}[thm]{Remark}
\newtheorem*{claim}{Claim}
\newtheorem*{aside*}{Aside}
\newtheorem*{rem*}{Remark}
\newtheorem*{hint*}{Hint}
\newtheorem*{note}{Note}
\newtheorem{rems}[thm]{Remarks}
\newtheorem{warn}[thm]{Warning}
\newtheorem{sch}[thm]{Scholium}
%
%--------Macros--------
\renewcommand{\qedsymbol}{$\blacksquare$}
\renewcommand{\sl}{\mathfrak{sl}}
\renewcommand{\hom}{\mathsf{Hom}}
\renewcommand{\emptyset}{\varnothing}
\renewcommand{\O}{\mathscr{O}}
\newcommand{\R}{\mathbb{R}}
\newcommand{\ib}[1]{\textbf{\textit{#1}}}
\newcommand{\Q}{\mathbb{Q}}
\newcommand{\Z}{\mathbb{Z}}
\newcommand{\N}{\mathbb{N}}
\newcommand{\C}{\mathbb{C}}
\newcommand{\A}{\mathbb{A}}
\newcommand{\F}{\mathbb{F}}
\newcommand{\M}{\mathcal{M}}
\renewcommand{\S}{\mathbb{S}}
\newcommand{\V}{\vec{v}}
\newcommand{\RP}{\mathbb{RP}}
\newcommand{\CP}{\mathbb{CP}}
\newcommand{\B}{\mathcal{B}}
\newcommand{\GL}{\mathsf{GL}}
\newcommand{\SL}{\mathsf{SL}}
\newcommand{\SP}{\mathsf{SP}}
\newcommand{\SO}{\mathsf{SO}}
\newcommand{\SU}{\mathsf{SU}}
\newcommand{\gl}{\mathfrak{gl}}
\newcommand{\g}{\mathfrak{g}}
\newcommand{\inv}{^{-1}}
\newcommand{\bra}[2]{ \left[ #1, #2 \right] }
\newcommand{\set}[1]{\left\lbrace #1 \right\rbrace}
\newcommand{\abs}[1]{\left\lvert#1\right\rvert}
\newcommand{\norm}[1]{\left\lVert#1\right\rVert}
\newcommand{\transv}{\mathrel{\text{\tpitchfork}}}
\newcommand{\enumbreak}{\ \\ \vspace{-\baselineskip}}
\let\oldexists\exists
\renewcommand\exists{\oldexists~}
\let\oldL\L
\renewcommand\L{\mathfrak{L}}
\makeatletter
\newcommand{\tpitchfork}{%
  \vbox{
    \baselineskip\z@skip
    \lineskip-.52ex
    \lineskiplimit\maxdimen
    \m@th
    \ialign{##\crcr\hidewidth\smash{$-$}\hidewidth\crcr$\pitchfork$\crcr}
  }%
}
\makeatother
\newcommand{\bd}{\partial}
\newcommand{\lang}{\begin{picture}(5,7)
\put(1.1,2.5){\rotatebox{45}{\line(1,0){6.0}}}
\put(1.1,2.5){\rotatebox{315}{\line(1,0){6.0}}}
\end{picture}}
\newcommand{\rang}{\begin{picture}(5,7)
\put(.1,2.5){\rotatebox{135}{\line(1,0){6.0}}}
\put(.1,2.5){\rotatebox{225}{\line(1,0){6.0}}}
\end{picture}}
\DeclareMathOperator{\id}{id}
\DeclareMathOperator{\im}{Im}
\DeclareMathOperator{\codim}{codim}
\DeclareMathOperator{\coker}{coker}
\DeclareMathOperator{\supp}{supp}
\DeclareMathOperator{\inter}{Int}
\DeclareMathOperator{\sign}{sign}
\DeclareMathOperator{\sgn}{sgn}
\DeclareMathOperator{\indx}{ind}
\DeclareMathOperator{\alt}{Alt}
\DeclareMathOperator{\Aut}{Aut}
\DeclareMathOperator{\trace}{trace}
\DeclareMathOperator{\ad}{ad}
\DeclareMathOperator{\End}{End}
\DeclareMathOperator{\Ad}{Ad}
\DeclareMathOperator{\Lie}{Lie}
\DeclareMathOperator{\spn}{span}
\DeclareMathOperator{\dv}{div}
\DeclareMathOperator{\grad}{grad}
\DeclareMathOperator{\Sym}{Sym}
\DeclareMathOperator{\sheafhom}{\mathscr{H}\text{\kern -3pt {\calligra\large om}}\,}
\newcommand*\myhrulefill{%
   \leavevmode\leaders\hrule depth-2pt height 2.4pt\hfill\kern0pt}
\newcommand\niceending[1]{%
  \begin{center}%
    \LARGE \myhrulefill \hspace{0.2cm} #1 \hspace{0.2cm} \myhrulefill%
  \end{center}}
\newcommand*\sectionend{\niceending{\decofourleft\decofourright}}
\newcommand*\subsectionend{\niceending{\decosix}}
\def\upint{\mathchoice%
    {\mkern13mu\overline{\vphantom{\intop}\mkern7mu}\mkern-20mu}%
    {\mkern7mu\overline{\vphantom{\intop}\mkern7mu}\mkern-14mu}%
    {\mkern7mu\overline{\vphantom{\intop}\mkern7mu}\mkern-14mu}%
    {\mkern7mu\overline{\vphantom{\intop}\mkern7mu}\mkern-14mu}%
  \int}
\def\lowint{\mkern3mu\underline{\vphantom{\intop}\mkern7mu}\mkern-10mu\int}
%
%--------Hypersetup--------
%
\hypersetup{
    colorlinks,
    citecolor=black,
    filecolor=black,
    linkcolor=blue,
    urlcolor=blacksquare
}
%
%--------Solution--------
%
\newenvironment{solution}
  {\begin{proof}[Solution]}
  {\end{proof}}
%
%--------Graphics--------
%
%\graphicspath{ {images/} }

\begin{document}
%
\author{Jeffrey Jiang}
%
\title{Representations of $\mathfrak{sl}_2(\C)$}
%
\setcounter{section}{1}
%
\maketitle
%
The representation theory of the Lie algebra $\sl_2(\C)$ is extremely important
in the representation theory of semisimple Lie algebras, and consequently, it
is also important in the representation theory of Lie algebras. It has an explicit
and relatively simple description of all the irreducible representations, which
we will lay out here.\\

The \ib{special linear group} $SL_n(\C)$ is the group of complex $n \times n$
matrices with determinant $1$. By differentiating the determinant map at the
identity, we get that it's Lie algebra, denoted $\sl_n(\C)$ is the space
of complex $n \times n$ matrices with trace 0, equipped with the commutator
bracket. It admits a basis given by the matrices
\[
H = \begin{pmatrix}
1 & 0 \\
0 & -1
\end{pmatrix} \qquad X = \begin{pmatrix}
0 & 1 \\
0 & 0
\end{pmatrix} \qquad Y = \begin{pmatrix}
0 & 0 \\
1 & 0
\end{pmatrix}
\]
It's easy to verify that these satisfy the commutation relations
\[
[H,X] - 2X \qquad [H,Y] = -2Y \qquad [X,Y] = H
\]
As it turns out, this basis will make the analysis of irreducible representations
quite simple. Let $V$ be a irreducible representation of $\sl_2(\C)$. From the
semisimplity of $\sl_2(\C)$, we have that the representation preserves the Jordan
decomposition, and since $H$ is diagonalizable, it's action on $V$ will also be
diagonalizable, giving us a direct sum decomposition $V = \oplus_\lambda V_\lambda$
of eigenspaces $V_\lambda$ of $H$. Let $v \in V_\lambda$ be an eigenvector of $H$
with eigenvalue $\lambda$. To analyze the action of $X$ on $v$, we note a few
helpful identities
%
\begin{align*}
HX &= XH + HX - XH = XH + [H,X]\\
HY &= YH + HY - YH = YH - [H,Y]
\end{align*}
%
This then gives us the identities
%
\begin{align*}
HXv &= XHv + [H,X]v = \lambda Xv + 2Xv = (\lambda + 2)Xv \\
HYv &= YHv + [H,Y]v = \lambda Yv - 2Yv = (\lambda - 2)Yv
\end{align*}
%
So the action of $X$ raises the eigenvalue by $2$, while the action of $Y$ lowers
the eigenvalue by $2$. Furthermore, all the eigenvalues must differ by some
multiple of $2$. If we fix an eigenvalue $\lambda$, then the subspace
\[
\bigoplus_{k \in \Z}V_{\lambda + 2k}
\]
is fixed by the actions of $H$, $X$, and $Y$, so it is invariant under the action
of $\sl_2(\C)$. Then by irreducibility of $V$, this must be the whole space.
Since $V$ is finite dimensional, there exists some maximal eigenvalue $n$ of $V$.
Let $v \in V_n$. Since $V_{n+2} = 0$, $Xv = 0$, so we now want to observe the action
of $Y$ on $v$. We claim that the set
\[
S = \set{Y^mv ~:~ m \in \Z^{\geq 0}}
\]
spans all of $V$. Due to the irreducibility of $V$, it suffices to show that
the span of these vectors is fixed by the action of $H$, $X$, and $Y$. It is
clearly fixed by $Y$, and since all elements of $S$ are eigenvectors of $H$, it
is clear that the subspace is preserved by $H$ as well. The final thing to
check is the action of $X$. We know $Xv = 0$, and for $XY$, we compute
%
\begin{align*}
XYv &= YXv - [Y,X]v \\
&= 0 - (-Hv) \\
&= nv
\end{align*}
%
\begin{prop}
\[
XY^mv = m(n-m+1)Y^{m-1}v
\]
\end{prop}
%
\begin{proof}
We do this by induction. The base case $m = 1$ is verified above. Then assuming
the proposition for $m$, we wish to verify the identity for $Y^{m+1}$. We compute
\begin{align*}
XY^{m+1}v &= [X,Y](Y^mv) + YX(Y^mv) \\
&= HY^mv + Y(XY^m v) \\
&= (n-2m)Y^mv + Y(m(n-m+1)Y^{m-1}v) \\
&= ((n-2m)+ (m(n-m+1)))Y^mv \\
\end{align*}
Expanding out the coefficient and the expression $(m+1)(n-(m+1)+1)$ then verifies
this identity.
\end{proof}
%
Therefore $X$ fixes this subspace, showing that the span of $S$ must be all of $V$.
We also note that this gives a basis of eigenvectors for $V$, with each eigenvector
$Y^mv$ having a different eigenvalue, which implies that it spans the eigenspace
with eigenvalue $n - 2m$. Finally, we note that since $V$ is finite dimensional,
there must exist some smallest integer $m$ such that $Y^mv = 0$. Then from our
proposition above, we get
\[
0 = Y^mv = m(n - m + 1)Y^{m-1}v \implies m(n-m+1) = 0 \implies n-m+1 = 0
\]
so $n$ is a nonnegative integer, and the fact that $n-m+1$ implies that the
eigenvalues of $H$ must be symmetric about $0$, and the dimension of $V$ must
be $n+1$. Since the representation is determined by this maximal eigenvalue $n$,
we let $V^n$ denote the irreducible representation in which the maximal eigenvalue
of $H$ is $n$.

Consider the standard representation of $\sl_2(\C)$, where the matrices
act on $\C^2$ in the standard way. Then the standard basis vectors $e_1$ and $e_2$
are eigenvectors of $H$ with eigenvalues $1$ and $-1$ respectively. We then note
that $Xe_1 = Ye_2 = 0$, and $Xe_2 = e_1$ and $Ye_1 = e_2$, so their action of
$\C^2$ satisfies the exact same relations as the irreducible representation $V^1$.
we can then take the symmetric power $\Sym^2\C^2$, which has basis
$\set{e_1^2, e_1e_2, e_2^2}$. Using the standard rules for symmetric powers
of a Lie algebra representation, we compute
\begin{align*}
H(e_1^2) &= e_1He_1 + (He_1) e_1 = 2e_1^2 \\
H(e_1e_2) &= e_1He_2 + (He_1) e_2 = -e_1e_2 + e_1e_2 = 0 \\
H(e_2^2) &= e_2He_2 + (He_2) e_2 = -e_2^2 - e_2^2 = -2e_2^2
\end{align*}
Therefore, $\Sym^2\C^2$ is the irreducible representation $V^2$. As it turns
out, the symmetric powers of the standard representation on $\C^2$ form the
complete set of irreducible representations of $\sl_2(\C)$. The symmetric power
$\Sym^n\C^2$ has basis $\set{e_1^{n-k}e_2^k ~:~ 0 \leq k \leq n}$. We compute
the action of $H$ on these vectors to be
\[
H(e_1^{n-k}e_2^k) =
\]
%
\end{document}
