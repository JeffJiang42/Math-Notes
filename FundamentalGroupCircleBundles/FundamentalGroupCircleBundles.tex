\documentclass[psamsfonts, 12pt]{amsart}
%
%-------Packages---------
%
\usepackage[h margin=1 in, v margin=1 in]{geometry}
\usepackage{amssymb,amsfonts}
\usepackage{amsmath}
\usepackage{accents}
\usepackage[all,arc]{xy}
\usepackage{tikz-cd}
\usepackage{enumerate}
\usepackage{mathrsfs}
\usepackage{amsthm}
\usepackage{mathpazo}
\usepackage{float}
%\usepackage[backend=biber]{biblatex}
%\addbibresource{bibliography.bib}
%\usepackage{charter} %another font
%\usepackage{eulervm} %Vakil font
\usepackage{yfonts}
\usepackage{mathtools}
\usepackage{enumitem}
\usepackage{mathrsfs}
\usepackage{fourier-orns}
\usepackage[all]{xy}
\usepackage{hyperref}
\usepackage{url}
\usepackage{mathtools}
\usepackage{graphicx}
\usepackage{pdfsync}
\usepackage{mathdots}
\usepackage{calligra}
\usepackage{import}
\usepackage{xifthen}
\usepackage{pdfpages}
\usepackage{transparent}

\usepackage{tgpagella}
\usepackage[T1]{fontenc}
%
\usepackage{listings}
\usepackage{color}

\definecolor{dkgreen}{rgb}{0,0.6,0}
\definecolor{gray}{rgb}{0.5,0.5,0.5}
\definecolor{mauve}{rgb}{0.58,0,0.82}

\lstset{frame=tb,
  language=Matlab,
  aboveskip=3mm,
  belowskip=3mm,
  showstringspaces=false,
  columns=flexible,
  basicstyle={\small\ttfamily},
  numbers=none,
  numberstyle=\tiny\color{gray},
  keywordstyle=\color{blue},
  commentstyle=\color{dkgreen},
  stringstyle=\color{mauve},
  breaklines=true,
  breakatwhitespace=true,
  tabsize=3
  }
%
%--------Theorem Environments--------
%
\newtheorem{thm}{Theorem}[section]
\newtheorem*{thm*}{Theorem}
\newtheorem{cor}[thm]{Corollary}
\newtheorem{prop}[thm]{Proposition}
\newtheorem{lem}[thm]{Lemma}
\newtheorem*{lem*}{Lemma}
\newtheorem{conj}[thm]{Conjecture}
\newtheorem{quest}[thm]{Question}
%
\theoremstyle{definition}
\newtheorem{defn}[thm]{Definition}
\newtheorem*{defn*}{Definition}
\newtheorem{defns}[thm]{Definitions}
\newtheorem{con}[thm]{Construction}
\newtheorem{exmp}[thm]{Example}
\newtheorem{exmps}[thm]{Examples}
\newtheorem{notn}[thm]{Notation}
\newtheorem{notns}[thm]{Notations}
\newtheorem{addm}[thm]{Addendum}
\newtheorem{exer}[thm]{Exercise}
%
\theoremstyle{remark}
\newtheorem{rem}[thm]{Remark}
\newtheorem*{claim}{Claim}
\newtheorem*{aside*}{Aside}
\newtheorem*{rem*}{Remark}
\newtheorem*{hint*}{Hint}
\newtheorem*{note}{Note}
\newtheorem{rems}[thm]{Remarks}
\newtheorem{warn}[thm]{Warning}
\newtheorem{sch}[thm]{Scholium}
%
%--------Macros--------
\renewcommand{\qedsymbol}{$\blacksquare$}
\renewcommand{\sl}{\mathfrak{sl}}
\newcommand{\Bord}{\mathsf{Bord}}
\renewcommand{\hom}{\mathsf{Hom}}
\renewcommand{\emptyset}{\varnothing}
\renewcommand{\O}{\mathcal{O}}
\newcommand{\R}{\mathbb{R}}
\newcommand{\ib}[1]{\textbf{\textit{#1}}}
\newcommand{\Q}{\mathbb{Q}}
\newcommand{\Z}{\mathbb{Z}}
\newcommand{\N}{\mathbb{N}}
\newcommand{\C}{\mathbb{C}}
\newcommand{\A}{\mathbb{A}}
\newcommand{\F}{\mathbb{F}}
\newcommand{\M}{\mathcal{M}}
\newcommand{\dbar}{\overline{\partial}}
\newcommand{\zbar}{\overline{z}}
\renewcommand{\S}{\mathbb{S}}
\newcommand{\V}{\vec{v}}
\newcommand{\RP}{\mathbb{RP}}
\newcommand{\CP}{\mathbb{CP}}
\newcommand{\B}{\mathcal{B}}
\newcommand{\GL}{\mathrm{GL}}
\newcommand{\SL}{\mathrm{SL}}
\newcommand{\SP}{\mathrm{SP}}
\newcommand{\SO}{\mathrm{SO}}
\newcommand{\SU}{\mathrm{SU}}
\newcommand{\gl}{\mathfrak{gl}}
\newcommand{\g}{\mathfrak{g}}
\newcommand{\Bun}{\mathsf{Bun}}
\newcommand*{\dt}[1]{%
   \accentset{\mbox{\large\bfseries .}}{#1}}
\newcommand{\inv}{^{-1}}
\newcommand{\bra}[2]{ \left[ #1, #2 \right] }
\newcommand{\set}[1]{\left\lbrace #1 \right\rbrace}
\newcommand{\abs}[1]{\left\lvert#1\right\rvert}
\newcommand{\norm}[1]{\left\lVert#1\right\rVert}
\newcommand{\transv}{\mathrel{\text{\tpitchfork}}}
\newcommand{\defeq}{\vcentcolon=}
\newcommand{\enumbreak}{\ \\ \vspace{-\baselineskip}}
\let\oldexists\exists
\renewcommand\exists{\oldexists~}
\let\oldL\L
\renewcommand\L{\mathfrak{L}}
\makeatletter
\newcommand{\incfig}[2]{%
    \fontsize{48pt}{50pt}\selectfont
    \def\svgwidth{\columnwidth}
    \scalebox{#2}{\input{#1.pdf_tex}}
}
%
\newcommand{\tpitchfork}{%
  \vbox{
    \baselineskip\z@skip
    \lineskip-.52ex
    \lineskiplimit\maxdimen
    \m@th
    \ialign{##\crcr\hidewidth\smash{$-$}\hidewidth\crcr$\pitchfork$\crcr}
  }%
}
\makeatother
\newcommand{\bd}{\partial}
\newcommand{\lang}{\begin{picture}(5,7)
\put(1.1,2.5){\rotatebox{45}{\line(1,0){6.0}}}
\put(1.1,2.5){\rotatebox{315}{\line(1,0){6.0}}}
\end{picture}}
\newcommand{\rang}{\begin{picture}(5,7)
\put(.1,2.5){\rotatebox{135}{\line(1,0){6.0}}}
\put(.1,2.5){\rotatebox{225}{\line(1,0){6.0}}}
\end{picture}}
\DeclareMathOperator{\id}{id}
\DeclareMathOperator{\im}{Im}
\DeclareMathOperator{\codim}{codim}
\DeclareMathOperator{\coker}{coker}
\DeclareMathOperator{\supp}{supp}
\DeclareMathOperator{\inter}{Int}
\DeclareMathOperator{\sign}{sign}
\DeclareMathOperator{\sgn}{sgn}
\DeclareMathOperator{\indx}{ind}
\DeclareMathOperator{\alt}{Alt}
\DeclareMathOperator{\Aut}{Aut}
\DeclareMathOperator{\trace}{trace}
\DeclareMathOperator{\ad}{ad}
\DeclareMathOperator{\End}{End}
\DeclareMathOperator{\Ad}{Ad}
\DeclareMathOperator{\Lie}{Lie}
\DeclareMathOperator{\spn}{span}
\DeclareMathOperator{\dv}{div}
\DeclareMathOperator{\grad}{grad}
\DeclareMathOperator{\Sym}{Sym}
\DeclareMathOperator{\tr}{tr}
\DeclareMathOperator{\sheafhom}{\mathscr{H}\text{\kern -3pt {\calligra\large om}}\,}
\newcommand*\myhrulefill{%
   \leavevmode\leaders\hrule depth-2pt height 2.4pt\hfill\kern0pt}
\newcommand\niceending[1]{%
  \begin{center}%
    \LARGE \myhrulefill \hspace{0.2cm} #1 \hspace{0.2cm} \myhrulefill%
  \end{center}}
\newcommand*\sectionend{\niceending{\decofourleft\decofourright}}
\newcommand*\subsectionend{\niceending{\decosix}}
\def\upint{\mathchoice%
    {\mkern13mu\overline{\vphantom{\intop}\mkern7mu}\mkern-20mu}%
    {\mkern7mu\overline{\vphantom{\intop}\mkern7mu}\mkern-14mu}%
    {\mkern7mu\overline{\vphantom{\intop}\mkern7mu}\mkern-14mu}%
    {\mkern7mu\overline{\vphantom{\intop}\mkern7mu}\mkern-14mu}%
  \int}
\def\lowint{\mkern3mu\underline{\vphantom{\intop}\mkern7mu}\mkern-10mu\int}
%
%--------Hypersetup--------
%
\hypersetup{
    colorlinks,
    citecolor=black,
    filecolor=black,
    linkcolor=blue,
    urlcolor=blacksquare
}
%
%--------Solution--------
%
\newenvironment{solution}
  {\begin{proof}[Solution]}
  {\end{proof}}
%
%--------Graphics--------
%
%\graphicspath{ {images/} }
%
\begin{document}
%
\author{Jeffrey Jiang}
%
\title{Fundamental Groups of Principal Circle Bundles Over Riemann Surfaces}
%
\maketitle
%
Let $\Sigma$ be a Riemann surface with genus $g > 0$, and $\pi : P \to \Sigma$
a principal $U(1)$ bundle over $\Sigma$. This gives rise to a long exact sequence
of homotopy groups
\[\begin{tikzcd}
\cdots \ar[r] & \pi_2(U(1)) \ar[r] & \pi_2(P) \ar[r] & \pi_2(\Sigma) \ar[dll]\\
& \pi_1(U(1)) \ar[r] & \pi_1(P) \ar[r] & \pi_1(\Sigma) \ar[r] & 1
\end{tikzcd}\]
Since the genus of $\Sigma$ is greater than $0$, the universal cover of $\Sigma$
is contractible, which implies that $\pi_2(\Sigma) = 1$, since the covering map
induces isomorphisms on higher homotopy groups. Then since $\pi_1(U(1)) \cong \Z$,
we get the short exact sequence of groups
\[\begin{tikzcd}
1 \ar[r] & \Z \ar[r] & \pi_1(P) \ar[r] & \pi_1(\Sigma) \ar[r] & 1
\end{tikzcd}\]
so the fundamental group of $P$ is an extension of $\pi_1(M)$ by $\Z$. \\

Any principal $U(1)$-bundle $P \to \Sigma$ can be characterized up to isomorphism by
its first Chern class $c_1(P) \in H^2(\Sigma,\Z) \cong \Z$, which can be explicitly realized
as the cohomology class
\[
c_1(P) = \left[\frac{i}{2\pi}\Omega\right]
\]
where $\Omega$ is the curvature $2$-form of any connection $\Theta \in \Omega^1_P$ on $P$,
where we make the identification of $\mathfrak{u}(1) \cong \R$. Given any connection
$\Theta$, we get a $U(1)$-invariant horizontal distribution on $TP$ given by
$\ker\Theta$, which allows us to lift paths $\gamma : [0,1] \to \Sigma$ to paths
$\widetilde{\gamma}_p : [0,1] \to P$ for every lift $p$ of $\gamma(0)$. In particular, if
$\gamma(0) = \gamma(1)$, (i.e. $\gamma$ is a  loop on $\Sigma$), the lifts may or may not
lift to loops on $P$ : in general, for each lift $p$ of $\gamma(0)$, we have
$\widetilde{\gamma}_p(1) = p\cdot \theta_\gamma$ for some $\theta_\gamma \in U(1)$,
called the \ib{holonomy} associated to the loops $\gamma$. Then suppose we have a
simple curve $\gamma$ bounding a disk $D \subset \Sigma$ contained
in some open set in which $P$ is trivial. In this case, the curvature can
be written locally as $dA$ for some $1$-form $A \in \Omega^1_M$, and the holonomy
is computed by the integral
\[
\int_\gamma A = \int_D dA = \int_D \Omega
\]
This can be extended to simple curves that don't necessarily lie in neighborhoods
where the bundle is trivial by applying a subdivision argument with a triangulation. \\

We now return to the situation of computing the fundamental group of $P$. Recall
that the fundamental group of $\Sigma$ is given by the presentation
\[
\langle a_1,b_1,\ldots a_g,b_g ~:~ [a_1,b_1]\cdots[a_g,b_g] = 1\rangle
\]
The image of the map $\Z \to \pi_1(P)$ is given by mapping $1$ to a curve $c$ traversing
once around a fiber of $P \to \Sigma$. In addition, if we view $\Sigma$ as a quotient
of a $2g$-gon with identifications at the boundaries, we have that the other
generators of $\pi_1(P)$ will be lifts of the loops on the boundary With a slight abuse
of notation, we also denote these lifted loops by $a_i,\ldots b_i$. Furthermore, we
may assume that the $a_i$ and $b_i$ are horizontal lifts of the generators on the base
with respect to some connection on $P$. This gives us the generators of for $\pi_1(P)$,
namely the $a_i,b_i$, and the new generator $c$. For relations, we fix a connection on
$P$ and consider the product of commutators $[a_1,b_1]\cdots[a_g,b_g]$, which traverses
the boundary of the polygon once. By pushing this into the interior of the polygon by an
arbitrarily small amount, we get a simple  curve that bounds a disk that covers almost
all of $\Sigma$. We then have that the holonomy about this curve is the integral of the
curvature. Taking the limit as the curve approaches the boundary, we get that this
integral becomes $2\pi$ times the first Chern class of $P$, which tells us that the
holonomy is traversing $c_1(P)$ times around $c$. This gives the relation
\[
[a_1,b_1]\cdots[a_g,b_g] = c_1(P)c
\]
We then claim that the element $c$ is central. To see this, we see that since $P$
is orientable, it restricts to a trivial bundle over the $1$-skeleton of $\Sigma$.
Since the bundle is trivial, the loop $c$ commutes with all the $a_i$ and $b_i$
over the $1$-skeleton. Consequently, it commutes in $\Sigma$ as well. \\

Putting everything together, we get a characterization of fundamental
groups of principal $U(1)$-bundles $P \to \Sigma$ : they are central extensions
of $\pi_1(M)$ by a cyclic group generated by a single loop $c$, along with
the additional relation that $[a_1,b_1]\cdots [a_g,b_g] = c_1(P)c$.
%
\end{document}