\documentclass[psamsfonts]{amsart}
%
%-------Packages---------
%
\usepackage[h margin=1 in, v margin=1 in]{geometry}
\usepackage{amssymb,amsfonts}
\usepackage[all,arc]{xy}
\usepackage{enumerate}
\usepackage{mathrsfs}
\usepackage{amsthm}
\usepackage{mathpazo}
\usepackage{yfonts}
\usepackage{enumitem}
\usepackage{mathrsfs}
\usepackage{fourier-orns}
\usepackage[all]{xy}
\usepackage{hyperref}
\usepackage{cite}
\usepackage{url}
\usepackage{mathtools}
\usepackage{graphicx}
\usepackage{pdfsync}
\usepackage{mathdots}
%
\usepackage{tgpagella}
\usepackage[T1]{fontenc}
%
\usepackage{listings}
\usepackage{color}

\definecolor{dkgreen}{rgb}{0,0.6,0}
\definecolor{gray}{rgb}{0.5,0.5,0.5}
\definecolor{mauve}{rgb}{0.58,0,0.82}

\lstset{frame=tb,
  language=Matlab,
  aboveskip=3mm,
  belowskip=3mm,
  showstringspaces=false,
  columns=flexible,
  basicstyle={\small\ttfamily},
  numbers=none,
  numberstyle=\tiny\color{gray},
  keywordstyle=\color{blue},
  commentstyle=\color{dkgreen},
  stringstyle=\color{mauve},
  breaklines=true,
  breakatwhitespace=true,
  tabsize=3
  }
%
%--------Theorem Environments--------
%
\newtheorem{thm}{Theorem}[section]
\newtheorem*{thm*}{Theorem}
\newtheorem{cor}[thm]{Corollary}
\newtheorem{prop}[thm]{Proposition}
\newtheorem{lem}[thm]{Lemma}
\newtheorem*{lem*}{Lemma}
\newtheorem{conj}[thm]{Conjecture}
\newtheorem{quest}[thm]{Question}
%
\theoremstyle{definition}
\newtheorem{defn}[thm]{Definition}
\newtheorem*{defn*}{Definition}
\newtheorem{defns}[thm]{Definitions}
\newtheorem{con}[thm]{Construction}
\newtheorem{exmp}[thm]{Example}
\newtheorem{exmps}[thm]{Examples}
\newtheorem{notn}[thm]{Notation}
\newtheorem{notns}[thm]{Notations}
\newtheorem{addm}[thm]{Addendum}
\newtheorem{exer}[thm]{Exercise}
%
\theoremstyle{remark}
\newtheorem{rem}[thm]{Remark}
\newtheorem*{claim}{Claim}
\newtheorem*{aside*}{Aside}
\newtheorem*{rem*}{Remark}
\newtheorem*{hint*}{Hint}
\newtheorem*{note}{Note}
\newtheorem{rems}[thm]{Remarks}
\newtheorem{warn}[thm]{Warning}
\newtheorem{sch}[thm]{Scholium}
%
%--------Macros--------
\renewcommand{\qedsymbol}{$\blacksquare$}
\renewcommand{\hom}{\mathsf{Hom}}
\renewcommand{\emptyset}{\varnothing}
\newcommand{\R}{\mathbb{R}}
\newcommand{\ib}[1]{\textbf{\textit{#1}}}
\newcommand{\Q}{\mathbb{Q}}
\newcommand{\Z}{\mathbb{Z}}
\newcommand{\N}{\mathbb{N}}
\newcommand{\C}{\mathbb{C}}
\newcommand{\A}{\mathbb{A}}
\newcommand{\F}{\mathbb{F}}
\newcommand{\M}{\mathcal{M}}
\renewcommand{\S}{\mathbb{S}}
\newcommand{\V}{\vec{v}}
\newcommand{\RP}{\mathbb{RP}}
\newcommand{\CP}{\mathbb{CP}}
\newcommand{\B}{\mathcal{B}}
\newcommand{\GL}{\mathsf{GL}}
\newcommand{\SL}{\mathsf{SL}}
\newcommand{\SP}{\mathsf{SP}}
\newcommand{\SO}{\mathsf{SO}}
\newcommand{\SU}{\mathsf{SU}}
\newcommand{\gl}{\mathfrak{gl}}
\newcommand{\g}{\mathfrak{g}}
\newcommand{\inv}{^{-1}}
\newcommand{\bra}[2]{ \left[ #1, #2 \right] }
\newcommand{\ind}{\lambda \in \Lambda}
\newcommand{\set}[1]{\left\lbrace#1 \right\rbrace}
\newcommand{\imp}[2]{ \underline{ #1 \implies #2} }
\newcommand{\abs}[1]{\left\lvert#1\right\rvert}
\newcommand{\norm}[1]{\left\lVert#1\right\rVert}
\newcommand{\transv}{\mathrel{\text{\tpitchfork}}}
\newcommand{\enumbreak}{\ \\ \vspace{-\baselineskip}}
\let\oldexists\exists
\renewcommand\exists{\oldexists~}
\let\oldL\L
\renewcommand\L{\mathfrak{L}}
\makeatletter
\newcommand{\tpitchfork}{%
  \vbox{
    \baselineskip\z@skip
    \lineskip-.52ex
    \lineskiplimit\maxdimen
    \m@th
    \ialign{##\crcr\hidewidth\smash{$-$}\hidewidth\crcr$\pitchfork$\crcr}
  }%
}
\makeatother
\newcommand{\bd}{\partial}
\newcommand{\lang}{\begin{picture}(5,7)
\put(1.1,2.5){\rotatebox{45}{\line(1,0){6.0}}}
\put(1.1,2.5){\rotatebox{315}{\line(1,0){6.0}}}
\end{picture}}
\newcommand{\rang}{\begin{picture}(5,7)
\put(.1,2.5){\rotatebox{135}{\line(1,0){6.0}}}
\put(.1,2.5){\rotatebox{225}{\line(1,0){6.0}}}
\end{picture}}
\DeclareMathOperator{\id}{id}
\DeclareMathOperator{\im}{Im}
\DeclareMathOperator{\grap}{graph}
\DeclareMathOperator{\codim}{codim}
\DeclareMathOperator{\supp}{supp}
\DeclareMathOperator{\inter}{Int}
\DeclareMathOperator{\sign}{sign}
\DeclareMathOperator{\sgn}{sgn}
\DeclareMathOperator{\indx}{ind}
\DeclareMathOperator{\alt}{Alt}
\DeclareMathOperator{\Aut}{Aut}
\DeclareMathOperator{\trace}{trace}
\DeclareMathOperator{\ad}{ad}
\DeclareMathOperator{\End}{End}
\DeclareMathOperator{\Ad}{Ad}
\DeclareMathOperator{\Lie}{Lie}
\DeclareMathOperator{\spn}{span}
\DeclareMathOperator{\dv}{div}
\DeclareMathOperator{\grad}{grad}
\newcommand*\myhrulefill{%
   \leavevmode\leaders\hrule depth-2pt height 2.4pt\hfill\kern0pt}
\newcommand\niceending[1]{%
  \begin{center}%
    \LARGE \myhrulefill \hspace{0.2cm} #1 \hspace{0.2cm} \myhrulefill%
  \end{center}}
\newcommand*\sectionend{\niceending{\decofourleft\decofourright}}
\newcommand*\subsectionend{\niceending{\decosix}}
\def\upint{\mathchoice%
    {\mkern13mu\overline{\vphantom{\intop}\mkern7mu}\mkern-20mu}%
    {\mkern7mu\overline{\vphantom{\intop}\mkern7mu}\mkern-14mu}%
    {\mkern7mu\overline{\vphantom{\intop}\mkern7mu}\mkern-14mu}%
    {\mkern7mu\overline{\vphantom{\intop}\mkern7mu}\mkern-14mu}%
  \int}
\def\lowint{\mkern3mu\underline{\vphantom{\intop}\mkern7mu}\mkern-10mu\int}
%
%--------Hypersetup--------
%
\hypersetup{
    colorlinks,
    citecolor=black,
    filecolor=black,
    linkcolor=blue,
    urlcolor=black
}
%
%--------Solution--------
%
\newenvironment{solution}
  {\begin{proof}[Solution]}
  {\end{proof}}
%
%--------Graphics--------
%
%\graphicspath{ {images/} }

\begin{document}
\author{Jeffrey Jiang}
\title{The Laplace-de Rham Operator on a Riemannian Manifold}
\maketitle
%
\setcounter{section}{1}
%
In $\R^2$, we know about the standard Laplace operator on $C^\infty(\R^2)$
$$\Delta = \frac{\partial^2}{\partial x^2} + \frac{\partial^2 }{\partial y^2}$$
In a more general setting, let $(M,g)$ be a Riemannian manifold. We can define an analogous operator
$$\Delta = \dv(\grad f) $$
In local coordinates $(x^i)$, we have that  for $f \in C^\infty(M)$  and $X \in \mathfrak{X}(M)$
\begin{align*}
\grad f &= g^{ij}\frac{\partial f}{\partial x^i}\frac{\partial }{\partial x^j} \\[6pt]
\dv X &= \frac{1}{\sqrt[•]{\det  g_{ij}}} \frac{\partial}{\partial x^i}\left((X^i \sqrt{\det g_{ij}}\right)
\end{align*}
Where $g_{ij}$ is the symmetric matrix given by $g_{ij} = \langle \partial_i ,\partial_j\rangle$ and $g^{ij}$ is the inverse of $g_{ij}$. This gives the coordinate formula for 
$$\Delta f = \frac{1}{\sqrt{\det g_{ij}}} \frac{\partial}{\partial x^i}\left( g^{ij} \sqrt{\det g_{ij}}\frac{\partial f}{\partial x^j} \right)$$
Which using the standard metric $g_{ij} = \delta_{ij}$ on $\R^2$ recovers the standard Laplacian. However, we want to generalize $\Delta$ to arbitrary differential forms, which requires us to construct a  bit of machinery.

To do this, we first note that the metric $g$ determines an inner product on each tangent space $T_pM$ where $\langle v, w \rangle = g_p(v,w)$. From this, we can construct an inner product on the alternating tensors $\Lambda^k(T_pM)$, which will give us a smoothly varying inner product on $\Omega^K(M)$. To  do this, we will use the fact that $g$ determines a bundle isomorphism $TM \to  T^*M$ via the mapping $(x,v) \mapsto (x, \langle v, \cdot \rangle )$.
%
\begin{prop}
For a Riemannian manifold $(M,g)$,  there is a unique inner product on each $\Lambda^k(T_pM)$ characterized by the formula
$$\langle \omega^1 \wedge \ldots \wedge w^k, \eta^1 \wedge \ldots \wedge \eta^k = \det\left( \big\langle (\omega^i)^\sharp, (\eta^j)^\sharp \big\rangle \right)$$
Where $\sharp$ is the index raising operator $\omega_i dx^i \mapsto g^{ij}\omega_j\frac{\partial}{\partial x^i}$.
\end{prop}
%
\begin{proof}
We define the inner product locally in terms of an orthonomal frame $E_i$, and show that it is independent of the choice of frame. Let $\varepsilon^i$  denote the coframe to  $E_i$. We first claim that the set of $\varepsilon^I$ where $I$  is a strictly increasing multi-index of length $k$ form an orthonormal basis. To see this, we compute 
\begin{align*}
\langle \varepsilon^I, \varepsilon^J \rangle &= \det\left( E_{i_k}, E_{j_\ell} \right) 
\end{align*}
We note that this is $1$ if and only if $I = J$, since then the matrix we are taking the determinant of is $\id_{\R^k}$, otherwise, $I$  contains some $i_k$ not in  $J$, which implies the $k^{th}$ row of the matrix is $0$, so the determinant is $0$. This then defines an inner product by extending linearly to arbitrary $k$-forms.

To show that this is independent of our choice of frame, let  $B_i$ be another orthonormal frame with coframe $\beta^i$. Then we know that $B_i = A^j_iE_j$ with smooth functions $A^i_j$ forming an orthogonal matrix every point. We then compute
\begin{align*}
\langle \beta^I, \beta^J \rangle &= \det \langle B_{i_k}, B_{j_\ell} \rangle \\
&= \det\langle A^j_{i_k}E_j, A^p_{j_\ell}E_p \rangle \\
\end{align*}
Noting that $A^j_{i_k}E_j$ is just the $i_k^{th}$ column of the matrix $A$, we have that this is equal to $\det \langle A_{i_k}, A_{j_\ell} \rangle$. Again, if $I = J$, this is just the identity matrix, but if $I \neq J$, there will be a row of zeroes in the matrix $\langle A_{i_k}, A_{j_\ell}\rangle$, so the determinant will be $0$. This shows that $\langle \cdot, \cdot \rangle$  is uniquely characterized. 
\end{proof}
\begin{rem*}
One observation is that the Riemannian volume form $dV_g$  is the unique $n$-form on $M$ with norm $1$.
\end{rem*}
We can then use this inner product to produce an important operator. Recall that given a function $f \in C^\infty(M)$, we can define the integral of $f$ over $M$ by integrating the $n$-form $fdV_g$, which is a bundle homomorphism $\Omega^0(M) \to \Omega^n(M)$. We can generalize this to arbitrary $k$ forms.
\begin{prop}
For every $k \in \set{0, \ldots  n}$, there exists a unique bundle homomorphism
$$\star: \Omega^{k}(M) \to \Omega^{n-k}(M)$$
called the \ib{Hodge star operator} such that for any $\omega, \eta  \in \Omega^k(M)$, we have that $\omega \wedge \star\eta = \langle \omega, \eta \rangle dV_g$ where $dV_g$ is the Riemannian volume form. The $n-k$-form $\star \omega$ is often referred to as the \ib{Hodge dual} to $\omega$
\end{prop}
\begin{proof}
We first prove uniqueness. Let $\varepsilon^i$  be the coframe to an orthonomal basis $E_i$. Then for and increasing index set $I$ of length $k$, we have that $\star$ must satisify 
$$\varepsilon^I \wedge \star \varepsilon^I = dV_g$$
Therefore, we must have that $\star \varepsilon^I = \pm \varepsilon^J$, where $I \cup J = \set{1, \ldots n}$ and $J$ is an increasing index and the sign are chosen such that when we permute $I$ and $J$ to be in increasing order, the sign chosen for $\star \varepsilon^I$ cancel the ones that come from the permutation, since otherwise, $\varepsilon \wedge \star \varepsilon^I = 0$. This uniquely characterizes $\star$ on a basis, so it uniquely extends linearly to $\Omega^k(M)$.
\end{proof}

One observation we make is that $\star\star \varepsilon^I = (-1)^{k(n-k)}\varepsilon^I$ ,which can be verified by shuffling the wedge products and carefully tracking signs. This extends to all $k$-forms, so $\star \star\omega = (-1)^{k(n-k)}\omega$. Another observation is that this determines a bundle \emph{isomorphism} $\Omega^k(M) \to \Omega^{n-k}(M)$, since it maps an orthonormal basis to an orthonormal basis.
\begin{exmp}
In $\R^n$ with the standard coordinates $x^i$  and the standard metric tensor  $g_{ij} = \delta_{ij}$, we have that the $dx^i$ form an global orthonormal coframe for $\R^n$. Given any $dx^i$, we have that 
$$\star dx^i =  (-1)^{i-1} dx^1 \wedge \ldots \wedge \hat{dx^i} \wedge \ldots \wedge dx^n $$
Where $\hat{dx^i}$ indicates that $dx^i$ is missing  from the wedge product. The sign comes from the fact that 
$$dx^i \wedge dx^1 \wedge \ldots \wedge \hat{dx^i} \wedge \ldots \wedge dx^n = (-1)^{i-1} dx^1\wedge \ldots \wedge dx^n$$
\end{exmp}
%
\begin{exmp}
For $\R^5$,  consider $\star\star dx^1 \wedge dx^3$. We first compute
$$\star dx^1 \wedge dx^3 = -dx^2 \wedge dx^4 \wedge dx^5 $$
$$\star\star dx^1\wedge dx^3 = \star -dx^2 \wedge dx^4 \wedge dx^5 = dx^1 \wedge dx^3$$
\end{exmp}
%
Finally, we can use the Hodge star to define yet another operator
\begin{defn}
Let $(M,g)$ be a compact oriented Riemannian manifold. Then the \ib{codifferential} $\delta$ (also denoted in the literature by $d^*$) is a map
\begin{align*}
\delta: \Omega^k(M) &\to \Omega^{k-1}(M) \\
\delta \omega &= (-1)^{n(k+1)+1}\star d \star \omega
\end{align*}
Where $\delta$ is  defined on $\Omega^0(M) = C^\infty(M)$ by  $\delta f = 0$ for all smooth functions $f$.
\end{defn}
%
\begin{prop}
The codifferential $\delta$ on  a Riemannian manifold $(M,g)$ without boundary satisfies the following properties:
\begin{enumerate}
\item $\delta^2 = 0$ 
\item For $\omega,\eta \in \Omega^k(M)$, let 
$$(\omega, \eta) = \int_M \langle \omega, \eta \rangle ~dV_g = \int_M \omega \wedge \star \eta$$
Then for $\omega \in \Omega^k(M)$ and $\eta \in \Omega^{k-1}(M)$, we have that 
$$(\delta \omega, \eta) = (\omega, d\eta) $$
where $d$ is the exterior derivative. In this way, we see that $\delta$  is the \ib{adjoint} of $d$ with respect to the inner product, justifying the name \ib{codifferential}.
\end{enumerate}
\end{prop}

\begin{proof}
\begin{enumerate}
\item We have that
\begin{align*}
\delta^2 &= (-1)^{n(k+1)+1} \delta \star d \star \\
&= (-1)^{n(k+1) + 1}(-1)^{nk + 1} \star d \star \star d \star
\end{align*}
We note that $\star\star = (-1)^{k(n-k)}\id_{\Omega^k(M)}$, so this simplifies to
$$(-1)^p \star d d \star = 0 $$
Since $d^2 = 0$.
\item We first verify that $(\cdot, \cdot)$ determines  an inner product. We note that it is symmetric since $\langle  \cdot, \cdot \rangle$ is symmetric, and it is also bilinear since integration is linear and $\langle \cdot, \cdot \rangle$ is as well. All that remains is to show that it is positive definite. We note that it is positive since $\langle \omega, \omega \rangle$ is positive for all $\omega$, so 
$$\int_M \langle\omega,\omega\rangle dV_g > 0$$
In addition, we have that $\langle \omega, \omega \rangle = 0$ if and only if $\omega = 0$, and $\int_M fdV_g = 0$ if and only  if $f = 0$. Therefore, $(\cdot,\cdot)$ is positive definite, so it defines an inner product on $\Omega^k(M)$.

We then want to show
$$\int_M \langle \delta\omega,\eta \rangle~dV_g = \int_M \langle\omega,d\eta\rangle~dV_g \iff\int_M  \delta\omega \wedge \star\eta = \int_M \omega \wedge \star d\eta $$
Then using symmetry of the inner product, this is equivalent to the statement
$$\int_M \eta \wedge \star\delta\omega = \int_M d\eta \wedge \star\omega $$
We then compute
\begin{align*}
d\eta \wedge \star\omega - \eta \wedge \star\delta\omega &= d\eta \wedge\star\omega - (-1)^{n(k+1)+1}\eta \wedge \star\star d\star\omega \\
&= d\eta \wedge \star\omega - (-1)^{n(k+1)+1}(-1)^{(n-k+1)(n-(n-k+1))}\eta \wedge d\star\omega \\
&= d\eta \wedge \star\omega + (-1)^{-k^2 + 1}  \eta \wedge d\star\omega \\
&= d\eta \wedge  \star\omega + (-1)^{k-1}\eta\wedge d\star\omega \\
&= d(\eta \wedge \star\omega)
\end{align*}
Where we use the fact that $-k^2  + 1$ has the opposite parity of $k$, and that $d$ is an antiderivation on $\Omega(M)$
Therefore, we have by Stokes' Theorem
$$\int_M d\eta \wedge \star\omega - \eta \wedge \star d\omega = \int_M d(\eta \wedge\star\omega) = \int_{\partial M} \eta \wedge  \star\omega = 0 $$
Which gives us that 
$$ (\delta\omega, \eta)= (\omega,d\eta)$$
\end{enumerate}
\end{proof}
Finally,  we have the necessary tools to define the fabled \ib{Laplace-de Rham Operator} (Also known as the  \ib{Laplace-Beltrami Operator}).
\begin{defn}
On a oriented compact Riemannian manifold $(M,g)$, define the \ib{Laplace-de Rham Operator}, denoted $\Delta$, as  the family of maps $\Omega^k(M) \to \Omega^k(M)$ satsifying
$$\Delta = \delta d + d\delta $$
A $k$-form $\omega$ satsifying $\Delta\omega = 0$ is called \ib{harmonic}, and the space of harmonic $k$-forms is denoted $\mathcal{H}^k(M)$
\end{defn}

\begin{prop}
The Laplace-de Rahm operator agrees with (up to sign) to the Laplacian on $C^\infty(M)$
\end{prop}
%
\begin{proof}
We have that for $f \in C^\infty(M)$ (letting $\Delta$ denote the Laplace-de Rham operator)
$$\Delta f  = d\delta f + \delta d f = \delta df$$
Since we defined $\delta$ to be $0$ on $\Omega^0(M) = C^\infty(M)$.
We then have that $\delta df = -\star d\star df$. We first compute $\star df$, which is the $n-1$-form satisfying 
$$df \wedge \star df = \langle df, df \rangle ~dV_g $$. Noting that $df^\sharp = \grad f$, we compute
\begin{align*}
\langle \grad f, \grad f\rangle ~dV_g &= \frac{\partial f}{\partial x^i} dx^i \left( g^{jk}\frac{\partial f}{\partial j}\frac{\partial}{\partial x^k} \right)~dV_g \\
&= g^{ij}\frac{\partial  f}{\partial x^i}\frac{\partial f}{\partial x^j} ~dV_g \\
&= \sqrt{\det g}\left( g^{ij}\frac{\partial  f}{\partial x^i}\frac{\partial f}{\partial x^j} \right)dx^1\wedge \ldots \wedge dx^n
\end{align*}
Therefore, we conclude that
$$\star df = (-1)^{i-1}\sqrt{\det g} \left(g^{ij}\frac{\partial f}{\partial x^j} \right)dx^1 \wedge \ldots \hat{dx^i} \wedge \ldots \wedge dx^n $$
We then compute $d\star df$ to be 
\begin{align*}
d\star df &= (-1)^{i-1} \frac{\partial}{\partial x^k}\left( g^{ij}\sqrt{\det g}\frac{\partial f}{\partial x^j} \right) dx^k \wedge dx^1 \wedge \ldots \wedge \hat{dx^i} \wedge \ldots \wedge dx^n \\
&= \frac{\partial}{\partial x^i}\left( g^{ij}\sqrt{\det g}\frac{\partial f}{\partial x^j} \right) dx^1 \wedge \ldots \wedge dx^n \\
&= \frac{1}{\sqrt{\det g}} \frac{\partial}{\partial x^i}\left( g^{ij}\sqrt{\det g}\frac{\partial f}{\partial x^j} \right) dV_g
\end{align*}

We then have that $\star d\star df$ must satsify $d\star f \wedge \star d\star df = \langle d\star f, d\star f \rangle~dV_g$. We then compute
\begin{align*}
\biggl\langle \frac{1}{\sqrt{\det g}} \frac{\partial}{\partial x^i}\left( g^{ij}\sqrt{\det g}\frac{\partial f}{\partial x^j} \right) dV_g, \frac{1}{\sqrt{\det g}} \frac{\partial}{\partial x^k}\left( g^{k\ell}\sqrt{\det g}\frac{\partial f}{\partial x^\ell} \right)~dV_g\biggr\rangle~dV_g  \\
= \frac{1}{\sqrt{\det g}} \frac{\partial}{\partial x^i}\left( g^{ij}\sqrt{\det g}\frac{\partial f}{\partial x^j} \right)    \frac{1}{\sqrt{\det g}} \frac{\partial}{\partial x^k}\left( g^{k\ell}\sqrt{\det g}\frac{\partial f}{\partial x^k} \right) ~dV_g
\end{align*}
Where we use the fact that $\langle dV_g, dV_g \rangle = 1$. Therefore, we conclude that 
$$\delta df = -\star d\star f = -\frac{1}{\sqrt{\det g}} \frac{\partial}{\partial x^i}\left( g^{ij}\sqrt{\det g}\frac{\partial f}{\partial x^j} \right) $$
\end{proof}
Due to our choice of sign convention, this is unfortunately the negative of the the original Laplacian we had. Some sign conventions define $\Delta f = -\dv(\grad f)$ for this reason.\\

We make several important observations regarding $\Delta$
\begin{prop}\enumbreak
\begin{enumerate}
\item $\Delta$ commutes with $\star$, i.e. for $\omega \in \Omega^k(M)$, we have $\Delta\star\omega = \star\Delta\omega$
\item $\Delta$ is self adjoint with respect to $(\cdot,\cdot)$, i.e. for $\omega,\eta \in \Omega^k(M)$, we have that
$$(\Delta\omega, \eta ) = (\omega, \Delta\eta)$$
\item A $k$-form $\omega$ is harmonic if and only if $d\omega = \delta\omega = 0$.
\end{enumerate}
\end{prop}
\begin{proof}\enumbreak
\begin{enumerate}
\item We compute
\begin{align*}
\Delta\star\omega &= \delta d\star\omega + d\delta\star\omega \\
&= (-1)^{n(k+2) + 1}\star d\star d\star \omega + (-1)^{n(n-k+1)+1}d\star d\star\star\omega \\
&= (-1)^{n(k+2) + 1}\star d\star d\star \omega + (-1)^{n(n-k+1)+1}(-1)^{k(n-k)}d\star d\omega
\end{align*}
We then compute, using linearity of $\star$,
\begin{align*}
\star\Delta\omega &= \star \delta d\omega + \star d\delta\omega \\
&= (-1)^{n(n-k)+1}\star\star d\star d\omega + (-1)^{n(k+1)+1} \star d\star d \star \omega \\
&= (-1)^{n(n-k)+1}(-1)^{(n-k)k} d\star d\omega + (-1)^{n(k+1)+1} \star d\star d \star \omega
\end{align*}
We then compare signs, giving us equality.
\item We compute
\begin{align*}
(\omega, \Delta\eta) &= (\omega, d\delta\eta + \delta d\eta) \\
&= (\omega, d\delta\eta) + (\omega, \delta d\eta) \\
&= (\delta\omega,\delta\eta) + (d\omega,d\eta)  \\
&= (d\delta\omega,\eta) + (\delta d\omega,\eta) \\
&= (d\delta\omega + \delta d\omega,\eta) \\
&= (\Delta\omega,\eta)
\end{align*}
\item The backwards direction is clear. For the forwards direction, suppose $\Delta \omega = 0$. Then
\begin{align*}
&(\Delta\omega,\omega) = (0,\omega) = 0 \\
\implies &(d\delta\omega + \delta d\omega,\omega) = 0 \\
\implies &(d\delta\omega,\omega) + (\delta d\omega,\omega) = 0\\
\implies &(\delta\omega,\delta\omega) + (d\omega,d\omega) = 0
\end{align*}
Since $(\cdot,\cdot)$ is positive definite, this implies that both $\delta\omega = 0$ and $d\omega = 0$.
\end{enumerate}
\end{proof}
While we don't have all the tools to do it, we state important applications of this operator
\begin{thm}[\ib{The Hodge Decomposition}]
The space $\Omega^k(M)$ decomposes as the orthogonal sum
$$\Omega^k(M) = \im \Delta \oplus \mathcal{H}^k(M) $$
\end{thm}
The proof usually involves some heavy functional analysis. We note that all the trouble here is to prove that $\im \Delta \supset (\ker\Delta)^\perp$. Since $\Delta$ is self adjoint, we can easily see that it's image is contained in the orthogonal complement of $\ker \Delta$. Let $\Delta \omega \in \im \Delta$ and $\Delta \eta = 0$. Then
$$(\Delta\omega, \eta) = (\omega,\Delta\eta) = 0 $$
Therefore, $\im\Delta \subset (\ker\Delta)^\perp$. In general, the image of a linear map is always contained in the orthogonal complement of the kernel of its adjoint. For finite dimensional vector spaces, this is actually equality, but the space of differential forms is infinite dimensional, so there is additional work to be done. In particular, we will need two big facts.
\begin{enumerate}
\item $\im \Delta = d\delta(\Omega^k(M)) \oplus \delta d(\Omega^k(M)) = d(\Omega^{k-1}(M)) \oplus \delta(\Omega^{k+1}(M))$
\item $\im \Delta = (\ker\Delta)^\perp = \mathcal{H}^k(M)^\perp$
\end{enumerate} 
\begin{cor}
There is a canonical isomorphism $\mathcal{H}^k(M) \to H^k_{dR}(M)$
\end{cor}
\begin{proof}
We have that 
$$\Omega^k(M) = \im\Delta \oplus \mathcal{H}^k(M) = d(\Omega^{k-1}(M)) \oplus \delta(\Omega^{k-1}(M)) \oplus \mathcal{H}^k(M)$$
and that $\ker d = (\im \delta)^\perp = \mathcal{H}^k(X) \oplus d(\Omega^{k-1}(M))$ This gives us that $\ker d / d(\Omega^{k-1}(M)) \cong \mathcal{H}^k(M)$.
\end{proof}
Another theorem that we'll state without proof.
\begin{thm}
Every cohomology class of $H^k_{dR}(M)$ has a unique harmonic representative.
\end{thm}
\begin{cor}[\ib{Poincar\'e duality}]
The pairing $\langle [\omega], [\eta]\rangle \mapsto \int_M \omega \wedge \star\eta$ is nondegenerate, giving an isomorphism $\mathcal{H}^k(M)\to(\mathcal{H}^{n-k}(M))^* \cong H_{n-k}(M, \R)$
\end{cor}
\begin{proof}
Given a nonzero harmonic form $\omega \in \mathcal{H}^k(M)$, we have that $\Delta\star\omega = \star\Delta\omega = 0$. So $\star\omega$ is harmonic as well. Then $\int_M \omega\wedge\star\omega = \int_M|\omega|dV_g \neq 0$. Therefore, this pairing is nondegenerate, giving us a canonical isomorphism.
\end{proof}
%
\end{document}