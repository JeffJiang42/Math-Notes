\documentclass[psamsfonts, 12pt]{amsart}
%
%-------Packages---------
%
\usepackage[h margin=1 in, v margin=1 in]{geometry}
\usepackage{amssymb,amsfonts}
\usepackage{amsmath}
\usepackage{accents}
\usepackage[all,arc]{xy}
\usepackage{tikz-cd}
\usepackage{enumerate}
\usepackage{mathrsfs}
\usepackage{amsthm}
\usepackage{mathpazo}
\usepackage{float}
%\usepackage[backend=biber]{biblatex}
%\addbibresource{bibliography.bib}
%\usepackage{charter} %another font
%\usepackage{eulervm} %Vakil font
\usepackage{yfonts}
\usepackage{mathtools}
\usepackage{enumitem}
\usepackage{mathrsfs}
\usepackage{fourier-orns}
\usepackage[all]{xy}
\usepackage{hyperref}
\usepackage{url}
\usepackage{mathtools}
\usepackage{graphicx}
\usepackage{pdfsync}
\usepackage{mathdots}
\usepackage{calligra}
\usepackage{import}
\usepackage{xifthen}
\usepackage{pdfpages}
\usepackage{transparent}

\usepackage{tgpagella}
\usepackage[T1]{fontenc}
%
\usepackage{listings}
\usepackage{color}

\definecolor{dkgreen}{rgb}{0,0.6,0}
\definecolor{gray}{rgb}{0.5,0.5,0.5}
\definecolor{mauve}{rgb}{0.58,0,0.82}

\lstset{frame=tb,
  language=Matlab,
  aboveskip=3mm,
  belowskip=3mm,
  showstringspaces=false,
  columns=flexible,
  basicstyle={\small\ttfamily},
  numbers=none,
  numberstyle=\tiny\color{gray},
  keywordstyle=\color{blue},
  commentstyle=\color{dkgreen},
  stringstyle=\color{mauve},
  breaklines=true,
  breakatwhitespace=true,
  tabsize=3
  }
%
%--------Theorem Environments--------
%
\newtheorem{thm}{Theorem}[section]
\newtheorem*{thm*}{Theorem}
\newtheorem{cor}[thm]{Corollary}
\newtheorem{prop}[thm]{Proposition}
\newtheorem{lem}[thm]{Lemma}
\newtheorem*{lem*}{Lemma}
\newtheorem{conj}[thm]{Conjecture}
\newtheorem{quest}[thm]{Question}
%
\theoremstyle{definition}
\newtheorem{defn}[thm]{Definition}
\newtheorem*{defn*}{Definition}
\newtheorem{defns}[thm]{Definitions}
\newtheorem{con}[thm]{Construction}
\newtheorem{exmp}[thm]{Example}
\newtheorem{exmps}[thm]{Examples}
\newtheorem{notn}[thm]{Notation}
\newtheorem{notns}[thm]{Notations}
\newtheorem{addm}[thm]{Addendum}
\newtheorem{exer}[thm]{Exercise}
%
\theoremstyle{remark}
\newtheorem{rem}[thm]{Remark}
\newtheorem*{claim}{Claim}
\newtheorem*{aside*}{Aside}
\newtheorem*{rem*}{Remark}
\newtheorem*{hint*}{Hint}
\newtheorem*{note}{Note}
\newtheorem{rems}[thm]{Remarks}
\newtheorem{warn}[thm]{Warning}
\newtheorem{sch}[thm]{Scholium}
%
%--------Macros--------
\renewcommand{\qedsymbol}{$\blacksquare$}
\renewcommand{\sl}{\mathfrak{sl}}
\newcommand{\Bord}{\mathsf{Bord}}
\renewcommand{\hom}{\mathsf{Hom}}
\renewcommand{\emptyset}{\varnothing}
\renewcommand{\O}{\mathcal{O}}
\newcommand{\R}{\mathbb{R}}
\newcommand{\ib}[1]{\textbf{\textit{#1}}}
\newcommand{\Q}{\mathbb{Q}}
\newcommand{\Z}{\mathbb{Z}}
\newcommand{\N}{\mathbb{N}}
\newcommand{\C}{\mathbb{C}}
\newcommand{\A}{\mathbb{A}}
\newcommand{\F}{\mathbb{F}}
\newcommand{\M}{\mathcal{M}}
\newcommand{\dbar}{\overline{\partial}}
\newcommand{\zbar}{\overline{z}}
\renewcommand{\S}{\mathbb{S}}
\newcommand{\V}{\vec{v}}
\newcommand{\RP}{\mathbb{RP}}
\newcommand{\CP}{\mathbb{CP}}
\newcommand{\B}{\mathcal{B}}
\newcommand{\GL}{\mathrm{GL}}
\newcommand{\SL}{\mathrm{SL}}
\newcommand{\SP}{\mathrm{SP}}
\newcommand{\SO}{\mathrm{SO}}
\newcommand{\SU}{\mathrm{SU}}
\newcommand{\gl}{\mathfrak{gl}}
\newcommand{\g}{\mathfrak{g}}
\newcommand{\Bun}{\mathsf{Bun}}
\newcommand*{\dt}[1]{%
   \accentset{\mbox{\large\bfseries .}}{#1}}
\newcommand{\inv}{^{-1}}
\newcommand{\bra}[2]{ \left[ #1, #2 \right] }
\newcommand{\set}[1]{\left\lbrace #1 \right\rbrace}
\newcommand{\abs}[1]{\left\lvert#1\right\rvert}
\newcommand{\norm}[1]{\left\lVert#1\right\rVert}
\newcommand{\transv}{\mathrel{\text{\tpitchfork}}}
\newcommand{\defeq}{\vcentcolon=}
\newcommand{\enumbreak}{\ \\ \vspace{-\baselineskip}}
\let\oldexists\exists
\renewcommand\exists{\oldexists~}
\let\oldL\L
\renewcommand\L{\mathfrak{L}}
\makeatletter
\newcommand{\incfig}[2]{%
    \fontsize{48pt}{50pt}\selectfont
    \def\svgwidth{\columnwidth}
    \scalebox{#2}{\input{#1.pdf_tex}}
}
%
\newcommand{\tpitchfork}{%
  \vbox{
    \baselineskip\z@skip
    \lineskip-.52ex
    \lineskiplimit\maxdimen
    \m@th
    \ialign{##\crcr\hidewidth\smash{$-$}\hidewidth\crcr$\pitchfork$\crcr}
  }%
}
\makeatother
\newcommand{\bd}{\partial}
\newcommand{\lang}{\begin{picture}(5,7)
\put(1.1,2.5){\rotatebox{45}{\line(1,0){6.0}}}
\put(1.1,2.5){\rotatebox{315}{\line(1,0){6.0}}}
\end{picture}}
\newcommand{\rang}{\begin{picture}(5,7)
\put(.1,2.5){\rotatebox{135}{\line(1,0){6.0}}}
\put(.1,2.5){\rotatebox{225}{\line(1,0){6.0}}}
\end{picture}}
\DeclareMathOperator{\id}{id}
\DeclareMathOperator{\im}{Im}
\DeclareMathOperator{\codim}{codim}
\DeclareMathOperator{\coker}{coker}
\DeclareMathOperator{\supp}{supp}
\DeclareMathOperator{\inter}{Int}
\DeclareMathOperator{\sign}{sign}
\DeclareMathOperator{\sgn}{sgn}
\DeclareMathOperator{\indx}{ind}
\DeclareMathOperator{\alt}{Alt}
\DeclareMathOperator{\Aut}{Aut}
\DeclareMathOperator{\trace}{trace}
\DeclareMathOperator{\ad}{ad}
\DeclareMathOperator{\End}{End}
\DeclareMathOperator{\Ad}{Ad}
\DeclareMathOperator{\Lie}{Lie}
\DeclareMathOperator{\spn}{span}
\DeclareMathOperator{\dv}{div}
\DeclareMathOperator{\grad}{grad}
\DeclareMathOperator{\Sym}{Sym}
\DeclareMathOperator{\tr}{tr}
\DeclareMathOperator{\sheafhom}{\mathscr{H}\text{\kern -3pt {\calligra\large om}}\,}
\newcommand*\myhrulefill{%
   \leavevmode\leaders\hrule depth-2pt height 2.4pt\hfill\kern0pt}
\newcommand\niceending[1]{%
  \begin{center}%
    \LARGE \myhrulefill \hspace{0.2cm} #1 \hspace{0.2cm} \myhrulefill%
  \end{center}}
\newcommand*\sectionend{\niceending{\decofourleft\decofourright}}
\newcommand*\subsectionend{\niceending{\decosix}}
\def\upint{\mathchoice%
    {\mkern13mu\overline{\vphantom{\intop}\mkern7mu}\mkern-20mu}%
    {\mkern7mu\overline{\vphantom{\intop}\mkern7mu}\mkern-14mu}%
    {\mkern7mu\overline{\vphantom{\intop}\mkern7mu}\mkern-14mu}%
    {\mkern7mu\overline{\vphantom{\intop}\mkern7mu}\mkern-14mu}%
  \int}
\def\lowint{\mkern3mu\underline{\vphantom{\intop}\mkern7mu}\mkern-10mu\int}
%
%--------Hypersetup--------
%
\hypersetup{
    colorlinks,
    citecolor=black,
    filecolor=black,
    linkcolor=blue,
    urlcolor=blacksquare
}
%
%--------Solution--------
%
\newenvironment{solution}
  {\begin{proof}[Solution]}
  {\end{proof}}
%
%--------Graphics--------
%
%\graphicspath{ {images/} }
%
\begin{document}
%
\author{Jeffrey Jiang}
%
\title{The Classifying Space of a Product Group}
%
\maketitle
%
Let $G = H \times K$ be a product group. The goal is to show that the product
space $BH \times BK$ has the homotopy type of $BG$.
%
\begin{lem}
Let $\pi_H : P \to M$ be a principal $H$ bundle and $\pi_l : Q \to M$ a principal
$K$-bundle. Then the pullback bundle $\pi_H^*Q$ is a principal $H \times K$ bundle
over $M$.
\end{lem}
%
\begin{proof}
We have the pullback diagram
\[\begin{tikzcd}
\pi_H^*Q \ar[r] \ar[d] & Q \ar[d, "\pi_K"] \\
P \ar[r, "\pi_H"'] & M
\end{tikzcd}\]
Let $U$ be a neighborhood where both $P$ and $Q$ are trivial. Then locally, the
pullback diagram becomes
\[\begin{tikzcd}
U\times H \times K \ar[r] \ar[d] & U \times K \ar[d] \\
U \times H \ar[r] & U
\end{tikzcd}\]
where all the maps are the obvious projections. This shows local triviality of
$\pi^*_HQ$ as a bundle over $M$, and also shows that the fibers are $H \times K$
torsors, so $\pi_H^*Q$ defines a principal $H\times K$ bundle over $M$.
\end{proof}
%
\iffalse
\begin{lem}
Let $G$ be a group, and $H$ a subgroup. Then a principal $G$-bundle $P \to M$
admits a reduction of structure group to $H$, i.e. a bundle isomorphism
\[
Q \times_H G \cong P
\]
where $Q \to M$ is a principal $H$ bundle, if and only if the associated bundle
$P \times_G G/H$ admits a section.
\end{lem}
%
\begin{proof}
Suppose we have a reduction of structure group to $H$, so $P \cong Q \times_H G$
for some principal $H$-bundle $Q$. This gives an isomorphism
$Q \times_H G/H \cong P \times_G G/H$. Identifying sections of $Q \times_H G/H$ with
$H$-equivariant maps $Q \to G/H$, we see that the constant map $q \mapsto H$ defines
a section. \\

In the other direction, suppose $P\times_G G/H$ admits a section
$\sigma : M \to P\times_G G/H$. We have a principal $H$-bundle
$P \times_G G \to P\times_G G/H$, so we can use $\sigma$ to get a pullback bundle
$Q$ over $M$ with structure group $H$.
\[\begin{tikzcd}
Q \ar[r]\ar[d] & P\times_G G \ar[d] \\
M \ar[r, "\sigma"'] & P\times_G G/H
\end{tikzcd}\]
We note that $P\times_G G \cong P$, so the map $Q \to P\times_G G$ defines
an embedding of $Q$ as a subbundle of $P$, giving us that $Q \times_H G \cong P$.
\end{proof}
\fi
%
\begin{prop}
Any principal $G = H \times K$ bundle $P \to M$ can be obtained as the pullback
of a principal $K$-bundle by a principal $H$ bundle.
\end{prop}
%
\begin{proof}
The principal $H\times K$-bundle $P$ has free actions of $H$ and $K$ via their
inclusions into the product $H \times K$, and the quotient spaces $P/K$ and $P/H$
have natural structures as principal $H$ and $K$-bundles respectively, where
we use the fact that the $H$-action and $K$-action on $P$ commute. We then want
to show that $P \to M$ is isomorphic to the pullback bundle obtained from
the pullback diagram
\[\begin{tikzcd}
\pi_H^*P_K \ar[r]\ar[d] & P/K \ar[d] \\
P/H \ar[r, "\pi_H"'] & M
\end{tikzcd}\]
We note that $P$ admits maps to $P/K$ and $P/H$ by quotienting by the actions
of $K$ and $H$ respectively, so by the universal property of the pullback, we get a
unique map $P \to \pi_H^*P/K$. Furthermore, this map is $H\times K$-equivariant, so
it is a map of principal $H\times K$-bundles over $M$, so it is an isomorphism.
\end{proof}
%
In this way, we see that a principal $H\times K$-bundle is equivalent to
the data of a principal $H$-bundle and a principal $K$-bundle. We know that
the data of a principal $H\times K$-bundle is equivalent to a homotopy class
of maps $M \to B(H\times K)$, and likewise for $H$ and $K$. Therefore, using the
universal property of the product, this tells us that $BH \times BK$ has the homotopy
type of $B(H \times K)$. \\

In fact, more can be said. Let $\pi : P \to M$ be a principal $H\times K$-bundle, and
let $P_H\to M$ and $P_K \to M$ be principal $H$ and $K$-bundles respectively such
that $P$ is isomorphic to the pullback of one along the other. Then a connection
on $P$ is equivalent to the data of connections on both $P_H$ and $P_K$. To see
this, we use the characterization of a connection as a splitting of the
exact sequence of vector bundles over $P$
\[\begin{tikzcd}
0 \ar[r] & \ker\pi_* \ar[r] & TP \ar[r] & \pi^*TM \ar[r] & 0
\end{tikzcd}\]
Since the bundle projections $P_H,P_K \to M$ are transverse (since they are both
submersions), we have that the tangent space of the fiber product
$P$ is the fiber product of the tangent spaces. Therefore, working fiberwise,
it suffices to prove the following linear algebra fact
%
\begin{prop}
Let $\alpha : A \to C$ and $\beta : B \to C$ be surjective linear maps, and
let $D \defeq A \times_C B$, giving us the pullback diagram
\[\begin{tikzcd}
D \ar[r, "p_A"] \ar[d, "p_B"'] \ar[dr, "p"] & A \ar[d,"\alpha"]\\
B \ar[r, "\beta"'] & C
\end{tikzcd}\]
Then a splitting of the exact sequence
\[\begin{tikzcd}
0 \ar[r] & \ker p \ar[r] & D \ar[r, "p"] & C \ar[r] & 0
\end{tikzcd}\]
is equivalent to the data of splittings of the analogous exact sequences for
$A$ and $B$.
\end{prop}
%
\begin{proof}
By the definition of the fiber product, we have that
\[
D = \set{(a,b) \in A \times B ~:~ \alpha(a) = \beta(b)}
\]
Therefore, if $(a,b) \in \ker p$, we must have that $\alpha(a) = \beta(b) = 0$.
This gives us a canonical isomorphism $\ker p \cong \ker\alpha\oplus\ker\beta$
and natural maps $\ker p \to \ker\alpha$ and $\ker p \to \ker\beta$. This
gives us the commutative diagram
\[\begin{tikzcd}
0 \ar[r] & \ker\alpha \ar[r] & A \ar[r,"\alpha"] & C \ar[d,equal] \ar[r] & 0 \\
0 \ar[r] & \ker p \ar[u]\ar[d]\ar[r] & D \ar[d, "p_B"']\ar[u, "p_A"]\ar[r,"p"]
& C \ar[r] \ar[d, equal] & 0 \\
0 \ar[r] & \ker\beta \ar[r] & B \ar[r,"\beta"] & C \ar[r] & 0
\end{tikzcd}\]
For one direction, suppose we have a splitting $j_D : C \to D$. Then we let
$j_A : C \to A$ be defined by $j_A \defeq p_A\circ j_D$, and we define $j_B : C \to B$
similarly. To see that $j_A$ is a splitting, we have that for $c \in C$
\begin{align*}
(\alpha \circ j_A)(c) &= (\alpha \circ p_A \circ j_D)(c) \\
&= (p \circ j_D)(c) \\
&= c
\end{align*}
So $j_A$ indeed defines a splitting. The same argument shows that $j_B$ is a
splitting. \\

In the other direction, suppose we are given splittings $j_A : C \to A$ and
$j_B : C \to B$. Then $C$ fits into the diagram
\[\begin{tikzcd}
C \ar[r, "j_A" ] \ar[d, "j_B"'] & A \ar[d, "\alpha"]\\
B \ar[r, "\beta"'] & C
\end{tikzcd}\]
So the universal property of the fiber product guarantees a map $j_D : C \to D$
such that the following diagram commutes:
\[\begin{tikzcd}
C \ar[dr, "j_D"] \ar[ddr, "j_B"', bend right=20] \ar[drr, "j_A", bend left=20]\\
& D \ar[r, "p_A" ] \ar[d, "p_B"']\ar[dr, "p"] & A \ar[d, "\alpha"]\\
& B \ar[r, "\beta"'] & C
\end{tikzcd}\]
Then since $\beta \circ j_B = \id_C$, we have that $p \circ j_D = \id_C$, so
$j_D$ defines a splitting.
\end{proof}
%
\end{document}