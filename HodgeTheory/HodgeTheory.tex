\documentclass[psamsfonts, 12pt]{amsart}
%
%-------Packages---------
%
\usepackage[h margin=1 in, v margin=1 in]{geometry}
\usepackage{amssymb,amsfonts}
\usepackage[all,arc]{xy}
\usepackage{tikz-cd}
\usepackage{enumerate}
\usepackage{mathrsfs}
\usepackage{amsthm}
\usepackage{mathpazo}
%\usepackage{charter} %another font
%\usepackage{eulervm} %Vakil font
\usepackage{yfonts}
\usepackage{mathtools}
\usepackage{enumitem}
\usepackage{mathrsfs}
\usepackage{fourier-orns}
\usepackage[all]{xy}
\usepackage{hyperref}
\usepackage{cite}
\usepackage{url}
\usepackage{mathtools}
\usepackage{graphicx}
\usepackage{pdfsync}
\usepackage{mathdots}
\usepackage{calligra}
\usepackage{import}
\usepackage{xifthen}
\usepackage{pdfpages}
\usepackage{transparent}

\newcommand{\incfig}[2]{%
    \fontsize{48pt}{50pt}\selectfont
    \def\svgwidth{\columnwidth}
    \scalebox{#2}{\input{#1.pdf_tex}}
}
%
\usepackage{tgpagella}
\usepackage[T1]{fontenc}
%
\usepackage{listings}
\usepackage{color}

\definecolor{dkgreen}{rgb}{0,0.6,0}
\definecolor{gray}{rgb}{0.5,0.5,0.5}
\definecolor{mauve}{rgb}{0.58,0,0.82}

\lstset{frame=tb,
  language=Matlab,
  aboveskip=3mm,
  belowskip=3mm,
  showstringspaces=false,
  columns=flexible,
  basicstyle={\small\ttfamily},
  numbers=none,
  numberstyle=\tiny\color{gray},
  keywordstyle=\color{blue},
  commentstyle=\color{dkgreen},
  stringstyle=\color{mauve},
  breaklines=true,
  breakatwhitespace=true,
  tabsize=3
  }
%
%--------Theorem Environments--------
%
\newtheorem{thm}{Theorem}[section]
\newtheorem*{thm*}{Theorem}
\newtheorem{cor}[thm]{Corollary}
\newtheorem{prop}[thm]{Proposition}
\newtheorem{lem}[thm]{Lemma}
\newtheorem*{lem*}{Lemma}
\newtheorem{conj}[thm]{Conjecture}
\newtheorem{quest}[thm]{Question}
%
\theoremstyle{definition}
\newtheorem{defn}[thm]{Definition}
\newtheorem*{defn*}{Definition}
\newtheorem{defns}[thm]{Definitions}
\newtheorem{con}[thm]{Construction}
\newtheorem{exmp}[thm]{Example}
\newtheorem{exmps}[thm]{Examples}
\newtheorem{notn}[thm]{Notation}
\newtheorem{notns}[thm]{Notations}
\newtheorem{addm}[thm]{Addendum}
\newtheorem{exer}[thm]{Exercise}
%
\theoremstyle{remark}
\newtheorem{rem}[thm]{Remark}
\newtheorem*{claim}{Claim}
\newtheorem*{aside*}{Aside}
\newtheorem*{rem*}{Remark}
\newtheorem*{hint*}{Hint}
\newtheorem*{note}{Note}
\newtheorem{rems}[thm]{Remarks}
\newtheorem{warn}[thm]{Warning}
\newtheorem{sch}[thm]{Scholium}
%
%--------Macros--------
\renewcommand{\qedsymbol}{$\blacksquare$}
\renewcommand{\sl}{\mathfrak{sl}}
\newcommand{\Bord}{\mathsf{Bord}}
\renewcommand{\hom}{\mathsf{Hom}}
\renewcommand{\emptyset}{\varnothing}
\renewcommand{\O}{\mathscr{O}}
\newcommand{\R}{\mathbb{R}}
\newcommand{\ib}[1]{\textbf{\textit{#1}}}
\newcommand{\Q}{\mathbb{Q}}
\newcommand{\Z}{\mathbb{Z}}
\newcommand{\N}{\mathbb{N}}
\newcommand{\C}{\mathbb{C}}
\newcommand{\A}{\mathbb{A}}
\newcommand{\F}{\mathbb{F}}
\newcommand{\M}{\mathcal{M}}
\newcommand{\dbar}{\overline{\partial}}
\newcommand{\zbar}{\overline{z}}
\renewcommand{\S}{\mathbb{S}}
\newcommand{\V}{\vec{v}}
\newcommand{\RP}{\mathbb{RP}}
\newcommand{\CP}{\mathbb{CP}}
\newcommand{\B}{\mathcal{B}}
\newcommand{\GL}{\mathsf{GL}}
\newcommand{\SL}{\mathsf{SL}}
\newcommand{\SP}{\mathsf{SP}}
\newcommand{\SO}{\mathsf{SO}}
\newcommand{\SU}{\mathsf{SU}}
\newcommand{\gl}{\mathfrak{gl}}
\newcommand{\g}{\mathfrak{g}}
\newcommand{\Bun}{\mathsf{Bun}}
\newcommand{\inv}{^{-1}}
\newcommand{\bra}[2]{ \left[ #1, #2 \right] }
\newcommand{\set}[1]{\left\lbrace #1 \right\rbrace}
\newcommand{\abs}[1]{\left\lvert#1\right\rvert}
\newcommand{\norm}[1]{\left\lVert#1\right\rVert}
\newcommand{\transv}{\mathrel{\text{\tpitchfork}}}
\newcommand{\defeq}{\vcentcolon=}
\newcommand{\enumbreak}{\ \\ \vspace{-\baselineskip}}
\let\oldexists\exists
\renewcommand\exists{\oldexists~}
\let\oldL\L
\renewcommand\L{\mathfrak{L}}
\makeatletter
\newcommand{\tpitchfork}{%
  \vbox{
    \baselineskip\z@skip
    \lineskip-.52ex
    \lineskiplimit\maxdimen
    \m@th
    \ialign{##\crcr\hidewidth\smash{$-$}\hidewidth\crcr$\pitchfork$\crcr}
  }%
}
\makeatother
\newcommand{\bd}{\partial}
\newcommand{\lang}{\begin{picture}(5,7)
\put(1.1,2.5){\rotatebox{45}{\line(1,0){6.0}}}
\put(1.1,2.5){\rotatebox{315}{\line(1,0){6.0}}}
\end{picture}}
\newcommand{\rang}{\begin{picture}(5,7)
\put(.1,2.5){\rotatebox{135}{\line(1,0){6.0}}}
\put(.1,2.5){\rotatebox{225}{\line(1,0){6.0}}}
\end{picture}}
\DeclareMathOperator{\id}{id}
\DeclareMathOperator{\im}{Im}
\DeclareMathOperator{\codim}{codim}
\DeclareMathOperator{\coker}{coker}
\DeclareMathOperator{\supp}{supp}
\DeclareMathOperator{\inter}{Int}
\DeclareMathOperator{\sign}{sign}
\DeclareMathOperator{\sgn}{sgn}
\DeclareMathOperator{\indx}{ind}
\DeclareMathOperator{\alt}{Alt}
\DeclareMathOperator{\Aut}{Aut}
\DeclareMathOperator{\trace}{trace}
\DeclareMathOperator{\ad}{ad}
\DeclareMathOperator{\End}{End}
\DeclareMathOperator{\Ad}{Ad}
\DeclareMathOperator{\Lie}{Lie}
\DeclareMathOperator{\spn}{span}
\DeclareMathOperator{\dv}{div}
\DeclareMathOperator{\grad}{grad}
\DeclareMathOperator{\Sym}{Sym}
\DeclareMathOperator{\sheafhom}{\mathscr{H}\text{\kern -3pt {\calligra\large om}}\,}
\newcommand*\myhrulefill{%
   \leavevmode\leaders\hrule depth-2pt height 2.4pt\hfill\kern0pt}
\newcommand\niceending[1]{%
  \begin{center}%
    \LARGE \myhrulefill \hspace{0.2cm} #1 \hspace{0.2cm} \myhrulefill%
  \end{center}}
\newcommand*\sectionend{\niceending{\decofourleft\decofourright}}
\newcommand*\subsectionend{\niceending{\decosix}}
\def\upint{\mathchoice%
    {\mkern13mu\overline{\vphantom{\intop}\mkern7mu}\mkern-20mu}%
    {\mkern7mu\overline{\vphantom{\intop}\mkern7mu}\mkern-14mu}%
    {\mkern7mu\overline{\vphantom{\intop}\mkern7mu}\mkern-14mu}%
    {\mkern7mu\overline{\vphantom{\intop}\mkern7mu}\mkern-14mu}%
  \int}
\def\lowint{\mkern3mu\underline{\vphantom{\intop}\mkern7mu}\mkern-10mu\int}
%
%--------Hypersetup--------
%
\hypersetup{
    colorlinks,
    citecolor=black,
    filecolor=black,
    linkcolor=blue,
    urlcolor=blacksquare
}
%
%--------Solution--------
%
\newenvironment{solution}
  {\begin{proof}[Solution]}
  {\end{proof}}
%
%--------Graphics--------
%
%\graphicspath{ {images/} }

\begin{document}
%
\author{Jeffrey Jiang}
%
\title{A Quick Survey of Hodge Theory}
%
\maketitle
%
\tableofcontents
%
\section{Linear Algebra}
%
\begin{defn}
Let $V$ be a finite dimensional $\R$-vector space. An \ib{almost complex structure} on
$V$ is a linear map $J : V \to V$ satisfying $J^2 = \id_V$. An \ib{almost complex
vector space} is tuple $(V,J)$, where $V$ is a finite dimensional $\R-$vector space
equipped with an almost complex structure $J$.
\end{defn}
%
The data of an almost complex structure $J$ is equivalent to giving $V$ the structure
of a complex vector space, where we define $(a + bi)\cdot v = ab + bJv$. Because of this,
we may call $J$ a \ib{complex structure}. We use the name \emph{almost} complex
structure to emphasize the differences between the analogous constructions in the
nonlinear world of manifolds. Note that since an almost complex structure is equivalent
to a complex structure, this immediately implies that $V$ is even dimensional. \\

One way to think about an almost complex structure is through the geometric
interpretation of complex multiplication. If we regard $\C$ as a $2$-dimensional vector
space over $\R$, multiplication by $i$ corresponds to a rotation by $\pi/2$ in the
counterclockwise direction, and multiplication by $-i$ corresponds to a rotation by
$\pi/2$ in the clockwise direction. From this we see that a choice of a square root
of $-1$ comes with a choice of clockwise or counterclockwise. This implies that
every complex vector space is canonically oriented as a real vector space. Given a
$\C$-vector space $V$ with $\C$-basis $\set{z_1, \ldots, z_n}$, we have that the ordered
basis $(z_1, iz_1, \ldots z_n, iz_n)$ defines a positively oriented basis for $V$ over
$\R$.
%
\begin{defn}
Let $V$ by any $\R$-vector space. The \ib{complexification} of $V$ is the complex
vector space $V_\C = \C \otimes_\R V$. The complexification $V_\C$ is naturally a
complex vector space, where for $\lambda \in \C$, the action of $\lambda$ on a
homogeneous element $\mu \otimes v$ is given by
\[
\lambda \cdot (\mu \otimes v) = \lambda\mu \otimes v
\]
\end{defn}
%
The complexification is a different way of obtaining a complex vector space from a real
one. The complexification $V_\C$ is a vector space of twice the dimension of $V$ as a
vector space over $\R$. In addition, there is a natural inclusion
$V \hookrightarrow V_\C$ given by $v \mapsto 1 \otimes v$. Complexification is an
instance of \emph{extension of scalars} -- every element of $V_\C$ is of the form
$av + biw$ where $a,b \in \R$, and $v,w \in V$. Therefore, we will denote the element
$(a + bi) \otimes v \in V_\C$ by $av + biv$, and we have a direct sum decomposition
$V_\C = V \oplus iV$.   Given a linear map $T : V \to W$ of $\R$-vector spaces, we can
extend $T$ to a complexified map $T_\C : V_\C \to W_\C$, where
$T_\C(av + biw) = aTv + biTv$. In other words, $T_\C = \id_\C \otimes T$. In this way,
we see that complexification defines a covariant functor
$\mathsf{Vect}_\R \to \mathsf{Vect}_\C$ from the category of $\R$-vector spaces to the
category of $\C$-vector spaces. \\

A natural question to ask is how an almost complex structure interacts with
the process of complexification. Let $(V,J)$ be an almost complex vector space. Then the
complexified map $J_\C : V_\C \to V_\C$ squares to $-1$, and admits eigenvalues
$\pm i$. For example, consider $V = \C$. We then can then make the natural
identification of $\C$ with the almost complex vector space $(\R^2, J)$, where $J$
is given by the matrix
\[
J = \begin{pmatrix}
0 & -1 \\
1 & 0
\end{pmatrix}
\]
using the ordered $\R$ basis $(1,i)$. Then $V_\C = \C \otimes_\R \C = \C^2$, and $J_\C$
is given by the matrix
\[
J_\C = \begin{pmatrix}
0 & -1 & 0 & 0 \\
1 & 0 & 0 & 0 \\
0 & 0 & 0 & -1 \\
0 & 0 & 1 & 0
\end{pmatrix}
\]
in the ordered $\R$ basis $(1, i, 1, i)$, where the first pair elements are elements of
one copy of $\C^2$, and the second pair of elements are elements of a separate copy of
$\C^2$, giving an $\R$ basis for $\C \cong \C \oplus \C$. The matrix $J_\C$
clearly has eigenvalues $\pm i$, with the $i$-eigenspace being spanned by the
vectors
\[
\begin{pmatrix}
\frac{1}{2} \\[5pt]
0 \\[5pt]
\frac{1}{2} \\[5pt]
0
\end{pmatrix}, \begin{pmatrix}
\frac{i}{2} \\[5pt]
0 \\[5pt]
-\frac{i}{2} \\[5pt]
0
\end{pmatrix}
\]
and the $-i$ eigenspace being the span of
\[
\begin{pmatrix}
0 \\[5pt]
\frac{1}{2} \\[5pt]
0 \\[5pt]
\frac{1}{2}
\end{pmatrix}, \begin{pmatrix}
0 \\[5pt]
\frac{i}{2} \\[5pt]
0 \\[5pt]
-\frac{i}{2}
\end{pmatrix}
\]
The case for general $V$ is similar, and amounts to fixing an $\R$ basis of the
form $\set{v_1, Jv_1, \ldots v_n, Jv_n}$ for $(V,J)$. The decomposition of $V_\C$ into
the $\pm i$-eigenspaces of $J_\C$ gives a direct sum decomposition
\[
V_\C = V^{1,0} \oplus V^{0,1}
\]
where $V^{1,0}$ denotes the $i$-eigenspace, and $V^{0,1}$ denotes the $-i$-eigenspace.
Note that in the bases we chose above for $\C^2$, complex conjugation is given by the
matrix
\[
\begin{pmatrix}
0 & 1 & 0 & 0\\
1 & 0 & 0 & 0 \\
0 & 0 & 0 & 1 \\
0 & 0 & 1 & 0
\end{pmatrix}
\]
and determines an isomorphism of complex vector spaces
$V^{1,0} \cong \overline{V^{0,1}}$ and vice versa. In addition, complex conjugation
picks out a distinguished subspace of $V_\C$ isomorphic to $V$ -- the $1$-eigenspace.
Elements of this eigenspace are called \ib{real} (much like how the $\R^n$
is left invariant under complex conjugation in $\C^n$). \\

An almost complex structure $J$ on $V$ induces a dual map $J^* : V^* \to V^*$. By
functoriality of taking dual spaces, $(J^*) ^2 = -\id_{V^*}$, giving $V^*$ the
structure of an almost complex vector space via $J^*$. Explicitly, given a linear
functional $\alpha \in V^*$, we have that the action of $J^*$ on $\alpha$ is given
by
\[
(J^*\alpha)(v) = \alpha(Jv)
\]
for every $v \in V$. We then get an analogous decomposition of the complexified
dual space$ V^*_\C = \hom_\R(V,\C)$ as
\[
V^*_\C = (V^*)^{1,0} \oplus (V^*)^{0,1}
\]
into the $\pm i$ eigenspaces of $J^*$, and have very natural interpretations in
terms of $V$. The subspace $(V^*)^{1,0}$ consists of the $\alpha$ such that
$\alpha(Jv) = i\alpha(v)$. We have a natural pairing $(V^*)^{1,0} \otimes V^{1,0} \to \C$
given by
\[
\langle \alpha,v \rangle = \alpha(v)
\]
which is nondegenerate, establishing the two vector spaces as dual to each other (as
complex vector spaces). A similar statement holds for $V^{0,1}$ and $(V^*)^{0,1}$,
putting them in duality as well. Another perspective to take is that the condition
that $\alpha(Jv) = i\alpha(v)$ is equivalent to $\alpha$ being \emph{complex linear}
with respect to the complex structure $J$ on $V$ and $i$ on $\C$, giving us an
isomorphism $(V^*)^{1,0} \cong \hom_\C(V,\C)$. Similarly, we have that
$(V^*)^{0,1} \cong \hom_{\C}(V,\overline{\C})$.\\

We the consider the effects of complexification on another linear algebraic construction
--the exterior algebra. Given a finite dimensional $\R$-vector space $V$, we have that
the exterior algebra $\Lambda^\bullet V$ has a natural $\Z$-grading
\[
\Lambda^\bullet V = \bigoplus_{k = 0}^n \Lambda^kV
\]
Upon complexification, we have a canonical isomorphism
$(\Lambda^\bullet V)_\C \cong \Lambda^\bullet V_\C$, where the right hand side is the
exterior algebra of $V_\C$ as a \emph{complex} vector space
(i.e. $\Lambda^0 V_\C \cong \C$, not $\R$), and $\Lambda^\bullet V$ is canonically
embedded in $\Lambda^\bullet V_\C$ as the subspace invariant under complex conjugation.
Further assume that we have an almost complex structure $J : V \to V$. This gives a
direct sum decomposition $V_\C = V^{1,0} \oplus V^{0,1}$, which then induces a
decomposition
\[
\Lambda^kV_\C \cong \bigoplus_{p+q=k}\Lambda^pV^{1,0} \otimes_\C \Lambda^q V^{0,1}
\]
giving the complexified exterior algebra a bigrading via the subspaces
\[
\Lambda^{p,q}V \defeq \Lambda^p V^{1,0} \otimes_\C \Lambda^q V^{0,1}
\]
elements of $\Lambda^{p,q}$ are said to have \ib{bidegree} $(p,q)$. We now make several
simple observations regarding the bigrading of $\Lambda^\bullet V_\C$, which stem from
our observations regarding the decomposition $V_\C = V^{1,0} \oplus V^{0,1}$.
%
\begin{enumerate}
  \item From the fact that $\overline{V^{1,0}} = V^{0,1}$, we get that
  $\overline{\Lambda^{p,q}V} = \Lambda^{q,p}V$.
  \item Given $\omega \in \Lambda^{p,q}V$ and $\eta \in \Lambda^{s,t}V$, we have that
  $\omega \wedge \eta \in \Lambda^{p+s,q+t}V$.
\end{enumerate}
%
\section{Complex Manifolds}
%
We now transfer our linear algebraic knowledge to complex manifolds. Let $U \subset \C$
be an open subset, and $z = x + iy$ the usual coordinate function on $U$. Considered
as a subset of $\R^2$, for any point $p \in U$, the tangent space $T_pU$ is spanned by
the coordinate vectors $\partial_x\vert_p$ and $\partial_y\vert_p$. If we then
complexify, the $\partial_x\vert_p$ and $\partial_y\vert_p$ also form a basis for
$(T_pU)_\C$ as a \emph{complex} vector space. Consequently, we can also form a
basis for $(T_pU)_\C$ with the vectors
%
\begin{align*}
\frac{\partial}{\partial z}\bigg\vert_p &\defeq
\frac{1}{2}\left(\frac{\partial}{\partial x} - i\frac{\partial}{\partial y}\right) \\
\frac{\partial}{\partial\overline{z}}\bigg\vert_p &\defeq
\frac{1}{2}\left( \frac{\partial}{\partial x} + i\frac{\partial}{\partial y}\right)
\end{align*}
%
which we abbreviate as $\partial_z\vert_p$ and $\partial{\overline{z}}\vert_p$
respectively. Doing this for all $p$, we obtain vector fields $\partial_z$ and
$\partial_{\overline{z}}$ on all of $U$, where a function $f \in C^\infty(U, \C)$ is
holomorphic if and only if $\partial_{\overline{z}}f = 0$. Functions $f$ annihilated by
$\partial_z$ are \ib{antiholomorphic}, i.e. $\overline{f}$ is holomorphic. These
vector fields give a framing for the complexified tangent bundle $TU_\C$, i.e. an
isomorphism $TU_\C \to U \times \C^2$. \\

We get a similar description of the complexified cotangent bundle $T^*U_\C$, which
is framed by the covector fields
%
\begin{align*}
dz = \frac{1}{2}\left(dx + idy\right) \\
d\overline{z} = \frac{1}{2}\left( dx -idy\right)
\end{align*}
%
which are easily verified to the duals of $\partial_z$ and $\partial_{\overline{z}}$
respectively. For an open subset $U \subset \C^n$ with complex coordinates
$z^i = x^i +iy^i$, we have analogous vector fields $\partial/\partial z^i$ and
$\partial/\partial\overline{z}^i$ and covector fields $dz^i$ and $d\overline{z}^i$.
%
\begin{defn}
An \ib{almost complex manifold} is the data of a smooth manifold $X$ and a smooth
section $J \in \Gamma_X(\End(TX))$ such that $J^2 = -\id_{TX}$.
\end{defn}
%
\begin{defn}
A \ib{complex manifold} is a smooth manifold $X$ such that there is an open cover
$\set{(U_\alpha, \varphi_\alpha)}$ of $X$ by charts $U_\alpha$ and diffeomorphisms
$\varphi: U_\alpha \to V_\alpha$ to open sets $V_\alpha \subset \R^{2n} \cong \C^n$
such that the transition maps $\varphi_\alpha \circ \varphi_\beta\inv$ are holomorphic.
\end{defn}
%
\begin{defn}
Let $X$ be a complex manifold. A complex valued function $f \in C^\infty(X,\C)$ is
\ib{holomorphic} if for any holomorphic chart $(\varphi, U)$, we have that
$f \circ \varphi\inv$ is holomorphic. The ring of holomorphic functions over $X$ is
denoted $\O_X$.
\end{defn}
%
Unlike the linear case, the notion of an almost complex manifold and a complex manifold
are \emph{not} equivalent. Let $(X,J)$ be an almost complex manifold. For a point
$x \in X$, we have that $J_x$ is an almost complex structure on the tangent space
$T_xX$, which gives a decomposition of the complexified tangent space
\[
T_xX_\C = T_x^{1,0}X \oplus T_x^{0,1}X
\]
into the $\pm i$-eigenspaces of complexified $J_x$. Doing this over all points $x$,
we get a decomposition of the complexified tangent bundle
\[
TX_\C = T^{1,0}X \oplus T^{0,1}X
\]
into the $\pm i$-eigenbundles of $J$.\\

Suppose $X$ is a complex manifold. Then every tangent space $T_xX$ inherits an almost
complex structure coming from multiplication by $i$ in $\C^n$, giving an almost complex
structure $J$ on $X$. To show this explicitly, let $z^i = x^i + iy^i$ be local
holomorphic coordinates for $X$. Then the coordinate functions
$(x^1, y^1, \ldots x^n, y^n)$ form a coordinate system for $X$ as a \emph{real} manifold,
and the action of $J$ in these coordinates is simply given by
%
\begin{align*}
\frac{\partial}{\partial x^i} &\mapsto \frac{\partial}{\partial y^i} \\
\frac{\partial}{\partial y^i} &\mapsto -\frac{\partial}{\partial x^i}
\end{align*}
%
and the subbundle $T^{1,0}X \subset TX_\C$ is spanned by the $\partial/\partial z^i$,
and $T^{0,1}X$ is spanned by the $\partial/\partial\overline{z}^i$. The composition
\[\begin{tikzcd}
TX \ar[r,hookrightarrow] & TX_\C \ar[r] & T^{1,0}X
\end{tikzcd}\]
consisting of the inclusion followed by projecting then defines an isomorphism of
the real tangent bundle with $T^{1,0}X$, which is called the
\ib{holomorphic tangent bundle}. Explicitly, this map is given in local coordinates by
%
\begin{align*}
\frac{\partial}{\partial x^i} &=
\frac{\partial}{\partial z^i} + \frac{\partial}{\partial\overline{z}^i}
\mapsto \frac{\partial}{\partial z^i} \\
\frac{\partial}{\partial y^i} &=
i\frac{\partial}{\partial z^i} - i\frac{\partial}{\partial\overline{z}^i}
\mapsto i\frac{\partial}{\partial z^i}
\end{align*}
%
and the $\partial/\partial z^i$ along with the $i\partial/\partial z^i$ form a local
frame for the holomorphic tangent bundle $T^{1,0}X$.\\

Given an almost complex manifold $(X,J)$, a natural question to ask is whether or not
$X$ admits the structure of a complex manifold where $J$ is multiplication by $i$.
The answer to this question is a celebrated theorem of Newlander and Nirenberg
%
\begin{thm}[\ib{Newlander-Nirenberg}]
An almost complex manifold $(X,J)$ is a complex manifold (where $J$ is multiplication by
$i$) if and only if the subbundle $T^{1,0}X \subset TX_\C$ is integrable in the sense
of Frobenius, i.e.
\[
[T^{1,0}X, T^{1,0}X] \subset T^{1,0}X
\]
via complex conjugation, this is equivalent to $T^{0,1}X$ being integrable.
\end{thm}
%
Intuitively, the existence of local \emph{holomorphic} coordinates amounts to
finding an integral submanifold for $T^{1,0}X$, since if $X$ was complex, $T^{1,0}X$
would arise as the holomorphic tangent bundle of $X$.
%
\section{Cohomology of Complex Manifolds.}
%
Let $X$ be a complex manifold, and let $J$ denote the canonical almost complex structure
on $X$. Applying our linear algebraic knowledge pointwise, we get a decomposition of
the complexified cotangent bundle
\[
T^*X_\C = (T^*X)^{1,0} \oplus (T^*X)^{0,1}
\]
%
which then gives a decomposition of the space $\Omega^k_X(\C)$ of complex valued
differential $k$-forms as
\[
\Omega^k_X(\C) = \bigoplus_{p+q=k}\Omega^{p,q}_X(\C)
\]
where $\Omega^{p,q}_X(\C)$ denotes the space of $(p,q)$-forms, i.e. sections of
$\Lambda^{p,q}T^*X$, and the direct sum is taken as modules over $C^\infty(X,\C)$.
Let $d$ denote the de Rham differential $\Omega^k_X \to \Omega^{k+1}_X$ after
complexifying it to obtain a map $\Omega^k_X(\C) \to \Omega^{k+1}_X(\C)$. In local
holomorphic coordinates $\set{z^i}$, we have that the $dz^i$ form a basis for the space
of smooth $(1,0)$-forms over the ring $C^\infty(M,\C)$, and the $d\overline{z}^i$ form a
basis for the space of smooth $(0,1)$-forms. Therefore, given $f \in C^\infty(X,\C)$, we
can uniquely write $df \in \Omega^1_X(\C)$ as
\[
df = \frac{\partial f}{\partial z^i}dz^i +
\frac{\partial f}{\partial\overline{z}^i}d\overline{z}^i
\]
where we use Einstein summation notation. Given an arbitrary $(p,q)$ form $\alpha$, it
has the local coordinate expression
\[
\alpha = \sum_{p,q} \alpha_{p,q} dz^{i_1} \wedge \ldots \wedge dz^{i_p} \wedge
d\overline{z}^{i_1}\ldots \wedge d\overline{z}^{i_q}
\]
for smooth $\alpha_{p,q} \in C^\infty(X,\C)$. Then we have that
%
\begin{align*}
d\alpha &= \left(\sum_{p,q}\frac{\partial\alpha_{p,q}}{\partial z^i} dz^i
\wedge dz^{i_1} \wedge \ldots \wedge dz^{i_p} \wedge
d\overline{z}^{i_1}\ldots \wedge d\overline{z}^{i_q}\right) \\[5pt]
&+ \left(
\sum_{p,q}\frac{\partial\alpha_{p,q}}{\partial \overline{z}^i} d\overline{z}^i
\wedge dz^{i_1} \wedge \ldots \wedge dz^{i_p} \wedge
d\overline{z}^{i_1}\ldots \wedge d\overline{z}^{i_q}
\right)
\end{align*}
%
which shows that $d\alpha$ is a $(p+1,q+1)$ form, i.e. $d$ is of bidegree $(1,1)$. This
expression for $d\alpha$ allows us to decompose $d$ as a sum $d = \partial + \dbar$,
where
%
\begin{align*}
\partial\alpha &= \sum_{p,q}\frac{\partial\alpha_{p,q}}{\partial z^i} dz^i
\wedge dz^{i_1} \wedge \ldots \wedge dz^{i_p} \wedge
d\overline{z}^{i_1}\ldots \wedge d\overline{z}^{i_q} \\[5pt]
\dbar\alpha &= \sum_{p,q}\frac{\partial\alpha_{p,q}}{\partial \overline{z}^i}
d\overline{z}^i \wedge dz^{i_1} \wedge \ldots \wedge dz^{i_p} \wedge
d\overline{z}^{i_1}\ldots \wedge d\overline{z}^{i_q}
\end{align*}
%
from the definition, it is clear that $\partial$ is of bidegree $(1,0)$ and $\dbar$ is
of bidegree $(0,1)$.
%
\begin{prop}\enumbreak
\begin{enumerate}
  \item $\partial^2 = 0$
  \item $\dbar^2 = 0$
  \item $\partial\dbar + \dbar\partial = 0$
\end{enumerate}
\end{prop}
%
\begin{proof}
We know that $d = \partial + \dbar$ satisfies $d^2 = 0$. Therefore, we have that
\[
d^2 = (\partial +\dbar)^2 = \partial^2 + \partial\dbar + \dbar\partial + \dbar^2
\]
we then note that the bidegrees of all the components. We have:
\begin{enumerate}
  \item $\partial^2$ has bidegree $(2,0)$.
  \item $\partial\dbar + \dbar\partial$ has bidegree $(1,1)$
  \item $\dbar^2$ has bidegree $(0,2)$
\end{enumerate}
in order for $d^2$ to vanish, all the terms of different bidegrees must vanish, proving
all three parts of the proposition.
\end{proof}
%
The operators $\partial$ and $\dbar$ satisfy many of the same identities as $d$. Another
helpful identity to note is that
\[
\partial\alpha = \overline{\dbar\overline{\alpha}}
\]
\begin{defn}
Let $X$ be a complex manifold. A \ib{holomorphic vector bundle} over $X$ is a
complex vector bundle $\pi : E \to X$ such that the total space $E$ is a complex manifold
and the projection $\pi$ is a holomorphic map.
\end{defn}
%
Note again that a holomorphic vector bundle is \emph{not} a complex vector bundle
(for example, a complex vector bundle can be odd dimensional if the base space is
odd dimensional). Instead, it is a complex bundle with extra structure. Let
$\sigma : X \to E$ be a \emph{smooth} section. In a local holomorphic trivialization of
$E$, we can write $\sigma$ as
\[
\sigma = (\sigma^1, \ldots, \sigma^k)
\]
for smooth complex-valued functions $\sigma^i$,  where $k$ is the rank of $E$ as a
complex vector bundle. In this trivialization, we can apply the operator $\dbar$
component-wise to get
\[
\dbar\sigma = (\dbar\sigma^1,\ldots,\dbar\sigma^k)
\]
and this local definition glues together to give a well defined operator
$\dbar_E : \Omega^{0,0}_X(E) \to \Omega^{0,1}_X(E)$, where $\Omega^{p,q}_X(E)$ denotes
the space of sections $\Gamma_X(T^{p,q}X \otimes_\C E)$. We then note that the
holomorphic sections $X \to E$ are exactly those annihilated by $\dbar_E$. Since
$\dbar_E$ is defined locally in terms of $\dbar$, we immediately see that
$\dbar_E^2 = 0$, giving us the
\ib{Dolbeault complex} of $E$.
\[\begin{tikzcd}
\Omega^{0,0}_X(E) \ar[r, "\dbar_E"] & \Omega^{0,1}_X(E) \ar[r, "\dbar_E"] &
\cdots\cdots \ar[r, "\dbar_E"] & \Omega^{0,n}_X(E) \ar[r,"\dbar_E"] & 0
\end{tikzcd}\]
%
The \ib{Dolbeault cohomology groups} of $E$ are the cohomology groups of the
Dolbeault complex, and are denoted $H^p(X,E)$.
%
\section{K\"ahler Manifolds}
%
K\"ahler manifolds will form a special class of manifolds where Hodge theory
will prove especially useful. Before diving into the world of K\"ahler manifolds,
we do some prerequisite linear algebra, which is closely tied to Hermitian geometry.
%
\begin{defn}
Let $V$ be a complex vector space. A \ib{Hermitian form} on $V$ is a bilinear map
$h : \overline{V} \otimes_\C V \to \C$ satisfying $h(v,w) = \overline{h(w,v)}$.
\end{defn}
%
Another way to define a Hermitian form is as a \ib{sesquilinear} form
$h : V \otimes_\C V \to \C$, i.e. conjugate linear in the first term, and linear in the
second term, along with the Hermitian symmetry condition $h(v,w) = \overline{h(v,w)}$.
In a basis for $V$, a Hermitian form is given by a Hermitian matrix $H$, i.e.
$H^\dagger = H$. \\

Given a Hermitian form $h$ on a complex vector space $V$, we can decompose $h$ into
its real and imaginary parts. Let $g = \mathfrak{Re}(h)$ and
$\omega = \mathfrak{Im}(h)$, so we have $h = g + i\omega$.
%
\begin{prop}
Let $h = g + i\omega$ be a hermitian form. Then we have
\begin{enumerate}
  \item $g$ is symmetric
  \item $\omega$ is skew-symmetric
\end{enumerate}
\end{prop}
%
\begin{proof}\enumbreak
\begin{enumerate}
  \item Since $h$ is Hermitian, we have that $h(v,w) = \overline{h(w,v)}$, which have
  the same real part. Therefore, $g(v,w) = g(w,v)$, so $g$ is symmetric
  \item We again use the fact that $h(v,w) = \overline{h(w,v)}$. Since conjugation
  reverses the sign on the imaginary component, we immediately get that
  $\omega(v,w) = -\omega(w,v)$.
\end{enumerate}
\end{proof}
%
\begin{rem*}
The decomposition of $h$ into its real and imaginary components is one way of seeing
the useful identity
\[
U_n = O_{2n} \cap Sp_{2n} \cap GL_n\C
\]
where we regard all of the groups as subgroups of $GL_{2n}\R$. Any matrix preserving
the standard Hermitian form on $\C^n$ must necessarily preserve both the real and
imaginary components, so it must preserve the standard Euclidean inner product on
$\R^{2n} \cong \C^n$ and the standard symplectic form on $\R^{2n} \cong \C$.
\end{rem*}
%
Given a complex vector space $V$ equipped with a hermitian metric $h$, let $J$
denote the natural almost complex structure induced by multiplication by $i$.
Writing $h = g + i\omega$, we see that $\omega$ and $g$ are intimately related, i.e.
\begin{enumerate}
  \item $g(v,Iw) = \omega(v,w)$
  \item $\omega(v,Iw) = g(v,w)$
\end{enumerate}
%
both of which come from the fact that $h$ is compatible with $J$ in the sense that
$h(Jv,Jw) = h(v,w)$. Finally, we note that $\omega$ can be naturally identified
with a real element of $(V^*)^{1,1}$.
%
\begin{defn}
Let $h$ be a Hermitian form, and $\omega = \mathfrak{Im}(h)$. Then $\omega$ is called
the \ib{K\"ahler form} of $h$.
\end{defn}
%
We now move back to the world of manifolds. Let $(X,J)$ be an almost complex manifold.
The almost complex structure gives each tangent space $T_xX$ the structure of a complex
vector space, so it makes sense to discuss a Hermitian form $h_x$ on $T_xX$.
%
\begin{defn}
A \ib{K\"ahler manifold} is an almost complex manifold $(X,J)$ with a smoothly varying
Hermitian metric (i.e. a positive definite Hermitian form) $h$ where $J$ is integrable
and the K\"ahler form $\omega$ obtained from $h$ is closed.
\end{defn}
%
From the decomposition $h = g + i\omega$, we see that a K\"ahler manifold is a
melting pot of several compatible structures, namely the Riemannian metric $g$, the
almost complex structure $J$, and the symplectic form $\omega$. \\

From the basic definition, we can already make some surprisingly deep statements about
K\"ahler manifolds.
%
\begin{prop}
Let $X$ be a K\"ahler manifold with Hermitian form $h = g + i\omega$. Then the volume
form of $X$ with respect to the Riemannian metric $g$ is $\omega^n/n!$, where
$n = \dim_\C X$.
\end{prop}
%
\begin{proof}
It suffices to verify this pointwise for a fixed point $x \in X$.
The Riemannian volume form $dV_g$ is uniquely characterized by the fact that
$(dV_g)_x(e_1, \ldots, e_{2n}) = 1$ for an oriented orthonormal basis
$\set{e_1, \ldots e_{2n}}$ for $T_xX$. Arrange local holomorphic coordinates
$\set{z^i}$ about $x$ such that at $x$, the Hermitian form $h_x$ is given by the
identity matrix, i.e.
\[
h_x\left(\frac{\partial}{\partial z^i}\bigg\vert_x,
\frac{\partial}{\partial z^j}\bigg\vert_x\right) = \delta_{ij}
\]
Let $\partial_i$ denote $\partial/\partial z^i\vert_x$. We then have that
$\set{\partial_1, i\partial_1, \ldots \partial_n, i\partial_n}$ is an oriented
orthonormal $\R$-basis for $T_xX$ with respect to $g$, so it suffices to evaluate
$\omega^n_x$ on this basis, and verify that it evaluates to $n!$. To do this, we
use a lemma.
%
\begin{lem}
In local holomorphic coordinates $\set{z^i}$, the K\"ahler form $\omega$ is given by
\[
\omega = \frac{i}{2}\sum_i H^i_j dz^i \wedge d\zbar^i
\]
where $H^i_j = h(\partial_i, \partial_j)$.
\end{lem}
%
The proof is an easy consequence of $\omega$ being of type $(1,1)$ and the
fact that it is the imaginary component of $h$. \\

In the coordinates we specified above, we have that $H_x$ is the identity matrix,
so the lemma gives us
\[
\omega_x = \frac{i}{2}\sum_i dz^i \wedge d\zbar^i
\]
we note that when we wedge $\omega_x$ with itself $n$ times, we continually pick
up duplications of $dz^i$ and $d\zbar^i$, so the only one term survives, leaving us with
\[
\omega_x^n = \frac{i}{2}\left(dz^1 \wedge d\zbar^i \wedge \cdots \wedge
dz^i \wedge d\zbar^i\right) = dx^1 \wedge dy^1 \wedge \cdots \wedge dx^n \wedge dy^n
= \frac{dV_g}{n!}
\]
where $z^i = x^i + iy^i$. We use the convention that
$v \wedge w = \mathrm{Alt}(v\otimes w)$ without the normalizing factor, which explains
the extra term of $n!$.
%
\end{proof}
%
\begin{prop}
Let $X$ be a K\"ahler manifold. Then all the even degree de Rham cohomology groups
$H^{2n}_{dR}(X,\R)$ are nontrivial.
\end{prop}
%
\begin{proof}
The fact that $\omega^n = dV_g/n!$ implies that
$\int_M \omega^n = \int_M dV_g = \mathrm{Vol}(M) > 0$, so by Stokes' theorem,
$\omega^n$ cannot be exact. From this, we can also conclude that $\omega^k$ for any
$k$ cannot be exact, for if $\omega^k = d\alpha$, then we would have that
\[
\omega^n = \omega^{n-k} \wedge d\alpha = d(\omega^{n-k} \wedge \alpha)
\]
Therefore, since $\omega$ is a degree $2$ form, we must have that all the even
dimensional cohomology groups are nontrivial
\end{proof}
%
This shows that the K\"ahler condition is quite rigid.
%
%TODO Maybe some comments on the Chern connection + interaction with Levi Civita?
%
\section{Laplacians and Harmonic Forms}
%
We start in the most general case, that of a Riemannian manifold $(X,g)$.
%
\begin{defn}
Let $(X,g)$ be a Riemannian manifold. Then the vector bundles $\Lambda^kT^*M$
inherit natural fiber metrics $(\cdot,\cdot)$, which are defined by the set
\[
\set{e_{i_1} \wedge \cdots \wedge e_{i_k} ~:~ 1 \leq i_1 < i_2 < \cdots < i_k}
\]
being an orthonormal basis for $\Lambda^kT^*_xM$ where $\set{e_i}$ is an orthonormal
basis for $T^*_xM$. This then introduces an \ib{$L^2$ inner product}
$(\cdot,\cdot)_{L^2}$ on the space $\Omega^k_M$ of differential $k$-forms, where we
define
\[
(\alpha,\beta)_{L^2} = \int_M (\alpha,\beta)~dV_g
\]
where $(\alpha,\beta)$ denotes the smooth function $x \mapsto (\alpha_x, \beta_x)$.
\end{defn}
%
\begin{defn}
The \ib{Hodge star} operator, denoted $\star$ (though many choose to denote it $*$), is
a bundle homomorphism $\star : \Lambda^kT^*M \to \Lambda^{n-k}T^*M$ defined by the
property such that for any $\alpha,\beta \in \Lambda^kT^*M$
\[
\alpha \wedge \star\beta = (\alpha,\beta)~dV_g
\]
this property uniquely characterizes $\star$.
\end{defn}
%
\begin{prop}
$\star$ exists.
\end{prop}
%
\begin{proof}
The defining property uniquely characterizes $\star$, so it suffices to specify a
local definition, which will then glue to a smooth bundle homomorphism. Let
$\set{x^i}$ be a local coordinate system for $X$.
\end{proof}
%
\end{document}