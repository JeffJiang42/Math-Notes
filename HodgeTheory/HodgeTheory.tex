\documentclass[psamsfonts, 12pt]{amsart}
%
%-------Packages---------
%
\usepackage[h margin=1 in, v margin=1 in]{geometry}
\usepackage{amssymb,amsfonts}
\usepackage[all,arc]{xy}
\usepackage{tikz-cd}
\usepackage{enumerate}
\usepackage{mathrsfs}
\usepackage{amsthm}
\usepackage{mathpazo}
%\usepackage{charter} %another font
%\usepackage{eulervm} %Vakil font
\usepackage{yfonts}
\usepackage{enumitem}
\usepackage{mathrsfs}
\usepackage{fourier-orns}
\usepackage[all]{xy}
\usepackage{hyperref}
\usepackage{cite}
\usepackage{url}
\usepackage{mathtools}
\usepackage{graphicx}
\usepackage{pdfsync}
\usepackage{mathdots}
\usepackage{calligra}
\usepackage{import}
\usepackage{xifthen}
\usepackage{pdfpages}
\usepackage{transparent}

\newcommand{\incfig}[2]{%
    \fontsize{48pt}{50pt}\selectfont
    \def\svgwidth{\columnwidth}
    \scalebox{#2}{\input{#1.pdf_tex}}
}
%
\usepackage{tgpagella}
\usepackage[T1]{fontenc}
%
\usepackage{listings}
\usepackage{color}

\definecolor{dkgreen}{rgb}{0,0.6,0}
\definecolor{gray}{rgb}{0.5,0.5,0.5}
\definecolor{mauve}{rgb}{0.58,0,0.82}

\lstset{frame=tb,
  language=Matlab,
  aboveskip=3mm,
  belowskip=3mm,
  showstringspaces=false,
  columns=flexible,
  basicstyle={\small\ttfamily},
  numbers=none,
  numberstyle=\tiny\color{gray},
  keywordstyle=\color{blue},
  commentstyle=\color{dkgreen},
  stringstyle=\color{mauve},
  breaklines=true,
  breakatwhitespace=true,
  tabsize=3
  }
%
%--------Theorem Environments--------
%
\newtheorem{thm}{Theorem}[section]
\newtheorem*{thm*}{Theorem}
\newtheorem{cor}[thm]{Corollary}
\newtheorem{prop}[thm]{Proposition}
\newtheorem{lem}[thm]{Lemma}
\newtheorem*{lem*}{Lemma}
\newtheorem{conj}[thm]{Conjecture}
\newtheorem{quest}[thm]{Question}
%
\theoremstyle{definition}
\newtheorem{defn}[thm]{Definition}
\newtheorem*{defn*}{Definition}
\newtheorem{defns}[thm]{Definitions}
\newtheorem{con}[thm]{Construction}
\newtheorem{exmp}[thm]{Example}
\newtheorem{exmps}[thm]{Examples}
\newtheorem{notn}[thm]{Notation}
\newtheorem{notns}[thm]{Notations}
\newtheorem{addm}[thm]{Addendum}
\newtheorem{exer}[thm]{Exercise}
%
\theoremstyle{remark}
\newtheorem{rem}[thm]{Remark}
\newtheorem*{claim}{Claim}
\newtheorem*{aside*}{Aside}
\newtheorem*{rem*}{Remark}
\newtheorem*{hint*}{Hint}
\newtheorem*{note}{Note}
\newtheorem{rems}[thm]{Remarks}
\newtheorem{warn}[thm]{Warning}
\newtheorem{sch}[thm]{Scholium}
%
%--------Macros--------
\renewcommand{\qedsymbol}{$\blacksquare$}
\renewcommand{\sl}{\mathfrak{sl}}
\newcommand{\Bord}{\mathsf{Bord}}
\renewcommand{\hom}{\mathsf{Hom}}
\renewcommand{\emptyset}{\varnothing}
\renewcommand{\O}{\mathscr{O}}
\newcommand{\R}{\mathbb{R}}
\newcommand{\ib}[1]{\textbf{\textit{#1}}}
\newcommand{\Q}{\mathbb{Q}}
\newcommand{\Z}{\mathbb{Z}}
\newcommand{\N}{\mathbb{N}}
\newcommand{\C}{\mathbb{C}}
\newcommand{\A}{\mathbb{A}}
\newcommand{\F}{\mathbb{F}}
\newcommand{\M}{\mathcal{M}}
\renewcommand{\S}{\mathbb{S}}
\newcommand{\V}{\vec{v}}
\newcommand{\RP}{\mathbb{RP}}
\newcommand{\CP}{\mathbb{CP}}
\newcommand{\B}{\mathcal{B}}
\newcommand{\GL}{\mathsf{GL}}
\newcommand{\SL}{\mathsf{SL}}
\newcommand{\SP}{\mathsf{SP}}
\newcommand{\SO}{\mathsf{SO}}
\newcommand{\SU}{\mathsf{SU}}
\newcommand{\gl}{\mathfrak{gl}}
\newcommand{\g}{\mathfrak{g}}
\newcommand{\Bun}{\mathsf{Bun}}
\newcommand{\inv}{^{-1}}
\newcommand{\bra}[2]{ \left[ #1, #2 \right] }
\newcommand{\set}[1]{\left\lbrace #1 \right\rbrace}
\newcommand{\abs}[1]{\left\lvert#1\right\rvert}
\newcommand{\norm}[1]{\left\lVert#1\right\rVert}
\newcommand{\transv}{\mathrel{\text{\tpitchfork}}}
\newcommand{\enumbreak}{\ \\ \vspace{-\baselineskip}}
\let\oldexists\exists
\renewcommand\exists{\oldexists~}
\let\oldL\L
\renewcommand\L{\mathfrak{L}}
\makeatletter
\newcommand{\tpitchfork}{%
  \vbox{
    \baselineskip\z@skip
    \lineskip-.52ex
    \lineskiplimit\maxdimen
    \m@th
    \ialign{##\crcr\hidewidth\smash{$-$}\hidewidth\crcr$\pitchfork$\crcr}
  }%
}
\makeatother
\newcommand{\bd}{\partial}
\newcommand{\lang}{\begin{picture}(5,7)
\put(1.1,2.5){\rotatebox{45}{\line(1,0){6.0}}}
\put(1.1,2.5){\rotatebox{315}{\line(1,0){6.0}}}
\end{picture}}
\newcommand{\rang}{\begin{picture}(5,7)
\put(.1,2.5){\rotatebox{135}{\line(1,0){6.0}}}
\put(.1,2.5){\rotatebox{225}{\line(1,0){6.0}}}
\end{picture}}
\DeclareMathOperator{\id}{id}
\DeclareMathOperator{\im}{Im}
\DeclareMathOperator{\codim}{codim}
\DeclareMathOperator{\coker}{coker}
\DeclareMathOperator{\supp}{supp}
\DeclareMathOperator{\inter}{Int}
\DeclareMathOperator{\sign}{sign}
\DeclareMathOperator{\sgn}{sgn}
\DeclareMathOperator{\indx}{ind}
\DeclareMathOperator{\alt}{Alt}
\DeclareMathOperator{\Aut}{Aut}
\DeclareMathOperator{\trace}{trace}
\DeclareMathOperator{\ad}{ad}
\DeclareMathOperator{\End}{End}
\DeclareMathOperator{\Ad}{Ad}
\DeclareMathOperator{\Lie}{Lie}
\DeclareMathOperator{\spn}{span}
\DeclareMathOperator{\dv}{div}
\DeclareMathOperator{\grad}{grad}
\DeclareMathOperator{\Sym}{Sym}
\DeclareMathOperator{\sheafhom}{\mathscr{H}\text{\kern -3pt {\calligra\large om}}\,}
\newcommand*\myhrulefill{%
   \leavevmode\leaders\hrule depth-2pt height 2.4pt\hfill\kern0pt}
\newcommand\niceending[1]{%
  \begin{center}%
    \LARGE \myhrulefill \hspace{0.2cm} #1 \hspace{0.2cm} \myhrulefill%
  \end{center}}
\newcommand*\sectionend{\niceending{\decofourleft\decofourright}}
\newcommand*\subsectionend{\niceending{\decosix}}
\def\upint{\mathchoice%
    {\mkern13mu\overline{\vphantom{\intop}\mkern7mu}\mkern-20mu}%
    {\mkern7mu\overline{\vphantom{\intop}\mkern7mu}\mkern-14mu}%
    {\mkern7mu\overline{\vphantom{\intop}\mkern7mu}\mkern-14mu}%
    {\mkern7mu\overline{\vphantom{\intop}\mkern7mu}\mkern-14mu}%
  \int}
\def\lowint{\mkern3mu\underline{\vphantom{\intop}\mkern7mu}\mkern-10mu\int}
%
%--------Hypersetup--------
%
\hypersetup{
    colorlinks,
    citecolor=black,
    filecolor=black,
    linkcolor=blue,
    urlcolor=blacksquare
}
%
%--------Solution--------
%
\newenvironment{solution}
  {\begin{proof}[Solution]}
  {\end{proof}}
%
%--------Graphics--------
%
%\graphicspath{ {images/} }

\begin{document}
%
\author{Jeffrey Jiang}
%
\title{A Quick Survey of Hodge Theory}
%
\maketitle
%
\tableofcontents
%
\section{Linear Algebra}
%
\begin{defn}
Let $V$ be a finite dimensional $\R$-vector space. An \ib{almost complex structure} on
$V$ is a linear map $J : V \to V$ satisfying $J^2 = \id_V$. An \ib{almost complex
vector space} is tuple $(V,J)$, where $V$ is a finite dimensional $\R-$vector space
equipped with an almost complex structure $J$.
\end{defn}
%
The data of an almost complex structure $J$ is equivalent to giving $V$ the structure
of a complex vector space, where we define $(a + bi)\cdot v = ab + bJv$. Because of this,
we may call $J$ a \ib{complex structure}. We use the name \emph{almost} complex
structure to emphasize the differences between the analogous constructions in the
nonlinear world of manifolds. Note that since an almost complex structure is equivalent
to a complex structure, this immediately implies that $V$ is even dimensional. \\

One way to think about an almost complex structure is through the geometric
interpretation of complex multiplication. If we regard $\C$ as a $2$-dimensional vector
space over $\R$, multiplication by $i$ corresponds to a rotation by $\pi/2$ in the
counterclockwise direction, and multiplication by $-i$ corresponds to a rotation by
$\pi/2$ in the clockwise direction. From this we see that a choice of a square root
of $-1$ comes with a choice of clockwise or counterclockwise. This implies that
every complex vector space is canonically oriented as a real vector space. Given a
$\C$-vector space $V$ with $\C$-basis $\set{z_1, \ldots, z_n}$, we have that the ordered
basis $(z_1, iz_1, \ldots z_n, iz_n)$ defines a positively oriented basis for $V$ over
$\R$.
%
\begin{defn}
Let $V$ by any $\R$-vector space. The \ib{complexification} of $V$ is the complex
vector space $V_\C = \C \otimes_\R V$. The complexification $V_\C$ is naturally a
complex vector space, where for $\lambda \in \C$, the action of $\lambda$ on a
homogeneous element $\mu \otimes v$ is given by
\[
\lambda \cdot (\mu \otimes v) = \lambda\mu \otimes v
\]
\end{defn}
%
The complexification is a different way of obtaining a complex vector space from a real
one. The complexification $V_\C$ is a vector space of twice the dimension of $V$ as a
vector space over $\R$. In addition, there is a natural inclusion
$V \hookrightarrow V_\C$ given by $v \mapsto 1 \otimes v$. Complexification is an
instance of \emph{extension of scalars} -- every element of $V_\C$ is of the form
$av + biw$ where $a,b \in \R$, and $v,w \in V$. Therefore, we will denote the element
$(a + bi) \otimes v \in V_\C$ by $av + biv$, and we have a direct sum decomposition
$V_\C = V \oplus iV$.   Given a linear map $T : V \to W$ of $\R$-vector spaces, we can
extend $T$ to a complexified map $T_\C : V_\C \to W_\C$, where
$T_\C(av + biw) = aTv + biTv$. In other words, $T_\C = \id_\C \otimes T$. In this way,
we see that complexification defines a covariant functor
$\mathsf{Vect}_\R \to \mathsf{Vect}_\C$ from the category of $\R$-vector spaces to the
category of $\C$-vector spaces. \\

A natural question to ask is how an almost complex structure interacts with
the process of complexification. Let $(V,J)$ be an almost complex vector space. Then the
complexified map $J_\C : V_\C \to V_\C$ squares to $-1$, and admits eigenvalues
$\pm i$. For example, consider $V = \C$. We then can then make the natural
identification of $\C$ with the almost complex vector space $(\R^2, J)$, where $J$
is given by the matrix
\[
J = \begin{pmatrix}
0 & -1 \\
1 & 0
\end{pmatrix}
\]
using the ordered $\R$ basis $(1,i)$. Then $V_\C = \C \otimes_\R \C = \C^2$, and $J_\C$
is given by the matrix
\[
J_\C = \begin{pmatrix}
0 & -1 & 0 & 0 \\
1 & 0 & 0 & 0 \\
0 & 0 & 0 & -1 \\
0 & 0 & 1 & 0
\end{pmatrix}
\]
in the ordered $\R$ basis $(1, i, 1, i)$, where the first pair elements are elements of
one copy of $\C^2$, and the second pair of elements are elements of a separate copy of
$\C^2$, giving an $\R$ basis for $\C \cong \C \oplus \C$. The matrix $J_\C$
clearly has eigenvalues $\pm i$, with the $i$-eigenspace being spanned by the
vectors
\[
\begin{pmatrix}
\frac{1}{2} \\[5pt]
0 \\[5pt]
\frac{1}{2} \\[5pt]
0
\end{pmatrix}, \begin{pmatrix}
\frac{i}{2} \\[5pt]
0 \\[5pt]
-\frac{i}{2} \\[5pt]
0
\end{pmatrix}
\]
and the $-i$ eigenspace being the span of
\[
\begin{pmatrix}
0 \\[5pt]
\frac{1}{2} \\[5pt]
0 \\[5pt]
\frac{1}{2}
\end{pmatrix}, \begin{pmatrix}
0 \\[5pt]
\frac{i}{2} \\[5pt]
0 \\[5pt]
-\frac{i}{2}
\end{pmatrix}
\]
The case for general $V$ is similar, and amounts to fixing an $\R$ basis of the
form $\set{v_1, Jv_1, \ldots v_n, Jv_n}$ for $(V,J)$. The decomposition of $V_\C$ into
the $\pm i$-eigenspaces of $J_\C$ gives a direct sum decomposition
\[
V_\C = V^{1,0} \oplus V^{0,1}
\]
where $V^{1,0}$ denotes the $i$-eigenspace, and $V^{0,1}$ denotes the $-i$-eigenspace.
Note that in the bases we chose above for $\C^2$, complex conjugation is given by the
matrix
\[
\begin{pmatrix}
0 & 1 & 0 & 0\\
1 & 0 & 0 & 0 \\
0 & 0 & 0 & 1 \\
0 & 0 & 1 & 0
\end{pmatrix}
\]
and determines an isomorphism of complex vector spaces
$V^{1,0} \cong \overline{V^{0,1}}$ and vice versa. \\

An almost complex structure $J$ on $V$ induces a dual map $J^* : V^* \to V^*$. By
functoriality of taking dual spaces, $(J^*) ^2 = -\id_{V^*}$, giving $V^*$ the
structure of an almost complex vector space via $J^*$. Explicitly, given a linear
functional $\alpha \in V^*$, we have that the action of $J^*$ on $\alpha$ is given
by
\[
(J^*\alpha)(v) = \alpha(Jv)
\]
for every $v \in V$. We then get an analogous decomposition of the complexified
dual space$ V^*_\C = \hom_\R(V,\C)$ as
\[
V^*_\C = (V^*)^{1,0} \oplus (V^*)^{0,1}
\]
into the $\pm i$ eigenspaces of $J^*$, and have very natural interpretations in
terms of $V$. The subspace $(V^*)^{1,0}$ consists of the $\alpha$ such that
$\alpha(Jv) = i\alpha(v)$. We have a natural pairing $(V^*)^{1,0} \otimes V^{1,0} \to \C$
given by
\[
\langle \alpha,v \rangle = \alpha(v)
\]
which is nondegenerate, establishing the two vector spaces as dual to each other (as
complex vector spaces). A similar statement holds for $V^{0,1}$ and $(V^*)^{0,1}$,
putting them in duality as well. Another perspective to take is that the condition
that $\alpha(Jv) = i\alpha(v)$ is equivalent to $\alpha$ being \emph{complex linear}
with respect to the complex structure $J$ on $V$ and $i$ on $\C$, giving us an
isomorphism $(V^*)^{1,0} \cong \hom_\C(V,\C)$. Similarly, we have that
$(V^*)^{0,1} \cong \hom_{\C}(V,\overline{\C})$
%
\section{Differential Forms}
%
\end{document}