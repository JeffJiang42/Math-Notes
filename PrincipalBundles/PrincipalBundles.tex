\documentclass[abstract=on,twoside]{scrreprt}

\usepackage[MyBox]{fancychap}
\usepackage{utmathjournal}

\begin{document}
\chapter{Principal Bundles} % your article title goes here
\thispagestyle{empty}
\vspace{-2cm}
\begin{flushright}
	\textbf{Jeffrey Jiang}\\ % your name goes here
	\texttt{jeffjiang@utexas.edu} % your email address goes here
\end{flushright}

\section*{Why Fiber Bundles?}

Suppose we want to study some space $M$, which for our purposes is a smooth manifold.
One way to study $M$ is to study functions $M \to F$ for some fixed target space $F$,
e.g. a manifold , $\R$, or a vector space. This is a perfectly good method
of studying $M$, but is sometimes not enough. More often then not, we want to
study ``function" from $M$ into a vector space that varies over the base
manifold $M$. For example, a vector field $X \in \mathfrak{X}(M)$ is not really
a map $M \to \R^n$, it is an assignment to each $p \in M$ a vector in $T_pM$.
In this way, we are led to the study of a smoothly parameterized family of
vector spaces -- the tangent bundle $TM$. This leads us to define a fiber bundle.
%
\begin{definition}
Let $M$ and $F$ be a smooth manifolds. A \textbf{\textit{fiber bundle}} over $M$ with
model fiber $F$ is the data of a smooth manifold $E$ with a smooth surjective map
$\pi : E \to M$ such that for each $p \in M$, there exists an open set $U$ and a
diffeomorphism $\varphi : \pi^{-1}(U) \to U \times F$ such that
\[\begin{tikzcd}
\pi^{-1}(U) \ar[dr, "\pi"']\ar[rr, "\varphi"] && U \times F \ar[dl, "p_1"] \\
& U
\end{tikzcd}\]
where $p_1 : U \times F \to U$ is projection onto the first factor. The map
$\varphi : \pi^{-1} \to U \times F$ is called a \textbf{\textit{local trivialization}}.
The space $E$ is called the \textbf{\textit{total space}}, while the manifold $M$
is called the \textbf{\textit{base space}}. We often omit naming the map $\pi$
and denote the fiber $\pi^{-1}(x)$ by $E_x$.
\end{definition}
%
This definition captures the notion of a family of manifolds diffeomorphic to $F$ that
are smoothly parameterized by the base space $M$. We also have a notion of a morphism
between bundles.
%
\begin{definition}
Let $\pi_E E \to M$ and $\pi_F : F \to M$ be fiber bundles with model fiber $X$. A
\textbf{\textit{bundle homomorphism}} is the data of a smooth map $\varphi : E \to F$
such that the diagram
\[\begin{tikzcd}
E \ar[dr, "\pi_E"']\ar[rr, "\varphi"] && F \ar[dl, "\pi_F"]\\
& M
\end{tikzcd}\]
commutes.
\end{definition}

Our original motivation for thinking about fiber bundles was for a generalized notion of
a function. To this end, we specify a special class of maps associated to a fiber
bundle.
%
\begin{definition}
Let $\pi : E \to M$ be a a fiber bundle with model fiber $F$.
A \textbf{\textit{local section}} of $\pi : E \to M$ is a smooth map $\sigma : U \to E$
such that $\pi \circ \sigma = \id_U$ for some open set $U \subset M$. If $U = M$, we
call $\sigma $ a \textbf{\textit{global section}}. Equivalently, it is a smooth assignment
to each $p \in U$ a point in the fiber $E_p$. We denote the space of sections over
an open set $U$ as $\Gamma_U(E)$.
\end{definition}
%
A section of $E \to M$ can be thought of as our desired generalization of a function.
A map $M \to F$ is the same data as a section of the trivial bundle $M \times F \to M$.
However, not every fiber bundle is trivial -- there can be a nontrivial ``twisting."
An example of this is the M\"obius band. Can you see why this bundle over $S^1$ is not
isomorphic to the trivial bundle $S^1 \times [0,1]$? \\

We are especially interested in two special classes of fiber bundles that carry
additional structure -- the fibers of a vector bundle carry the extra structure of a
vector space, and the fibers of a principal bundle have the extra structure of a
$G$-torsor for a Lie group $G$.
%
\begin{definition}
A \textbf{\textit{vector bundle of rank}} $k$ is a fiber bundle $E \to M$ such that each
fiber $E_x$ has the structure of a $k$-dimensional vector space (usually over $\R$ or
$\C$). A \textbf{\textit{vector bundle homomorphism}} is a bundle homomorphism that
restricts to a linear map on each fiber.
\end{definition}
%
Vector bundles form a familiar family of fiber bundles, as tangent bundles,
cotangent bundles, and their associated tensor bundles are all vector bundles.
%
\end{document}