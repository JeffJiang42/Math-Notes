\documentclass[abstract=on,twoside]{scrreprt}

\usepackage[MyBox]{fancychap}
\usepackage{utmathjournal}

\begin{document}
\chapter{Principal Bundles} % your article title goes here
\thispagestyle{empty}
\vspace{-2cm}
\begin{flushright}
	\textbf{Jeffrey Jiang}\\ % your name goes here
	\texttt{jeffjiang@utexas.edu} % your email address goes here
\end{flushright}

\section*{Why Fiber Bundles?}

Suppose we want to study some space $M$, which for our purposes is a smooth manifold.
One way to study $M$ is to study functions $M \to F$ for some fixed target space $F$,
e.g. a manifold , $\R$, or a vector space. This is a perfectly good method
of studying $M$, but is sometimes not enough. More often then not, we want to
study ``function" from $M$ into a vector space that varies over the base
manifold $M$. For example, a vector field $X \in \mathfrak{X}(M)$ is not really
a map $M \to \R^n$, it is an assignment to each $p \in M$ a vector in $T_pM$.
In this way, we are led to the study of a smoothly parameterized family of
vector spaces -- the tangent bundle $TM$. This leads us to define a fiber bundle.
%
\begin{definition}
Let $M$ and $F$ be a smooth manifolds. A \textbf{\textit{fiber bundle}} over $M$ with
model fiber $F$ is the data of a smooth manifold $E$ with a smooth surjective map
$\pi : E \to M$ such that for each $p \in M$, there exists an open set $U$ and a
diffeomorphism $\varphi : \pi^{-1}(U) \to U \times F$ such that
\[\begin{tikzcd}
\pi^{-1}(U) \ar[dr, "\pi"']\ar[rr, "\varphi"] && U \times F \ar[dl, "p_1"] \\
& U
\end{tikzcd}\]
where $p_1 : U \times F \to U$ is projection onto the first factor. The map
$\varphi : \pi^{-1} \to U \times F$ is called a \textbf{\textit{local trivialization}}.
The space $E$ is called the \textbf{\textit{total space}}, while the manifold $M$
is called the \textbf{\textit{base space}}. We often omit naming the map $\pi$
and denote the fiber $\pi^{-1}(x)$ by $E_x$.
\end{definition}
%
This definition captures the notion of a family of manifolds diffeomorphic to $F$ that
are smoothly parameterized by the base space $M$. We also have a notion of a morphism
between bundles.
%
\begin{definition}
Let $\pi_E E \to M$ and $\pi_F : F \to M$ be fiber bundles with model fiber $X$. A
\textbf{\textit{bundle homomorphism}} is the data of a smooth map $\varphi : E \to F$
such that the diagram
\[\begin{tikzcd}
E \ar[dr, "\pi_E"']\ar[rr, "\varphi"] && F \ar[dl, "\pi_F"]\\
& M
\end{tikzcd}\]
commutes.
\end{definition}

Our original motivation for thinking about fiber bundles was for a generalized notion of
a function. To this end, we specify a special class of maps associated to a fiber
bundle.
%
\begin{definition}
Let $\pi : E \to M$ be a a fiber bundle with model fiber $F$.
A \textbf{\textit{local section}} of $\pi : E \to M$ is a smooth map $\sigma : U \to E$
such that $\pi \circ \sigma = \id_U$ for some open set $U \subset M$. If $U = M$, we
call $\sigma $ a \textbf{\textit{global section}}. Equivalently, it is a smooth assignment
to each $p \in U$ a point in the fiber $E_p$. We denote the space of sections over
an open set $U$ as $\Gamma_U(E)$.
\end{definition}
%
A section of $E \to M$ can be thought of as our desired generalization of a function.
A map $M \to F$ is the same data as a section of the trivial bundle $M \times F \to M$.
However, not every fiber bundle is trivial -- there can be a nontrivial ``twisting."
An example of this is the M\"obius band. Can you see why this bundle over $S^1$ is not
isomorphic to the trivial bundle $S^1 \times [0,1]$? \\

We are especially interested in two special classes of fiber bundles that carry
additional structure -- the fibers of a vector bundle carry the extra structure of a
vector space, and the fibers of a principal bundle have the extra structure of a
$G$-torsor for a Lie group $G$.
%
\begin{definition}
A \textbf{\textit{vector bundle of rank}} $k$ is a fiber bundle $E \to M$ such that each
fiber $E_x$ has the structure of a $k$-dimensional vector space (usually over $\R$ or
$\C$). A \textbf{\textit{vector bundle homomorphism}} is a bundle homomorphism that
restricts to a linear map on each fiber.
\end{definition}
%
Vector bundles form a familiar family of fiber bundles, as tangent bundles,
cotangent bundles, and their associated tensor bundles are all vector bundles.
%
\begin{definition}
Let $G$ be a Lie group. a (left) \textbf{\textit{$G$-torsor}} is a smooth manifold $M$
equipped with a smooth (left) $G$-action that is free in transitive i.e.
\begin{enumerate}
	\item For any $g \in G$ and $p \in M$, if $g\cdot p = p$, then $g = e$.
	\item For any $p \in M$, the orbit $G \cdot p$ is all of $M$
\end{enumerate}
combined, these conditions give that for any fixed point $p$, there exists a unique
group element $g \in G$ that takes $p$ to any other point of $M$. In other words,
the map $g \mapsto g \cdot p$ is a diffeomorphism.
\end{definition}
%
\begin{example}
Let $V$ be a vector space. A \textbf{\textit{basis}} (also called a
\textbf{\textit{frame}}) of $V$ is a linear isomorphism $\R^n \to V$, and the
set of frames of $V$ is denoted $\\mathcal{B}(V)$. There is a
natural right action of $GL_n\R$ on $\mathcal{B}(V)$ by precomposition, i.e.
for $g \in GL_n\R$ and $b \in \mathcal{B}(V)$, we have $b \cdot g = b \circ g$.
This action is free and transitive, which allows us to define a topology and
smooth structure on $\mathcal{B}(V)$, giving it the structure of a
right $GL_n\R$-torsor.
\end{example}
%
$G$-torsors can be thought of as Lie groups without a fixed choice of identity element.
Indeed, if we fix a point $p$ of a $G$-torsor $M$, this determines a unique Lie group
structure on $M$ where $p$ is the identity element. This allows us to define
a principal $G$-bundle. For example, there is no distinguished basis for a vector
space $V$, which is reflected by the fact that $\mathcal{B}(V)$ is a torsor, rather
than a group.
%
\begin{definition}
A \textbf{\textit{principal $G$-bundle}} is a fiber bundle $\pi : P \to M$ with a
smooth \emph{right} $G$-action such that every fiber $\pi^{-1}(x)$ is a $G$-torsor,
and every point $p \in M$ in the base has a local trivialization
$\pi^{-1}(U) \to U \times G$ that is $G$-equivariant, where the $G$-action on
$U \times G$ is right multiplication on the second factor. The group $G$ is often
called the \textbf{\textit{structure group}}.
\end{definition}
%
\begin{example}[\textbf{\textit{Bundles of frames}}]
Let $M$ be a smooth manifold, and $p \in M$. Let $\mathcal{B}_p$ denote the
$GL_n\R$-torsor of frames of the tangent space $T_pM$, and let
$\mathcal{B}(M)$ be the set
\[
\mathcal{B}(M) = \coprod_{p \in M} \mathcal{B}_p
\]
This set comes with a natural map $\pi : \mathcal{B}(M) \to M$ that maps a frame
$b \in \mathcal{B}_p$ to $p$, so the fiber $\pi^{-1}(p)$ over any point is
$\mathcal{B}_p$. Local coordinates $x^i$ on an open set $U \subset M$ induce bijections
$\pi^{-1}(U) \to U \times GL_n\R$, where we map $b \in \mathcal{B}_p$ to
$(p, A_b)$, where $A_b \in GL_n\R $ is the matrix where the $i^{th}$ column are the
components of $b(e_i)$ with respect to the coordinate vectors $\partial_i$. This
then induces a topology and smooth structure on $\mathcal{B}(M)$ such that
the projection map $\pi : \mathcal{B}(M) \to M$ is smooth and the maps
$\pi^{-1}(U) \to U \times GL_n\R$ are smooth local trivializations. \\

This construction works for vector bundles as well. For a rank $k$ vector bundle
$E \to M$, there is a principal $GL_n\R$-bundle $\mathcal{B}(E) \to M$, where the fiber
over a point $p$ is the $GL_k\R$-torsor of frames for the fiber $E_x$.
\end{example}
%
Heuristically, principal bundles can be thought of bundles of symmetries of some
object, and in many cases, this object is a vector bundle, as we will soon see.
%
\section*{Groups Control the Geometry}
%
Given any geometric object $M$ (e.g. a vector space, manifold, fiber bundle), an
extremely important question to as is -- "what are the symmetries of $M$?" More
concretely, we would like to know the \emph{automorphisms} of $M$. Any quantity
intrinsic to the object $M$ must be invariant under these automorphisms, sine
we have no "preferred frame of reference" for the object $M$. In this way, we
see that any geometric properties of $M$ are defined by the symmetry group $G$,
and we say the group $G$ "controls" the geometry of $M$. For example, if we let
$V$ be any finite dimensional vector space, the group of symmetries is the general
linear group $GL(V)$ of invertible linear transformations. If we further fix an
inner product $\langle\cdot,\cdot\rangle$ on $V$, we have a smaller class of
symmetries -- the orthogonal group $O(V) \subset GL(V)$ of linear automorphisms
preserving the inner product. Additional structure (e.g. orientation, symplectic
form, complex structure) allows us to pick out subgroups  of automorphisms, which gives
a more restrictive class of symmetries for our vector spaces, \\

This proves to be a very fruitful philosophy for approaching geometry. Smooth manifolds
are locally modeled on vector spaces, and given a structure on the linear world
of vector spaces, we often get an analogous structure in the nonlinear world of
manifolds. For example, a Riemannian metric $g$ on a manifold $M$ is the nonlinear
analogue of an inner product on a vector space. Because of this, we would expect
the orthogonal group $O_n$ to play an important role in the geometry of a Riemannaian
manifold. The interaction of the group with the geometry comes in play through the
language of principal bundles. Before we mentioned that principal bundles can be
thought of as bundles of symmetries, which we make precise with the construction of
a \textbf{\textit{associated bundle}}.
%
\begin{definition}
Let $P \to M$ be a principal $G$-bundle, and $F$ a smooth manifold with a smooth
left $G$-action. The \textbf{\textit{associated bundle}} is the bundle
\[
P \times_G F = P \times F / (p, f) \sim (p \cdot g, g^{-1} \cdot f)
\]
This is a fiber bundle over $M$ with model fiber $F$, and it is a good exercise to see
why this is true by explicitly constructing the local trivializations in terms
of local trivializations of $\mathcal{B}(M)$.
\end{definition}
%
The construction is a bit obtuse, so we make a few observations to make sense of
why we want to care about associated bundles.
%
\begin{example}[\textbf{\textit{The tangent bundle}}]
Let $M$ be a smooth manifold. From $M$, we have a principal $GL_n$-bundle
$\mathcal{B}(M) \to M$ of frames for $TM$. In addition, $GL_n\R$ admits a natural
left action on $\R^n$ via matrix multiplication. The associated bundle
$E = \mathcal{B}(M) \times_{GL_n\R} \R^n$ is isomorphic to the tangent bundle
$TM$. To see this, we construct maps in both directions. \\

Let $[b, v]$ denote an equivalence class in $E$, where $b \in \mathcal{B}_p$ is a
frame for $T_pM$ and $v \in \R^n$. Then define $\varphi : E \to TM$ by
$\varphi[b,v] = b(v)$. This is well defined, since
\[
\varphi[b\cdot g, g^{-1}\cdot v] = b \circ g(g^{-1}(v)) = b(v)
\]
In the other direction, let $(p,v) \in TM$, i.e. $p \in M$ and $v \in T_pM$. Fix
a basis $b : \R^n \to T_pM$, and define $\psi : TM \to E$ by
$\psi(p,v) = [b,b^{-1}(v)]$. It's an easy exercise to check that these two maps compose
to identity in both directions.
\end{example}
%
The above construction works in the setting of vector bundles as well. Given
a rank $k$ vector bundle $E \to M$, we can construct the principal $GL_k\R$-bundle
$\mathcal{B}(E)$, and the associated bundle $\mathcal{B}(E) \times_{GL_k\R} \R^k$ is
isomorphic to the bundle $E$. \\

This construction shows why we might care about associated bundles -- the quotient by
relation we specified exactly gives the correct transformation law for tangent
vectors. You may have heard the joke that physicists define a vector as ``something
that transforms like a vector." A more precise joke here would perhaps be ``a tangent
vector is something that transforms like a tangent vector." What do we mean by this?
Physically, the only ``real" quantities are those invariant under a change of reference
frame. In terms our example, we see that a tangent vector $v \in T_pM$ is
\emph{not} just an $n$-tuple of numbers, it's the collection of all coordinate
representations of $v$ with respect to any basis of $T_pM$, which is exactly what
the associated bundle construction captures. We can see this another way. Fix
an element $b \in \mathcal{B}_p \subset \mathcal{B}(M)$. This then determines an
isomorphism of $\R^n$ to the fiber of the associated bundle
$\mathcal{B}(M) \times_{GL_n\R} \R^n$ by $v \mapsto [b,v]$, this exactly what happens
when we fix a basis for $T_pM$! In this way, we see that fixing the first component
in the equivalence class of an element of an associated bundle is essentially a
choice of basis or reference frame, and doing the computations with the equivalence
classes themselves is in essense, working in a coordinate-free manner. \\
%TODO REDUCTION OF STRUCTURE GROUP

We now address how fixing additional structures on our manifold $M$ changes this
picture. The best example here is that of a Riemannian metric $g$ on $M$, which gives
us a notion of an \textbf{\textit{orthonormal frame}} of the tangent space $T_pM$,
which is a linear \emph{isometry} $b : (\R^n,\langle\cdot,\cdot\rangle) \to (T_pM,g_p)$
where $\langle\cdot,\cdot\rangle$ is the standard inner product on $\R^n$, and
$g_p$ is the Riemannian metric evaluated at the point $p$. We then construct the
\textbf{\textit{orthonormal frame bundle}} $\mathcal{B}_O(M)$ in the same way we
constructed the frame bundle $\mathcal{B}(M)$, which is a principal $O_n$-bundle,
which reflects that we now have a more restrictive view on symmmetry -- our
automorphisms now take values in $O_n$ instead of $GL_n\R$.
%
\begin{definition}
Let $M$ be a smooth manifold, and $\mathcal{B}(M)$ its $GL_n\R$-bundle of frames.
Let $G$ be a Lie group, and $\rho : G \to GL_n\R$ be a homomorphism. Then a
\textbf{\textit{reduction of structure group}} to $G$ is the data of a principal
$G$-bundle $Q \to M$ equipped with a $G$-equivariant map $Q \to \mathcal{B}(M)$, where
$G$ acts on the left of $\mathcal{B}(M)$ by $g \cdot b = b \circ \rho(g)^{-1}$.
\end{definition}
%
In the case we illustrated above with a Riemannian metric $g$, the homomorphism
$\rho$ is just the inclusion $O_n \hookrightarrow GL_n\R$, the bundle $Q$ is
$\mathcal{B}_O(M)$, and the $O_n$-equivariant map is the inclusion
$\mathcal{B}_O(M) \hookrightarrow \mathcal{B}(M)$. However, the homomorphism $\rho$
does \emph{not} need to be injective, which makes the name a bit a misnomer -- the
group does not need to be a subgroup of $GL_n\R$. For example, when working with
the Spin group $\mathrm{Spin}_n$, the map $\rho$ is often the double cover
$\mathrm{Spin}_n \to SO_n$.

Our construction of the tangent bundle as an associated bundle can also be generalized
to other tensor bundles built out of the tangent bundle (e.g. $\mathcal{T}^k_\ell(M)$,
$T^*M$, $\Lambda^k(T^*M)$, etc), by using the induced representations of the
stucture group $G$ on tensor products, the dual representation, and the induced
representations on exterior powers respectively. From this, we see that the associated
bundle construction is the bridge between the theory of principal bundles and vector
bundles with representation theory, and reductions of structure group allows us to
import our linear model geometries (e.g. an inner product space, a vector space with
a complex structure $I$, an oriented vector space, etc.) to the nonlinear world of
manifolds by asking for a reduction of structure group to the appropriate group. For
example, an orientation is equivalent to a reduction of structure group to
$GL_n^+\R = \{A \in GL_n\R ~:~ \det A > 0\}$, and an almost complex structure on a
$2n$-dimensional manifold is equivalent  to a reduction of structure group from
$GL_{2n}\R$ to $GL_n\C$.
%
\section*{References}
%
Most of the content of this article came from many discussions with Professor Dan
Freed. If you want to learn more, I heavily recommend finding some time to talk to
him. A good reference for the theory of principal bundles can be found in Kobayashi
and Nomizu's books \emph{Foundations of Differential Geometry}.
%
\end{document}