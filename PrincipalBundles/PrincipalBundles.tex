\documentclass[abstract=on,twoside]{scrreprt}

\usepackage[MyBox]{fancychap}
\usepackage{utmathjournal}

\begin{document}
\chapter{Principal Bundles} % your article title goes here
\thispagestyle{empty}
\vspace{-2cm}
\begin{flushright}
	\textbf{Jeffrey Jiang}\\ % your name goes here
	\texttt{jeffjiang@utexas.edu} % your email address goes here
\end{flushright}

\section*{Why Fiber Bundles?}

Suppose we want to study some space $M$, which for our purposes is a smooth manifold.
One way to study $M$ is to study functions $M \to F$ for some fixed target space $F$,
e.g. a manifold , $\R$, or a vector space. This is a perfectly good method
of studying $M$, but is sometimes not enough. More often then not, we want to
study ``function" from $M$ into a vector space that varies over the base
manifold $M$. For example, a vector field $X \in \mathfrak{X}(M)$ is not really
a map $M \to \R^n$, it is an assignment to each $p \in M$ a vector in $T_pM$.
In this way, we are led to the study of a smoothly parameterized family of
vector spaces -- the tangent bundle $TM$. This leads us to define a fiber bundle.
%
\begin{definition}
Let $M$ and $F$ be a smooth manifolds. A \textbf{\textit{fiber bundle}} over $M$ with
model fiber $F$ is the data of a smooth manifold $E$ with a smooth surjective map
$\pi : E \to M$ such that for each $p \in M$, there exists an open set $U$ and a
diffeomorphism $\varphi : \pi^{-1}(U) \to U \times F$ such that
\[\begin{tikzcd}
\pi^{-1}(U) \ar[dr, "\pi"]\ar[rr, "\varphi"] && U \times F \ar[dl, "p_1"] \\
& U
\end{tikzcd}\]
where $p_1 : U \times F \to U$ is projection onto the first factor.
\end{definition}
%
\end{document}