\documentclass[psamsfonts, 12pt]{amsart}
%
%-------Packages---------
%
\usepackage[h margin=1 in, v margin=1 in]{geometry}
\usepackage{amssymb,amsfonts}
\usepackage{amsmath}
\usepackage{accents}
\usepackage[all,arc]{xy}
\usepackage{tikz-cd}
\usepackage{enumerate}
\usepackage{mathrsfs}
\usepackage{amsthm}
\usepackage{mathpazo}
\usepackage{float}
%\usepackage[backend=biber]{biblatex}
%\addbibresource{bibliography.bib}
%\usepackage{charter} %another font
%\usepackage{eulervm} %Vakil font
\usepackage{yfonts}
\usepackage{mathtools}
\usepackage{enumitem}
\usepackage{mathrsfs}
\usepackage{fourier-orns}
\usepackage[all]{xy}
\usepackage{hyperref}
\usepackage{url}
\usepackage{mathtools}
\usepackage{graphicx}
\usepackage{pdfsync}
\usepackage{mathdots}
\usepackage{calligra}
\usepackage{import}
\usepackage{xifthen}
\usepackage{pdfpages}
\usepackage{transparent}

\usepackage{tgpagella}
\usepackage[T1]{fontenc}
%
\usepackage{listings}
\usepackage{color}

\definecolor{dkgreen}{rgb}{0,0.6,0}
\definecolor{gray}{rgb}{0.5,0.5,0.5}
\definecolor{mauve}{rgb}{0.58,0,0.82}

\lstset{frame=tb,
  language=Matlab,
  aboveskip=3mm,
  belowskip=3mm,
  showstringspaces=false,
  columns=flexible,
  basicstyle={\small\ttfamily},
  numbers=none,
  numberstyle=\tiny\color{gray},
  keywordstyle=\color{blue},
  commentstyle=\color{dkgreen},
  stringstyle=\color{mauve},
  breaklines=true,
  breakatwhitespace=true,
  tabsize=3
  }
%
%--------Theorem Environments--------
%
\newtheorem{thm}{Theorem}[section]
\newtheorem*{thm*}{Theorem}
\newtheorem{cor}[thm]{Corollary}
\newtheorem{prop}[thm]{Proposition}
\newtheorem{lem}[thm]{Lemma}
\newtheorem*{lem*}{Lemma}
\newtheorem{conj}[thm]{Conjecture}
\newtheorem{quest}[thm]{Question}
%
\theoremstyle{definition}
\newtheorem{defn}[thm]{Definition}
\newtheorem*{defn*}{Definition}
\newtheorem{defns}[thm]{Definitions}
\newtheorem{con}[thm]{Construction}
\newtheorem{exmp}[thm]{Example}
\newtheorem{exmps}[thm]{Examples}
\newtheorem{notn}[thm]{Notation}
\newtheorem{notns}[thm]{Notations}
\newtheorem{addm}[thm]{Addendum}
\newtheorem{exer}[thm]{Exercise}
%
\theoremstyle{remark}
\newtheorem{rem}[thm]{Remark}
\newtheorem*{claim}{Claim}
\newtheorem*{aside*}{Aside}
\newtheorem*{rem*}{Remark}
\newtheorem*{hint*}{Hint}
\newtheorem*{note}{Note}
\newtheorem{rems}[thm]{Remarks}
\newtheorem{warn}[thm]{Warning}
\newtheorem{sch}[thm]{Scholium}
%
%--------Macros--------
\renewcommand{\qedsymbol}{$\blacksquare$}
\renewcommand{\sl}{\mathfrak{sl}}
\newcommand{\Bord}{\mathsf{Bord}}
\renewcommand{\hom}{\mathrm{Hom}}
\renewcommand{\emptyset}{\varnothing}
\renewcommand{\O}{\mathcal{O}}
\newcommand{\R}{\mathbb{R}}
\newcommand{\ib}[1]{\textbf{\textit{#1}}}
\newcommand{\Q}{\mathbb{Q}}
\newcommand{\Z}{\mathbb{Z}}
\newcommand{\N}{\mathbb{N}}
\newcommand{\C}{\mathbb{C}}
\newcommand{\A}{\mathbb{A}}
\newcommand{\F}{\mathbb{F}}
\newcommand{\M}{\mathcal{M}}
\newcommand{\dbar}{\overline{\partial}}
\newcommand{\zbar}{\overline{z}}
\renewcommand{\S}{\mathbb{S}}
\newcommand{\V}{\vec{v}}
\newcommand{\RP}{\mathbb{RP}}
\newcommand{\CP}{\mathbb{CP}}
\newcommand{\B}{\mathcal{B}}
\newcommand{\GL}{\mathrm{GL}}
\newcommand{\SL}{\mathrm{SL}}
\newcommand{\SP}{\mathrm{SP}}
\newcommand{\SO}{\mathrm{SO}}
\newcommand{\SU}{\mathrm{SU}}
\newcommand{\gl}{\mathfrak{gl}}
\newcommand{\g}{\mathfrak{g}}
\newcommand{\Bun}{\mathsf{Bun}}
\newcommand*{\dt}[1]{%
   \accentset{\mbox{\large\bfseries .}}{#1}}
\newcommand{\inv}{^{-1}}
\newcommand{\bra}[2]{ \left[ #1, #2 \right] }
\newcommand{\set}[1]{\left\lbrace #1 \right\rbrace}
\newcommand{\abs}[1]{\left\lvert#1\right\rvert}
\newcommand{\norm}[1]{\left\lVert#1\right\rVert}
\newcommand{\transv}{\mathrel{\text{\tpitchfork}}}
\newcommand{\defeq}{\vcentcolon=}
\newcommand{\enumbreak}{\ \\ \vspace{-\baselineskip}}
\let\oldexists\exists
\renewcommand\exists{\oldexists~}
\let\oldL\L
\renewcommand\L{\mathfrak{L}}
\makeatletter
\newcommand{\incfig}[2]{%
    \fontsize{48pt}{50pt}\selectfont
    \def\svgwidth{\columnwidth}
    \scalebox{#2}{\input{#1.pdf_tex}}
}
%
\newcommand{\tpitchfork}{%
  \vbox{
    \baselineskip\z@skip
    \lineskip-.52ex
    \lineskiplimit\maxdimen
    \m@th
    \ialign{##\crcr\hidewidth\smash{$-$}\hidewidth\crcr$\pitchfork$\crcr}
  }%
}
\makeatother
\newcommand{\bd}{\partial}
\newcommand{\lang}{\begin{picture}(5,7)
\put(1.1,2.5){\rotatebox{45}{\line(1,0){6.0}}}
\put(1.1,2.5){\rotatebox{315}{\line(1,0){6.0}}}
\end{picture}}
\newcommand{\rang}{\begin{picture}(5,7)
\put(.1,2.5){\rotatebox{135}{\line(1,0){6.0}}}
\put(.1,2.5){\rotatebox{225}{\line(1,0){6.0}}}
\end{picture}}
\DeclareMathOperator{\id}{id}
\DeclareMathOperator{\im}{Im}
\DeclareMathOperator{\codim}{codim}
\DeclareMathOperator{\coker}{coker}
\DeclareMathOperator{\supp}{supp}
\DeclareMathOperator{\inter}{Int}
\DeclareMathOperator{\sign}{sign}
\DeclareMathOperator{\sgn}{sgn}
\DeclareMathOperator{\indx}{ind}
\DeclareMathOperator{\alt}{Alt}
\DeclareMathOperator{\Aut}{Aut}
\DeclareMathOperator{\trace}{trace}
\DeclareMathOperator{\ad}{ad}
\DeclareMathOperator{\End}{End}
\DeclareMathOperator{\Ad}{Ad}
\DeclareMathOperator{\Lie}{Lie}
\DeclareMathOperator{\spn}{span}
\DeclareMathOperator{\dv}{div}
\DeclareMathOperator{\grad}{grad}
\DeclareMathOperator{\Sym}{Sym}
\DeclareMathOperator{\tr}{tr}
\DeclareMathOperator{\sheafhom}{\mathscr{H}\text{\kern -3pt {\calligra\large om}}\,}
\newcommand*\myhrulefill{%
   \leavevmode\leaders\hrule depth-2pt height 2.4pt\hfill\kern0pt}
\newcommand\niceending[1]{%
  \begin{center}%
    \LARGE \myhrulefill \hspace{0.2cm} #1 \hspace{0.2cm} \myhrulefill%
  \end{center}}
\newcommand*\sectionend{\niceending{\decofourleft\decofourright}}
\newcommand*\subsectionend{\niceending{\decosix}}
\def\upint{\mathchoice%
    {\mkern13mu\overline{\vphantom{\intop}\mkern7mu}\mkern-20mu}%
    {\mkern7mu\overline{\vphantom{\intop}\mkern7mu}\mkern-14mu}%
    {\mkern7mu\overline{\vphantom{\intop}\mkern7mu}\mkern-14mu}%
    {\mkern7mu\overline{\vphantom{\intop}\mkern7mu}\mkern-14mu}%
  \int}
\def\lowint{\mkern3mu\underline{\vphantom{\intop}\mkern7mu}\mkern-10mu\int}
%
%--------Hypersetup--------
%
\hypersetup{
    colorlinks,
    citecolor=black,
    filecolor=black,
    linkcolor=blue
}
%
%--------Solution--------
%
\newenvironment{solution}
  {\begin{proof}[Solution]}
  {\end{proof}}
%
%--------Graphics--------
%
%\graphicspath{ {images/} }
%
\begin{document}
%
\author{Jeffrey Jiang}
%
\title{Hodge Theory and $\mathrm{U}(1)$ Yang-Mills}
%
\maketitle
%
\section{Hodge Theory on Riemannian Manifolds}
%
Fix an $n$-dimensional compact boundaryless Riemannian manifold $(X,g)$. The Riemannian
metric gives us a volume form $\mathrm{Vol}_g \in \Omega^n_X$, which is uniquely
characterized as the $n$-form whose representation in a local orthonormal frame
$\set{E^i}$ of $T^*X$ is given by
\[
\mathrm{Vol}_g = E^1 \wedge\cdots\wedge E^n
\]

The Riemannian metric also induces an fiber metric on the bundles $\Lambda^kT^*M$ by
declaring the sets
\[
\set{E^{i_1} \wedge \cdots \wedge E^{i_k} ~:~ 1 \leq i_1 < \cdots < i_k \leq n}
\]
to be orthonormal frames for the bundles $\Lambda^kT^*X$. The \ib{Hodge star}
is an operator $\star : \Omega^k_X \to \Omega^{n-k}$, whose action on these
orthonormal frames can be described as ``whatever is missing to get the volume form."
For example, in $\R^4$, the volume form is $dx^1 \wedge dx^2 \wedge dx^3 \wedge dx^4$,
and
\[
\star(dx^1\wedge dx^3) = -dx^2\wedge dx^4
\]
The negative sign accounts for the fact that we need to move $dx^2$ past
the $dx^3$ to get the volume form, which picks up a negative sign due to
skew symmetry of the wedge product of $1$-forms. Using the Hodge star, we
define an inner product $(\cdot,\cdot)_g$ on the space $\Omega^k_X$ where
\[
(\omega,\eta)_g \defeq \int_X \omega\wedge\star\eta
\]
We then define the operator $d^* : \Omega^k_X \to \Omega^{k-1}_X$ by
$d^* = (-1)^{n(k+1)+1}\star d\star$. As the notation suggests, $d^*$ is the formal
adjoint to $d$, i.e.
\[
(\omega,d\eta)_g = (d^*\omega,\eta)_g
\]
%
\begin{defn}
The \ib{Hodge Laplacian} is the operator $\Delta \defeq dd^* + d^*d$. Forms $\omega$
satisfying $\Delta\omega = 0$ are said to be \ib{harmonic}, and the space
of harmonic $k$-forms is denoted $\mathcal{H}^k(X)$.
\end{defn}
%
\begin{prop}
A $k$ form $\omega$ is harmonic if and only if $d\omega = d^*\omega = 0$.
\end{prop}
%
\begin{thm}[\ib{The Hodge Theorem}]
Every de Rham cohomology class $[\omega] \in H^k_{\mathrm{dR}}(X)$ has a unique
harmonic representative, i.e. the map
\begin{align*}
\mathcal{H}^k(X) &\to H^k_{\mathrm{dR}}(X) \\
\alpha &\mapsto [\alpha]
\end{align*}
is an isomorphism.
\end{thm}
%
One way to prove this is to characterize the harmonic forms as the unique minimizers of
the norm $\norm{\cdot}$ induced by $(\cdot,\cdot)_g$, and this characterization
is the one that fits best into the Yang-Mills picture, and is also the main
takeaway from Hodge theory on a Riemannian manifold : every cohomology class has
a distinguished representative that ``minimizes energy."
%
\section{Yang-Mills with Principal $\mathrm{U}(1)$-bundles}
%
In the more general story, Yang-Mills theory is a machine in which you specify
some compact connected Lie group $G$ called the \ib{structure group} (though
physics will often call this the gauge group, or global gauge group if they're feeling
nice.) (e.g. $\mathrm{U}(n)$, $\mathrm{SO}(n)$), and the machine spits out some
differential equations for you to solve. The story here involves principal $G$-bundles
and connections on them. In the case where  $G = \mathrm{U}(1)$, the story simplifies
greatly. The main reason for this is  the fact that $\mathrm{U}(1)$ is \emph{abelian}.
When the structure group $G$ is nonabelian, many of the ideas in the $\mathrm{U}(1)$
case generalize, but are in some sense \emph{twisted} by the noncommutativity of
the group $G$. We will focus on the $\mathrm{U}(1)$ story.
%
\begin{defn}
A \ib{principal $\mathrm{U}(1)$-bundle} over $X$ is a space $P$ equipped with a
$\mathrm{U}(1)$ action and a smooth submersion $\pi : P \to X$ such that:
\begin{enumerate}
  \item The action of $\mathrm{U}(1)$ preserves the fibers of $\pi$.
  \item $\mathrm{U}(1)$ acts freely and transitively on the fibers -
  in particular, every fiber is diffeomorphic to $\mathrm{U}(1)$.
  \item For every $x \in X$, there exists an open neighborhood $U$ of $x$
  and a $\mathrm{U}(1)$-equivariant diffeomorphism
  $\varphi : \pi\inv(U) \to U \times \mathrm{U}(1)$ such that the following diagram
  commutes:
  \[\begin{tikzcd}
  \pi\inv(U) \ar[rr, "\varphi"] \ar[dr] && U \times \mathrm{U}(1) \ar[dl] \\
  & U
  \end{tikzcd}\]
\end{enumerate}
\end{defn}
%
Topologically, $\mathrm{U}(1)$ is a circle, so in particular, a principal
$\mathrm{U}(1)$-bundle $P \to X$ is a circle bundle over $X$, which is a smoothly
varying family of circles parameterized by the base space of $X$. However, the
$\mathrm{U}(1)$ action is additional data -- not every circle bundle is a principal
$\mathrm{U}(1)$-bundle. The distinguishing feature is orientability -- the
$\mathrm{U}(1)$ action on $P$ gives a \emph{global} notion of ``rotate each fiber
clockwise by angle $\theta$." However, there exist non-orientable circle bundles. For
example, the Klein bottle is a nonorientable circle bundle over the circle. If you
walked around the base circle of the Klein bottle and rotated the fiber above clockwise
you as you walked, once you've gone around once you would be rotating counterclockwise!
%
\begin{defn}
A \ib{connection} on a principal circle bundle $\pi : P \to X$ is a $1$-form
$A \in \Omega^1_P$ such that the restriction of $A$ to any fiber $\pi\inv(x) \cong S^1$
is $d\theta \in \Omega^1_{S^1}$\footnote{The name $d\theta$ for this form is a bit
unfortunate -- it is not exact.}.
\end{defn}
%
In the more general setting, a connection on a principal $G$-bundle is a Lie
algebra valued $1$ form such that the restriction to a fiber is something called
the \ib{Maurer-Cartan form}, and it must satisfy an additional property that doesn't
appear here due to the fact that $\mathrm{U}(1)$ is abelian. \\

One way to think of a connection is in terms of the kernel of the $1$-form $A$.
This gives a rank $n$ subbundle of $TP$ called the \ib{horizontal distribution},
and can be thought of as a distinguished ``manifold direction" on the space $P$,
which locally looks like $X \times \mathrm{U}(1)$. The notion of a horizontal
distribution generalizes to arbitrary fiber bundles. \\

The space of connection on $P$ is denoted $\mathscr{A}(P)$, and it is an
infinite dimensional affine space over the vector space $\Omega^1_X$. In
other words, any two connections $A_1,A_2$ differ by the pullback of an element
in $\Omega^1_X$ along $\pi$. However, the sum of two connections need not
be a connection -- $\mathscr{A}(P)$ is not a vector space.
%
\begin{defn}
Let $\pi : P \to X$ be a principal $\mathrm{U}(1)$-bundle with connection $A$.
The \ib{curvature form} of $A$ is a $2$-form $F_A \defeq dA$.
\end{defn}
%
In general, there is another term, but this term vanishes because $\mathrm{U}(1)$ is
abelian. One thing to note is that $F_A$ actually descends to the base manifold $X$,
i.e. $dA \in \Omega^2_P$ is actually the pullback of some $2$-form in $\Omega^2_X$.
Because of this, we often use $F_A$ to refer to both the $2$-form on $P$ and the
$2$-form on $X$, though technically we should be calling the $2$-form on $P$ something
like $\pi^*F_A$. In most cases, the distinction will be unnecessary. You can think of
the curvature form $F_A$ a measure of the failure of the existence of an integral
submanifold to the horizontal distribution defined by $A$.  \\

One consequence of $F_A = dA$ descending to a $2$-form on $X$ is that it defines
a \emph{closed} form on $X$, which follows from $d$ commuting with $\pi^*$. Therefore,
it defines a de Rham cohomology class $[F_A] \in H^2_{\mathrm{dR}}(X)$. Surprisingly,
this cohomology class is independent of the connection, and is equal
to $2\pi i$ times the \ib{first Chern class} of $P$. One other thing we will need is a
formula for how the curvature of a connection changes when we vary the connection by a
$1$-form $\eta \in \Omega^1_X$. This formula is given by
\[
F_{A+\eta} = F_A + d\eta
\]
which is another way to see that the cohomology class of $F_A$ is independent
of the choice of $A$. In particular, note that if $\eta$ is closed, $F_{A+\eta} = F_A$.
We now connect this discussion to Hodge theory.
%
\begin{defn}
Let $P \to X$ be a principal $\mathrm{U}(1)$-bundle. The \ib{Yang-Mills functional}
is the map $L : \mathscr{A}(P) \to \R$ defined by
\[
L(A) \defeq \norm{F_A} = \int_X F_A \wedge \star F_A
\]
\end{defn}
%
The Yang-Mills equations come from the Euler-Lagrange equations of the Yang-Mills
functional.
%
\begin{prop}[\ib{The first variation}]
Let $A \in \mathscr{A}(P)$ be a local extremum for the Yang-Mills functional
Then $A$ satisfies
\[
d^*F_A = 0
\]
If $A$ is a \emph{global} minimum of $L$, the we say $A$ is a \ib{Yang-Mills connection}.
The space of such connections is denoted by $\mathscr{A}_{\mathrm{YM}}(P)$.
\end{prop}
%
\begin{proof}
Let $\eta \in \Omega^1_X$. Then if we take a line of connections $A + t\eta$ in
the direction $\eta$, we have that
\[
F_{A+t\eta} = F_A + td\eta
\]
If we compute $L(F_{A+t\eta})$, the term linear in $t$ will be $2(F_A,d\eta)_g$.
So differentiating $\norm{F_{A+t\eta}}$ with respect to $t$ gives $2(d^*F_A,\eta)_g$.
Since $A$ is a local extremum, we must have that $(d^*F_A,\eta)$ vanishes
for all $\eta$, so $d^*F = 0$.
\end{proof}
%
Recall that the curvature $F_A$ of a is always closed, so $dF_A = 0$.
%
\begin{defn}
The \ib{Yang-Mills equations} are
\begin{align*}
dF_A = 0 \\
d^*F_A = 0
\end{align*}
\end{defn}
%
From our earlier discussion, this exactly the condition that the curvature $F_A$
of $A$ is harmonic! It is in this sense why people might say that $U(1)$ Yang-Mills
is ``just Hodge Theory." \\

As an aside, this also relates to physics. Another phrase that is sometimes said
is that $U(1)$ Yang-Mills is ``just electromagnetism." If you know physics,
this Yang-Mills equations are exactly Maxwell's equations in a vacuum. The connection
$A$ is the electromagnetic potential, the curvature $F_A$ is the electromagnetic field,
the equation $dF_A = 0$ is the homogeneous equation, and the equation $d^*F_A$ is the
inhomogeneous equation. It's interesting to see another physical phenomenon (the other
being gravity) being realized as curvature.
%
\section{The Gauge Group}
%
In physics you might have heard something along the lines of ``Maxwell's equations
are gauge invariant." We will make this notion precise and explain what it means.
%
\begin{defn}
Let $P \to X$ be a principal $\mathrm{U}(1)$-bundle. The \ib{gauge group} is the group
\[
\mathscr{G}(P) \defeq \mathrm{Map}(X,\mathrm{U}(1))
\]
where the group operation is pointwise multiplication.
\end{defn}
%
One way to think of $\mathscr{G}(P)$ is as the group of bundle automorphism of $P$,
i.e. $\mathrm{U}(1)$-equivariant maps $\varphi : P \to P$ such that
\[\begin{tikzcd}
P \ar[dr]\ar[rr,"\varphi"] && P \ar[dl]\\
& X
\end{tikzcd}\]
commutes. A function $f : X \to \mathrm{U}(1)$ can be thought of as an automorphism
of $P$ by saying ``rotate the fiber over $x$ counterclockwise by angle $f(x)$."
Because of this, we also refer to elements of the gauge group as
\ib{gauge transformations}. This is again a situation where $\mathrm{U}(1)$ being
abelian simplifies the discussion. If the structure group was nonabelian, then the gauge
group is more complicated -- the group of automorphisms becomes ``twisted" by the
noncommutativity. \\

The gauge group $\mathscr{G}(P)$ has a right action on $\mathscr{A}(P)$ by pullback,
and can be described in relatively concrete terms
%
\begin{prop}
Let $\varphi \in \mathscr{G}(P)$ be a gauge transformation associated to
a map $g_\varphi : X \to \mathrm{U}(1)$ and $A \in \mathrm{A}(P)$ any connection.
Then
\[
\varphi^*A = A + \pi^*g_\varphi^*d\theta
\]
where $d\theta$ is the unfortunately named $1$-form on the circle.
\end{prop}
%
This yields the following observation
%
\begin{prop}
The Yang-Mills functional is \ib{gauge invariant}, i.e. for a connection
$A \in \mathscr{A}(P)$ and a gauge transformation $\varphi \in \mathscr{G}(P)$,
we have
\[
L(A) = L(\varphi^*A)
\]
In particular, if $A$ is a Yang-Mills connection, $\varphi^*A$ is also a Yang-Mills
connection.
\end{prop}
%
\begin{proof}
We compute
\begin{align*}
L(\varphi^*A) &= \int_X F_{\varphi^*A}\wedge\star F_{\varphi^*A} \\
&= \int_X F_{A + \pi^*g_\varphi^*d\theta}\wedge\star
F_{A + \varphi^*g_\varphi^*d\theta} \\
&= \int_X F_A \wedge\star F_A \\
&= L(A)
\end{align*}
where we use the formula for how the curvature changes and fact that $d\theta$ is closed.
\end{proof}
%
Gauge invariant of $L$ suggests that we should only really care about Yang-Mills
connections up to gauge equivalence. This is a common philosophy in physics. If your
equations exhibit a lot of symmetry (in our case, we have an infinite dimensional
group of symmetries!), then you can reduce the dimensionality of your problem to
make finding solutions easier. One way this philosophy is made rigorous is
through symplectic geometry via a process called \ib{symplectic reduction},
which invovles something called a moment map for the group action
(named after angular momentum). Yang-Mills theory is closely related to
this notion of symplectic reduction -- for example, the mapping $A \mapsto F_A$
can be realized as the moment map for the action of $\mathscr{G}(P)$ on $\mathscr{A}(P)$,
and the Yang-Mills functional acts as the norm squared of the moment map. \\
%
\section{The Moduli Space}
%
With the previous discussion in mind, we want to answer the question:
\begin{quest}
What does the space $\mathscr{A}_{\mathrm{YM}}(P) / \mathscr{G}(P)$ look like?
\end{quest}
%
The answer turns out to be surprisingly simple! \\

Recall that the cohomology class  of the curvature form $F_A$ is independent
of our choice of $A$. From Hodge theory, we know that there exists a unique
$2$-form $\omega$ minimizing the norm in the class $[F_A]$. The Yang-Mills connections
will be exactly the connections $A$ satisfying $F_A = \omega$. Furthermore,
recall that adding a closed form $\eta \in \Omega^1_X$ to $A$ does not change
the curvature form, which tells us that the space $\mathscr{A}_{\mathrm{YM}}(P)$
is an affine space over the vector space $Z^1_X$ of closed $1$-forms on $X$. Furthermore,
the gauge group acts on $Z^1_X$ on the right by
\[
\eta \cdot f = \eta + f^*d\theta
\]
Therefore, if we fix a reference connection $A_0 \in \mathscr{A}_{\mathrm{YM}}(P)$, we
obtain an isomorphism $\mathscr{A}_{\mathrm{YM}}(P) \cong Z^1_X$, and this isomorphism
is $\mathscr{G}(P)$-equivariant. As a result, to identify the space
$\mathscr{A}_{\mathrm{YM}}(P)/\mathscr{G}(P)$, it suffices to identify the quotient
space $Z^1_X/\mathscr{G}(P)$. \\

We will do this in two steps. We first quotient $Z^1_X$ by the connected component of
the identify, denoted $\mathscr{G}^0(P)$, which is a normal subgroup. We then quotient
the resulting space by the action of the quotient group
$\mathscr{G}(P)/\mathscr{G}^0(P)$, which is the group of components of $\mathscr{G}(P)$.
We can identify paths in $\mathscr{G}(P)$ as homotopies between maps
$X \to \mathrm{U}(1)$, so the identify component is the group of all
nullhomotopic maps $X \to \mathrm{U}(1)$, while the group of components is the
group of homotopy classes of maps. \\

For the first step, let $f : X \to \mathrm{U}(1)$ be a nullhomotopic map. Since
it is nullhomotpic, we can lift $f$ to a function $\widetilde{f} : X \to \R$,
which satisfies $e^{i\widetilde{f}} = f$. The action of $f$ on a closed
form $\eta$ is
\[
\eta \cdot f = \eta + f^*d\theta = \eta + \widetilde{f}^*dx = \eta + d\widetilde{f}
\]
which follows from the fact that $d\theta$ pulls back to $dx$ along the
exponential map $\R \to \mathrm{U}(1)$. Since any function $h : X \to \R$
descends to a nullhomotopic map $e^h : X \to \mathrm{U}(1)$, this tells us that
the orbit of a closed form $\eta$ under the action of $\mathscr{G}^0(P)$ is
exactly the cohomology class $[\eta]$, so the quotient $Z^1_X/\mathscr{G}^0(P)$ is
$H^1_{\mathrm{dR}}(X) = H^1(X,\R)$. \\

For the second step, we use the fact that $\mathrm{U}(1)$ is a $K(\Z,1)$, which
tells us that homotopy classes of maps $X \to \mathrm{U}(1)$ are in bijection
with the integral cohomology group $H^1(X,\Z)$. Quotienting by this group
gives us that $Z^1_X / \mathscr{G}(P) = H^1(X,\R)/H^1(X,\Z)$, which is a torus
of dimension $b_1(X) \defeq \dim H^1(X,\R)$. In other words, the space of
solutions to Maxwell's equations (up to gauge equivalence) on a closed manifold
is a torus!
%
\section{Final Remarks}
%
The Yang-Mills equations have found many uses in mathematics, and are beautifully
interdisciplinary, attracting the attention of mathematicians working in algebraic
geometry, symplectic geometry, mathematical physics, partial differential equations,
and four manifold topology. \\

On the analysis front, the Yang-Mills equations have provided many interesting
problems. The space of connections and the gauge group are both infinite dimensional
spaces, which requires one to be careful when working with these spaces, especially
with issues of convergence. Much of the analytic underpinnings of gauge theory were
developed by Karen Uhlenbeck --  notably in her ``Uhlenbeck compactness" result, which
allows one to deduce convergence in situations of bounded curvature. \\

Another fascinating development was done by Atiyah and Bott in their paper
\emph{The Yang-Mills Equations Over Riemann Surfaces}. By studying Yang-Mills
connections on principal bundles over Riemann surfaces with structure group
$\mathrm{U}(n)$, Atiyah and Bott observed that the theory looked a lot like
another topic of interest -- the moduli space of holomorphic bundles. Indeed,
the Yang-Mills equations ended up being the ``symplectic" perpspective of
an algebro-geometric problem, and parallels the relationship between a Geometric
Invariant Theory (GIT) quotient and a symplectic quotient. Using this relationship,
they used Morse theoretic ideas to compute the cohomology of the moduli spaces
of holomorphic bundles. \\

The ideas in this paper were expanded upon in further years. In one direction,
Uhlenbeck and Yau used these ideas to equated the existence of special metrics
on holomorphic vector bundles called \ib{Hermitian Yang-Mills metrics} (or
\ib{Hermitian-Einstein metrics}) to a stability condition coming from GIT. This
has been further built upon in recent years, where the existence of K\"ahler-Einstein
metrics on Fano varieties has been related to something called $K$-\ib{stability},
which was inspired by the stability criterion in the Hermitian-Yang-Mills case. In
another direction, Hitchin and Simpson expanded these ideas to study objects called
\ib{Higgs bundles}, and the work in this direction has branched out into fields like
Hodge theory, integrable systems, and mirror symmetry. \\

Another subject revolutionized by the Yang-Mills equations was four dimensional
topology. By studying a modified version of the Yang-Mills equations with structure
group $\mathrm{SU}(2)$ over a smooth four manifold, Donaldson was able to prove his
famous theorem about intersection forms of four manifolds. Using results of
Taubes and Uhlenbeck, Donaldson showed that on a simply connected smooth four manifold
$X$ with definite intersection form, the space of solutions to these equations
could be used to construct a cobordism from $X$ to a disjoint union of copies
of $\CP^2$. Then using the fact that the intersection form (up to integer equivalence)
is a cobordism invariant, this allowed him to conclude that any such smooth manifold
must have an intersection form that is diagonalizable over the integers. This was
exremely surprising, and combined with results of Freedman, proved the existence of
many topological four manifolds that admit no smooth structure.
%
\end{document}