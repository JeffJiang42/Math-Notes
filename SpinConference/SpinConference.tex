\documentclass[psamsfonts]{amsart}
%
%-------Packages---------
%
\usepackage[h margin=1 in, v margin=1 in]{geometry}
\usepackage{amssymb,amsfonts}
\usepackage[all,arc]{xy}
\usepackage{tikz-cd}
\usepackage{enumerate}
\usepackage{mathrsfs}
\usepackage{amsthm}
\usepackage{mathpazo}
\usepackage{yfonts}
\usepackage{enumitem}
\usepackage{mathrsfs}
\usepackage{fourier-orns}
\usepackage[all]{xy}
\usepackage{hyperref}
\usepackage{cite}
\usepackage{url}
\usepackage{mathtools}
\usepackage{graphicx}
\usepackage{pdfsync}
\usepackage{mathdots}
\usepackage{calligra}
%
\usepackage{tgpagella}
\usepackage[T1]{fontenc}
%
\usepackage{listings}
\usepackage{color}

\definecolor{dkgreen}{rgb}{0,0.6,0}
\definecolor{gray}{rgb}{0.5,0.5,0.5}
\definecolor{mauve}{rgb}{0.58,0,0.82}

\lstset{frame=tb,
  language=Matlab,
  aboveskip=3mm,
  belowskip=3mm,
  showstringspaces=false,
  columns=flexible,
  basicstyle={\small\ttfamily},
  numbers=none,
  numberstyle=\tiny\color{gray},
  keywordstyle=\color{blue},
  commentstyle=\color{dkgreen},
  stringstyle=\color{mauve},
  breaklines=true,
  breakatwhitespace=true,
  tabsize=3
  }
%
%--------Theorem Environments--------
%
\newtheorem{thm}{Theorem}[section]
\newtheorem*{thm*}{Theorem}
\newtheorem{cor}[thm]{Corollary}
\newtheorem{prop}[thm]{Proposition}
\newtheorem{lem}[thm]{Lemma}
\newtheorem*{lem*}{Lemma}
\newtheorem{conj}[thm]{Conjecture}
\newtheorem{quest}[thm]{Question}
%
\theoremstyle{definition}
\newtheorem{defn}[thm]{Definition}
\newtheorem*{defn*}{Definition}
\newtheorem{defns}[thm]{Definitions}
\newtheorem{con}[thm]{Construction}
\newtheorem{exmp}[thm]{Example}
\newtheorem{exmps}[thm]{Examples}
\newtheorem{notn}[thm]{Notation}
\newtheorem{notns}[thm]{Notations}
\newtheorem{addm}[thm]{Addendum}
\newtheorem{exer}[thm]{Exercise}
%
\theoremstyle{remark}
\newtheorem{rem}[thm]{Remark}
\newtheorem*{claim}{Claim}
\newtheorem*{aside*}{Aside}
\newtheorem*{rem*}{Remark}
\newtheorem*{hint*}{Hint}
\newtheorem*{note}{Note}
\newtheorem{rems}[thm]{Remarks}
\newtheorem{warn}[thm]{Warning}
\newtheorem{sch}[thm]{Scholium}
%
%--------Macros--------
\renewcommand{\qedsymbol}{$\blacksquare$}
\renewcommand{\hom}{\mathsf{Hom}}
\renewcommand{\emptyset}{\varnothing}
\renewcommand{\O}{\mathscr{O}}
\newcommand{\R}{\mathbb{R}}
\newcommand{\ib}[1]{\textbf{\textit{#1}}}
\newcommand{\Q}{\mathbb{Q}}
\newcommand{\Z}{\mathbb{Z}}
\newcommand{\N}{\mathbb{N}}
\newcommand{\C}{\mathbb{C}}
\newcommand{\A}{\mathbb{A}}
\newcommand{\F}{\mathbb{F}}
\newcommand{\M}{\mathcal{M}}
\renewcommand{\S}{\mathbb{S}}
\newcommand{\V}{\vec{v}}
\newcommand{\RP}{\mathbb{RP}}
\newcommand{\CP}{\mathbb{CP}}
\newcommand{\B}{\mathcal{B}}
\newcommand{\GL}{\mathsf{GL}}
\newcommand{\SL}{\mathsf{SL}}
\newcommand{\SP}{\mathsf{SP}}
\newcommand{\SO}{\mathsf{SO}}
\newcommand{\SU}{\mathsf{SU}}
\newcommand{\gl}{\mathfrak{gl}}
\newcommand{\g}{\mathfrak{g}}
\newcommand{\inv}{^{-1}}
\newcommand{\bra}[2]{ \left[ #1, #2 \right] }
\newcommand{\ind}{\lambda \in \Lambda}
\newcommand{\set}[1]{\left\lbrace #1 \right\rbrace}
\newcommand{\abs}[1]{\left\lvert#1\right\rvert}
\newcommand{\norm}[1]{\left\lVert#1\right\rVert}
\newcommand{\transv}{\mathrel{\text{\tpitchfork}}}
\newcommand{\enumbreak}{\ \\ \vspace{-\baselineskip}}
\let\oldexists\exists
\renewcommand\exists{\oldexists~}
\let\oldL\L
\renewcommand\L{\mathfrak{L}}
\makeatletter
\newcommand{\tpitchfork}{%
  \vbox{
    \baselineskip\z@skip
    \lineskip-.52ex
    \lineskiplimit\maxdimen
    \m@th
    \ialign{##\crcr\hidewidth\smash{$-$}\hidewidth\crcr$\pitchfork$\crcr}
  }%
}
\makeatother
\newcommand{\bd}{\partial}
\newcommand{\lang}{\begin{picture}(5,7)
\put(1.1,2.5){\rotatebox{45}{\line(1,0){6.0}}}
\put(1.1,2.5){\rotatebox{315}{\line(1,0){6.0}}}
\end{picture}}
\newcommand{\rang}{\begin{picture}(5,7)
\put(.1,2.5){\rotatebox{135}{\line(1,0){6.0}}}
\put(.1,2.5){\rotatebox{225}{\line(1,0){6.0}}}
\end{picture}}
\DeclareMathOperator{\id}{id}
\DeclareMathOperator{\im}{Im}
\DeclareMathOperator{\grap}{graph}
\DeclareMathOperator{\codim}{codim}
\DeclareMathOperator{\coker}{coker}
\DeclareMathOperator{\supp}{supp}
\DeclareMathOperator{\inter}{Int}
\DeclareMathOperator{\sign}{sign}
\DeclareMathOperator{\sgn}{sgn}
\DeclareMathOperator{\indx}{ind}
\DeclareMathOperator{\alt}{Alt}
\DeclareMathOperator{\Aut}{Aut}
\DeclareMathOperator{\trace}{trace}
\DeclareMathOperator{\ad}{ad}
\DeclareMathOperator{\End}{End}
\DeclareMathOperator{\Ad}{Ad}
\DeclareMathOperator{\Lie}{Lie}
\DeclareMathOperator{\spn}{span}
\DeclareMathOperator{\dv}{div}
\DeclareMathOperator{\grad}{grad}
\DeclareMathOperator{\sheafhom}{\mathscr{H}\text{\kern -3pt {\calligra\large om}}\,}
\newcommand*\myhrulefill{%
   \leavevmode\leaders\hrule depth-2pt height 2.4pt\hfill\kern0pt}
\newcommand\niceending[1]{%
  \begin{center}%
    \LARGE \myhrulefill \hspace{0.2cm} #1 \hspace{0.2cm} \myhrulefill%
  \end{center}}
\newcommand*\sectionend{\niceending{\decofourleft\decofourright}}
\newcommand*\subsectionend{\niceending{\decosix}}
\def\upint{\mathchoice%
    {\mkern13mu\overline{\vphantom{\intop}\mkern7mu}\mkern-20mu}%
    {\mkern7mu\overline{\vphantom{\intop}\mkern7mu}\mkern-14mu}%
    {\mkern7mu\overline{\vphantom{\intop}\mkern7mu}\mkern-14mu}%
    {\mkern7mu\overline{\vphantom{\intop}\mkern7mu}\mkern-14mu}%
  \int}
\def\lowint{\mkern3mu\underline{\vphantom{\intop}\mkern7mu}\mkern-10mu\int}
%
%--------Hypersetup--------
%
\hypersetup{
    colorlinks,
    citecolor=black,
    filecolor=black,
    linkcolor=blue,
    urlcolor=black
}
%
%--------Solution--------
%
\newenvironment{solution}
  {\begin{proof}[Solution]}
  {\end{proof}}
%
%--------Graphics--------
%
%\graphicspath{ {images/} }

\begin{document}
\author{Jeffrey Jiang}
\title{Spin Geometry Conference Course}
\maketitle
%
\setcounter{section}{1}
\section*{Week 1}
%
\begin{exer}
Prove $SL_n(\R)$ and $O(n)$ are manifolds
\end{exer}
%
\begin{exer}
What is the ``shape" of $SL_2(\R)$?
\end{exer}
%
\begin{exer}
Prove that
$$O(2) = \set{\begin{pmatrix}
\cos\theta -\sin\theta \\
\sin\theta \cos\theta
\end{pmatrix}} \cup \set{\begin{pmatrix}
\cos\theta \sin\theta \\
\sin\theta -\cos\theta 
\end{pmatrix}}$$
The first set consists of rotations and the second set consists of reflections. Which rotations commute? Which reflections commute? Do reflections commute with reflections?
\end{exer}
%
\begin{exer}
Investigate $O(3)$. What is it's ``shape?"
\end{exer}
%
\setcounter{section}{2}
%
\setcounter{thm}{0}
%
\section*{Week 2}
\begin{exer}
What is the derivative of $\det$?
\end{exer}
\begin{exer}
Explore the exponetial map $\mathfrak{sl}_2(\R) \to SL_2(\R)$
\end{exer}
\begin{exer}
Prove that every element of $O(n)$ can be written as the composition of at most $n$ reflections about hyperplanes in $\R^n$.
\end{exer}
%
\begin{proof}
We do this by induction. For $n = 1$, this is obvious, since $O(1) \cong \pm 1$. The assuming that this holds for dimension $n-1$, Let $A \in O(n)$, and let $v \in \R$. We want to construct a hyperplane reflection $R$ such that $RAv = v$, which is obtained by taking $R$ to be the hyperplane reflection about the bisector of $v$ and $Av$. More explicitly, take $R$ to be the hyperplane reflection about the vector 
$$\frac{Av - v}{\norm{Av - v}} $$
which is given by the equation
$$Rw = w - 2 \frac{\langle Av - v, v \rangle}{\langle Av - v, Av - v \rangle}(Av - v) $$
Computing its action on $v$, we get 
\begin{align*}
Rv &= v - 2 \frac{\langle Av - v, v \rangle}{\langle Av - v, Av - v \rangle}(Av - v) \\
&= v - \frac{2\langle Av, v \rangle - 2\langle v, v \rangle}{2 \langle v, v \rangle - 2 \langle Av, v \rangle}(Av - v) \\
&= v + Av -v \\
&= Av
\end{align*}
Then since $R$ is its own inverse (being a reflection), we have that $RAv = v$, so $RAv$ fixes $v$ and its orthogonal complement.
\end{proof}
TODO insert motivation of $A^\pm_n$
\begin{defn}
Define $A^\pm_n$ to be the unital algebra generated by $\R^n$ such that $\xi^2 = \pm 1$. and $\xi\eta = ?~\eta\xi$. Determine the sign of $\eta\xi$. Explore these algebras. Find $A\pm_1$, $A^\pm_2 \ldots$. What are they isomorphic to? Can you identify $O(n)$ as a subgroup?
\end{defn}
%
\setcounter{section}{3}
%
\section*{Week 3}
\begin{exer}
Classify the algebras $A^+_n$ (we messed these up week 2).
\end{exer}
%
\begin{exer}
Prove that 
$$\set{e_{i_1}e_{i_2}\ldots e_{i+k} ~|~ 1 \leq i_1 < i_2 < \ldots < i_k \leq n} $$
is a basis for $A^\pm_n$.
\end{exer}
%
\begin{exer}
Modify the isomorphisms found for $A^-_n$ by choosing $\Z/2\Z$ gradings for the domains and codomains such that the isomorphisms are now isomorphisms as superalgebras.
\end{exer}
%
\begin{exer}
Construct a tensor product for super vector spaces and superalgebras.
\end{exer}
\begin{exer}
Explore the ``shape" of the group 
$$G = \langle v ~|~ \norm{v} = 1 \rangle \subset (A_n^-)^\times $$
and the nature of the surjection $G \twoheadrightarrow O(n)$. What is the kernel of this map?
\end{exer}
%
\end{document}