\documentclass[psamsfonts, 11pt]{amsart}
%
%-------Packages---------
%
\usepackage[h margin=1 in, v margin=1 in]{geometry}
\usepackage{amssymb,amsfonts}
\usepackage[all,arc]{xy}
\usepackage{tikz-cd}
\usepackage{enumerate}
\usepackage{mathrsfs}
\usepackage{amsthm}
\usepackage{mathpazo}
\usepackage{yfonts}
\usepackage{enumitem}
\usepackage{mathrsfs}
\usepackage{fourier-orns}
\usepackage[all]{xy}
\usepackage{hyperref}
\usepackage{cite}
\usepackage{url}
\usepackage{mathtools}
\usepackage{graphicx}
\usepackage{pdfsync}
\usepackage{mathdots}
\usepackage{calligra}
\usepackage{import}
\usepackage{xifthen}
\usepackage{pdfpages}
\usepackage{transparent}

\newcommand{\incfig}[2]{%
    \def\svgwidth{\columnwidth}
    \scalebox{#2}{\input{#1.pdf_tex}}
}
%
\usepackage{tgpagella}
\usepackage[T1]{fontenc}
%
\usepackage{listings}
\usepackage{color}

\definecolor{dkgreen}{rgb}{0,0.6,0}
\definecolor{gray}{rgb}{0.5,0.5,0.5}
\definecolor{mauve}{rgb}{0.58,0,0.82}

\lstset{frame=tb,
  language=Matlab,
  aboveskip=3mm,
  belowskip=3mm,
  showstringspaces=false,
  columns=flexible,
  basicstyle={\small\ttfamily},
  numbers=none,
  numberstyle=\tiny\color{gray},
  keywordstyle=\color{blue},
  commentstyle=\color{dkgreen},
  stringstyle=\color{mauve},
  breaklines=true,
  breakatwhitespace=true,
  tabsize=3
  }
%
%--------Theorem Environments--------
%
\newtheorem{thm}{Theorem}[section]
\newtheorem*{thm*}{Theorem}
\newtheorem{cor}[thm]{Corollary}
\newtheorem{prop}[thm]{Proposition}
\newtheorem{lem}[thm]{Lemma}
\newtheorem*{lem*}{Lemma}
\newtheorem{conj}[thm]{Conjecture}
\newtheorem{quest}[thm]{Question}
%
\theoremstyle{definition}
\newtheorem{defn}[thm]{Definition}
\newtheorem*{defn*}{Definition}
\newtheorem{defns}[thm]{Definitions}
\newtheorem{con}[thm]{Construction}
\newtheorem{exmp}[thm]{Example}
\newtheorem{exmps}[thm]{Examples}
\newtheorem{notn}[thm]{Notation}
\newtheorem{notns}[thm]{Notations}
\newtheorem{addm}[thm]{Addendum}
\newtheorem{exer}[thm]{Exercise}
%
\theoremstyle{remark}
\newtheorem{rem}[thm]{Remark}
\newtheorem*{claim}{Claim}
\newtheorem*{aside*}{Aside}
\newtheorem*{rem*}{Remark}
\newtheorem*{hint*}{Hint}
\newtheorem*{note}{Note}
\newtheorem{rems}[thm]{Remarks}
\newtheorem{warn}[thm]{Warning}
\newtheorem{sch}[thm]{Scholium}
%
%--------Macros--------
\renewcommand{\qedsymbol}{$\blacksquare$}
\renewcommand{\sl}{\mathfrak{sl}}
\newcommand{\Bord}{\mathsf{Bord}}
\renewcommand{\hom}{\mathsf{Hom}}
\renewcommand{\emptyset}{\varnothing}
\renewcommand{\O}{\mathscr{O}}
\newcommand{\R}{\mathbb{R}}
\newcommand{\ib}[1]{\textbf{\textit{#1}}}
\newcommand{\Q}{\mathbb{Q}}
\newcommand{\Z}{\mathbb{Z}}
\newcommand{\N}{\mathbb{N}}
\newcommand{\C}{\mathbb{C}}
\newcommand{\A}{\mathbb{A}}
\newcommand{\F}{\mathbb{F}}
\newcommand{\M}{\mathcal{M}}
\renewcommand{\S}{\mathbb{S}}
\newcommand{\V}{\vec{v}}
\newcommand{\RP}{\mathbb{RP}}
\newcommand{\CP}{\mathbb{CP}}
\newcommand{\B}{\mathcal{B}}
\newcommand{\GL}{\mathsf{GL}}
\newcommand{\SL}{\mathsf{SL}}
\newcommand{\SP}{\mathsf{SP}}
\newcommand{\SO}{\mathsf{SO}}
\newcommand{\SU}{\mathsf{SU}}
\newcommand{\gl}{\mathfrak{gl}}
\newcommand{\g}{\mathfrak{g}}
\newcommand{\Bun}{\mathsf{Bun}}
\newcommand{\inv}{^{-1}}
\newcommand{\bra}[2]{ \left[ #1, #2 \right] }
\newcommand{\set}[1]{\left\lbrace #1 \right\rbrace}
\newcommand{\abs}[1]{\left\lvert#1\right\rvert}
\newcommand{\norm}[1]{\left\lVert#1\right\rVert}
\newcommand{\transv}{\mathrel{\text{\tpitchfork}}}
\newcommand{\enumbreak}{\ \\ \vspace{-\baselineskip}}
\let\oldexists\exists
\renewcommand\exists{\oldexists~}
\let\oldL\L
\renewcommand\L{\mathfrak{L}}
\makeatletter
\newcommand{\tpitchfork}{%
  \vbox{
    \baselineskip\z@skip
    \lineskip-.52ex
    \lineskiplimit\maxdimen
    \m@th
    \ialign{##\crcr\hidewidth\smash{$-$}\hidewidth\crcr$\pitchfork$\crcr}
  }%
}
\makeatother
\newcommand{\bd}{\partial}
\newcommand{\lang}{\begin{picture}(5,7)
\put(1.1,2.5){\rotatebox{45}{\line(1,0){6.0}}}
\put(1.1,2.5){\rotatebox{315}{\line(1,0){6.0}}}
\end{picture}}
\newcommand{\rang}{\begin{picture}(5,7)
\put(.1,2.5){\rotatebox{135}{\line(1,0){6.0}}}
\put(.1,2.5){\rotatebox{225}{\line(1,0){6.0}}}
\end{picture}}
\DeclareMathOperator{\id}{id}
\DeclareMathOperator{\im}{Im}
\DeclareMathOperator{\codim}{codim}
\DeclareMathOperator{\coker}{coker}
\DeclareMathOperator{\supp}{supp}
\DeclareMathOperator{\inter}{Int}
\DeclareMathOperator{\sign}{sign}
\DeclareMathOperator{\sgn}{sgn}
\DeclareMathOperator{\indx}{ind}
\DeclareMathOperator{\alt}{Alt}
\DeclareMathOperator{\Aut}{Aut}
\DeclareMathOperator{\trace}{trace}
\DeclareMathOperator{\ad}{ad}
\DeclareMathOperator{\End}{End}
\DeclareMathOperator{\Ad}{Ad}
\DeclareMathOperator{\Lie}{Lie}
\DeclareMathOperator{\spn}{span}
\DeclareMathOperator{\dv}{div}
\DeclareMathOperator{\grad}{grad}
\DeclareMathOperator{\Sym}{Sym}
\DeclareMathOperator{\sheafhom}{\mathscr{H}\text{\kern -3pt {\calligra\large om}}\,}
\newcommand*\myhrulefill{%
   \leavevmode\leaders\hrule depth-2pt height 2.4pt\hfill\kern0pt}
\newcommand\niceending[1]{%
  \begin{center}%
    \LARGE \myhrulefill \hspace{0.2cm} #1 \hspace{0.2cm} \myhrulefill%
  \end{center}}
\newcommand*\sectionend{\niceending{\decofourleft\decofourright}}
\newcommand*\subsectionend{\niceending{\decosix}}
\def\upint{\mathchoice%
    {\mkern13mu\overline{\vphantom{\intop}\mkern7mu}\mkern-20mu}%
    {\mkern7mu\overline{\vphantom{\intop}\mkern7mu}\mkern-14mu}%
    {\mkern7mu\overline{\vphantom{\intop}\mkern7mu}\mkern-14mu}%
    {\mkern7mu\overline{\vphantom{\intop}\mkern7mu}\mkern-14mu}%
  \int}
\def\lowint{\mkern3mu\underline{\vphantom{\intop}\mkern7mu}\mkern-10mu\int}
%
%--------Hypersetup--------
%
\hypersetup{
    colorlinks,
    citecolor=black,
    filecolor=black,
    linkcolor=blue,
    urlcolor=blacksquare
}
%
%--------Solution--------
%
\newenvironment{solution}
  {\begin{proof}[Solution]}
  {\end{proof}}
%
%--------Graphics--------
%
%\graphicspath{ {images/} }

\begin{document}
%
\author{Jeffrey Jiang}
%
\title{Representation Theory and Topological Quantum Field Theories}
%
\maketitle
%
\section{Bordism and TQFTs}
%
\begin{defn}
Let $X$ and $Y$ be $n$-dimensional closed manifolds (i.e. compact and without boundary).
A \ib{bordism} from $X$ to $Y$ is an $n+1$ dimensional manifold $M$ such that
the boundary is diffeomorphic to the disjoint union $X \coprod Y$. The
\ib{Bordism category} $\Bord_n$ is the category where the objects are closed
$n$-dimensional manifolds, and the morphisms are bordisms between them.
\end{defn}
%
\begin{figure}[ht]
  \centering
  \incfig{bordism}{0.20}
  \caption{An example of a bordism $S^1 \coprod S^1 \to S^1$.}
\end{figure}
%
We make a distinction between the \emph{incoming} and \emph{outgoing} manifolds.
In the bordism drawn above, we think of the two circles as the incoming manifolds,
and single circle as the outgoing manifold. One way to think of a bordism is
the time evolution of the incoming manifold to the outgoing one (think of the
figure above extending outwards from the two circles and growing towards the
single circle over some fixed amount of time). \\

The category of bordisms comes with a natural product -- the disjoint union $\coprod$,
where given two $n$-manifolds $X$ and $Y$, we obtain a new $n$-manifold $X \coprod Y$.
This operation is \emph{symmetric}, i.e. there is a natural isomorphism
$X \coprod Y \to Y \coprod X$, and gives the set of objects the structure of a
commutative monoid. We say that $(\Bord_n, \coprod)$ is a \ib{symmetric monoidal
category}. Another important example of a symmetric monoidal category is the
category $\mathsf{Vect}_\F$ of finite dimensional $\F$-vector spaces, where the
operation is the tensor product $\otimes$.
%
\begin{defn}
An $n+1$-dimensional \ib{topological quantum field theory (TQFT)} is a symmetric
monoidal functor $Z : (\Bord_n, \coprod) \to (\mathsf{Vect}_\C, \otimes)$, i.e. a
functor $\Bord_n \to \mathsf{Vect}_\C$ satisfying:
\begin{enumerate}
  \item $Z(\emptyset) = \C$
  \item $Z(X \coprod Y) = Z(X) \otimes Z(Y)$
\end{enumerate}
\end{defn}
%
Note that the conditions on $Z$ essentially state that it is a homomorphism of symmetric
monoidal categories, hence the name. Note that the empty set is vacuously a manifold of
\emph{any} dimension. While this sounds silly at first, it ends up being very important,
as it is the unit element under the disjoint union. In addition, we can interpret a
closed manifold $M$ as a bordism $\emptyset \to \emptyset$. Applying $Z$ to $M$,
we get a linear map $\C \to \C$, which is just multiplication by a complex number
$\lambda$. The number is called the \ib{partition function} of $M$. \\

While the definition might seem somewhat abstract, it is very grounded in physical
motivation. The incoming manifold of a bordism can be thought of as a space or system at
an initial time, and the outgoing manifold can be thought of as the end state after
undergoing the time evolution specified by the bordism. The functor
$Z$ then assigns to the initial and final states a state space. The fact
that the state space of a disjoint union is the tensor product of the state spaces
also matches the physical model.
%
\section{Dijkgraaf-Witten Theory With Finite Gauge Group}
%
We now define a specific TQFT, which is a toy model developed by Dijkgraaf and Witten.
Fix a finite group $G$. For a fixed manifold $M$, let $\Bun_G(M)$ denote the
category of principal $G$-bundles over $M$. Any morphism $P \to Q$ of principal
$G$-bundles is an isomorphism, so this category is a \ib{groupoid}. For any given
groupoid $\mathcal{G}$, we let $\pi_0(\mathcal{G})$ denote the set of isomorphism
classes of objects in $\mathcal{G}$. Given a basepoint
$x \in M$, we let $\Bun_G(M)$ denote the category of pointed $G$-bundles over $M$,
which are pairs $(P,p)$ where $P \to M$ is a principal $G$-bundle, and $p\in P$ is
an element of the fiber $P_x$ over $x$.
%
\begin{prop}
There is a bijective correspondence
\[
\hom(\pi_1(M, x), G) \longleftrightarrow \Bun_G(M, x)
\]
\end{prop}
%
\begin{proof}
Given a pointed bundle $(P,p) \to (M,x)$, we obtain a map $\varphi : \pi_1(M,x) \to G$
as follows: For a homotopy class of a loop $\sigma : I \to M$ based at $x$, we
can lift $\sigma$ to a path $\tilde{\sigma}$ on $P$ starting at $p$. Then the endpoint
$\tilde{\sigma}(1)$ is an element of the fiber $P_x$, so it can be uniquely written
as $p \cdot g$ for some $g \in G$. Then defining $\varphi(\sigma) = g$ gives the
desired homomorphism (though one should check that this construction is well defined
on homotopy classes) called the \ib{holonomy} of the bundle. This map is also
called the \ib{monodromy}. In the other direction, given a homomorphism
$\varphi : \pi_1(M,x) \to G$, we want to construct a pointed bundle
$(P,p) \to (M, x)$ with holonomy $\varphi$. The manifold $M$ admits a universal
cover $(\widetilde{M}, \tilde{x})$, which is a principal $\pi_1(M, x)$-bundle over $M$.
The homomorphism $\varphi$ induces a left action of $\pi_1(M, x)$ on $G$, so we
construct the associated bundle
\[
P = \widetilde{M} \times_{\pi_1(M, x)} G
\]
where we choose the basepoint of $P$ to be $p = [\tilde{x}, e]$. It is easy to
verify that the holonomy of this bundle is $\varphi$, and it is also easy to verify
that the holonomy of this bundle is indeed $\varphi$, giving us the desired bijection.
\end{proof}
%
The group $G$ acts on the category $\Bun_G(M,x)$ by permuting basepoints. Given
a pointed bundle $(P, p) \to (M, x)$ and a group element $g \in G$, the action of
$g$ on $(P,p)$ is the bundle $(P, p \cdot g)$. In addition, $g$ acts on maps
$(P,p) \to (Q,q)$ by precomposition. Taking the quotient by this group action,
we obtain the groupoid $\Bun_G(M)$, since quotienting by the group action amounts
to forgetting the basepoint. It is also useful to see the group action from
the perspective of homomorphisms $\varphi : \pi_1(M, x) \to G$, which we can determine
by studying how the holonomy of a bundle changes when we permute its basepoint.
Let $\varphi$ be the holonomy of a bundle $(P,p) \to (M, x)$, and let $\gamma : I \to M$
be a loop based at $x$. Then let $\widetilde{\gamma}$ denote its lift to a loop
based at $p$. By the uniqueness of path lifting, the lift of $\gamma$ to the
a loop based at $p\cdot g$ is the loop
$\widetilde{\gamma}_g(t) = \widetilde{\gamma(t)} \cdot g$. We the compute
%
\begin{align*}
\widetilde{\gamma}_g(1) &= \widetilde{\gamma}(1) \cdot g \\
&= p \cdot \varphi[\gamma] \cdot g \\
&= p \cdot g \cdot g\inv \cdot \varphi[\gamma] \cdot g
\end{align*}
%
So the holonomy $\varphi$ transforms under the group action by conjugation,
i.e. the holonomy $\varphi_g$ of $(P, p\cdot g) \to (M, x)$ is
$g\inv \cdot \varphi \cdot g$. \\

We now have the necessary tools to define the TQFT. Given a closed $n$-manifold $M$,
define $Z(M) = \C_G(M)$, where $\C_G(M)$ denotes the vector space generated by complex
valued functions on $\pi_0(\Bun_G(M))$. Then given a bordism $X$
between manifolds $M$, and $N$, we need to produce a map $Z(X) : \C_G(M) \to \C_G(N)$.
We do so as follows. We have inclusions $M \hookrightarrow X$ and $N \hookrightarrow X$
of $M$ and $N$ into $\partial X$, giving us the diagram
\[\begin{tikzcd}
& X \\
M\ar[ur] && N\ar[ul]
\end{tikzcd}\]
the inclusion maps induce maps $\Bun_G(X) \to \Bun_G(M)$ and $\Bun_G(X)\to \Bun(Y)$
by pullback (also called restriction), where we map a bundle $P \to X$ to the
pullback bundle $P\vert_M$ and $P\vert_N$ respectively, where we identify $M$
and $N$ with the images under the inclusion into $X$, giving the diagram
\[\begin{tikzcd}
& \Bun_G(X) \ar[dl]\ar[dr] \\
\Bun_G(M) && \Bun_G(N)
\end{tikzcd}\]
Then given a function $f \in \C_G(M)$ on principal bundles over $M$, we need to
produce a function $Z(X)(f) \in \C_G(N)$. To do so, we need to define one more
object. Let $Q \to N$ be a principal bundle over $N$. We then construct a
groupoid $\mathcal{G}_Q$, where the objects are pairs $(P, \varphi)$, where
$P \to X$ is a principal bundle, and $\varphi$ is an isomorphism $P\vert_N \to Q$,
and the morphisms from $(P, \varphi) \to (P', \varphi')$ are bundle morphisms
$\psi : P \to P'$ such that the diagram
\[\begin{tikzcd}
P\vert_N \ar[dr, "\varphi"']\ar[rr, "\psi"] && P'\vert_N \ar[dl, "\varphi'"]\\
& Q
\end{tikzcd}\]
commutes, where we pullback the map $\psi$ using the inclusion map $N \hookrightarrow X$.
We now construct the linear map $Z(X) : \C_G(M) \to \C_G(N)$. Let $f \in \C_G(M)$,
Then define $Z(X)(f) \in \C_G(N)$ by
\[
Z(X)(f)(Q) = \sum_{(P,\varphi) \in \pi_0(\mathcal{G}_Q)}
\frac{f(P\vert_M)}{|\Aut(P,\varphi)|}
\]
While a bit opaque, you can think of this as a sort of categorified Fourier transform.
Given a function over $M$, we pull it back to a function over $X$, convolve with
some sort of kernel (where we sum instead of integrate because everything is finite), and
then push down to $N$. Such push-pull constructions arise often as a sort of generalized
Fourier transform.
%
\section{Dijkgraaf-Witten Theory in Two Dimensions}
%
We now restrict our attention to the $2$-dimensional version of the TQFT $Z$
we constructed above. First, we recall some of the details of the classification of
$2d$-TQFTs, which are in bijection with \ib{commutative Frobenius algebras}. It is well
known that the only connected closed $1$-manifold is the circle $S^1$, up to
diffeomorphism. Therefore, all the objects in the category $\Bord_n$ (again, up to
diffeomorphism) are just disjoint unions of a finite number of circles. Using the fact
that $Z(M \coprod N) = Z(M) \otimes Z(N)$, this means we can interpret the morphisms
in $\Bord_n$ as operations on $Z(S^1)$, which will form the underlying vector space
of our commuative Frobenius algebra. To classify the algebra, it suffice to know the
following list of bordisms and the operations they induce on $Z(S^1)$
\begin{enumerate}
  \item The \ib{pair of pants} $P$ is a bordism $S^1 \coprod S^1 \to S^1$. Applying
  the functor $Z$, we get a \ib{multiplication}
  $Z(P) : Z(S^1) \otimes Z(S^1) \to Z(S^1)$.
  \begin{figure}[ht]
      \centering
      \incfig{pair_of_pants}{0.15}
      \caption{A pair of pants}
  \end{figure}
  \item We can flip the pair of pants around to get a bordism $S^1 \to S^1 \coprod S^1$,
  which then induces a map $Z(S^1) \to Z(S^1) \otimes Z(S^1)$ called
  \ib{comultiplication}.
  \begin{figure}[ht]
      \centering
      \incfig{flipped_pair_of_pants}{0.15}
      \caption{The pair of pants, flipped around}
  \end{figure}
  \item The \ib{cap} $C$ is a bordism $S^1 \to \emptyset$. Applying the functor $Z$, we
  get a linear map $Z(C) : Z(S^1) \to Z(\emptyset) = \C$, called the \ib{trace}.
  \begin{figure}[ht]
      \centering
      \incfig{cap}{0.15}
      \caption{A cap}
  \end{figure}
  \item By flipping the cap around, we get a bordism $\emptyset \to \S^1$, called
  the \ib{cocap}, which gives us a map $\C \to Z(S^1)$. The identity element then
  determines the identity of $Z(S^1)$ under the multiplication by the pair of pants.
  \begin{figure}[ht]
      \centering
      \incfig{cocap}{0.15}
      \caption{A cocap}
  \end{figure}
\end{enumerate}
%
The operations induced by the bordisms above are essentially the defining ingredients
for a Frobenius algebra (along with some relations they satisfy, which we will
not prove).
\iffalse
An example of a Frobenius algebra is the group algebra $\C[G]$ for a
finite group $g$. The maps are as follows:
\begin{enumerate}
  \item The multiplication is the usual one in the group alebra.
  \item the comultiplication is the map $g \mapsto g \otimes g$.
  \item
\end{enumerate}
\fi
%
\end{document}