\documentclass[psamsfonts, 12pt]{amsart}
%
%-------Packages---------
%
\usepackage[h margin=1 in, v margin=1 in]{geometry}
\usepackage{amssymb,amsfonts}
\usepackage{amsmath}
\usepackage{accents}
\usepackage[all,arc]{xy}
\usepackage{tikz-cd}
\usepackage{enumerate}
\usepackage{mathrsfs}
\usepackage{amsthm}
\usepackage{mathpazo}
\usepackage{float}
%\usepackage[backend=biber]{biblatex}
%\addbibresource{bibliography.bib}
%\usepackage{charter} %another font
%\usepackage{eulervm} %Vakil font
\usepackage{yfonts}
\usepackage{mathtools}
\usepackage{enumitem}
\usepackage{mathrsfs}
\usepackage{fourier-orns}
\usepackage[all]{xy}
\usepackage{hyperref}
\usepackage{url}
\usepackage{mathtools}
\usepackage{graphicx}
\usepackage{pdfsync}
\usepackage{mathdots}
\usepackage{calligra}
\usepackage{import}
\usepackage{xifthen}
\usepackage{pdfpages}
\usepackage{transparent}

\usepackage{tgpagella}
\usepackage[T1]{fontenc}
%
\usepackage{listings}
\usepackage{color}

\definecolor{dkgreen}{rgb}{0,0.6,0}
\definecolor{gray}{rgb}{0.5,0.5,0.5}
\definecolor{mauve}{rgb}{0.58,0,0.82}

\lstset{frame=tb,
  language=Matlab,
  aboveskip=3mm,
  belowskip=3mm,
  showstringspaces=false,
  columns=flexible,
  basicstyle={\small\ttfamily},
  numbers=none,
  numberstyle=\tiny\color{gray},
  keywordstyle=\color{blue},
  commentstyle=\color{dkgreen},
  stringstyle=\color{mauve},
  breaklines=true,
  breakatwhitespace=true,
  tabsize=3
  }
%
%--------Theorem Environments--------
%
\newtheorem{thm}{Theorem}[section]
\newtheorem*{thm*}{Theorem}
\newtheorem{cor}[thm]{Corollary}
\newtheorem{prop}[thm]{Proposition}
\newtheorem{lem}[thm]{Lemma}
\newtheorem*{lem*}{Lemma}
\newtheorem{conj}[thm]{Conjecture}
\newtheorem{quest}[thm]{Question}
%
\theoremstyle{definition}
\newtheorem{defn}[thm]{Definition}
\newtheorem*{defn*}{Definition}
\newtheorem{defns}[thm]{Definitions}
\newtheorem{con}[thm]{Construction}
\newtheorem{exmp}[thm]{Example}
\newtheorem{exmps}[thm]{Examples}
\newtheorem{notn}[thm]{Notation}
\newtheorem{notns}[thm]{Notations}
\newtheorem{addm}[thm]{Addendum}
\newtheorem{exer}[thm]{Exercise}
%
\theoremstyle{remark}
\newtheorem{rem}[thm]{Remark}
\newtheorem*{claim}{Claim}
\newtheorem*{aside*}{Aside}
\newtheorem*{rem*}{Remark}
\newtheorem*{hint*}{Hint}
\newtheorem*{note}{Note}
\newtheorem{rems}[thm]{Remarks}
\newtheorem{warn}[thm]{Warning}
\newtheorem{sch}[thm]{Scholium}
%
%--------Macros--------
\renewcommand{\qedsymbol}{$\blacksquare$}
\renewcommand{\sl}{\mathfrak{sl}}
\newcommand{\Bord}{\mathsf{Bord}}
\renewcommand{\hom}{\mathsf{Hom}}
\renewcommand{\emptyset}{\varnothing}
\renewcommand{\O}{\mathcal{O}}
\newcommand{\R}{\mathbb{R}}
\newcommand{\ib}[1]{\textbf{\textit{#1}}}
\newcommand{\Q}{\mathbb{Q}}
\newcommand{\Z}{\mathbb{Z}}
\newcommand{\N}{\mathbb{N}}
\newcommand{\C}{\mathbb{C}}
\newcommand{\A}{\mathbb{A}}
\newcommand{\F}{\mathbb{F}}
\newcommand{\M}{\mathcal{M}}
\newcommand{\dbar}{\overline{\partial}}
\newcommand{\zbar}{\overline{z}}
\renewcommand{\S}{\mathbb{S}}
\newcommand{\V}{\vec{v}}
\newcommand{\RP}{\mathbb{RP}}
\newcommand{\CP}{\mathbb{CP}}
\newcommand{\B}{\mathcal{B}}
\newcommand{\GL}{\mathrm{GL}}
\newcommand{\SL}{\mathrm{SL}}
\newcommand{\SP}{\mathrm{SP}}
\newcommand{\SO}{\mathrm{SO}}
\newcommand{\SU}{\mathrm{SU}}
\newcommand{\gl}{\mathfrak{gl}}
\newcommand{\g}{\mathfrak{g}}
\newcommand{\Bun}{\mathsf{Bun}}
\newcommand*{\dt}[1]{%
   \accentset{\mbox{\large\bfseries .}}{#1}}
\newcommand{\inv}{^{-1}}
\newcommand{\bra}[2]{ \left[ #1, #2 \right] }
\newcommand{\set}[1]{\left\lbrace #1 \right\rbrace}
\newcommand{\abs}[1]{\left\lvert#1\right\rvert}
\newcommand{\norm}[1]{\left\lVert#1\right\rVert}
\newcommand{\transv}{\mathrel{\text{\tpitchfork}}}
\newcommand{\defeq}{\vcentcolon=}
\newcommand{\enumbreak}{\ \\ \vspace{-\baselineskip}}
\let\oldexists\exists
\renewcommand\exists{\oldexists~}
\let\oldL\L
\renewcommand\L{\mathfrak{L}}
\makeatletter
\newcommand{\incfig}[2]{%
    \fontsize{48pt}{50pt}\selectfont
    \def\svgwidth{\columnwidth}
    \scalebox{#2}{\input{#1.pdf_tex}}
}
%
\newcommand{\tpitchfork}{%
  \vbox{
    \baselineskip\z@skip
    \lineskip-.52ex
    \lineskiplimit\maxdimen
    \m@th
    \ialign{##\crcr\hidewidth\smash{$-$}\hidewidth\crcr$\pitchfork$\crcr}
  }%
}
\makeatother
\newcommand{\bd}{\partial}
\newcommand{\lang}{\begin{picture}(5,7)
\put(1.1,2.5){\rotatebox{45}{\line(1,0){6.0}}}
\put(1.1,2.5){\rotatebox{315}{\line(1,0){6.0}}}
\end{picture}}
\newcommand{\rang}{\begin{picture}(5,7)
\put(.1,2.5){\rotatebox{135}{\line(1,0){6.0}}}
\put(.1,2.5){\rotatebox{225}{\line(1,0){6.0}}}
\end{picture}}
\DeclareMathOperator{\id}{id}
\DeclareMathOperator{\im}{Im}
\DeclareMathOperator{\codim}{codim}
\DeclareMathOperator{\coker}{coker}
\DeclareMathOperator{\supp}{supp}
\DeclareMathOperator{\inter}{Int}
\DeclareMathOperator{\sign}{sign}
\DeclareMathOperator{\sgn}{sgn}
\DeclareMathOperator{\indx}{ind}
\DeclareMathOperator{\alt}{Alt}
\DeclareMathOperator{\Aut}{Aut}
\DeclareMathOperator{\trace}{trace}
\DeclareMathOperator{\ad}{ad}
\DeclareMathOperator{\End}{End}
\DeclareMathOperator{\Ad}{Ad}
\DeclareMathOperator{\Lie}{Lie}
\DeclareMathOperator{\spn}{span}
\DeclareMathOperator{\dv}{div}
\DeclareMathOperator{\grad}{grad}
\DeclareMathOperator{\Sym}{Sym}
\DeclareMathOperator{\tr}{tr}
\DeclareMathOperator{\sheafhom}{\mathscr{H}\text{\kern -3pt {\calligra\large om}}\,}
\newcommand*\myhrulefill{%
   \leavevmode\leaders\hrule depth-2pt height 2.4pt\hfill\kern0pt}
\newcommand\niceending[1]{%
  \begin{center}%
    \LARGE \myhrulefill \hspace{0.2cm} #1 \hspace{0.2cm} \myhrulefill%
  \end{center}}
\newcommand*\sectionend{\niceending{\decofourleft\decofourright}}
\newcommand*\subsectionend{\niceending{\decosix}}
\def\upint{\mathchoice%
    {\mkern13mu\overline{\vphantom{\intop}\mkern7mu}\mkern-20mu}%
    {\mkern7mu\overline{\vphantom{\intop}\mkern7mu}\mkern-14mu}%
    {\mkern7mu\overline{\vphantom{\intop}\mkern7mu}\mkern-14mu}%
    {\mkern7mu\overline{\vphantom{\intop}\mkern7mu}\mkern-14mu}%
  \int}
\def\lowint{\mkern3mu\underline{\vphantom{\intop}\mkern7mu}\mkern-10mu\int}
%
%--------Hypersetup--------
%
\hypersetup{
    colorlinks,
    citecolor=black,
    filecolor=black,
    linkcolor=blue,
    urlcolor=blacksquare
}
%
%--------Solution--------
%
\newenvironment{solution}
  {\begin{proof}[Solution]}
  {\end{proof}}
%
%--------Graphics--------
%
%\graphicspath{ {images/} }
%
\begin{document}
%
\author{Jeffrey Jiang}
%
\title{Abelian Yang-Mills}
%
\maketitle
%
\tableofcontents
%
\section{The General Case}
%
In the case where the structure group is $U(1)$, the Yang-Mills equations for
a connection $A$ on principal $U(1)$ bundle $\pi : P \to M$ over a Riemannian manifold $M$
reduce to
\begin{align*}
dF_A &= 0 \\
d^*F_A &= 0
\end{align*}
%
Where $F_A \in \Omega^2_M$ is the curvature form of $A$. The Yang-Mills equations are
equivalent to $\Delta F_A = 0$, where $\Delta$ is the Laplacian on $M$. By Hodge theory,
the cohomology class of $F_A$ has a unique harmonic minimizer $\Theta$, and Yang-Mills
connections are the connections $A$ satisfying $F_A = \Theta$. For a
connection $A$ and a $1$-form $\eta \in \Omega^1_M$, we have the identity
\[
F_{A + \eta} = F_A + d\eta
\]
from which we can conclude that the space $\mathscr{A}_{\text{YM}}(P)$ of
Yang-Mills connections over $M$ are a torsor over the vector space
$Z^1_M$ of closed $1$-forms on $M$. \\

The gauge group in this situation is the group
$\mathscr{G}(P) \defeq \mathrm{Map}(M,U(1))$, which follows from the fact that
$U(1)$ is abelian. The right action of $\mathscr{G}(P)$ on $\mathscr{A}_{\text{YM}}(P)$
is given by the mapping
\[
A \cdot f = A + \pi^*f^*\theta
\]
where $\theta \in \Omega^1_{U(1)}$ is the Maurer-Cartan form. The Yang-Mills
equations are invariant under the action of $\mathscr{G}(P)$, so we are interested
in the space $\mathscr{A}_{\text{YM}}(P) / \mathscr{G}(P)$ of Yang-Mills connections
up to gauge equivalence. The gauge group also acts on $Z^1_M$, where
$\eta \cdot f = \eta + f^*\theta$. Therefore, upon fixing a reference
connection $A_0 \in \mathscr{A}_{\text{YM}}(P)$ to identify $\mathscr{A}_{\text{YM}}(P)$
with $Z^1_M$, we may instead compute the quotient $Z^1_M / \mathscr{G}(P)$. To
do so we first quotient by the identity component $\mathscr{G}_0(P)$, and
the quotient by the component group $\pi_0\mathscr{G}(P) = \mathscr{G}(P)/\mathscr{G}_0(P)$
The components of $\mathscr{G}(P)$ are given by homotopy classes of maps
$M \to U(1)$, so $\mathscr{G}_0(P)$ is the space of nullhomotopic maps. Any such
map $f : M \to S^1$ lifts to a map $\tilde{f}$ such that $e^{\tilde{f}} = f$,
and since $\theta$ pulls back to $dx$ along the exponential $\R \to U(1)$, we have that
the action of $f$ on a closed form $\eta$ is just $\eta + d\tilde{f}$. In
particular, since any function $h : M \to \R$ descends to a nullhomotopic map
$e^h : M \to S^1$, this tells us that $Z^1_M / \mathscr{G}_0(P) = H^1(M,\R)$.
To quotient this space by $\pi_0\mathscr{G}(P)$, we note that $U(1)$ is a
$K(\Z,1)$, so homotopy classes of maps $M \to U(1)$ are classified by
$H^1(X,\Z)$. Therefore, upon quotienting by $\pi_0\mathscr{G}(P)$, we get
\[
\mathscr{A}_{\text{YM}}(P)/\mathscr{G}(P) \cong Z^1_M/\mathscr{G}(P)
\cong H^1(X,\R)/H^1(X,\Z)
\]
which is a torus $\mathbb{T}^{b_1(M)}$, where $b_1(M)$ is the first Betti number.
It is important to note that the isomorphism
$\mathscr{A}_{\text{YM}}(P)/\mathscr{G}(P) \cong H^1(X,\R)/H^1(X,\Z)$ is only
as topological spaces, as we will see later that the Yang-Mills moduli
space is a torsor over $H^1(X,\R)/H^1(X,\Z)$. \\

The torus $H^1(X,\R)/H^1(X,\Z)$ can also be realized \ib{Jacobian}
$\mathrm{Jac}(M)$ of $M$, which parameterizes flat $U(1)$ bundles over $M$ up to gauge
equivalence. The space $\mathscr{A}_{\text{Flat}}(M)$ of flat $U(1)$ bundles up to
gauge equivalence over $M$ is well known to be the space of unitary representations
$\rho : \pi_1(M) \to U(1)$. Since $U(1)$ is abelian, this factors through the
abelianization, which, upon doing so, gives the same identification of $\mathrm{Jac}(M)$
as a torus of dimension $b_1(M)$. \\

To realize $\mathscr{A}_{\text{YM}}(P)$ as a torsor over $\mathrm{Jac}(M)$, we
pass though the correspondence
\[
\set{U(1)\text{-bundles } P \to M \text{ with connection}}
\leftrightarrow \set{\text{Line bundles } L \to M \text{ with unitary connection}}
\]
The correspondence is obtained in one direction by taking associated bundles with the
defining representation of $U(1)$, and the other direction comes from taking
the unitary frame bundle with respect to some Hermitian fiber metric, along with
the Chern connection. Then fix any principal $U(1)$ bundle $P \to M$, and let
$L \to M$ be its associated line bundle. Taking the tensor product with a trivial
bundle clearly results in an isomorphic line bundle, and for a Yang-Mills
connection $A$ on $L$, it is also clear that tensoring $A$ with a flat connection
yields another Yang-Mills connection, giving us an action $\mathscr{A}_{\text{Flat}}(M)$
on $\mathscr{A}_{\text{YM}}(P)$. Furthermore, tensoring by gauge equivalent flat
connections clearly results in gauge equivalent connections, so this action factors
through to an action of $\mathrm{Jac}(M)$ on $\mathscr{A}_{\text{YM}}(P)/\mathscr{G}(P)$.
To show that this gives $\mathscr{A}_{\text{YM}}(P)$ the structure of a
$\mathrm{Jac}(M)$-torsor, it suffices to show that for topologically isomorphic line
bundles $L_1,L_2 \to M$ equipped with Yang-Mills connections $A_1$ and $A_2$,
the product bundle $L_1 \otimes L_2^* \to M$ equipped with connection $A_1 \otimes A_2^*$
is a flat bundle. This follows immediately from the fact that $A_2 = A_1 + \eta$
for some closed form $\eta$, so $A_1 \otimes A_2^* = d + \eta$, which is a flat
connection on the trivial bundle. \\

The case of principal $\mathbb{T}^n$ bundles is only a minor extension of
the $U(1)$ case. We have a similar correspondence between principal
$\mathbb{T}^n$-bundles and rank $n$ vector bundles $E \to M$ along with the
data of a direct sum decomposition $E = L_1 \oplus \cdots \oplus L_n$ of
$E$ into Hermitian line bundles. Identifying the Lie algebra of
$\mathbb{T}^n$ with $\R^n$, we see that the curvature form $F_A$ of a Yang-Mills
connection $A$ on $E$ can be viewed as a vector $2$-forms with each component being the
curvature form for a connection on a direct summand $L_i$. The condition that a
connection $A$ on $E$ is a Yang-Mills connection is then seen to be equivalent to each
component of $F_A$ being a harmonic $2$-form on $M$. Therefore, the data of a Yang-Mills
connection on a principal $\mathbb{T}^n$-bundle $P \to M$ is equivalent to the data of
Yang-Mills connections $A_i$ for each of the line bundles $L_i$, noting that the direct
sum of Yang-Mills connections is also a Yang-Mills connection.
%
\section{The Case of Riemann Surfaces}
%
Let $\Sigma$ be a Riemann surface of genus $g \geq 1$, and fix a Riemannian metric
on $\Sigma$ such that the Riemannian volume form $\omega$ satisfies
\[
\int_\Sigma \omega = 1
\]
In this case, the first Chern class $c_1(P)$ of a principal $U(1)$-bundle is an integer,
using the identification $H^2(\Sigma,\Z) \cong \Z$ using the orientation induced by the
complex structure. We abuse notation and let $c_1(P)$ denote the integer under this
correspondence, and let $[c_1(P)]$ denote the cohomology class. Fix once and for
all a principal $U(1)$-bundle $Q \to \Sigma$ with $c_1(Q) = 1$ and a Yang-Mills connection
$A_0$, which will serve as our reference bundle with connection. The standard Hodge
theory argument shows that the curvature $F_{A_0}$ must be $2\pi i\omega$.
Let $p : \widetilde{\Sigma} \to \Sigma$ be the universal cover of $\Sigma$, and consider
the pullback bundle
\[\begin{tikzcd}
p^*Q \ar[d]\ar[r] & Q \ar[d]\\
\widetilde{\Sigma} \ar[r, "p"'] & \Sigma
\end{tikzcd}\]
Since the genus of $\Sigma$ is greater than $0$, we know $\widetilde{\Sigma}$ is
contractible, so $p^*Q$ is a trivial bundle $\widetilde{\Sigma} \times U(1)$, and
the pullback connection still has curvature $2\pi i \omega$. We have a covering
map $\widetilde{\Sigma} \times \R \to \widetilde{\Sigma} \times U(1)$ given
by exponentiation in the second factor, which gives us
\[\begin{tikzcd}
\widetilde{\Sigma} \times \R \ar[d] \\
\widetilde{\Sigma} \times U(1) \ar[d]\ar[r] & Q \ar[d]\\
\widetilde{\Sigma} \ar[r, "p"'] & \Sigma
\end{tikzcd}\]
Then since $\widetilde{\Sigma} \times U(1)$ is a trivial bundle, the composite
map $\widetilde{\Sigma} \times \R \to \Sigma$ is a principal bundle. Denote the structure
group of this bundle by $\Gamma_\R$. We then determine the structure of $\Gamma_\R$.
Since the action of $\pi_1(M)$ on $\widetilde{\Sigma}$ commutes with the $\R$ action on
$\widetilde{\Sigma} \times \R$, it follows that $\Gamma_\R$ is a central extension of
$\pi_1(M)$ by $\R$ so it fits into the short exact sequence of groups
\[\begin{tikzcd}
1 \ar[r] & \R \ar[r] & \Gamma_\R \ar[r] & \pi_1(M) \ar[r] & 1
\end{tikzcd}\]
Finally, the connection on $\widetilde{\Sigma} \times U(1)$ lifts to a connection
on $\widetilde{\Sigma} \times \R$, which follows from the fact that we can lift
horizontal distributions along covering spaces. By a slight abuse of notation, we
also refer to this connection as $A_0$. Once more, the curvature of the
connection on $\widetilde{\Sigma} \times \R$ remains equal to $2\pi i \omega$.
To determine the group $\Gamma_\R$ we identify $\Sigma$ as a quotient of the
$2g$-gon with edges labeled by $a_1,b_1 \ldots, a_g,b_g$ and their inverses. Then
it suffices to compute the holonomy of the connection $A_0$ about
the boundary path $\prod_i [a_i,b_i]$, since the holonomy about this path
determines the $U(1)$ action on $\widetilde{\Sigma} \times U(1)$, which determines the
action on $\widetilde{\Sigma} \times \R$ by lifting along the covering map. By pushing
the path $\prod_i [a_i,b_i]$ into the interior of the $2g$-gon, we obtain a closed
loops that bounds a disk in $\Sigma$, so the holonomy about the boundary of the disk
is given by the integral of the curvature form $2\pi i \omega$. Taking the limit
as we push this path out to the boundary, we find that the holonomy is computed by
the integral
\[
\int_\Sigma 2\pi i\omega = 2\pi c_1(Q) = 2\pi
\]
which tells us the holonomy traverses the fiber once. Putting everything together,
we get that $\Gamma_\R$ is the central extension of $\pi_1(M)$ obtained by adjoining
a central element $J$ that generates a subgroup isomorphic to $\R$, along with the
relation $\prod_i [a_i,b_i] = J$. \\

The purpose of this construction is to realize the bijective correspondence:
\[
\set{U(1)\text{-bundles }P \to \Sigma\text{ with Yang-Mills connection}}/\mathscr{G}(P)
\longleftrightarrow \hom(\Gamma_\R, U(1))
\]
One direction is clear-- given a homomorphism $\rho : \Gamma_\R \to U(1)$, we can form
the associated bundle $(\widetilde{\Sigma} \times \R) \times_{\Gamma_\R} \times U(1)$
with connection $\dt{\rho}(A_0)$. The fact that $\dt{\rho}(A_0)$ is a
Yang-Mills connection follows from the observation
\[
\dt{\rho}(d\star F_{A_0}) = d\star\dt{\rho}(F_A)
\]
where $\dt{\rho} : \R \to \R$ is the derivative of $\rho$ at the identity after making
the identifications of
$\mathrm{Lie}(\Gamma_\R) \cong \R$ and $\mathfrak{u}(1) \cong \R$. \\

For the other direction, let $P \to \Sigma$ be a $U(1)$-bundle with Yang-Mills
connection $A$. By passing to the line bundle perspective, we have a line
bundle $L \to \Sigma$ with Yang-Mills connection, and the original reference bundle
$Q$ corresponds to another line bundle $L_0 \to \Sigma$ with Yang-Mills connection
$A_0$. The usual Hodge theory gives us that $F_A = 2\pi i c_1(L) \omega$, so
up to gauge equivalence, we may write $L = L_0^{\otimes^{c_1(L)}}$, equipped with
the connection $A_0^{\otimes^{c_1(L)}} \otimes \Theta$, where $\Theta$ is a flat
connection on the trivial line bundle. The flat bundle with connection $\Theta$
furnishes us with a homomorphism $\varphi : \pi_1(M) \to U(1)$. The rest of the
argument follows from the following observation : A group homomorphism
$\rho : \Gamma_\R \to U(1)$ necessarily factors through $\pi_1(M)$, since
$U(1)$ being abelian implies that
\[
\rho\left(\prod_i [a_i,b_i]\right) = 1
\]
So any such homomorphism must necessarily map the central generator $J$ to the
identity. However, we note that a map $\Gamma_\R \to U(1)$ is still more
data than a map $\pi_1(\Sigma) \to U(1)$, since we are also provided the information
of the differential $\dt{\rho} : \R \to \R$. The fact that $\rho(J) = 1$ implies
that $\dt{\rho}(1)$ must be integral, since $e^{\dt{\rho}(1)} = \rho(J)$. Therefore,
we construct the group homomorphism $\rho : \Gamma_\R \to U(1)$ maps all
of $\R$ to $1$, and agrees with $\varphi : \pi_1(\Sigma) \to U(1)$ on the generators
$a_1,b_1,\ldots a_g,b_g$, but satisfies $\dt{\rho}(1) = c_1(L)$. In this way, we
see that the central element $J$ determines the topological type of the bundle,
while the map $\pi_1(M) \to U(1)$ determines the connection up to gauge equivalence. \\

In the general case, recall that for a fixed bundle $P$, we fixed a reference
connection $A_0$ on $P$, which allowed us to identify $\mathscr{A}_{\text{YM}}(P)$ with
the space of closed forms. In the special case of a Riemann surface, fixing a connection
on the bundle $Q$ provided a reference connection on \emph{all} principal
$U(1)$-bundles, since $U(1)$-bundles over $\Sigma$ are classified by integers via the
first Chern class, so they can all be written as tensor powers of $L$ and $L^*$.
%
\end{document}