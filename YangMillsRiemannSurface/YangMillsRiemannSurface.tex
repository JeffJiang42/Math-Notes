\documentclass[psamsfonts, 12pt]{amsart}
%
%-------Packages---------
%
\usepackage[h margin=1 in, v margin=1 in]{geometry}
\usepackage{amssymb,amsfonts}
\usepackage{amsmath}
\usepackage{accents}
\usepackage[all,arc]{xy}
\usepackage{tikz-cd}
\usepackage{enumerate}
\usepackage{mathrsfs}
\usepackage{amsthm}
\usepackage{mathpazo}
\usepackage{float}
%\usepackage[backend=biber]{biblatex}
%\addbibresource{bibliography.bib}
%\usepackage{charter} %another font
%\usepackage{eulervm} %Vakil font
\usepackage{yfonts}
\usepackage{mathtools}
\usepackage{enumitem}
\usepackage{mathrsfs}
\usepackage{fourier-orns}
\usepackage[all]{xy}
\usepackage{hyperref}
\usepackage{url}
\usepackage{mathtools}
\usepackage{graphicx}
\usepackage{pdfsync}
\usepackage{mathdots}
\usepackage{calligra}
\usepackage{import}
\usepackage{xifthen}
\usepackage{pdfpages}
\usepackage{transparent}

\usepackage{tgpagella}
\usepackage[T1]{fontenc}
%
\usepackage{listings}
\usepackage{color}

\definecolor{dkgreen}{rgb}{0,0.6,0}
\definecolor{gray}{rgb}{0.5,0.5,0.5}
\definecolor{mauve}{rgb}{0.58,0,0.82}

\lstset{frame=tb,
  language=Matlab,
  aboveskip=3mm,
  belowskip=3mm,
  showstringspaces=false,
  columns=flexible,
  basicstyle={\small\ttfamily},
  numbers=none,
  numberstyle=\tiny\color{gray},
  keywordstyle=\color{blue},
  commentstyle=\color{dkgreen},
  stringstyle=\color{mauve},
  breaklines=true,
  breakatwhitespace=true,
  tabsize=3
  }
%
%--------Theorem Environments--------
%
\newtheorem{thm}{Theorem}[section]
\newtheorem*{thm*}{Theorem}
\newtheorem{cor}[thm]{Corollary}
\newtheorem{prop}[thm]{Proposition}
\newtheorem{lem}[thm]{Lemma}
\newtheorem*{lem*}{Lemma}
\newtheorem{conj}[thm]{Conjecture}
\newtheorem{quest}[thm]{Question}
%
\theoremstyle{definition}
\newtheorem{defn}[thm]{Definition}
\newtheorem*{defn*}{Definition}
\newtheorem{defns}[thm]{Definitions}
\newtheorem{con}[thm]{Construction}
\newtheorem{exmp}[thm]{Example}
\newtheorem{exmps}[thm]{Examples}
\newtheorem{notn}[thm]{Notation}
\newtheorem{notns}[thm]{Notations}
\newtheorem{addm}[thm]{Addendum}
\newtheorem{exer}[thm]{Exercise}
%
\theoremstyle{remark}
\newtheorem{rem}[thm]{Remark}
\newtheorem*{claim}{Claim}
\newtheorem*{aside*}{Aside}
\newtheorem*{rem*}{Remark}
\newtheorem*{hint*}{Hint}
\newtheorem*{note}{Note}
\newtheorem{rems}[thm]{Remarks}
\newtheorem{warn}[thm]{Warning}
\newtheorem{sch}[thm]{Scholium}
%
%--------Macros--------
\renewcommand{\qedsymbol}{$\blacksquare$}
\renewcommand{\sl}{\mathfrak{sl}}
\newcommand{\Bord}{\mathsf{Bord}}
\renewcommand{\hom}{\mathsf{Hom}}
\renewcommand{\emptyset}{\varnothing}
\renewcommand{\O}{\mathcal{O}}
\newcommand{\R}{\mathbb{R}}
\newcommand{\ib}[1]{\textbf{\textit{#1}}}
\newcommand{\Q}{\mathbb{Q}}
\newcommand{\Z}{\mathbb{Z}}
\newcommand{\N}{\mathbb{N}}
\newcommand{\C}{\mathbb{C}}
\newcommand{\A}{\mathbb{A}}
\newcommand{\F}{\mathbb{F}}
\newcommand{\M}{\mathcal{M}}
\newcommand{\dbar}{\overline{\partial}}
\newcommand{\zbar}{\overline{z}}
\renewcommand{\S}{\mathbb{S}}
\newcommand{\V}{\vec{v}}
\newcommand{\RP}{\mathbb{RP}}
\newcommand{\CP}{\mathbb{CP}}
\newcommand{\B}{\mathcal{B}}
\newcommand{\GL}{\mathrm{GL}}
\newcommand{\SL}{\mathrm{SL}}
\newcommand{\SP}{\mathrm{SP}}
\newcommand{\SO}{\mathrm{SO}}
\newcommand{\SU}{\mathrm{SU}}
\newcommand{\gl}{\mathfrak{gl}}
\newcommand{\g}{\mathfrak{g}}
\newcommand{\Bun}{\mathsf{Bun}}
\newcommand*{\dt}[1]{%
   \accentset{\mbox{\large\bfseries .}}{#1}}
\newcommand{\inv}{^{-1}}
\newcommand{\bra}[2]{ \left[ #1, #2 \right] }
\newcommand{\set}[1]{\left\lbrace #1 \right\rbrace}
\newcommand{\abs}[1]{\left\lvert#1\right\rvert}
\newcommand{\norm}[1]{\left\lVert#1\right\rVert}
\newcommand{\transv}{\mathrel{\text{\tpitchfork}}}
\newcommand{\defeq}{\vcentcolon=}
\newcommand{\enumbreak}{\ \\ \vspace{-\baselineskip}}
\let\oldexists\exists
\renewcommand\exists{\oldexists~}
\let\oldL\L
\renewcommand\L{\mathfrak{L}}
\makeatletter
\newcommand{\incfig}[2]{%
    \fontsize{48pt}{50pt}\selectfont
    \def\svgwidth{\columnwidth}
    \scalebox{#2}{\input{#1.pdf_tex}}
}
%
\newcommand{\tpitchfork}{%
  \vbox{
    \baselineskip\z@skip
    \lineskip-.52ex
    \lineskiplimit\maxdimen
    \m@th
    \ialign{##\crcr\hidewidth\smash{$-$}\hidewidth\crcr$\pitchfork$\crcr}
  }%
}
\makeatother
\newcommand{\bd}{\partial}
\newcommand{\lang}{\begin{picture}(5,7)
\put(1.1,2.5){\rotatebox{45}{\line(1,0){6.0}}}
\put(1.1,2.5){\rotatebox{315}{\line(1,0){6.0}}}
\end{picture}}
\newcommand{\rang}{\begin{picture}(5,7)
\put(.1,2.5){\rotatebox{135}{\line(1,0){6.0}}}
\put(.1,2.5){\rotatebox{225}{\line(1,0){6.0}}}
\end{picture}}
\DeclareMathOperator{\id}{id}
\DeclareMathOperator{\im}{Im}
\DeclareMathOperator{\codim}{codim}
\DeclareMathOperator{\coker}{coker}
\DeclareMathOperator{\supp}{supp}
\DeclareMathOperator{\inter}{Int}
\DeclareMathOperator{\sign}{sign}
\DeclareMathOperator{\sgn}{sgn}
\DeclareMathOperator{\indx}{ind}
\DeclareMathOperator{\alt}{Alt}
\DeclareMathOperator{\Aut}{Aut}
\DeclareMathOperator{\trace}{trace}
\DeclareMathOperator{\ad}{ad}
\DeclareMathOperator{\End}{End}
\DeclareMathOperator{\Ad}{Ad}
\DeclareMathOperator{\Lie}{Lie}
\DeclareMathOperator{\spn}{span}
\DeclareMathOperator{\dv}{div}
\DeclareMathOperator{\grad}{grad}
\DeclareMathOperator{\Sym}{Sym}
\DeclareMathOperator{\tr}{tr}
\DeclareMathOperator{\sheafhom}{\mathscr{H}\text{\kern -3pt {\calligra\large om}}\,}
\newcommand*\myhrulefill{%
   \leavevmode\leaders\hrule depth-2pt height 2.4pt\hfill\kern0pt}
\newcommand\niceending[1]{%
  \begin{center}%
    \LARGE \myhrulefill \hspace{0.2cm} #1 \hspace{0.2cm} \myhrulefill%
  \end{center}}
\newcommand*\sectionend{\niceending{\decofourleft\decofourright}}
\newcommand*\subsectionend{\niceending{\decosix}}
\def\upint{\mathchoice%
    {\mkern13mu\overline{\vphantom{\intop}\mkern7mu}\mkern-20mu}%
    {\mkern7mu\overline{\vphantom{\intop}\mkern7mu}\mkern-14mu}%
    {\mkern7mu\overline{\vphantom{\intop}\mkern7mu}\mkern-14mu}%
    {\mkern7mu\overline{\vphantom{\intop}\mkern7mu}\mkern-14mu}%
  \int}
\def\lowint{\mkern3mu\underline{\vphantom{\intop}\mkern7mu}\mkern-10mu\int}
%
%--------Hypersetup--------
%
\hypersetup{
    colorlinks,
    citecolor=black,
    filecolor=black,
    linkcolor=blue,
    urlcolor=blacksquare
}
%
%--------Solution--------
%
\newenvironment{solution}
  {\begin{proof}[Solution]}
  {\end{proof}}
%
%--------Graphics--------
%
%\graphicspath{ {images/} }
%
\begin{document}
%
\author{Jeffrey Jiang}
%
\title{Yang-Mills on Riemann Surfaces}
%
\maketitle
%
\tableofcontents
%
\section{Preliminary Setup}
%
To discuss the Yang-Mills functional, we must first fix some data. The setup will
consist of the following ingredients
\begin{enumerate}
  \item A compact manifold $M$.
  \item A compact connected Lie group $G$.
  \item A principal $G$-bundle $P \to M$.
\end{enumerate}
%
With this data, we have two associated bundles
\begin{align*}
\Ad P &\defeq P \times_G G \\
\g_P &\defeq P \times_G \g
\end{align*}
%
Where the action of $G$ on $G$ is by conjugation, and the action of $G$ on $\g$
is the adjoint action. We note that these bundles both contain additional structure --
$\Ad P$ is a bundle of groups (not a principal bundle), and $\g_P$ is a bundle of Lie
algebras. The space of sections $\Gamma(M, \Ad P)$ has
a natural group structure given by pointwise multiplication, and is called the
\ib{gauge group} $\mathscr{G}(P)$. Likewise, the space of sections
$\Gamma(M, \g_P)$ has a natural Lie algebra structure given by the pointwise Lie
bracket, and can be naturally identified with the Lie algebra of $\mathscr{G}(P)$,
as we shall see. An alternate characterization of these spaces of sections comes from
a general characterization of sections of associated bundles
%
\begin{prop}
We have natural correspondences
\begin{align*}
\Gamma(M, \Ad P) &\longleftrightarrow \set{f : P \to G ~:~ f(p\cdot g) = g\inv f(p)g} \\
\Gamma(M, \g_P) &\longleftrightarrow \set{f : P \to \g ~:~ f(p\cdot g) = \Ad_{g\inv}f(p)}
\end{align*}
\end{prop}
%
From the above correspondence we a group isomorphism
$\mathscr{G}(P) \to \Aut(P)$, where $\Aut(P)$ denotes the group of
$G$-equivariant diffeomorphisms $\varphi : P \to P$ such that
the following diagram commutes :
\[\begin{tikzcd}
P \ar[dr] \ar[rr, "\varphi"] && P \ar[dl] \\
& M
\end{tikzcd}\]
The isomorphism is given by mapping a $G$-equivariant map $f : P \to g$
to the automorphism $\varphi_f : P \to P$ defined by $\varphi_f(p) = p\cdot f(p)$.
In a similar fashion, we get a Lie algebra isomorphism of $\Gamma(M,\g_P)$ with
the vertical vector fields on $P$, which lets us explicitly realize
$\Gamma(M,\g_P)$ as the Lie algebra of the gauge group. Regarding a section
$\phi \in \Gamma(M,\g_P)$ as a $G$-equivariant map $P \to \g$, we get
a bundle automorphism $\exp(\phi)$ defined by $p\mapsto p\cdot\exp(\phi(p))$. From
this perspective, we see that $\Gamma(M,\g_P)$ can be identified with the $G$-invariant
vertical vector fields on $P$, which are the infinitesimal generators of the action
of $\mathscr{G}(P)$ on $P$. \\

In the the case of $\Gamma(M,\g_P)$, we can extend the correspondence to the spaces of
$\g_P$ valued forms. The kernel of the differential of the projection $P \to M$ gives a
subbundle of $TP$, which has a natural identification with the trivial bundle
$\underline{\g} = P \times \g$. Then we can identify the space of sections of
$\Lambda^kT^*M \otimes \g_P$, (i.e. the space $\Omega^k_M(\g_p)$ of $\g_p$ valued
$k$-forms with a subspace of the space $\Omega^k_P(\g)$ of $\g$-valued $k$-forms $\omega$
on $P$ satisfying :
\begin{enumerate}
  \item $R_g^*\omega = \Ad_{g\inv}\omega$, where $R_g : P \to P$ denotes the right
  action of $g \in G$.
  \item $\iota_\xi\omega = 0$ for any $\xi \in \g$, where $\iota$ denotes interior
  multiplication, and we identify $\xi$ with the constant vector field
  $\xi$ under the identification of the vertical space with $\g$.
\end{enumerate}
%
We have a maps
\begin{align*}
\Omega^p_M(\g_p) \otimes \Omega^q_M(\g_P) &\to \Omega^{p+q}_M(\g_p \otimes \g_p) \\
(\omega_1 \otimes \xi_1) \otimes (\omega_2 \otimes \xi_2)
&\mapsto (\omega_1 \wedge \omega_2) \otimes (\xi_1 \otimes \xi_2)
\end{align*}
From now on, we will usually omit the tensor symbol for $\g_P$-valued forms in favor of
juxtaposition, i.e. we write $\omega\xi$ instead of $\omega \otimes \xi$. Using the Lie
bracket, we then get
\begin{align*}
\Omega^p_M(\g_p) \otimes \Omega^q_M(\g_P) &\to \Omega^{p+q}_M(\g_p) \\
\omega \otimes \eta \mapsto [\omega,\eta]
\end{align*}
We note that this is \emph{not} skew-symmetric, instead, given
$\omega \in \Omega^p_M(\g_P)$  and $\eta \in \Omega^q_M(\g_P)$, we have
\[
[\omega,\eta] = (-1)^{pq+1} [\eta,\omega]
\]
For any semisimple Lie group $G$ (in particular, for any compact Lie group $G$), we have
an inner product $\langle\cdot,\cdot\rangle : \g \times \g \to \R$ that is invariant
under the adjoint action (e.g. the Killing form). Invariance under the adjoint action
gives us
\begin{align*}
\langle [\xi_1,\xi_2],\xi_3 \rangle &= \langle[-\xi_2,\xi_1],\xi_3 \rangle \\
&= \frac{d}{dt}\bigg\vert_{t=0}\langle \Ad_{\exp(-t\xi_2)}\xi_1,\xi_3 \\
&= \frac{d}{dt}\bigg\vert_{t=}\langle \Ad_{\exp(t\xi_2)}\Ad_{\exp(-t\xi_2)}\xi_1,
\Ad_{\exp(t\xi_2)}\xi_3\rangle \\
&= \langle \xi_1,[\xi_2,\xi_3]\rangle
\end{align*}

Fixing one such inner product induces a fiber product on the trivial bundle $P \times \g$,
and invariance guarantees that this descends to a fiber product on $\g_P$. This give us
pairings
\begin{align*}
\Omega^p_M(\g_P) \otimes \Omega^q_M(\g_P) &\to \Omega^{p+q}_M \\
\omega \otimes \eta &\mapsto \langle \omega, \eta \rangle
\end{align*}
%
which also satisfies the identity
\[
\langle[\omega,\eta],\xi\rangle = \langle \omega,[\eta,\xi]\rangle
\]
We again note that this is not symmetric or skew symmetric, and instead behaves
like the wedge product, i.e. for $\omega \in \Omega^p_M(\g_P)$ and
$\eta \in \Omega^q_M(\g_P)$, we have
\[
\langle \omega,\eta \rangle = (-1)^{pq}\langle \eta,\omega\rangle
\]
which can be seen by writing $\omega = \omega^i\xi_i$ and $\eta = \eta^i\xi_i$
for an orthonormal basis $\set{\xi_i}$ for $\g$.
%
We then fix an orientation and Riemannian metric on $M$, which gives us a Hodge
star operator $\star : \Omega^p_M \to \Omega^{n-p}_M$ and a Riemannian volume form
$dV_g$. The Hodge star extends to $\g_P$-valued $k$-forms, where given
$\omega \in \Omega^p_M$ and $\xi \in \Gamma(M,\g_P)$, we define
$\star(\omega\xi) = (\star\omega)\xi$. Then given
$\omega_1\xi_1,\omega_2\xi_2 \in \Omega^p_M(\g_P)$, we have
\[
\langle\omega_1\xi_1,\star\omega_2\xi_2\rangle =
\langle \omega_1,\omega_2\rangle_g\langle\xi_1,\xi_2\rangle
\]
where $\langle\cdot,\cdot\rangle_g$ denotes the fiber metric on $\Lambda^pT^*M$
induced by $g$. This gives us an inner product on each $\Omega^p_M(\g_P)$ defined
by
\[
(\theta,\varphi) = \int_M \langle \theta,\star\varphi\rangle
\]
Which gives us the $L_2$ norm on $\Omega^p_M(\g_P)$ with $\norm{F}_{L^2}^2 = (F,F)$.
\begin{defn}
A connection on a principal bundle $\pi : P \to M$ is a choice of $G$-invariant splitting
of the exact sequence of vector bundles over $P$
\[\begin{tikzcd}
0 \ar[r] & \underline{\g} \ar[r] & TP \ar[r] & \pi^*TM \ar[r] & 0
\end{tikzcd}\]
i.e. a distribution $H \subset TP$ such that
\begin{enumerate}
  \item $(R_g)_*H_p = H_{p\cdot g}$
  \item $H \oplus \underline{\g} = TP$
\end{enumerate}
Equivalently, it is the data of a $\g$-valued $1$-form $A \in \Omega^1_P(\g)$
satisfying
\begin{enumerate}
  \item $R_g^*A = \Ad_{g\inv} A$
  \item $\iota_\xi A = \xi$ for all $\xi \in \g$.
\end{enumerate}
\end{defn}
%
Note in particular that by a dimension count, we have that $\pi_*\vert_H : H \to TM$ is
an isomorphism. This implies that given a tangent vector $v$ at $x$ and a point
$p \in P$ in the fiber over $x$, we get a unique horizontal lift
$\tilde{v} \in H_p$. For a fixed principal $G$-bundle $\pi : P \to M$, we let
$\mathscr{A}(P)$ denote the space of all connections on $P$, which is an affine space
over $\Omega^1_M(\g_P)$. The connection form $A$ on $P$ induces an exterior covariant
derivative on any associated vector bundle $E = P \times_G V$ arising from a linear
representation $\rho : G \to \GL(V)$. Let $\dt{\rho} : \g \to \End(V)$ be the
derivative of $\rho$ at the identity. Then the exterior covariant derivative is given
by
\begin{align*}
d_A : \Omega^p_M(E) &\to \Omega^{p+1}_M(E) \\
\psi &\mapsto d\psi + \dt{\rho}(A) \wedge \psi
\end{align*}
%
In particular, we get an exterior covariant derivative on $\g_P$, which is given by
\[
d_A\psi = d\psi + [A,\psi]
\]
\begin{prop}
Let $\phi \in \Omega^p_M(\g_P)$ and $\psi \in \Omega^q_M(\g_P)$. Then
\[
d\langle\phi,\psi\rangle = \langle d_A\phi,\psi\rangle
+ (-1)^p\langle \phi,d_A\psi\rangle
\]
\end{prop}
%
\begin{proof}
We compute
\begin{align*}
\langle d_A\phi,\psi\rangle + (-1)^p\langle\phi,d_A\psi\rangle
&= \langle d\phi,\psi\rangle + \langle[A,\phi],\psi\rangle
+ (-1)^p (\langle \phi,d\psi\rangle + \langle \phi,[A,\psi]) \\
&= \langle d\phi,\psi\rangle + \langle[A,\phi],\psi\rangle
+ (-1)^p (\langle\phi,d\psi\rangle + \langle[\phi,A],\psi) \\
&= \langle d\phi,\psi\rangle + \langle [A,\phi],\psi\rangle
+ (-1)^{2p+1}\langle [A,\phi],\psi + (-1)^p\langle\phi,d\psi\rangle \\
&= \langle d\phi,\psi\rangle + (-1)^p\langle\phi,d\psi\rangle
\end{align*}
Writing $\phi = \phi^i\xi_i$ and $\psi=\psi^i\xi_i$ in an orthonormal basis
$\set{\xi_i}$ for $\g$, this becomes
\begin{align*}
\langle d_A\phi,\psi\rangle + (-1)^p\langle\phi,d_A\psi\rangle
&= \sum_i d\phi^i \wedge \psi^i + (-1)^p \phi^i\wedge d\psi^i \\
&= d\langle\phi,\psi\rangle
\end{align*}
\end{proof}
%
Given any distribution $E \subset TP$, we get a Frobenius tensor
$\phi_E : E \otimes E \to TP/E$ given by $X \otimes Y \to [X,Y] \mod E$ where
we extend $X$ and $Y$ to local vector fields. The Frobenius tensor should be
thought of as the obstruction to the existence of an integral submanifold for the
distribution $E$. In the case of a connection $H$ on a principal bundle $P \to M$,
we can extend to all of $TP$ by first projecting onto $H$, and
have an identification of $TP/H \cong \underline{\g}$, and the Frobenius
tensor is given by $X \otimes Y \mapsto A([X,Y])$, where $A$ is the connection $1$-form,
and is called the \ib{curvature form} of the connection, and is denoted $F_A$. In
terms of differential forms, we have that for horizontal vectors $\xi_1,\xi_2$ on
$TP$,
\[
dA(\xi_1,\xi_2) = \xi_1A(\xi_2) - \xi_2A(\xi_1) - A([\xi_1,\xi_2])
\]
The fact that $\xi_1$ and $\xi_2$ are horizontal implies that they are in the
kernel of $A$, which gives us $dA(\xi_1,\xi_2) = -F_A(\xi_1,\xi_2)$. We also know
that $F_A$ vanishes on vertical vectors, and since $A(X) = X$ for $X \in \g$,
we get that
\[
dA + \frac{1}{2}[A,A] = -F_A\footnote{Our convention for the sign of the curvature
is opposite from many other conventions, which usually sets
$F_A = dA + \frac{1}{2}[A,A]$}
\]
It can be shown that $F_A$ transforms by the adjoint action under pullback, and
vanishes on vertical vectors, so it descends to a $\g_P$-valued $2$-form on the base
manifold $M$. \\

Another thing to note is that there is a natural action of the gauge group
$\mathscr{G}(P)$ on the space of connections $\mathscr{A}(P)$. Interpreting the
elements of $\mathscr{G}(P)$ as bundle automorphisms $\varphi : P \to P$ and
elements of $\mathscr{A}(P)$ as $\g$-valued $1$-forms $A$ on $P$, the action
is simply pullback, $(\varphi,A) \mapsto \varphi^*A$. To show that
this defines an action, we must check that $\varphi^*A$ satisfies the conditions
%
\begin{enumerate}
  \item $R_g^*\varphi^*A = \Ad_{g\inv} \varphi^*A$
  \item $\iota_\xi \varphi^*A = \xi$ for all $\xi \in \g$.
\end{enumerate}
%
Which are all simple consequences of the $G$-equivariance of $\varphi$ and
the transformation law for $A$. For a specific formula, let
$\varphi : P \to P$ be an element of the gauge group, and let
$g_\varphi : P \to G$ be its associated $G$-equivariant map. Then
\[
\varphi^*A = \Ad_{g_\varphi\inv} A + g_\varphi^*\theta
\]
where $\theta \in \Omega^1_G(\g)$ denotes the \ib{Maurer-Cartan form}
\[
\theta_g(v) = (dL_{g\inv})_g(v)
\]
which satisfies the \ib{Maurer-Cartan equation}
\[
d\theta + \frac{1}{2}[\theta,\theta] = 0
\]
%
\begin{prop}
Let $A \in \mathscr{A}(P)$ be a connection and $\varphi : P \to P$ an element
of $\mathscr{G}(P)$ with associated $G$-equivariant map $g_\varphi : P \to G$.
Then
\[
F_{\varphi^*A} = \Ad_{g_\varphi\inv} F_A
\]
\end{prop}
%
\begin{proof}
Using the transformation law for $\varphi^*A$ we compute
\begin{align*}
F_{\varphi^*A} &= d(\Ad_{g_\varphi\inv} A + g_\varphi^*\theta)
+ \frac{1}{2}[\Ad_{g_\varphi\inv} A + g_\varphi^*\theta,
\Ad_{g_\varphi\inv} A + g_\varphi^*\theta] \\
&= \Ad_{g_\varphi\inv}dA + g_\varphi^*d\theta + \frac{1}{2}
\left([\Ad_{g_\varphi\inv}A, \Ad_{g_\varphi\inv} A] + [\Ad_{g_\varphi\inv} A,
g_\varphi^*\theta] + [g^*_\varphi\theta, \Ad_{g_\varphi\inv}]
+ [g_\varphi^*\theta,g_\varphi^*\theta] \right) \\
&= \Ad_{g_\varphi\inv} dA + \frac{1}{2}[\Ad_{g_\varphi\inv}A, \Ad_{g_\varphi\inv} A]]
\end{align*}
Where we use skew-symmetry and the Maurer-Cartan equation.
\end{proof}
%
We similarly compute the infinitesimal action of the Lie algebra $\Gamma(M,\g_P)$.
%
\begin{prop}
The vector field corresponding to $\phi \in \Gamma(M,\g_P)$ is
$A \mapsto d_A\phi \in \Omega_M^1(\g_P)$
\end{prop}
%
\begin{proof}
We compute the vector field at a connection $A \in \mathscr{A}(P)$ to be
\begin{align*}
&\frac{d}{dt}\bigg\vert_{t=0} \Ad_{\exp(t\phi)\inv} A + \exp(t\phi)^*\theta
= -[\phi, A] + \frac{d}{dt}\bigg\vert_{t=0}(dL_{\exp(-t\phi)} d(\exp(t\phi))) \\
&= [A,\phi] + \left(\frac{d}{dt}\bigg\vert_{t=0} dL_{\exp(-t\phi)}\right)d(\exp(0))
+ dL_{\exp(0)}\left(\frac{d}{dt}\bigg\vert_{t=0}d(\exp(t\phi))\right) \\[5pt]
&= [A,\phi] + d\phi\\
&= d_A\phi
\end{align*}
where for the third equality we use the product rule, and in the fourth equality
we use the fact that $\exp(0) = \id$ and that the derivative of
$\exp(t\phi)$ as $t \to 0$ is $\phi$.
\end{proof}
%
For any other connection $A + \eta$ with $\eta \in \Omega^1_M(\g_P)$,
a quick computation yields
\[
F_{A+\eta} = F_A + \frac{1}{2}[\eta,\eta] + d_A\eta
\]
From this description, we can relate the curvature $F_A$ with the covariant derivative.
Note that for the line of connections $A + t\eta$, we have that
\[
\frac{d}{dt}\bigg\vert_{t=0} F_{a+t\eta} =
\frac{d}{dt}\bigg\vert_{t=0} F_A + \frac{t^2}{2}[\eta,\eta] + td_A\eta
= d_A\eta
\]
So $d_A\eta$ measures the infinitesimal change of the curvature $F_A$ in the direction
$\eta$.
%
\section{The Yang-Mills Functional}
%
With the setup done, we have the ingredients necessary to define the Yang-Mills
functional.
%
\begin{defn}
The \ib{Yang-Mills functional} is the map
$L : \mathscr{A}(P) \to \R$ given by
\[
L(A) = \norm{F_A}_{L^2}^2 = \int_M \langle F_A,\star F_A \rangle
\]
\end{defn}
%
We immediately see that the Yang-Mills equations are invariant under $\mathscr{G}(P)$
in the following sense -- if we have any gauge transformation $\varphi$ with
associated map $g_\varphi : P \to G$, we have that $L(\varphi^*A) = L(A)$,
which follows immediately from the invariance of $\langle\cdot,\cdot\rangle$
and the transformation law for curvature. \\

Our goal now will be to find the Euler-Lagrange
equations for the Yang-Mills functional by computing the first and second variations.
Using the Hodge star operator, we construct the formal adjoint with respect to
the inner product $d_A^* : \Omega^{p}_M(\g_P) \to \Omega^{p-1}_M(\g_P)$ in the same manner
as for classical Hodge theory on a Riemannian manifold. Explicitly, the formula
on $p$-forms is given by
\[
d^*_A = (-1)^{n(p+1) + 1}\star d_A \star
\]
where $n = \dim M$. We then compute the first variation of $L$.
%
\begin{prop}[\ib{The First Variation}]
For a local extremum $A \in \mathscr{A}(P)$ of the Yang-Mills functional, we have
\[
d_A\star F_A = 0
\]
The local extremum connection $A$ is then called a \ib{Yang-Mills connection}, and the
space of Yang-Mills connections is denoted $\mathscr{A}_{\mathrm{YM}}(P)$.
\end{prop}
%
\begin{proof}
Consider a variation $A + t\eta$ with $t \in \R$ and $\eta \in \Omega^1_M(\g_P)$.
We have that the curvature is given by
\[
F_{A+t\eta} = F_A + \frac{t^2}{2}[\eta,\eta] + td_A\eta
\]
This then gives us
\begin{align*}
\norm{F_{A+t\eta}}_{L^2} &= \int_M \langle F_{A+t\eta},F_{A+t\eta}\rangle \\
&= \int_M \langle F_A + \frac{t^2}{2}[\eta,\eta]
+ td_A\eta,\star (F_A + \frac{t^2}{2}[\eta,\eta] + td_A\eta)\rangle \\
\end{align*}
Expanding this out, we get that the term that is linear in $t$ is

\[
\int_M \langle F_A, \star d_A\eta\rangle + \langle d_A\eta, \star F_A \rangle
= 2(F_A,d_A\eta)
\]
where we use symmetry of $(\cdot,\cdot)$. Since $A$ is extremal, we have that
this term must vanish, giving us that $(F_A,d_A\eta) = (d^*_A F,\eta) = 0$ for
every $\eta$. Then since we have (up to sign) $d^*_A = \star d_A \star$, and
$\star$ is an isomorphism, this implies $d_A\star F_A = 0$.
\end{proof}
%
\begin{prop}[\ib{The Second Variation}]
At a Yang-Mills connection $A \in \mathscr{A}(P)$, we have
\[
d^*_A d_A\eta + \star[\eta,\star F_A] = 0
\]
\end{prop}
%
\begin{proof}
We differentiate the first variational equation with respect to $t$, i.e.
we compute
\[
\frac{d}{dt}\bigg\vert_{t=0} d_{A+t\eta}^*F_{A+t\eta}
\]
We expand out
\begin{align*}
d_{A+t\eta}^*F_{A+t\eta} &= \pm \star d_{A+t\eta}\star F_{A+t\eta} \\
&= \pm\left(\star d_A\star\left(F_A + td_A\eta + \frac{t^2}{2}[\eta,\eta]\right)
+ t\star
\left[\eta, \star\left(F_A + td_A\eta + \frac{t^2}{2}[\eta,\eta]\right)\right]\right)
\end{align*}
Taking the term linear in $t$ yields
\[
\pm\left( \star d_A\star d_A\eta + \star[\eta,\star F_A] \right)
\]
Giving us that at an extremal connection $A$, we have
\[
d^*_A d_A\eta + \star[\eta,\star F_A] = 0
\]
\end{proof}
%
%TODO Hessian?
%
\section{The $U(1)$ Case}
%
We first restrict to the special case $G = U(1)$. In this case, the
Lie algebra is abelian, so the adjoint action of $U(1)$ on $\mathfrak{u}(1)$ is trivial,
giving us that $\g_P$ is a trivial bundle. Identifying $\mathfrak{u}(1)$ with
$\R$, we can then identify $\mathfrak{u}(1)$ valued forms on $P$ with ordinary
differential forms. Likewise, using triviality of $\g_P$, we can identify
$\g_P$-valued forms with ordinary differential forms on $M$. The vertical bundle
in this case is a trivial line bundle over $P$, and there is a unique $U(1)$-invariant
vertical vector field on $P$, which on each fiber restricts to the vector field dual
to $d\theta$. Then given a connection $A$ on $P$, we have that $dA = \pi^*F_A$,
since $[A,A] = 0$. This immediately tells us that $F_A$ is closed, since $d$
commutes with pullback. Furthermore, for any other connection $A + \eta$, we have that
\[
F_{A+\eta} = F_A + \frac{1}{2}[\eta,\eta] + d_A\eta
\]
Then since $d_A = d$ and $[\eta,\eta] = 0$, this gives us that
$F_{A+\eta} = F_A + d\eta$, which tells us that the cohomology class of $F_A$ is
independent of our choice of $A$. Using our sign convention, this is equal to
$-2\pi i c_1(P)$. Furthermore, in this situation, the Yang-Mills functional reduces
to the standard Hodge theory picture, since the $L^2$ norm will coincide with the
$L^2$ norm on differential forms. Therefore, a Yang-Mills connection on $P$ is equivalent
to finding the unique connection that minimizes the $L^2$ norm in the cohomology class
$2\pi i c_1(P)$. By standard Hodge theory, this gives us that every $U(1)$-bundle $P$
admits a Yang-Mills connection.
%
\section{Yang-Mills Over a Riemann Surface}
%
\iffalse
The Hodge star operator maps $\Omega^1_M \to \Omega^1_M$, and satisfies
$\star^2 = -\id$, which induces an almost complex structure on $M$, giving us a
decomposition $\Omega^1_M(\C) = \Omega^{1,0}_M(\C) \oplus \Omega^{0,1}_M(\C)$
into the $\pm i$ eigenspaces of the complexified Hodge star. The operator
$\dbar \defeq \pi^{0,1} \circ d$ (where $\pi^{0,1}$ denotes projection onto
$\Omega^{0,1}_M(\C)$) satisfies $\dbar^2 = 0$ since by dimension reasons,
$\Omega^{0,2}_M(\C) = 0$, so the induced almost complex structure is integrable by the
Newlander-Nirenberg theorem. The same argument with projection onto $\Omega^{1,0}_M(\C)$
gives an operator $\partial$ satisfying $\partial^2 = 0$, and we get a decomposition
$d = \partial + \dbar$. Then given a principal bundle $P \to M$, We get a similar
decomposition for $\Omega^1_M(\g_P)$ after complexification giving a decomposition
$d_A = \partial_A + \dbar_A$ for any connection $A \in \mathscr{A}(P)$.
\fi
We now restrict our attention to when $M$ is a orientable surface, with genus $g > 0$.
Let $Q \to M$ be a principal $U(1)$ bundle with $c_1(Q) = 1$, i.e.
\[
\frac{1}{2\pi i} \int_M c_1(Q) = 1
\]
Then fix a Riemannian metric on $M$ with volume form $\omega$ such that
$\int_M \omega = 1$, and Yang-Mills connection $A$ on $Q$. Since $c_1(Q) = 1$, we have
that $[c_1(Q)] = [\omega]$. Furthermore, since $\star\omega = 1$ and
$[-2\pi i c_1(Q)] = [F_A]$, we get that the curvature of $A$ must be equal to
$-2\pi i \omega$ to minimize the Yang-Mills functional. Then let $\widetilde{M} \to M$
be the universal cover of $M$. Since the genus of $M$ is at least $1$, $\widetilde{M}$
is contractible, so the pullback of $Q$ along the covering projection gives us a
trivial $U(1)$ bundle over $\widetilde{M}$
\[\begin{tikzcd}
\widetilde{M} \times U(1)\ar[d] \ar[r] & Q \ar[d]\\
\widetilde{M} \ar[r] & M
\end{tikzcd}\]
%
Then we have a covering map $\widetilde{M} \times \R \to U(1)$, using the usual
covering $\R \to U(1)$, giving a principal $\R$-bundle over $\widetilde{M}$. Then
if we consider the composite map $\widetilde{M} \times \R \to \widetilde{M} \to M$,
this is a fiber bundle over $M$. Furthermore, since the action of $\R$
on on $\widetilde{M} \times \R$ commutes with the $\pi_1(M)$ action on $\widetilde{M}$,
we know that this is a principal bundle with structure group $\Gamma_\R$, where
$\Gamma_\R$ is a central extension of $\pi_1(M)$ by $\R$. Let $J$ denote the
element of $\R \subset \Gamma_\R$ corresponding to $1 \in \R$. Then consider
$M$ as the quotient of the $2g$-gon. The holonomy about the path traversed by
the boundary is exactly the product $\prod_i [a_i,b_i]$ of the commutators of
representatives of generators of $\pi_1(M)$, and has holonomy equal to $2\pi$, which
follows from the fact that the holonomy about any loop bounding a disk is equal to the
integral of the curvature, and the fact that $c_1(Q)$ is represented by
the curvature class of any connection, divided by $2\pi i$. Therefore, if we
consider the pullback connection on $\widetilde{M} \times U(1)$, the holonomy
about the lifts of the boundary paths to $\widetilde{M}$ will also be $2\pi$,
which then lifts to translation by $1$ in the bundle $\widetilde{M} \times \R$.
This gives the relation that $\prod_i [a_i,b_i] = J$, which gives us a presentation
of the group $\Gamma_\R$. Since $\pi_1(M)$ is discrete, $\widetilde{M}$ is a flat
bundle, so the pullback connection on $\widetilde{M} \times U(1)$ still has curvature
$-2\pi i \omega$, and the curvature also remains unchanged after lifting to
$\widetilde{M} \times \R$. \\

Suppose we have a homomorphism $\rho : \Gamma_\R \to G$ to a compact group $G$.
This then gives us an associated bundle
$P = (\widetilde{M}\times \R) \times_{\Gamma_\R} G$, which is a principal $G$ bundle
In addition to $\rho$, we get a Lie algebra homomorphism
$\dt{\rho} : \R \to \g$. Using this, $\dt{\rho}(A) \in \Omega^1_P(\g)$ determines
a connection on $P$, which has curvature $\dt{\rho}(F_A)$. Furthermore, we have that
\[
\dt{\rho}(d_A\star F_A) = d_{\dt{\rho}(A)}\star\dt{\rho}(F_A)
\]
which tells us that $\dt{\rho}(A)$ is a Yang-Mills connection on $P$. The main theorem
is that every Yang-Mills connection on every principal bundle arises in this way.
%
\begin{thm}
The above construction gives a bijective correspondence
\[
\hom(\Gamma_\R, G) / G \longleftrightarrow
\set{G\text{-bundles } P \to M
\text{ equipped with a Yang-Mills connection } A }/\mathscr{G}(P)
\]
Where the action of $G$ on $\hom(\Gamma_\R,G)$ is by conjugation.
\end{thm}
%
\begin{proof}
There are several things here to show. First, we must show that for any compact
group $G$, any Yang-Mills connection on an associated bundle comes from such a
homomorphism. Furthermore, we also must show that for any compact
group $G$, every principal $G$-bundle can be realized as an associated bundle
of $\widetilde{M} \times \R \to M$. \\

We first tackle the first claim. Since $M$ is $2$-dimensional, we have that
$\star F_A \in \Omega^0_M(\g_P)$, so we may regard it as a $G$-equivariant map
$P \to \g$, i.e. $\star F_A(p\cdot g) = \Ad_{g\inv}\star F_A(p)$. This then
tells us that the image of $\star F_A$ is exactly one orbit of $\g$ under the adjoint
action.  Fix a nonzero element $X \in \g$ lying in the image of $\star F$, and then
consider the preimage $P_X \defeq (\star F_A)\inv(X)$. Let $G_X \subset G$ be the
stabilizer of $X$ under the adjoint action. Then $G_X$ acts on $P_X$, since given any
$p \in P_X$, and $g \in G_X$, we have that $\star F_A(p\cdot g) = \Ad_{g\inv} F_A(p) = X$.
This action is clearly transitive and free, so $P_X$ defines a reduction of structure
group from $G$ to $G_X$, giving us a bundle isomorphism $P_X \times_{G_X} G \cong P$.
Furthermore, since $d_A \star F = 0$, we have that $\star F$ is constant in
the horizontal directions, so the differential of $\star F$ vanishes in the horizontal
directions, so the horizontal distribution is contained in the tangent bundle
of $P_X$. Therefore, the connection $A$ on $P$ restricts to a Yang-Mills connection on
$P_X$ (which we also denote $A$). This restricted connection has the property
that $\star F_A$ is the constant map with value $X \in \g$, so $F_A$ is
just the $2$-form $X \otimes \omega \in \Omega^2_M(\g_{P_X})$. This tells us that
every Yang-Mills connection on any bundle $P$ arises from such a connection on
the reduced bundle $P_X$ for some $X \in \g$. \\

Then suppose we have a homomorphism $\rho : \Gamma_\R \to G$ with derivative
$\dt{\rho} : \R \to \g$. The image of $1 \in \R$ under $\dt{\rho}$ determines an
element $X_\rho \in \g$. Centrality of $\R$ then implies that
the image of $\Gamma_\R$ preserves $X_\rho$ under the adjoint action, so we may
regard $\rho$ as a homomorphism $\Gamma_\R \to G_{X_\rho}$. Combining this observation
with the previous one, we can then reduce to the case where $X$ is preserved
by all of $G$ (i.e. $G = G_X$). \\

To complete the proof of the first claim, we want to reduce to the cases where
$G$ is either a torus or a semisimple group. This follows from the fact
that any compact group $G$ arises as $H \times_D S$, where $H$ is a maximal torus
and $S = [G,G]$ is the maximal connected semisimple subgroup, and $D = H \cap S$ is
a finite subgroup of the center of $S$. Quotienting by $D$, we get a finite covering
$G \to \overline{G} = \overline{H} \times \overline{S}$, where
$\overline{H} = H/D$ and $\overline{S} = S/D$. We then claim that we can reduce to
the case where the structure group is $\overline{G}$. To see this, we note that
if we quotient $P$ by the action of $D$ to obtain $\overline{P}$, we get a finite
sheeted covering $P \to \overline{P}$. Since this covering is a local diffeomorphism,
we get an identification $TP \cong \pi^*T\overline{P}$ where $\pi : P \to \overline{P}$
is the covering projection. Therefore, we get that the horizontal distribution on
$P$ induces a horizontal distribution on $\overline{P}$, and conversely, we get
that a connection on $\overline{P}$ lifts to a horizontal distribution on $P$.
\end{proof}
%
\section{The Symplectic Viewpoint}
%
Even though the space $\mathscr{A}(P)$ is infinite dimensional, it has enough structure
to be viewed as a symplectic ``manifold," but we will gloss over the formal details.
Because $\mathcal{A}(P)$ is an affine space over $\Omega_M^1(\g_P)$, we may
work with it as a manifold, where the tangent space at any point is $\Omega_M^1(\g_P)$.
We will be relatively cavalier with the details, though all we are doing can be made
formal by passing to Sobolev completions of spaces of sections. \\

In the case that $M$ is a Riemann surface, then the Hodge star
$\star : \Omega^1_M(\g_P) \to \Omega^1_M(\g_P)$ can be viewed as a complex
structure on $\mathscr{A}(P)$. In addition, after fixing a Riemannian metric on $M$,
we get a trivialization $\Omega_M^2 \cong \R$ using the natural orientation induced
by the complex structure. This allows us to view the pairing
\[
(\omega,\eta) \mapsto \int_M \langle\omega,\eta\rangle
\]
as a symplectic form on $\mathscr{A}(P)$. In addition, these structures are
visibly compatible, which gives $\mathscr{A}(P)$ a K\"ahler structure. \\

Recall that if we have a symplectic left action of a group $G$ on a symplectic manifold
$(M,\omega)$ we get an induced map $\g \to \mathfrak{X}(M)$ mapping $\xi$ to the vector
field $X_\xi$ defined by
\[
(X_\xi)_p = \frac{d}{dt}\bigg\vert_{t=0} \exp(t\xi)\cdot p
\]
The action is \ib{Hamiltonian} if for all $\xi \in \g$ there exists a function
$H_\xi : M \to \R$ called a \ib{Hamiltonian function} such that the vector field
$X_\xi$ satisfies the identity
\[
\omega_p((X_\xi)_p, v) = (dH_\xi)_p v
\]
for all points $p \in M$ and tangent vectors $v \in T_pM$, and the mapping
$\xi \mapsto H_\xi$ is $G$-equivariant with respect to the right actions
$\xi\cdot g= \Ad_{g\inv}\xi$ and $f \cdot g = f \circ L_g$, where $L_g : M \to M$
is the symplectomorphism determined by left multiplication by $g$. Given a
Hamiltonian action of $G$ on $M$, a \ib{moment map} for the action is a map
$\mu : M \to \g^*$ such that for any $p \in M$ and $\xi \in \g$, we have
\[
H_\xi(p) = \mu(p)((X_\xi)_p)
\]
We claim that the action of $\mathscr{G}(P)$ on $\mathscr{A}(P)$ is Hamiltonian.
We first note that the action is symplectic, since
$\langle\cdot,\cdot\rangle$ is $\Ad$-invariant, and the action of
a gauge transformation $\varphi$ on a tangent vector $\eta \in \Omega^1_M(\g_P)$
is by $\varphi\cdot\eta = \Ad_{g_\varphi\inv}\eta$ where
$g_\varphi : P \to G$ is the associated $G$-equivariant map. To show that the
action is Hamiltonian, we note that each $\phi \in \Omega^0_M(\g_P)$ determines
a map $H_\phi : \mathscr{A}(P) \to \R$ given by
\[
H_\phi(A) = \int_M \langle F_A,\phi\rangle
\]
and the mapping $\phi \mapsto H_\phi$ is clearly $\mathscr{G}(P)$-equivariant.
We then claim that mapping $\mu(A) = F_A$ defines a moment map. We see this, we first
note that the map is $\mathscr{G}(P)$-equivariant by our formulas for how the connection
and its curvature transform under a gauge transformation. We then compute
\begin{align*}
d(H_\phi)_A(\psi) &= \frac{d}{dt}\bigg\vert_{t=0}\int_M\langle F_{A+t\psi},\phi\rangle \\
&= \int_M \langle d_A\psi,\phi\rangle \\
&= \int_M d\langle\psi,\phi\rangle - \int_M \langle \psi,d_A\phi\rangle \\
&= \int_M\langle d_A\phi,\psi\rangle
\end{align*}
%
Then noting that $d_A\phi$ is the vector field determined by $\phi$, this
shows that $\mu(A) = F_A$ is a moment map for the action.
%
\end{document}