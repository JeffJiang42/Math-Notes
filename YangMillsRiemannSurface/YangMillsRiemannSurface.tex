\documentclass[psamsfonts, 12pt]{amsart}
%
%-------Packages---------
%
\usepackage[h margin=1 in, v margin=1 in]{geometry}
\usepackage{amssymb,amsfonts}
\usepackage{amsmath}
\usepackage{accents}
\usepackage[all,arc]{xy}
\usepackage{tikz-cd}
\usepackage{enumerate}
\usepackage{mathrsfs}
\usepackage{amsthm}
\usepackage{mathpazo}
\usepackage{float}
\usepackage[backend=biber]{biblatex}
\addbibresource{bibliography.bib}
%\usepackage{charter} %another font
%\usepackage{eulervm} %Vakil font
\usepackage{yfonts}
\usepackage{mathtools}
\usepackage{enumitem}
\usepackage{mathrsfs}
\usepackage{fourier-orns}
\usepackage[all]{xy}
\usepackage{hyperref}
\usepackage{url}
\usepackage{mathtools}
\usepackage{graphicx}
\usepackage{pdfsync}
\usepackage{mathdots}
\usepackage{calligra}
\usepackage{import}
\usepackage{xifthen}
\usepackage{pdfpages}
\usepackage{transparent}

\usepackage{tgpagella}
\usepackage[T1]{fontenc}
%
\usepackage{listings}
\usepackage{color}

\definecolor{dkgreen}{rgb}{0,0.6,0}
\definecolor{gray}{rgb}{0.5,0.5,0.5}
\definecolor{mauve}{rgb}{0.58,0,0.82}

\lstset{frame=tb,
  language=Matlab,
  aboveskip=3mm,
  belowskip=3mm,
  showstringspaces=false,
  columns=flexible,
  basicstyle={\small\ttfamily},
  numbers=none,
  numberstyle=\tiny\color{gray},
  keywordstyle=\color{blue},
  commentstyle=\color{dkgreen},
  stringstyle=\color{mauve},
  breaklines=true,
  breakatwhitespace=true,
  tabsize=3
  }
%
%--------Theorem Environments--------
%
\newtheorem{thm}{Theorem}[section]
\newtheorem*{thm*}{Theorem}
\newtheorem{cor}[thm]{Corollary}
\newtheorem{prop}[thm]{Proposition}
\newtheorem{lem}[thm]{Lemma}
\newtheorem*{lem*}{Lemma}
\newtheorem{conj}[thm]{Conjecture}
\newtheorem{quest}[thm]{Question}
%
\theoremstyle{definition}
\newtheorem{defn}[thm]{Definition}
\newtheorem*{defn*}{Definition}
\newtheorem{defns}[thm]{Definitions}
\newtheorem{con}[thm]{Construction}
\newtheorem{exmp}[thm]{Example}
\newtheorem{exmps}[thm]{Examples}
\newtheorem{notn}[thm]{Notation}
\newtheorem{notns}[thm]{Notations}
\newtheorem{addm}[thm]{Addendum}
\newtheorem{exer}[thm]{Exercise}
%
\theoremstyle{remark}
\newtheorem{rem}[thm]{Remark}
\newtheorem*{claim}{Claim}
\newtheorem*{aside*}{Aside}
\newtheorem*{rem*}{Remark}
\newtheorem*{hint*}{Hint}
\newtheorem*{note}{Note}
\newtheorem{rems}[thm]{Remarks}
\newtheorem{warn}[thm]{Warning}
\newtheorem{sch}[thm]{Scholium}
%
%--------Macros--------
\renewcommand{\qedsymbol}{$\blacksquare$}
\renewcommand{\sl}{\mathfrak{sl}}
\newcommand{\Bord}{\mathsf{Bord}}
\renewcommand{\hom}{\mathsf{Hom}}
\renewcommand{\emptyset}{\varnothing}
\renewcommand{\O}{\mathcal{O}}
\newcommand{\R}{\mathbb{R}}
\newcommand{\ib}[1]{\textbf{\textit{#1}}}
\newcommand{\Q}{\mathbb{Q}}
\newcommand{\Z}{\mathbb{Z}}
\newcommand{\N}{\mathbb{N}}
\newcommand{\C}{\mathbb{C}}
\newcommand{\A}{\mathbb{A}}
\newcommand{\F}{\mathbb{F}}
\newcommand{\M}{\mathcal{M}}
\newcommand{\dbar}{\overline{\partial}}
\newcommand{\zbar}{\overline{z}}
\renewcommand{\S}{\mathbb{S}}
\newcommand{\V}{\vec{v}}
\newcommand{\RP}{\mathbb{RP}}
\newcommand{\CP}{\mathbb{CP}}
\newcommand{\B}{\mathcal{B}}
\newcommand{\GL}{\mathrm{GL}}
\newcommand{\SL}{\mathrm{SL}}
\newcommand{\SP}{\mathrm{SP}}
\newcommand{\SO}{\mathrm{SO}}
\newcommand{\SU}{\mathrm{SU}}
\newcommand{\gl}{\mathfrak{gl}}
\newcommand{\g}{\mathfrak{g}}
\newcommand{\Bun}{\mathsf{Bun}}
\newcommand*{\dt}[1]{%
   \accentset{\mbox{\large\bfseries .}}{#1}}
\newcommand{\inv}{^{-1}}
\newcommand{\bra}[2]{ \left[ #1, #2 \right] }
\newcommand{\set}[1]{\left\lbrace #1 \right\rbrace}
\newcommand{\abs}[1]{\left\lvert#1\right\rvert}
\newcommand{\norm}[1]{\left\lVert#1\right\rVert}
\newcommand{\transv}{\mathrel{\text{\tpitchfork}}}
\newcommand{\defeq}{\vcentcolon=}
\newcommand{\enumbreak}{\ \\ \vspace{-\baselineskip}}
\let\oldexists\exists
\renewcommand\exists{\oldexists~}
\let\oldL\L
\renewcommand\L{\mathfrak{L}}
\makeatletter
\newcommand{\incfig}[2]{%
    \fontsize{48pt}{50pt}\selectfont
    \def\svgwidth{\columnwidth}
    \scalebox{#2}{\input{#1.pdf_tex}}
}
%
\newcommand{\tpitchfork}{%
  \vbox{
    \baselineskip\z@skip
    \lineskip-.52ex
    \lineskiplimit\maxdimen
    \m@th
    \ialign{##\crcr\hidewidth\smash{$-$}\hidewidth\crcr$\pitchfork$\crcr}
  }%
}
\makeatother
\newcommand{\bd}{\partial}
\newcommand{\lang}{\begin{picture}(5,7)
\put(1.1,2.5){\rotatebox{45}{\line(1,0){6.0}}}
\put(1.1,2.5){\rotatebox{315}{\line(1,0){6.0}}}
\end{picture}}
\newcommand{\rang}{\begin{picture}(5,7)
\put(.1,2.5){\rotatebox{135}{\line(1,0){6.0}}}
\put(.1,2.5){\rotatebox{225}{\line(1,0){6.0}}}
\end{picture}}
\DeclareMathOperator{\id}{id}
\DeclareMathOperator{\im}{Im}
\DeclareMathOperator{\codim}{codim}
\DeclareMathOperator{\coker}{coker}
\DeclareMathOperator{\supp}{supp}
\DeclareMathOperator{\inter}{Int}
\DeclareMathOperator{\sign}{sign}
\DeclareMathOperator{\sgn}{sgn}
\DeclareMathOperator{\indx}{ind}
\DeclareMathOperator{\alt}{Alt}
\DeclareMathOperator{\Aut}{Aut}
\DeclareMathOperator{\trace}{trace}
\DeclareMathOperator{\ad}{ad}
\DeclareMathOperator{\End}{End}
\DeclareMathOperator{\Ad}{Ad}
\DeclareMathOperator{\Lie}{Lie}
\DeclareMathOperator{\spn}{span}
\DeclareMathOperator{\dv}{div}
\DeclareMathOperator{\grad}{grad}
\DeclareMathOperator{\Sym}{Sym}
\DeclareMathOperator{\tr}{tr}
\DeclareMathOperator{\sheafhom}{\mathscr{H}\text{\kern -3pt {\calligra\large om}}\,}
\newcommand*\myhrulefill{%
   \leavevmode\leaders\hrule depth-2pt height 2.4pt\hfill\kern0pt}
\newcommand\niceending[1]{%
  \begin{center}%
    \LARGE \myhrulefill \hspace{0.2cm} #1 \hspace{0.2cm} \myhrulefill%
  \end{center}}
\newcommand*\sectionend{\niceending{\decofourleft\decofourright}}
\newcommand*\subsectionend{\niceending{\decosix}}
\def\upint{\mathchoice%
    {\mkern13mu\overline{\vphantom{\intop}\mkern7mu}\mkern-20mu}%
    {\mkern7mu\overline{\vphantom{\intop}\mkern7mu}\mkern-14mu}%
    {\mkern7mu\overline{\vphantom{\intop}\mkern7mu}\mkern-14mu}%
    {\mkern7mu\overline{\vphantom{\intop}\mkern7mu}\mkern-14mu}%
  \int}
\def\lowint{\mkern3mu\underline{\vphantom{\intop}\mkern7mu}\mkern-10mu\int}
%
%--------Hypersetup--------
%
\hypersetup{
    colorlinks,
    citecolor=black,
    filecolor=black,
    linkcolor=blue
}
%
%--------Solution--------
%
\newenvironment{solution}
  {\begin{proof}[Solution]}
  {\end{proof}}
%
%--------Graphics--------
%
%\graphicspath{ {images/} }
%
\begin{document}
%
\author{Jeffrey Jiang}
%
\title{Yang-Mills on Riemann Surfaces}
%
\maketitle
%
These are notes I've made while working through the paper \emph{The Yang-Mills
Equations over Riemann Surfaces} \cite{10.2307/37156} by Atiyah and Bott.
%
\tableofcontents
%
\section{Preliminary Setup}
%
To discuss the Yang-Mills functional, we must first fix some data. The setup will
consist of the following ingredients
\begin{enumerate}
  \item A compact manifold $M$.
  \item A compact connected Lie group $G$.
  \item A principal $G$-bundle $P \to M$.
\end{enumerate}
%
With this data, we have two associated bundles
\begin{align*}
\Ad P &\defeq P \times_G G \\
\g_P &\defeq P \times_G \g
\end{align*}
%
Where the action of $G$ on $G$ is by conjugation, and the action of $G$ on $\g$
is the adjoint action. We note that these bundles both contain additional structure --
$\Ad P$ is a bundle of groups (not a principal bundle), and $\g_P$ is a bundle of Lie
algebras. The space of sections $\Gamma(M, \Ad P)$ has
a natural group structure given by pointwise multiplication, and is called the
\ib{gauge group} $\mathscr{G}(P)$. Likewise, the space of sections
$\Gamma(M, \g_P)$ has a natural Lie algebra structure given by the pointwise Lie
bracket, and can be naturally identified with the Lie algebra of $\mathscr{G}(P)$,
as we shall see. An alternate characterization of these spaces of sections comes from
a general characterization of sections of associated bundles
%
\begin{prop}
We have natural correspondences
\begin{align*}
\Gamma(M, \Ad P) &\longleftrightarrow \set{f : P \to G ~:~ f(p\cdot g) = g\inv f(p)g} \\
\Gamma(M, \g_P) &\longleftrightarrow \set{f : P \to \g ~:~ f(p\cdot g) = \Ad_{g\inv}f(p)}
\end{align*}
\end{prop}
%
From the above correspondence we a group isomorphism
$\mathscr{G}(P) \to \Aut(P)$, where $\Aut(P)$ denotes the group of
$G$-equivariant diffeomorphisms $\varphi : P \to P$ such that
the following diagram commutes :
\[\begin{tikzcd}
P \ar[dr] \ar[rr, "\varphi"] && P \ar[dl] \\
& M
\end{tikzcd}\]
The isomorphism is given by mapping a $G$-equivariant map $f : P \to g$
to the automorphism $\varphi_f : P \to P$ defined by $\varphi_f(p) = p\cdot f(p)$.
We get a similar identification for $\Gamma(M, \g_P)$. Regarding a section
$\phi \in \Gamma(M,\g_P)$ as a $G$-equivariant map $P \to \g$, we get
a bundle automorphism $\exp(\phi)$ defined by $p\mapsto p\cdot\exp(\phi(p))$. From
this perspective, we see that $\Gamma(M,\g_P)$ can be identified with the $G$-invariant
vertical vector fields on $P$, which are the infinitesimal generators of the action
of $\mathscr{G}(P)$ on $P$. \\

In the the case of $\Gamma(M,\g_P)$, we can extend the correspondence to the spaces of
$\g_P$ valued forms. The kernel of the differential of the projection $P \to M$ gives a
subbundle of $TP$, which has a natural identification with the trivial bundle
$\underline{\g} = P \times \g$. Then we can identify the space of sections of
$\Lambda^kT^*M \otimes \g_P$, (i.e. the space $\Omega^k_M(\g_p)$ of $\g_p$ valued
$k$-forms with a subspace of the space $\Omega^k_P(\g)$ of $\g$-valued $k$-forms $\omega$
on $P$ satisfying :
\begin{enumerate}
  \item $R_g^*\omega = \Ad_{g\inv}\omega$, where $R_g : P \to P$ denotes the right
  action of $g \in G$.
  \item $\iota_\xi\omega = 0$ for any $\xi \in \g$, where $\iota$ denotes interior
  multiplication, and we identify $\xi$ with the constant vector field
  $\xi$ under the identification of the vertical space with $\g$.
\end{enumerate}
%
We have maps
\begin{align*}
\Omega^p_M(\g_p) \otimes \Omega^q_M(\g_P) &\to \Omega^{p+q}_M(\g_p \otimes \g_p) \\
(\omega_1 \otimes \xi_1) \otimes (\omega_2 \otimes \xi_2)
&\mapsto (\omega_1 \wedge \omega_2) \otimes (\xi_1 \otimes \xi_2)
\end{align*}
From now on, we will usually omit the tensor symbol for $\g_P$-valued forms in favor of
juxtaposition, i.e. we write $\omega\xi$ instead of $\omega \otimes \xi$. Using the Lie
bracket, we then get
\begin{align*}
\Omega^p_M(\g_p) \otimes \Omega^q_M(\g_P) &\to \Omega^{p+q}_M(\g_p) \\
\omega \otimes \eta \mapsto [\omega,\eta]
\end{align*}
We note that this is \emph{not} skew-symmetric, instead, given
$\omega \in \Omega^p_M(\g_P)$  and $\eta \in \Omega^q_M(\g_P)$, we have
\[
[\omega,\eta] = (-1)^{pq+1} [\eta,\omega]
\]
For any semisimple Lie group $G$ (in particular, for any compact Lie group $G$), we have
an inner product $\langle\cdot,\cdot\rangle : \g \times \g \to \R$ that is invariant
under the adjoint action (e.g. the Killing form). Invariance under the adjoint action
gives us
\begin{align*}
\langle [\xi_1,\xi_2],\xi_3 \rangle &= \langle[-\xi_2,\xi_1],\xi_3 \rangle \\
&= \frac{d}{dt}\bigg\vert_{t=0}\langle \Ad_{\exp(-t\xi_2)}\xi_1,\xi_3 \\
&= \frac{d}{dt}\bigg\vert_{t=}\langle \Ad_{\exp(t\xi_2)}\Ad_{\exp(-t\xi_2)}\xi_1,
\Ad_{\exp(t\xi_2)}\xi_3\rangle \\
&= \langle \xi_1,[\xi_2,\xi_3]\rangle
\end{align*}

Fixing one such inner product induces a fiber product on the trivial bundle
$P \times \g$, and invariance guarantees that this descends to a fiber product on
$\g_P$. This give us pairings
\begin{align*}
\Omega^p_M(\g_P) \otimes \Omega^q_M(\g_P) &\to \Omega^{p+q}_M \\
\omega \otimes \eta &\mapsto \langle \omega, \eta \rangle
\end{align*}
%
which also satisfies the identity
\[
\langle[\omega,\eta],\xi\rangle = \langle \omega,[\eta,\xi]\rangle
\]
We again note that this is not symmetric or skew symmetric, and instead behaves
like the wedge product, i.e. for $\omega \in \Omega^p_M(\g_P)$ and
$\eta \in \Omega^q_M(\g_P)$, we have
\[
\langle \omega,\eta \rangle = (-1)^{pq}\langle \eta,\omega\rangle
\]
which can be seen by writing $\omega = \omega^i\xi_i$ and $\eta = \eta^i\xi_i$
for an orthonormal basis $\set{\xi_i}$ for $\g$.
%
We then fix an orientation and Riemannian metric on $M$, which gives us a Hodge
star operator $\star : \Omega^p_M \to \Omega^{n-p}_M$ and a Riemannian volume form
$dV_g$. The Hodge star extends to $\g_P$-valued $k$-forms, where given
$\omega \in \Omega^p_M$ and $\xi \in \Gamma(M,\g_P)$, we define
$\star(\omega\xi) = (\star\omega)\xi$. Then given
$\omega_1\xi_1,\omega_2\xi_2 \in \Omega^p_M(\g_P)$, we have
\[
\langle\omega_1\xi_1,\star\omega_2\xi_2\rangle =
\langle \omega_1,\omega_2\rangle_g\langle\xi_1,\xi_2\rangle
\]
where $\langle\cdot,\cdot\rangle_g$ denotes the fiber metric on $\Lambda^pT^*M$
induced by $g$. This gives us an inner product on each $\Omega^p_M(\g_P)$ defined
by
\[
(\theta,\varphi) = \int_M \langle \theta,\star\varphi\rangle
\]
Which gives us the $L^2$ norm on $\Omega^p_M(\g_P)$ with $\norm{F}_{L^2}^2 = (F,F)$.
\begin{defn}
A connection on a principal bundle $\pi : P \to M$ is a choice of $G$-invariant splitting
of the exact sequence of vector bundles over $P$
\[\begin{tikzcd}
0 \ar[r] & \underline{\g} \ar[r] & TP \ar[r] & \pi^*TM \ar[r] & 0
\end{tikzcd}\]
i.e. a distribution $H \subset TP$ such that
\begin{enumerate}
  \item $(R_g)_*H_p = H_{p\cdot g}$
  \item $H \oplus \underline{\g} = TP$
\end{enumerate}
Equivalently, it is the data of a $\g$-valued $1$-form $A \in \Omega^1_P(\g)$
satisfying
\begin{enumerate}
  \item $R_g^*A = \Ad_{g\inv} A$
  \item $\iota_\xi A = \xi$ for all $\xi \in \g$.
\end{enumerate}
\end{defn}
%
Note in particular that by a dimension count, we have that $\pi_*\vert_H : H \to TM$ is
an isomorphism. This implies that given a tangent vector $v$ at $x$ and a point
$p \in P$ in the fiber over $x$, we get a unique horizontal lift
$\tilde{v} \in H_p$. For a fixed principal $G$-bundle $\pi : P \to M$, we let
$\mathscr{A}(P)$ denote the space of all connections on $P$, which is an affine space
over $\Omega^1_M(\g_P)$. The connection form $A$ on $P$ induces an exterior covariant
derivative on any associated vector bundle $E = P \times_G V$ arising from a linear
representation $\rho : G \to \GL(V)$. Let $\dt{\rho} : \g \to \End(V)$ be the
derivative of $\rho$ at the identity. Then the exterior covariant derivative is given
by
\begin{align*}
d_A : \Omega^p_M(E) &\to \Omega^{p+1}_M(E) \\
\psi &\mapsto d\psi + \dt{\rho}(A) \wedge \psi
\end{align*}
%
In particular, we get an exterior covariant derivative on $\g_P$, which is given by
\[
d_A\psi = d\psi + [A,\psi]
\]
\begin{prop}
Let $\phi \in \Omega^p_M(\g_P)$ and $\psi \in \Omega^q_M(\g_P)$. Then
\[
d\langle\phi,\psi\rangle = \langle d_A\phi,\psi\rangle
+ (-1)^p\langle \phi,d_A\psi\rangle
\]
\end{prop}
%
\begin{proof}
We compute
\begin{align*}
\langle d_A\phi,\psi\rangle + (-1)^p\langle\phi,d_A\psi\rangle
&= \langle d\phi,\psi\rangle + \langle[A,\phi],\psi\rangle
+ (-1)^p (\langle \phi,d\psi\rangle + \langle \phi,[A,\psi]) \\
&= \langle d\phi,\psi\rangle + \langle[A,\phi],\psi\rangle
+ (-1)^p (\langle\phi,d\psi\rangle + \langle[\phi,A],\psi) \\
&= \langle d\phi,\psi\rangle + \langle [A,\phi],\psi\rangle
+ (-1)^{2p+1}\langle [A,\phi],\psi + (-1)^p\langle\phi,d\psi\rangle \\
&= \langle d\phi,\psi\rangle + (-1)^p\langle\phi,d\psi\rangle
\end{align*}
Writing $\phi = \phi^i\xi_i$ and $\psi=\psi^i\xi_i$ in an orthonormal basis
$\set{\xi_i}$ for $\g$, this becomes
\begin{align*}
\langle d_A\phi,\psi\rangle + (-1)^p\langle\phi,d_A\psi\rangle
&= \sum_i d\phi^i \wedge \psi^i + (-1)^p \phi^i\wedge d\psi^i \\
&= d\langle\phi,\psi\rangle
\end{align*}
\end{proof}
%
Given any distribution $E \subset TP$, we get a Frobenius tensor
$\phi_E : E \otimes E \to TP/E$ given by $X \otimes Y \to [X,Y] \mod E$ where
we extend $X$ and $Y$ to local vector fields. The Frobenius tensor should be
thought of as the obstruction to the existence of an integral submanifold for the
distribution $E$. In the case of a connection $H$ on a principal bundle $P \to M$,
we can extend to all of $TP$ by first projecting onto $H$, and
have an identification of $TP/H \cong \underline{\g}$, and the Frobenius
tensor is given by $X \otimes Y \mapsto A([X,Y])$, where $A$ is the connection $1$-form,
and is called the \ib{curvature form} of the connection, and is denoted $F_A$. In
terms of differential forms, we have that for horizontal vectors $\xi_1,\xi_2$ on
$TP$,
\[
dA(\xi_1,\xi_2) = \xi_1A(\xi_2) - \xi_2A(\xi_1) - A([\xi_1,\xi_2])
\]
The fact that $\xi_1$ and $\xi_2$ are horizontal implies that they are in the
kernel of $A$, which gives us $dA(\xi_1,\xi_2) = -F_A(\xi_1,\xi_2)$. We also know
that $F_A$ vanishes on vertical vectors, and since $A(X) = X$ for $X \in \g$,
we get that
\[
dA + \frac{1}{2}[A,A] = -F_A\footnote{Our convention for the sign of the curvature
is opposite from many other conventions, which usually sets
$F_A = dA + \frac{1}{2}[A,A]$}
\]
It can be shown that $F_A$ transforms by the adjoint action under pullback, and
vanishes on vertical vectors, so it descends to a $\g_P$-valued $2$-form on the base
manifold $M$. \\

Another thing to note is that there is a natural right action of the gauge group
$\mathscr{G}(P)$ on the space of connections $\mathscr{A}(P)$. Interpreting the
elements of $\mathscr{G}(P)$ as bundle automorphisms $\varphi : P \to P$ and
elements of $\mathscr{A}(P)$ as $\g$-valued $1$-forms $A$ on $P$, the action
is simply pullback, $(\varphi,A) \mapsto \varphi^*A$. To show that
this defines an action, we must check that $\varphi^*A$ satisfies the conditions
%
\begin{enumerate}
  \item $R_g^*\varphi^*A = \Ad_{g\inv} \varphi^*A$
  \item $\iota_\xi \varphi^*A = \xi$ for all $\xi \in \g$.
\end{enumerate}
%
Which are all simple consequences of the $G$-equivariance of $\varphi$ and
the transformation law for $A$. For a specific formula, let
$\varphi : P \to P$ be an element of the gauge group, and let
$g_\varphi : P \to G$ be its associated $G$-equivariant map. Then
\[
\varphi^*A = \Ad_{g_\varphi\inv} A + g_\varphi^*\theta
\]
where $\theta \in \Omega^1_G(\g)$ denotes the \ib{Maurer-Cartan form}
\[
\theta_g(v) = (dL_{g\inv})_g(v)
\]
which satisfies the \ib{Maurer-Cartan equation}
\[
d\theta + \frac{1}{2}[\theta,\theta] = 0
\]
%
\begin{prop}
Let $A \in \mathscr{A}(P)$ be a connection and $\varphi : P \to P$ an element
of $\mathscr{G}(P)$ with associated $G$-equivariant map $g_\varphi : P \to G$.
Then
\[
F_{\varphi^*A} = \Ad_{g_\varphi\inv} F_A
\]
\end{prop}
%
\begin{proof}
Using the transformation law for $\varphi^*A$ we compute
\begin{align*}
F_{\varphi^*A} &= d(\Ad_{g_\varphi\inv} A + g_\varphi^*\theta)
+ \frac{1}{2}[\Ad_{g_\varphi\inv} A + g_\varphi^*\theta,
\Ad_{g_\varphi\inv} A + g_\varphi^*\theta] \\
&= \Ad_{g_\varphi\inv}dA + g_\varphi^*d\theta + \frac{1}{2}
\left([\Ad_{g_\varphi\inv}A, \Ad_{g_\varphi\inv} A] + [\Ad_{g_\varphi\inv} A,
g_\varphi^*\theta] + [g^*_\varphi\theta, \Ad_{g_\varphi\inv}]
+ [g_\varphi^*\theta,g_\varphi^*\theta] \right) \\
&= \Ad_{g_\varphi\inv} dA + \frac{1}{2}[\Ad_{g_\varphi\inv}A, \Ad_{g_\varphi\inv} A]]
\end{align*}
Where we use skew-symmetry and the Maurer-Cartan equation.
\end{proof}
%
We similarly compute the infinitesimal action of the Lie algebra $\Gamma(M,\g_P)$.
%
\begin{prop}
The vector field corresponding to $\phi \in \Gamma(M,\g_P)$ is
$A \mapsto d_A\phi \in \Omega_M^1(\g_P)$
\end{prop}
%
\begin{proof}
We compute the vector field at a connection $A \in \mathscr{A}(P)$ to be
\begin{align*}
&\frac{d}{dt}\bigg\vert_{t=0} \Ad_{\exp(t\phi)\inv} A + \exp(t\phi)^*\theta
= -[\phi, A] + \frac{d}{dt}\bigg\vert_{t=0}(dL_{\exp(-t\phi)} d(\exp(t\phi))) \\
&= [A,\phi] + \left(\frac{d}{dt}\bigg\vert_{t=0} dL_{\exp(-t\phi)}\right)d(\exp(0))
+ dL_{\exp(0)}\left(\frac{d}{dt}\bigg\vert_{t=0}d(\exp(t\phi))\right) \\[5pt]
&= [A,\phi] + d\phi\\
&= d_A\phi
\end{align*}
where for the third equality we use the product rule, and in the fourth equality
we use the fact that $\exp(0) = \id$ and that the derivative of
$\exp(t\phi)$ as $t \to 0$ is $\phi$.
\end{proof}
%
For any other connection $A + \eta$ with $\eta \in \Omega^1_M(\g_P)$,
a quick computation yields
\[
F_{A+\eta} = F_A + \frac{1}{2}[\eta,\eta] + d_A\eta
\]
From this description, we can relate the curvature $F_A$ with the covariant derivative.
Note that for the line of connections $A + t\eta$, we have that
\[
\frac{d}{dt}\bigg\vert_{t=0} F_{a+t\eta} =
\frac{d}{dt}\bigg\vert_{t=0} F_A + \frac{t^2}{2}[\eta,\eta] + td_A\eta
= d_A\eta
\]
So $d_A\eta$ measures the infinitesimal change of the curvature $F_A$ in the direction
$\eta$.
%
\section{The Yang-Mills Functional}
%
With the setup done, we have the ingredients necessary to define the Yang-Mills
functional.
%
\begin{defn}
The \ib{Yang-Mills functional} is the map
$L : \mathscr{A}(P) \to \R$ given by
\[
L(A) = \norm{F_A}_{L^2}^2 = \int_M \langle F_A,\star F_A \rangle
\]
\end{defn}
%
We immediately see that the Yang-Mills functional is invariant under $\mathscr{G}(P)$
in the following sense -- for any gauge transformation $\varphi$ we have that
$L(\varphi^*A) = L(A)$, which follows immediately from the invariance of
$\langle\cdot,\cdot\rangle$ and the transformation law for curvature. \\

Our goal now will be to find the Euler-Lagrange
equations for the Yang-Mills functional by computing the first and second variations.
Using the Hodge star operator, we construct the formal adjoint with respect to
the inner product $d_A^* : \Omega^{p}_M(\g_P) \to \Omega^{p-1}_M(\g_P)$ in the same manner
as for classical Hodge theory on a Riemannian manifold. Explicitly, the formula
on $p$-forms is given by
\[
d^*_A = (-1)^{n(p+1) + 1}\star d_A \star
\]
where $n = \dim M$. We then compute the first variation of $L$.
%
\begin{prop}[\ib{The First Variation}]
For a local extremum $A \in \mathscr{A}(P)$ of the Yang-Mills functional, we have
\[
d_A\star F_A = 0
\]
The local extremum connection $A$ is then called a \ib{Yang-Mills connection}, and the
space of Yang-Mills connections is denoted $\mathscr{A}_{\mathrm{YM}}(P)$.
\end{prop}
%
\begin{proof}
Consider a variation $A + t\eta$ with $t \in \R$ and $\eta \in \Omega^1_M(\g_P)$.
We have that the curvature is given by
\[
F_{A+t\eta} = F_A + \frac{t^2}{2}[\eta,\eta] + td_A\eta
\]
This then gives us
\begin{align*}
\norm{F_{A+t\eta}}_{L^2} &= \int_M \langle F_{A+t\eta},F_{A+t\eta}\rangle \\
&= \int_M \langle F_A + \frac{t^2}{2}[\eta,\eta]
+ td_A\eta,\star (F_A + \frac{t^2}{2}[\eta,\eta] + td_A\eta)\rangle \\
\end{align*}
Expanding this out, we get that the term that is linear in $t$ is

\[
\int_M \langle F_A, \star d_A\eta\rangle + \langle d_A\eta, \star F_A \rangle
= 2(F_A,d_A\eta)
\]
where we use symmetry of $(\cdot,\cdot)$. Since $A$ is extremal, we have that
this term must vanish, giving us that $(F_A,d_A\eta) = (d^*_A F,\eta) = 0$ for
every $\eta$. Then since we have (up to sign) $d^*_A = \star d_A \star$, and
$\star$ is an isomorphism, this implies $d_A\star F_A = 0$.
\end{proof}
%
\begin{prop}[\ib{The Second Variation}]
At a Yang-Mills connection $A \in \mathscr{A}(P)$, we have
\[
d^*_A d_A\eta + \star[\eta,\star F_A] = 0
\]
\end{prop}
%
\begin{proof}
We differentiate the first variational equation with respect to $t$, i.e.
we compute
\[
\frac{d}{dt}\bigg\vert_{t=0} d_{A+t\eta}^*F_{A+t\eta}
\]
We expand out
\begin{align*}
d_{A+t\eta}^*F_{A+t\eta} &= \pm \star d_{A+t\eta}\star F_{A+t\eta} \\
&= \pm\left(\star d_A\star\left(F_A + td_A\eta + \frac{t^2}{2}[\eta,\eta]\right)
+ t\star
\left[\eta, \star\left(F_A + td_A\eta + \frac{t^2}{2}[\eta,\eta]\right)\right]\right)
\end{align*}
Taking the term linear in $t$ yields
\[
\pm\left( \star d_A\star d_A\eta + \star[\eta,\star F_A] \right)
\]
Giving us that at an extremal connection $A$, we have
\[
d^*_A d_A\eta + \star[\eta,\star F_A] = 0
\]
\end{proof}
%
The second variation should be thought of as the Hessian to the Yang-Mills functional,
which will allow us to apply Morse theory techniques to the space of connections.
%
\section{The $U(1)$ Case}
%
We first restrict to the special case $G = U(1)$. In this case, the
Lie algebra is abelian, so the adjoint action of $U(1)$ on $\mathfrak{u}(1)$ is trivial,
giving us that $\g_P$ is a trivial bundle. Identifying $\mathfrak{u}(1)$ with
$\R$, we can then identify $\mathfrak{u}(1)$ valued forms on $P$ with ordinary
differential forms. Likewise, using triviality of $\g_P$, we can identify
$\g_P$-valued forms with ordinary differential forms on $M$. The vertical bundle
in this case is a trivial line bundle over $P$, and there is a unique $U(1)$-invariant
vertical vector field on $P$, which on each fiber restricts to the vector field dual
to the Maurer-Cartan form $\theta$. Then given a connection $A$ on $P$, we have that
$dA = \pi^*F_A$, since $[A,A] = 0$. This immediately tells us that $F_A$ is closed,
since $d$ commutes with pullback. Furthermore, for any other connection $A + \eta$, we
have that
\[
F_{A+\eta} = F_A + \frac{1}{2}[\eta,\eta] + d_A\eta
\]
Then since $d_A = d$ and $[\eta,\eta] = 0$, this gives us that
$F_{A+\eta} = F_A + d\eta$, which tells us that the cohomology class of $F_A$ is
independent of our choice of $A$. Using our sign convention, this is equal to
$-2\pi i c_1(P)$. In addition, it tells us that every representative
of the curvature class can be realized as the curvature of some connection.
Furthermore, in this situation, the Yang-Mills functional reduces
to the standard Hodge theory picture, since the $L^2$ norm will coincide with the
$L^2$ norm on differential forms. Therefore, a Yang-Mills connection on $P$ is equivalent
to finding the unique connection whose curvature minimizes the $L^2$ norm in the
cohomology class $2\pi i c_1(P)$. By standard Hodge theory, there exists a unique
harmonic representative of the curvature class, and the Yang-Mills connections for $P$
are a torsor over the space $Z^1_M$ of closed $1$-forms. In total, this gives us the
fibration
\[\begin{tikzcd}
Z^1_M \ar[r] & \mathscr{A}_{\text{YM}}(P) \ar[d] \\
& 2\pi i c_1(P)
\end{tikzcd}\]
In the flat case $c_1(P) = 0$, the Yang-Mills connections
are just flat connections, which are parameterized by conjugacy classes of homomorphisms
$\pi_1(M) \to U(1)$. \\

With this information, we can construct the Yang-Mills moduli space
$\mathscr{A}_{\text{YM}}(P) / \mathscr{G}(P)$ in this simple case. Since the
conjugation action is trivial, the bundle $\mathrm{Ad}(P)$ is a trivial bundle,
so the gauge group is just $\mathscr{G}(P) = \mathrm{Map}(M,U(1))$. Given
$f : M \to U(1)$, the action of $f$ on a connection $A$ is given by
\[
A \mapsto A + \pi^*f^*\theta
\]
The gauge group acts on $Z^1_M$ in the same way, so if we fix some reference Yang-Mills
connection $A_0$ to identify $\mathscr{A}_{\text{YM}}(P)$ with $Z^1_M$, the
group actions are identified. Therefore, it suffices to compute the quotient
of $Z^1_M/\mathscr{G}(P)$. To compute this quotient, we do it in two steps,
first quotienting by the identity component $\mathscr{G}_0(P)$, and then quotienting
by the component group $\pi_0\mathscr{G}(P)$. The components of $\mathscr{G}(P)$
are simply the homotopy classes of maps $M \to S^1$, so $\mathscr{G}_0(P)$ consists
of all nullhomotopic maps $M \to S^1$. Let $dx$ denote the standard form
on $\R$. Then any nullhomotopic map $f \in \mathscr{G}_0(P)$ lifts to a function
$\tilde{f} : M \to \R$ that exponentiates to $f$. Since the Maurer-Cartan form
on $S^1$ pulls back to $dx$, we find that $f^*\theta = d\tilde{f}$. Therefore,
the action of $\mathscr{G}_0(P)$ on $Z^1_M$ is given by the addition of exact
$1$-forms, giving us that the quotient $Z^1_M / \mathscr{G}_0(P)$ is
$H^1(M,\R)$. Then since $S^1$ is a $K(\Z,1)$, we know that homotopy classes
of maps $M \to S^1$ are in bijection with $H^1(M,\Z)$. Therefore, by quotieting
by $\mathscr{G}_0(P)$ and putting everything together, we get isomorphism
\[
\mathscr{A}_{\text{YM}}(P) /\mathscr{G}(P) \cong Z^1_M/\mathscr{G}(P)
\cong \mathrm{Jac}(M) \defeq H^1(X,\R)/H^1(X,\Z)
\]
which is isomorphic to a torus $\mathbb{T}^{b_1(M)}$. However, we note that in general,
this isomorphism is highly non-canonical -- to make the identification of
$\mathscr{A}_{\text{YM}}(P)$ with $Z^1_M$, we need to fix a reference connection. In
general, there is no canonical choice of reference except in the case where $P$ is a
trivial bundle, in which case the trivial connection defines a canonical reference
connection. As we'll see soon, this reflects the fact that $\mathscr{A}_{YM}(P)$
is a $\mathrm{Jac}(M)$-torsor when $P$ is a $U(1)$-bundle.
%
%
\section{Yang-Mills Over a Riemann Surface}
%
\iffalse
The Hodge star operator maps $\Omega^1_M \to \Omega^1_M$, and satisfies
$\star^2 = -\id$, which induces an almost complex structure on $M$, giving us a
decomposition $\Omega^1_M(\C) = \Omega^{1,0}_M(\C) \oplus \Omega^{0,1}_M(\C)$
into the $\pm i$ eigenspaces of the complexified Hodge star. The operator
$\dbar \defeq \pi^{0,1} \circ d$ (where $\pi^{0,1}$ denotes projection onto
$\Omega^{0,1}_M(\C)$) satisfies $\dbar^2 = 0$ since by dimension reasons,
$\Omega^{0,2}_M(\C) = 0$, so the induced almost complex structure is integrable by the
Newlander-Nirenberg theorem. The same argument with projection onto $\Omega^{1,0}_M(\C)$
gives an operator $\partial$ satisfying $\partial^2 = 0$, and we get a decomposition
$d = \partial + \dbar$. Then given a principal bundle $P \to M$, We get a similar
decomposition for $\Omega^1_M(\g_P)$ after complexification giving a decomposition
$d_A = \partial_A + \dbar_A$ for any connection $A \in \mathscr{A}(P)$.
\fi
%TODO Identification of space of Yang-Mills connections with
%homomorphisms from central extension.
%Idea : Use reference connection on Q to make sense of a "flat part"
%Observe that the gauge transformations only act on the "flat part"
%Central element determines the topology, rest is a flat connection
%Remark of action of Jacobian on the moduli space
We now restrict our attention to when $M$ is a orientable surface, with genus $g > 0$.
Let $Q \to M$ be a principal $U(1)$ bundle with $c_1(Q) = 1$, i.e.
\[
\frac{1}{2\pi i} \int_M c_1(Q) = 1
\]
Then fix a Riemannian metric on $M$ with volume form $\omega$ such that
$\int_M \omega = 1$ and Yang-Mills connection $A$ on $Q$. Since $c_1(Q) = 1$, we have
that $[c_1(Q)] = [\omega]$. Furthermore, since $\star\omega = 1$ and
$-2\pi i c_1(Q) = [F_A]$, we get that the curvature of $A$ must be equal to
$-2\pi i \omega$ to minimize the Yang-Mills functional. Similarly, for any other
$U(1)$-bundle $P$, the curvature of a Yang-Mills connection must be
$-2\pi i c_1(P)\omega$. Then let $\widetilde{M} \to M$ be the universal cover of $M$.
Since the genus of $M$ is at least $1$, $\widetilde{M}$
is contractible, so the pullback of $Q$ along the covering projection gives us a
trivial $U(1)$ bundle over $\widetilde{M}$
\[\begin{tikzcd}
\widetilde{M} \times U(1)\ar[d] \ar[r] & Q \ar[d]\\
\widetilde{M} \ar[r] & M
\end{tikzcd}\]
%
Then we have a covering map $\widetilde{M} \times \R \to U(1)$, using the usual
covering $\R \to U(1)$, giving a principal $\R$-bundle over $\widetilde{M}$. Then
if we consider the composite map $\widetilde{M} \times \R \to \widetilde{M} \to M$,
this is a fiber bundle over $M$. Furthermore, since the action of $\R$
on on $\widetilde{M} \times \R$ commutes with the $\pi_1(M)$ action on $\widetilde{M}$,
we know that this is a principal bundle with structure group $\Gamma_\R$, where
$\Gamma_\R$ is a central extension of $\pi_1(M)$ by $\R$. Let $J$ denote the
element of $\R \subset \Gamma_\R$ corresponding to $1 \in \R$. Then consider
$M$ as the quotient of the $2g$-gon. The holonomy about the path traversed by
the boundary is exactly the product $\prod_i [a_i,b_i]$ of the commutators of
representatives of generators of $\pi_1(M)$, and has holonomy equal to $2\pi$, which
follows from the fact that the holonomy about any loop bounding a disk is equal to the
integral of the curvature, and the fact that $c_1(Q)$ is represented by
the curvature class of any connection, divided by $2\pi i$. Therefore, if we
consider the pullback connection on $\widetilde{M} \times U(1)$, the holonomy
about the lifts of the boundary path to $\widetilde{M}$ will also be $2\pi$,
which then lifts to translation by $1$ in the bundle $\widetilde{M} \times \R$.
This gives the relation that $\prod_i [a_i,b_i] = J$, which gives us a presentation
of the group $\Gamma_\R$. We let $\Gamma_\Z$ denote the central extension obtained
in a similar manner using $\Z$ instead of $\R$. Since $\pi_1(M)$ is discrete,
$\widetilde{M}$ is a flat bundle, so the pullback connection on
$\widetilde{M} \times U(1)$ still has curvature $-2\pi i \omega$, and the curvature
also remains unchanged after lifting to $\widetilde{M} \times \R$. \\

Suppose we have a homomorphism $\rho : \Gamma_\R \to G$ to a compact group $G$.
This then gives us an associated bundle
$P = (\widetilde{M}\times \R) \times_{\Gamma_\R} G$, which is a principal $G$ bundle
In addition to $\rho$, we get a Lie algebra homomorphism
$\dt{\rho} : \R \to \g$. Using this, $\dt{\rho}(A) \in \Omega^1_P(\g)$ determines
a connection on $P$, which has curvature $\dt{\rho}(F_A)$. Furthermore, we have that
\[
\dt{\rho}(d_A\star F_A) = d_{\dt{\rho}(A)}\star\dt{\rho}(F_A)
\]
which tells us that $\dt{\rho}(A)$ is a Yang-Mills connection on $P$. The main theorem
is that every Yang-Mills connection on every principal bundle arises in this way.
%
\begin{thm}
The above construction gives a bijective correspondence
\[
\hom(\Gamma_\R, G) / G \longleftrightarrow
\set{G\text{-bundles } P \to M
\text{ equipped with a Yang-Mills connection}}/\sim
\]
Where the action of $G$ on $\hom(\Gamma_\R,G)$ is by conjugation, and the
equivalence relation $\sim$ is given by $(P_1,A_1) \sim (P_2,A_2)$ if
there exists an isomorphism $\varphi : P_1 \to P_2$ of principal bundles such that
$\varphi^*A_2 = A_1$.
\end{thm}
%
\begin{proof}
We first give an outline of the proof strategy, since the proof is rather long
and involved. Before that, we note that we have not fixed a principal bundle, so a
homomorphism $\Gamma_\R \to G$ must provided us with the data of a principal
$G$-bundle $P \to M$, as well as a Yang-Mills connection on $P$. Our strategy will
be as follows:
\begin{enumerate}
  \item For any principal $G$-bundle $P$ with a Yang-Mills connection, use
  $\star F_A$ to identify an orbit of the adjoint action on $\g$. Then reduce
  to the case where $\star F_A$ takes the constant value $X \in \g$, where
  $X$ is fixed by the adjoint action.
  \item Pass the case where we can replace $G$ with a quotient group $\overline{G}$
  that is the product of a torus and a semisimple group, which are easy cases that
  are easily established.
\end{enumerate}
Since $M$ is $2$-dimensional, we have that $\star F_A \in \Omega^0_M(\g_P)$, so we
may regard it as a $G$-equivariant map $P \to \g$, i.e.
$\star F_A(p\cdot g) = \Ad_{g\inv}\star F_A(p)$. This then tells us that the image
of $\star F_A$ is exactly one orbit of $\g$ under the adjoint action.  Fix a nonzero
element $X \in \g$ lying in the image of $\star F$, and then consider the preimage
$P_X \defeq (\star F_A)\inv(X)$. Let $G_X \subset G$ be the stabilizer of $X$ under
the adjoint action. Then $G_X$ acts on $P_X$, since given any $p \in P_X$, and
$g \in G_X$, we have that $\star F_A(p\cdot g) = \Ad_{g\inv} F_A(p) = X$. This action
is clearly transitive and free, so $P_X$ defines a reduction of structure group from
$G$ to $G_X$, giving us a bundle isomorphism $P_X \times_{G_X} G \cong P$. Furthermore,
since $d_A \star F = 0$, we have that $\star F$ is constant in the horizontal
directions, so the differential of $\star F$ vanishes in the horizontal directions,
so the horizontal distribution is contained in the tangent bundle of $P_X$. Therefore,
the connection $A$ on $P$ restricts to a Yang-Mills connection on $P_X$
(which we also denote $A$). This restricted connection has the property that
$\star F_A$ is the constant map with value $X \in \g$, so $F_A$ is just the $2$-form
$X \otimes \omega \in \Omega^2_M(\g_{P_X})$. This tells us that every Yang-Mills
connection on any bundle $P$ arises from such a connection on the reduced bundle
$P_X$ for some $X \in \g$. Then suppose we have a homomorphism
$\rho : \Gamma_\R \to G$ with derivative $\dt{\rho} : \R \to \g$. The image of
$1 \in \R$ under $\dt{\rho}$ determines an element $X_\rho \in \g$. Centrality of
$\R$ in $\Gamma_\R$ then implies that the image of $\Gamma_\R$ preserves
$X_\rho$ under the adjoint action, so we may regard $\rho$ as a homomorphism
$\Gamma_\R \to G_{X_\rho}$. Combining this observation with the previous one, we can
then reduce to the case where $X$ is preserved by all of $G$ (i.e. $G = G_X$, which
is equivalent to $X$ lying in the center of $\g$). \\

We then want to reduce to the cases where $G$ is either a torus or a semisimple group.
This follows from the fact that any compact group $G$ arises as $H \times_D S$, where
$H$ is a maximal torus and $S = [G,G]$ is the maximal connected semisimple subgroup,
and $D = H \cap S$ is a finite subgroup of the center of $S$. Quotienting by $D$, we
get a finite covering $G \to \overline{G} = \overline{H} \times \overline{S}$, where
$\overline{H} = H/D$ and $\overline{S} = S/D$. We then claim that we can reduce to
the case where the structure group is $\overline{G}$. To see this, we note that
if we quotient $P$ by the action of $D$ to obtain $\overline{P}$, we get a finite
sheeted covering $P \to \overline{P}$. Since this covering is a local diffeomorphism,
we get an identification $TP \cong \pi^*T\overline{P}$ where $\pi : P \to \overline{P}$
is the covering projection. Therefore, we get that the horizontal distribution on
$P$ induces a horizontal distribution on $\overline{P}$, and conversely, we get
that a connection on $\overline{P}$ lifts to a horizontal distribution on $P$, so we
may reduce to the case $P = \overline{P}$. Then since $\overline{P}$ is a principal
bundle where the structure group is a product group, $\overline{P}$ is isomorphic
to a principal $\overline{H} \times \overline{K}$-bundle obtained by taking the
fiber product of an $\overline{H}$-bundle with a $\overline{S}$-bundle, and
the connection on $\overline{P}$ is equivalent to the data of a connection
on each of these two bundles. Then since we assumed that the Lie algebra
element $X \in \mathfrak{h} \oplus \mathfrak{s}$ is central, and the center of
$S$ is finite, we have that $X \in \mathfrak{h}$. Therefore, we see that a Yang-Mills
connection for $\overline{P}$ is equivalent to a flat connection for a principal
$\overline{S}$-bundle and a Yang-Mills connection for a principal
$\overline{H}$-bundle. Furthermore, any map  $\Gamma_\R \to G$ gives a map
$\overline{\rho} : \Gamma_\R \to \overline{G}$ by composition with the quotient map.
Then since we assume that $\dt{\rho}(1) = X$ is central, we have that the image of
$\R \subset \Gamma_\R$ is a $1$-parameter subgroup of $G$ lying in the center.
Furthermore, we have that $\Z$ is contained in the commutator subgroup
$[\Gamma_\R,\Gamma_\R]$, so its image under $\rho$ is central in
$G$ and lies in the maximal semisimple subgroup $S = [G,G]$, so $\rho(\Z) \subset D$.
Therefore, the map $\overline{\rho}$ descends to a map
$\Gamma_\R / \Z \to \overline{G}$. We note $\Gamma_\R / \Z \cong U(1) \times \pi_1(M)$.
Then since $\overline{G}$ is the product group $\overline{H} \times \overline{S}$, we
have that $\overline{\rho}$ is given by a pair of homomorphisms
$\alpha : U(1) \times \pi_1(M) \to \overline{H}$ and
$\beta : U(1) \times \pi_1(M) \to \overline{S}$. Furthermore, since $\overline{S}$
is semisimple, it has a finite center, so centrality of $X$ implies that $\beta$
is trivial on the $U(1)$ factor, so it is actually a map
$\beta : \pi_1(M) \to \overline{S}$, which is exactly the data of a principal
$\overline{S}$-bundle with flat connection, which in particular, is a
$\overline{S}$-bundle with Yang-Mills connection. \\

We are left to consider the homomorphism $\alpha$, which amounts to understanding
Yang-Mills connections when structure group is a torus $U(1) \times \cdots \times U(1)$.
By passing to associated bundles, this is equivalent to passing to a direct sum
of Hermitian line bundles, each equipped with a Yang-Mills connection. Therefore, it
suffices to verify this for the case where the torus is just $U(1)$. In this case,
the map $\rho$ necessarily factors through $\pi_1(M)$, since $U(1)$ is abelian, and
the derivative $\dt{\rho}$ evaluated at the identity gives the first Chern class
of the line bundle. Using our fixed reference bundle, this allows us to construct
a Yang-Mills connection by tensoring our reference bundle tensored with
itself $\dt{\rho}(1)$ many times with the flat line bundle determined by the induced map
$\pi_1(M) \to U(1)$.
\end{proof}
%
While we have shown that the data of a principal $G$-bundle $P \to M$ with
Yang-Mills connection is equivalent to a homomorphism $\Gamma_\R \to G$, we have
not yet shown how to recover the isomorphism class of the principal bundle associated
to such a homomorphism. Moreover, we have not yet shown that an arbitrary
principal $G$-bundle admits a single Yang-Mills connection.
%
\begin{thm}
Every principal $G$-bundle admits a Yang-Mills connection.
\end{thm}
%
\begin{proof}
We use the fact that for a group $\overline{G}$, we have that
principal $\overline{G}$-bundles over the surface $M$ are classified by
$H^2(M, \pi_1(G)) = \pi_1(G)$. In our case, by the reductions we made in the proof
of the previous theorem, $\overline{G} = \overline{H} \times \overline{S}$, so we have
\[
\pi_1(G) = \pi_1(\overline{H}) \oplus \pi_1(\overline{S})
\]
By restricting to a copy of $U(1)$, the map $\alpha$ determines
a class in $\pi_1(\overline{H}) \cong \Z^n$, which can be thought of as
choosing the first Chern class for each direct summand of a vector bundle.
For $\beta$, we need to do some additional work. The homomorphism
$\beta : \pi_1(M) \to \overline{S}$ defines a group action of $\pi_1(M)$ on
$\overline{S}$, which can be lifted to an action of $\Gamma_\Z$ on the universal
cover of $\overline{S}$, giving a homomorphism of $\Gamma_\Z$ to the universal covering
group. The image of the central element $J \in \Gamma_\Z$ in the
universal cover is an element of the center (since we assumed that $G = G_X$),
which is equivalent to an element of $\pi_1(\overline{S})$, since $\overline{S}$ is
the quotient of the universal cover by a subgroup of the center. We then note
that since $\overline{S}$ is compact and semisimple, every element of $\overline{S}$
is a commutator. Therefore, we can always find a homomorphism
$\pi_1(M) \to \overline{S}$ that maps $J$ to any element of the center
$Z(\overline{S})$. Therefore, we can always find a map $\pi_1(M) \to \overline{S}$
whose lift to $\Gamma_Z$ maps the central element $J$ to any element of
$\pi_1(\overline{S})$, considered as a subgroup of the center of the universal cover.
In addition, it is clear that any element of $\pi_1(\overline{H})$ is realized
as the restriction of a map $U(1) \times \pi_1(M) \to \overline{H}$. Therefore,
every element of $\pi_1(\overline{G})$ can be realized via the maps
$\alpha$ and $\beta$ coming from the homomorphism
$\overline{\rho} : \Gamma_\R \to \overline{G}$, so this shows that every
isomorphism class of principal $G$-bundle can be obtained from a homomorphism
$\Gamma_\R \to G$, and also shows that every principal $G$-bundle admits a Yang-Mills
connection.
\end{proof}
%
We now specialize to the case where $G = \mathrm{U}(n)$, which will be important for
our study of holomorphic vector bundles over $M$. Given a principal
$\mathrm{U}(n)$-bundle $P \to M$, we obtain a complex vector bundle $E \to M$ by
taking the associated bundle corresponding to the defining representation
$\mathrm{U}(n) \hookrightarrow \GL_n\C$. In the other direction, given a complex
vector bundle $E \to M$, we obtain a principal $\mathrm{U}(n)$-bundle by fixing
a Hermitian form on $E$ and taking the unitary frame bundle of $E$. Let $P \to M$ be
a principal $\mathrm{U}(n)$ bundle with Yang-Mills connection $A$, and let
$X \in \mathfrak{u}_n$ be the Lie algebra element determined by the curvature of $A$.
Writing $X = -2\pi i \Lambda$ for a Hermitian matrix $\Lambda$, the Yang-Mills condition
implies that the trace of $\Lambda$ is integral equal to the Chern class of $P$.
Furthermore, using the eigenspace decomposition of $X$, we find that the stabilizer
group $G_X$ is a product of $U(n_i)$, where $n_i$ is the multiplicity of the
$i^{th}$ eigenvalue $\lambda_i$. It can then be shown (though we don't show it
here) that we must have that $n_i\lambda_i \in \Z$. \\

From the perspective of vector bundles, these constraints have a more natural
interpretation. The reduction of structure group to $G_X$ corresponds to a direct
sum decomposition of the vector bundle, while the constraint on the eigenvalues
corresponds to the Chern classes of the factors coinciding with the Chern class
of the total bundle, which follows from the formula for the determinant line bundle
of a direct sum of vector bundles.
%
\section{The Holomorphic Viewpoint}
%
Fix a smooth complex vector bundle $E \to M$ of rank $n$ and first Chern class $k$.
The data of a holomorphic structure on $E$ is equivalent to the data of a first
order differential operator $\dbar_E : \Omega^0_M(E) \to \Omega^1_M(E)$ satisfying
$\dbar_E^2 = 0$, and the holomorphic sections of $E$ are obtained by taking the
kernel of $\dbar_E$. Much like connections, the any such operator may be written in
a smooth local trivialization of $E$ as
\[
\dbar_E = \dbar + B
\]
where $\dbar$ is the usual operator on $\C^n$ and $B$ is $\GL_n\C$-valued
$(0,1)$-form. This corresponds to the space of holomorphic structures being an
affine space over $\Omega_M^{0,1}(\End E)$. The group $\Aut(E)$ of smooth bundle
automorphisms of $E$ acts on the space of holomorphic structures by conjugation, i.e.
under an automorphism $g \in \mathrm{Aut}(E)$, the action is given by the mapping
$\dbar_E \mapsto g\dbar_E g\inv$. Furthermore, the orbits of this group action are
exactly the isomorphism classes of holomorphic structures on $E$, so the quotient by
this action would be the moduli space of holomorphic structures on $E$. However,
the quotient by this group action is not very nice, so we must restrict ourselves to
structures that are stable in the sense of GIT.
%
\begin{defn}
The \ib{slope} of a complex vector bundle $E \to M$ is defined to be
\[
\mu(E) \defeq \frac{c_1(E)}{\mathrm{rank}(E)}
\]
where we use the orientation on $M$ to determine the isomorphism $H^2(M,\Z) \cong \Z$.
\end{defn}
%
\begin{defn}
A holomorphic vector bundle $E \to M$ is
\begin{enumerate}
  \item \ib{Stable} if for all holomorphic subbundles $F \subset E$, we have the
  strict inequality $\mu(F) < \mu(E)$.
  \item \ib{Semistable} if for all holomorphic subbundles $F \subset E$, we have
  inequality $\mu(F) \leq \mu(E)$.
\end{enumerate}
\end{defn}
%
If we let $\mathscr{C}(E)$ denote the space of holomorphic structures on $E$ and let
$\mathscr{C}_s(E)$ and $\mathscr{C}_{ss}(E)$ denote the subspace of stable and
semistable holomorphic structures respectively, then we can construct the moduli space
of of holomorphic vector bundles of rank $n$ and degree $k$ as
\[
\mathcal{N}(n,k) \defeq \mathscr{C}_{ss}(E)/\Aut(E)
\]
In the case that $n$ and $k$ are coprime, we have that stability and semistability
coincide, and this will be the main situation of interest. \\

Understanding the moduli space of semistable bundles will also give
information regarding unstable holomorphic bundles. One reason for this is
the existence of a canonical filtration of an arbitrary holomorphic vector bundle.
%
\begin{thm}
Let $E \to M$ be a holomorphic vector bundle. Then there exists a filtration
\[
0 = E_0 \subset E_1 \subset \cdots \subset E_m = E
\]
of $E$ such that $E_{i+1}/E_i$ is semistable and we have
\[
\mu(E_1/E_0) > \mu(E_2/E_1) > \cdots > \mu(E_n/E_{n-1})
\]
\end{thm}
%
\section{The Symplectic Viewpoint}
%
Even though the space $\mathscr{A}(P)$ is infinite dimensional, it has enough structure
to be viewed as a symplectic ``manifold," but we will gloss over the formal details.
Because $\mathcal{A}(P)$ is an affine space over $\Omega_M^1(\g_P)$, we may
work with it as a manifold, where the tangent space at any point is $\Omega_M^1(\g_P)$.
We will be relatively cavalier with the details, though all we are doing can be made
formal by passing to Sobolev completions of spaces of sections. \\

In the case that $M$ is a Riemann surface, then the Hodge star
$\star : \Omega^1_M(\g_P) \to \Omega^1_M(\g_P)$ can be viewed as a complex
structure on $\mathscr{A}(P)$. In addition, after fixing a Riemannian metric on $M$,
we get a trivialization $\Omega_M^2 \cong \R$ using the natural orientation induced
by the complex structure. This allows us to view the pairing
\[
(\omega,\eta) \mapsto \int_M \langle\omega,\eta\rangle
\]
as a symplectic form on $\mathscr{A}(P)$. In addition, these structures are
visibly compatible, which gives $\mathscr{A}(P)$ a K\"ahler structure. \\

Recall that if we have a symplectic left action of a group $G$ on a symplectic manifold
$(M,\omega)$ we get an induced map $\g \to \mathfrak{X}(M)$ mapping $\xi$ to the vector
field $X_\xi$ defined by
\[
(X_\xi)_p = \frac{d}{dt}\bigg\vert_{t=0} \exp(t\xi)\cdot p
\]
The action is \ib{Hamiltonian} if for all $\xi \in \g$ there exists a function
$H_\xi : M \to \R$ called a \ib{Hamiltonian function} such that the vector field
$X_\xi$ satisfies the identity
\[
\omega_p((X_\xi)_p, v) = (dH_\xi)_p v
\]
for all points $p \in M$ and tangent vectors $v \in T_pM$, and the mapping
$\xi \mapsto H_\xi$ is $G$-equivariant with respect to the right actions
$\xi\cdot g= \Ad_{g\inv}\xi$ and $f \cdot g = f \circ L_g$, where $L_g : M \to M$
is the symplectomorphism determined by left multiplication by $g$. Given a
Hamiltonian action of $G$ on $M$, a \ib{moment map} for the action is a
$G$-equivariant map $\mu : M \to \g^*$ such that for any $p \in M$ and $\xi \in \g$,
we have
\[
d\mu_p(v)(\xi) = \omega_p((X_\xi)_p, v)
\]
in which case the Hamiltonian function is recovered by the formula
\[
H_\xi(p) = \mu(p)(\xi)
\]
We claim that the action of $\mathscr{G}(P)$ on $\mathscr{A}(P)$ is Hamiltonian.
We first note that the action is symplectic, since
$\langle\cdot,\cdot\rangle$ is $\Ad$-invariant, and the action of
a gauge transformation $\varphi$ on a tangent vector $\eta \in \Omega^1_M(\g_P)$
is by $\varphi\cdot\eta = \Ad_{g_\varphi\inv}\eta$ where
$g_\varphi : P \to G$ is the associated $G$-equivariant map. To show that the
action is Hamiltonian, we note that each
$\phi \in \mathrm{Lie}(\mathscr{G}(P)) = \Omega^0_M(\g_P)$ determines
a map $H_\phi : \mathscr{A}(P) \to \R$ given by
\[
H_\phi(A) = \int_M \langle F_A,\phi\rangle
\]
and the mapping $\phi \mapsto H_\phi$ is clearly $\mathscr{G}(P)$-equivariant.
We then compute for
$A \in \mathscr{A}(P)$ and $\psi \in T_A\mathscr{A}(P) = \Omega^1_M(\g_P)$
\begin{align*}
d(H_\phi)_A(\psi) &= \frac{d}{dt}\bigg\vert_{t=0}\int_M\langle F_{A+t\psi},\phi\rangle \\
&= \int_M \langle d_A\psi,\phi\rangle \\
&= \int_M d\langle\psi,\phi\rangle - \int_M \langle \psi,d_A\phi\rangle \\
&= \int_M\langle d_A\phi,\psi\rangle
\end{align*}
%
Then noting that $d_A\phi$ is the vector field determined by $\phi$, this
shows that the action is Hamiltonian, with the functions $H_\phi$ defining the
Hamiltonian functions. We then claim that mapping $\mu(A) = F_A$ defines a moment map.
We see this, we first note that $\mu$ is $\mathscr{G}(P)$-equivariant by our formulas
regarding how a connection and its curvature transform under gauge transformation.
For $\psi \in \Omega^1_M(\g_P)$, we have that
\begin{align*}
d\mu_A(\psi)(\phi) = \int_M \langle d_A\psi,\phi\rangle
\end{align*}
which from our above computation is equal to $\int_M \langle d_A\phi,\psi\rangle$,
so $\mu$ defines a moment map for this Hamiltonian action.
%
\section{Comparison of the Holomorphic and Symplectic Perspectives}
%
Given a smooth complex vector bundle $E \to M$, a Hermitian metric on $E$ gives rise
to the unitary frame bundle $\mathcal{B}_U(E)$, which is a principal $U(n)$-bundle.
Furthermore, the isomorphism class of the bundle $\mathcal{B}_U(E)$ is
independent of the choice of Hermitian metric, and is determined by the first
Chern class of $E$. For notational compactness, let $P \to M$ denote the
principal $U(n)$-bundle of unitary frames of $E$ with respect to a fixed
Hermitian metric on $E$. This Hermitian structure allows us to relate holomorphic
stuctures on $E$ with connections on $P$ -- given a holomorphic structure
$\dbar_E$ on $E$, there exists a unique connection on $E$ called the
\ib{Chern connection} that is compatible with both the holomorphic and Hermitian
structures. Furthermore, a connection on $P$ induces a connection on $E$,
and the $(0,1)$ component of this connection necessarily squares to $0$, since
$\Omega^{2,0}_M = 0$, so it defines a holomorphic structure on $E$. Therefore, we
have a bijection $\mathscr{A}(P) \to \mathscr{C}(E)$. \\

Using this comparison, we can compare the groups $\mathscr{G}(P)$ and $\Aut(E)$.
In a similar fashion to $\mathscr{G}(P)$, an element of $\Aut(E)$ can be
thought of as a global section of the associated bundle
$\mathcal{B}(E) \times_{\GL_n\C} \GL_n\C$ constructed with the conjugation action.
Ignoring analytic difficulties, this identifies the Lie algebra of $\Aut(E)$ with
the global sections of $\mathcal{B}(E) \times_{\GL_n\C} \mathfrak{gl}_n\C$. In this
way, we see that $\Aut(E)$ is the complexification of $\mathscr{G}(P)$, which
essentially boils down to the fact that every complex matrix $A$ can be expressed
as $A = B + iC$ with $B$ Hermitian and $C$ skew-Hermitian. \\

With this in mind, we want to appeal to the usual relationship between a GIT
quotient by a complex group and the symplectic quotient by its maximal compact
subgroup. The analogies can be summarized as follows:
\begin{align*}
\text{Complex reductive group } G &\longleftrightarrow \Aut(E) \\
\text{Maximal compact subgroup } K &\longleftrightarrow \mathscr{G}(P) \\
\text{Moment map } \mu &\longleftrightarrow \text{Curvature }
F_A \text{ of a connection} \\
\text{Norm square of the moment map} &\longleftrightarrow \text{Yang-Mills functional}
\end{align*}
%
In the finite dimensional case, the Kempf-Ness theorem establishes a homeomorphism
between the GIT quotient by $G$ and the symplectic quotient by $K$, which is a
consequence of every $G$-orbit containing a unique $K$-orbit minimizing the norm square
of the moment map. In the Yang-Mills situation, we have an infinite dimensional
example, which is the Narasimhan-Seshadri theorem. Reformulating the statement in
a more differential geometric way, the statement of the theorem is:
%
\begin{thm}[\ib{Narasimhan-Seshadri}]
Let $\mathscr{A}_s(P) \subset \mathscr{A}(P)$ denote the subspace of connections of $P$
that are absolute minima of the Yang-Mills and arise from irreducible representations
$\Gamma_\R \to U(n)$. Then the identification of $\mathscr{A}(P)$ with
$\mathscr{C}(E)$ induces a homeomorphism
$\mathscr{A}_s(P)/\mathscr{G}(P) \to \mathscr{C}_s(E)/\Aut(E)$.
\end{thm}
%
The reformulation and differential geometric proof of Narasimhan-Seshadri is due
to Donaldson \cite{donaldson1983}, where he proved that a stability of a holomorphic
vector bundle $E \to M$ is equivalent to the existence of a unitary connection $A$
with central curvature satisfying the condition that $\star F_A = -2\pi i \mu(E)$.
In the case that $E$ is degree $0$, this agrees with the classical statement of
Narasimhan-Seshadri by passing to the correspondence between flat bundles and
representations of the fundamental group. \\

To continue this analogy, in the traditional GIT setting, the norm square of the
moment map serves as a Morse function, which then gives a Morse stratification
into stable and unstable submanifolds by looking at the gradient flow lines, which
flow towards or away from the critical sets of the norm square of the moment map.
In our case, we have a good candidate for the Morse strata of the Yang-Mills
functional, which comes from the existence of Harder-Narasimhan filtrations.
As before, let $E \to M$ be a holomorphic vector bundle with a fixed Hermitian metric
and let $P \to M$ be its principal $U(n)$-bundle of unitary frames. Recall that
the curvature of a Yang-Mills connection on $P$ is equivalent to the data of a
skew-Hermitian matrix $X$, and writing at $X = -2\pi i \Lambda$ with $\Lambda$
Hermitian, the eigenvalues of $X$ determine a reduction of structure group
from $U(n)$ to the stabilizer subgroup o $X$, which is a subgroup isomorphic to
$U(n_1) \times \ldots U(n_m)$ depending on the multiplicities of the eigenvalues.
Comparing this with the holomorphic perspective, this exactly corresponds with the
Harder-Narasimhan filtration of the bundle $E \to M$. In a paper of Shatz
\cite{shatz1976}, the Harder-Narasimhan filtration gives a stratification of
the space $\mathscr{C}(E)$ of holomorphic structures of $E$, where a holomorphic
structure on $E$ is of type $\mu \in \Q^k$ (where $k$ is the length of the
filtration) if the slopes of the quotients in the Harder-Narasimhan filtration
(arranged in decreasing order) are given by the entries of $\mu$. We let
$\mathscr{C}_\mu(E) \subset \mathscr{C}(E)$ denote the subspace of holomorphic
structures on $E$ of type $\mu$. Using the type, one can construct the
Harder-Narasimhan polygon using the slopes, which is a convex polygon whose
vertices are determined by the numerators and denominators of the slopes
specified by $\mu$. These polygons determine a partial ordering on the slope
vectors, where we say $\lambda \geq \mu$ if the polygon corresponding to $\lambda$
lies above the polygon corresponding to $\mu$. Then the subspaces
$\mathscr{C}_\mu(E)$ give a stratification of $\mathscr{C}(E)$ where
\[
\mathscr{C}_\mu(E) \subset \bigcup_{\lambda \geq \mu} \mathscr{C}_\lambda(E)
\]
called the \ib{Harder-Narasimhan stratification}. Furthermore, since the
Harder-Narasimhan filtration is canonical, it is preserved by the
action of $\Aut(E)$, so the action of $\Aut(E)$ on $\mathscr{C}$ restricts
to actions of $\Aut(E)$ on the strata $\mathscr{C}_\mu(E)$, which we will
need to compute the equivariant cohomology of $\mathscr{C}(E)$.\\

Using the identification of $\mathscr{C}(E)$ with $\mathscr{A}(P)$ to
transport the Harder-Narasimhan stratification to $\mathscr{A}(P)$, and we can then
give a more differential geometric interpretation of the strata $\mathscr{C}_\mu(E)$.
Let $\mathscr{A}_\mu(P) \subset \mathscr{A}(P)$ denote the stratum corresponding
to $\mathscr{C}_\mu(E)$, which under the identification
$\mathscr{C}(E) \cong \mathscr{A}(P)$ corresponds to Yang-Mills minima whose Lie
algebra element has eigenvalues coinciding with the type vector $\mu$.
Our goal will be to draw analogies between these Harder-Narasimhan strata and the
hypothetical Morse strata for the Yang-Mills functional, and use the strata to
compute the cohomology of the moduli space $\mathcal{N}(n,k)$ when $n$ and $k$ are
coprime. In the differential geometric perspective, we consider a
larger class of functionals on $\mathscr{A}(P)$ arising from convex
functions on $\mathfrak{u}(n)$ that are invariant under the adjoint action,
which will have the same critical sets as the Yang-Mills functional. Using
these functionals, one can show that the gradient flow of these functionals is
tangential to the $\Aut(E)$ orbits, and applying some elliptic theory to
the connection Laplacian $d_A^*d_A + d_Ad_A^*$ allows one to show that
the critical sets have finite Morse index. Using these convex invariant
functions also gives an alternative proof of the stratification
\[
\mathscr{A}_\mu(P) \subset \bigcup_{\lambda \geq \mu} \mathscr{A}_\lambda(P)
\]
Which suggests that the $\mathscr{A}_\mu$ play the role of Morse strata for the
Yang-Mills functional modulo analytic difficulties with the critical sets, and
problems regarding convergence of the gradient flow of the Yang-Mills functional.
It was later shown by Daskalopoulos \cite{daskalopoulos1992} that these analytic
difficulties can be resolved, and that the Yang-Mills functional is an equivariantly
perfect Morse function on $\mathscr{A}(P)$. However, it is not strictly necessary to
use this fact to compute the cohomology of $\mathcal{N}(n,k)$, provided that we can
show that the Harder-Narasimhan stratification is something called
\ib{equivariantly perfect}, which we will discuss later.
%
\section{The Cohomology of $\mathcal{N}(n,k)$}
%
We restrict to the case where $n$ and $k$ are coprime, so stable bundles coincide
with semistable bundles, and most of the work will be done using the holomorphic
perspective. The strategy will be to compute the $\Aut(E)$-equivariant
cohomology of the strata $\mathscr{C}_\mu(E)$ and then establish the
equivariant perfection of the stratification to use an inductive Mayer-Vietoris
and Kunneth formula procedure to compute the equivariant cohomology of
$\mathscr{C}(E)$. Then by quotienting $\Aut(E)$ by a certain subgroup, we obtain
a free action on $\mathscr{C}_s(E)$, and the equivariant cohomology by this
quotient group will be the ordinary cohomology of the quotient space
$\mathcal{N}(n,k)$, since in the coprime case, $\mathscr{C}_{ss}(E)$ and
$\mathscr{C}_s(E)$ coicide. \\

We first tackle the problem of computing the $\Aut(E)$-equivariant cohomology of
the strata $\mathscr{C}_\mu(E)$. To do this, we will first make some reductions
to replace $\Aut(E)$ and $\mathscr{C}_\mu(E)$ with a smaller group and space
that are homotopy equivalent to $\Aut(E)$ and $\mathscr{C}_\mu(E)$ respectively.
Let $\mathscr{F}_\mu(E)$ denote the space of smooth filtrations of $E$ of
type $\mu$, which can be interpreted as a subspace of the space of smooth
sections of a flag bundle of $E$. There is a natural surjection
$\mathscr{C}_\mu(E) \to \mathscr{F}_\mu(E)$ given by forgetting the holomorphic
structures of the subbundles in the filtration, and the fiber $\mathscr{B}(E_\mu)$
of this map over a fixed filtration $E_\mu$ is the space of holomorphic structures
on the terms of the filtration. Let $\Aut(E_\mu)$ denote the group of smooth
automorphisms of $E$ preserving the filtration $E_\mu$. Then we can identify
$\mathscr{F}_\mu(E)$ with the quotient space $\Aut(E)/\Aut(E_\mu)$, which is
analogous to the identification of the space of partial flags in $\C^n$ with
$\GL_n\C/P$ for a parabolic subgroup $P$. This identification gives $\Aut(E)$ the
structure of a principal $\Aut(E_\mu)$-bundle over $F_\mu(E)$. Furthermore,
this gives an identification
\[
\mathscr{C}_\mu(E) \cong \Aut(E) \times_{\Aut(E_\mu)} \mathscr{B}(E_\mu)
\]
Then let $E\Aut(E) \to B\Aut(E)$ be a universal bundle for $\Aut(E)$. Then
the $\Aut(E)$-equivariant cohomology of $\mathscr{C}_\mu(E)$ is defined to the
the ordinary cohomology of the total space of the associated bundle
$E\Aut(E) \times_{\Aut(E)} \mathscr{C}_\mu(E)$. By our above observation, this
is the same as the space
$E\Aut(E) \times_{\Aut(E)} (\Aut(E) \times_{\Aut(E_\mu)} \mathscr{B}(E_\mu))$.
Noting that $E\Aut(E) \times_{\Aut(E)} \Aut(E) \cong E\Aut(E)$, we have that
this is the same as the space $E\Aut(E) \times_{\Aut(E_\mu)} \mathscr{B}(E_\mu)$.
We then note that $E\Aut(E)$ is a contractible space with an action of
$\Aut(E_\mu)$ (coming from restriction), so it may serve as the total space
for a universal bundle for $\Aut(E_\mu)$. Thus, we have have
\[
H^\bullet_{\Aut(E)}(\mathscr{C}_\mu(E)) \cong
H^\bullet_{\Aut(E_\mu)}(\mathscr{B}(E_\mu))
\]
We then reduce further. Let $\set{E_i}$ denote the subbundles appearing in
the filtration $E_\mu$ of $E$. We have exact sequences of smooth vector bundles
\[\begin{tikzcd}
0 \ar[r] & E_i \ar[r] & E_{i+1} \ar[r] & E_{i+1}/E_i \ar[r] & 0
\end{tikzcd}\]
which split in the smooth category. By fixing splittings for each of these
sequences, we get a direct sum decomposition
\[
E = D_1 \oplus \cdots \oplus D_r
\]
compatible with the filtration $E_\mu$, which we denote by $E_\mu^0$. We
then let $\Aut(E_\mu^0) \subset \Aut(E_\mu)$ be the automorphisms of $E$ that
respect this direct sum decomposition, and let $\mathscr{B}(E_\mu^0)$ denote
the holomorphic structures that give the $D_i$ the structure of semistable
holomorphic bundles. We then clearly have that $\Aut(E^0_\mu) \cong \prod_i\Aut(D_i)$,
which corresponds to writing an automorphism in block form. Similarly, we
have that $\mathscr{B}(E_\mu^0) = \prod_i\mathscr{C}_{ss}(D_i)$. We then make
two observations:
\begin{enumerate}
  \item $\Aut(E_\mu)$ deformations retracts onto $\Aut(E^0_\mu)$, which
  roughly corresponds to a parabolic subgroup of $\GL_n\C$ deformation retracting
  onto a product of $\GL_{n_i}\C$ via a Gram-Schmidt like procedure.
  \item The forgetful map $\mathscr{B}(E_\mu) \to \mathscr{B}(E_\mu^0)$ is a
  homotopy equivalence, which roughly corresponds to the space of splittings
  of an exact sequence of vector spaces being affine.
\end{enumerate}
%
These two observations give us the second reduction, which when combined with the first
gives us
\[
H^\bullet_{\Aut(E)}(\mathscr{C}_\mu(E)) \cong
H^\bullet_{\Aut(E_\mu^0)}(\mathscr{B}(E_\mu^0))
\]
Using Kunneth and the identification of $\Aut(E_\mu^0)$ and $\mathscr{B}(E_\mu^0)$
with products, this gives us
\[
H^\bullet_{\Aut(E)}(\mathscr{C}_\mu(E),\Q) \cong
\bigotimes_{i=1}^r H^\bullet_{\Aut(D_i)}(\mathscr{C}_{ss}(D_i),\Q)
\]
Which tells us that computing the equivariant cohomology for the semistable
locus for lower rank bundles will give us the equivariant cohomology of
the Harder-Narasimhan strata. \\

With this in hand, our next goal is to establish equivariant perfection of the
Harder-Narasimhan stratification, which will allow us to conclude that the
equivariant Poincar\'e polyomial for $\mathscr{C}_{ss}(E)$ is equal to the
``Morse polynomial" coming from the stratification, which is the polynomial
\[
\sum_\lambda t^{k_\mu} P_{t,\Aut(E_\mu)}(\mathscr{C}_\mu(E))
\]
where $k_\mu$ is the codimension of $\mathscr{C}_\mu(E)$ in $\mathscr{C}(E)$
and $P_{t,\Aut(E_\mu}(\mathscr{C}_\mu(E))$ denotes the equivariant Poincar\'e
polynomial of the stratum $\mathscr{C}_\mu(E)$. To do this, we will need to take
some results on faith, which give conditions for equivariant perfection.
%
\begin{prop}
If the equivariant Euler classes to the normal bundles $N_\mu$ of the strata
$\mathscr{C}_\mu(E)$ are not zero divisors in $H^\bullet_{\Aut(E_\mu)}(N_\mu, k)$
for a coefficient field $k$, then the stratification is equivariantly perfect
with respect to $k$.
\end{prop}
%
\begin{prop}
Let $X$ be a connected $G$ space where a subtorus of $G$ acts trivially,
and let $N \to X$ be an equivariant vector bundle. Then if the action of the
subtorus on each fiber is a primitive representation and $H^\bullet(G,\Z)$
has no torsion, then multiplication by the equivariant Euler class of
$N$ on $H_G^\bullet(X,\F_p)$ is injective.
\end{prop}
%
With this in mind, we want to identify the normal bundles to the strata
and compute their equivariant Euler classes.
%
\newpage
%
\nocite{*}
%
\printbibliography
%
\end{document}