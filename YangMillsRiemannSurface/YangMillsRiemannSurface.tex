\documentclass[psamsfonts, 12pt]{amsart}
%
%-------Packages---------
%
\usepackage[h margin=1 in, v margin=1 in]{geometry}
\usepackage{amssymb,amsfonts}
\usepackage{amsmath}
\usepackage{accents}
\usepackage[all,arc]{xy}
\usepackage{tikz-cd}
\usepackage{enumerate}
\usepackage{mathrsfs}
\usepackage{amsthm}
\usepackage{mathpazo}
\usepackage{float}
%\usepackage[backend=biber]{biblatex}
%\addbibresource{bibliography.bib}
%\usepackage{charter} %another font
%\usepackage{eulervm} %Vakil font
\usepackage{yfonts}
\usepackage{mathtools}
\usepackage{enumitem}
\usepackage{mathrsfs}
\usepackage{fourier-orns}
\usepackage[all]{xy}
\usepackage{hyperref}
\usepackage{url}
\usepackage{mathtools}
\usepackage{graphicx}
\usepackage{pdfsync}
\usepackage{mathdots}
\usepackage{calligra}
\usepackage{import}
\usepackage{xifthen}
\usepackage{pdfpages}
\usepackage{transparent}

\usepackage{tgpagella}
\usepackage[T1]{fontenc}
%
\usepackage{listings}
\usepackage{color}

\definecolor{dkgreen}{rgb}{0,0.6,0}
\definecolor{gray}{rgb}{0.5,0.5,0.5}
\definecolor{mauve}{rgb}{0.58,0,0.82}

\lstset{frame=tb,
  language=Matlab,
  aboveskip=3mm,
  belowskip=3mm,
  showstringspaces=false,
  columns=flexible,
  basicstyle={\small\ttfamily},
  numbers=none,
  numberstyle=\tiny\color{gray},
  keywordstyle=\color{blue},
  commentstyle=\color{dkgreen},
  stringstyle=\color{mauve},
  breaklines=true,
  breakatwhitespace=true,
  tabsize=3
  }
%
%--------Theorem Environments--------
%
\newtheorem{thm}{Theorem}[section]
\newtheorem*{thm*}{Theorem}
\newtheorem{cor}[thm]{Corollary}
\newtheorem{prop}[thm]{Proposition}
\newtheorem{lem}[thm]{Lemma}
\newtheorem*{lem*}{Lemma}
\newtheorem{conj}[thm]{Conjecture}
\newtheorem{quest}[thm]{Question}
%
\theoremstyle{definition}
\newtheorem{defn}[thm]{Definition}
\newtheorem*{defn*}{Definition}
\newtheorem{defns}[thm]{Definitions}
\newtheorem{con}[thm]{Construction}
\newtheorem{exmp}[thm]{Example}
\newtheorem{exmps}[thm]{Examples}
\newtheorem{notn}[thm]{Notation}
\newtheorem{notns}[thm]{Notations}
\newtheorem{addm}[thm]{Addendum}
\newtheorem{exer}[thm]{Exercise}
%
\theoremstyle{remark}
\newtheorem{rem}[thm]{Remark}
\newtheorem*{claim}{Claim}
\newtheorem*{aside*}{Aside}
\newtheorem*{rem*}{Remark}
\newtheorem*{hint*}{Hint}
\newtheorem*{note}{Note}
\newtheorem{rems}[thm]{Remarks}
\newtheorem{warn}[thm]{Warning}
\newtheorem{sch}[thm]{Scholium}
%
%--------Macros--------
\renewcommand{\qedsymbol}{$\blacksquare$}
\renewcommand{\sl}{\mathfrak{sl}}
\newcommand{\Bord}{\mathsf{Bord}}
\renewcommand{\hom}{\mathsf{Hom}}
\renewcommand{\emptyset}{\varnothing}
\renewcommand{\O}{\mathcal{O}}
\newcommand{\R}{\mathbb{R}}
\newcommand{\ib}[1]{\textbf{\textit{#1}}}
\newcommand{\Q}{\mathbb{Q}}
\newcommand{\Z}{\mathbb{Z}}
\newcommand{\N}{\mathbb{N}}
\newcommand{\C}{\mathbb{C}}
\newcommand{\A}{\mathbb{A}}
\newcommand{\F}{\mathbb{F}}
\newcommand{\M}{\mathcal{M}}
\newcommand{\dbar}{\overline{\partial}}
\newcommand{\zbar}{\overline{z}}
\renewcommand{\S}{\mathbb{S}}
\newcommand{\V}{\vec{v}}
\newcommand{\RP}{\mathbb{RP}}
\newcommand{\CP}{\mathbb{CP}}
\newcommand{\B}{\mathcal{B}}
\newcommand{\GL}{\mathrm{GL}}
\newcommand{\SL}{\mathrm{SL}}
\newcommand{\SP}{\mathrm{SP}}
\newcommand{\SO}{\mathrm{SO}}
\newcommand{\SU}{\mathrm{SU}}
\newcommand{\gl}{\mathfrak{gl}}
\newcommand{\g}{\mathfrak{g}}
\newcommand{\Bun}{\mathsf{Bun}}
\newcommand*{\dt}[1]{%
   \accentset{\mbox{\large\bfseries .}}{#1}}
\newcommand{\inv}{^{-1}}
\newcommand{\bra}[2]{ \left[ #1, #2 \right] }
\newcommand{\set}[1]{\left\lbrace #1 \right\rbrace}
\newcommand{\abs}[1]{\left\lvert#1\right\rvert}
\newcommand{\norm}[1]{\left\lVert#1\right\rVert}
\newcommand{\transv}{\mathrel{\text{\tpitchfork}}}
\newcommand{\defeq}{\vcentcolon=}
\newcommand{\enumbreak}{\ \\ \vspace{-\baselineskip}}
\let\oldexists\exists
\renewcommand\exists{\oldexists~}
\let\oldL\L
\renewcommand\L{\mathfrak{L}}
\makeatletter
\newcommand{\incfig}[2]{%
    \fontsize{48pt}{50pt}\selectfont
    \def\svgwidth{\columnwidth}
    \scalebox{#2}{\input{#1.pdf_tex}}
}
%
\newcommand{\tpitchfork}{%
  \vbox{
    \baselineskip\z@skip
    \lineskip-.52ex
    \lineskiplimit\maxdimen
    \m@th
    \ialign{##\crcr\hidewidth\smash{$-$}\hidewidth\crcr$\pitchfork$\crcr}
  }%
}
\makeatother
\newcommand{\bd}{\partial}
\newcommand{\lang}{\begin{picture}(5,7)
\put(1.1,2.5){\rotatebox{45}{\line(1,0){6.0}}}
\put(1.1,2.5){\rotatebox{315}{\line(1,0){6.0}}}
\end{picture}}
\newcommand{\rang}{\begin{picture}(5,7)
\put(.1,2.5){\rotatebox{135}{\line(1,0){6.0}}}
\put(.1,2.5){\rotatebox{225}{\line(1,0){6.0}}}
\end{picture}}
\DeclareMathOperator{\id}{id}
\DeclareMathOperator{\im}{Im}
\DeclareMathOperator{\codim}{codim}
\DeclareMathOperator{\coker}{coker}
\DeclareMathOperator{\supp}{supp}
\DeclareMathOperator{\inter}{Int}
\DeclareMathOperator{\sign}{sign}
\DeclareMathOperator{\sgn}{sgn}
\DeclareMathOperator{\indx}{ind}
\DeclareMathOperator{\alt}{Alt}
\DeclareMathOperator{\Aut}{Aut}
\DeclareMathOperator{\trace}{trace}
\DeclareMathOperator{\ad}{ad}
\DeclareMathOperator{\End}{End}
\DeclareMathOperator{\Ad}{Ad}
\DeclareMathOperator{\Lie}{Lie}
\DeclareMathOperator{\spn}{span}
\DeclareMathOperator{\dv}{div}
\DeclareMathOperator{\grad}{grad}
\DeclareMathOperator{\Sym}{Sym}
\DeclareMathOperator{\tr}{tr}
\DeclareMathOperator{\sheafhom}{\mathscr{H}\text{\kern -3pt {\calligra\large om}}\,}
\newcommand*\myhrulefill{%
   \leavevmode\leaders\hrule depth-2pt height 2.4pt\hfill\kern0pt}
\newcommand\niceending[1]{%
  \begin{center}%
    \LARGE \myhrulefill \hspace{0.2cm} #1 \hspace{0.2cm} \myhrulefill%
  \end{center}}
\newcommand*\sectionend{\niceending{\decofourleft\decofourright}}
\newcommand*\subsectionend{\niceending{\decosix}}
\def\upint{\mathchoice%
    {\mkern13mu\overline{\vphantom{\intop}\mkern7mu}\mkern-20mu}%
    {\mkern7mu\overline{\vphantom{\intop}\mkern7mu}\mkern-14mu}%
    {\mkern7mu\overline{\vphantom{\intop}\mkern7mu}\mkern-14mu}%
    {\mkern7mu\overline{\vphantom{\intop}\mkern7mu}\mkern-14mu}%
  \int}
\def\lowint{\mkern3mu\underline{\vphantom{\intop}\mkern7mu}\mkern-10mu\int}
%
%--------Hypersetup--------
%
\hypersetup{
    colorlinks,
    citecolor=black,
    filecolor=black,
    linkcolor=blue,
    urlcolor=blacksquare
}
%
%--------Solution--------
%
\newenvironment{solution}
  {\begin{proof}[Solution]}
  {\end{proof}}
%
%--------Graphics--------
%
%\graphicspath{ {images/} }
%
\begin{document}
%
\author{Jeffrey Jiang}
%
\title{Yang-Mills on Riemann Surfaces}
%
\maketitle
%
\tableofcontents
%
\section{Preliminary Setup}
%
To discuss the Yang-Mills functional, we must first fix some data. The setup will
consist of the following ingredients
\begin{enumerate}
  \item A compact manifold $M$.
  \item A compact connected Lie group $G$.
  \item A principal $G$-bundle $P \to M$.
\end{enumerate}
%
With this data, we have two associated bundles
\begin{align*}
\Ad P &\defeq P \times_G G \\
\g_P &\defeq P \times_G \g
\end{align*}
%
Where the action of $G$ on $G$ is by conjugation, and the action of $G$ on $\g$
is induced by the differential of conjugation. We note that these bundles both
contain additional structure -- $\Ad P$ is a bundle of groups (not a principal bundle),
and $\g_P$ is a bundle of Lie algebras. The space of sections $\Gamma(M, \Ad P)$ has
a natural group structure given by pointwise multiplication, and is called the
\ib{gauge group} $\mathscr{G}(P)$. Likewise, the space of sections
$\Gamma(M, \g_P)$ has a natural Lie algebra structure given by the pointwise Lie
bracket, and can be naturally identified with the Lie algebra of $\mathscr{G}(P)$.
An alternate characterization of these spaces of sections comes from a general
characterization of sections of associated bundles
%
\begin{prop}
We have natural correspondences
\begin{align*}
\Gamma(M, \Ad P) &\longleftrightarrow \set{f : P \to G ~:~ f(p\cdot g) = g\inv f(p)g} \\
\Gamma(M, \g_P) &\longleftrightarrow \set{g : P \to \g ~:~ f(p\cdot g) = \Ad_{g\inv}f(p)}
\end{align*}
\end{prop}
%
From the above correspondence we a group isomorphism
$\mathscr{G}(P) \to \Aut(P)$, where $\Aut(P)$ denotes the group of
$G$-equivariant diffeomorphisms $\varphi : P \to P$ such that
the following diagram commutes :
\[\begin{tikzcd}
P \ar[dr] \ar[rr, "\varphi"] && P \ar[dl] \\
& M
\end{tikzcd}\]
The isomorphism is given by mapping a $G$-equivariant map $f : P \to g$
to the automorphism $\varphi_f : P \to P$ defined by $\varphi_f(p) = p\cdot f(p)$.
%
In the the case of $\Gamma(M,\g_P)$, we can extend this to the spaces of
$\g_P$ valued forms. The kernel of the differential of the projection $P \to M$ gives a
subbundle of$TP$, which has a natural identification with the trivial bundle
$\underline{\g} = P \times \g$. Then we can identify the space of sections of
$\Lambda^kT^*M \otimes \g_P$, (i.e. the space $\Omega^k_M(\g_p)$ of $\g_p$ valued
$k$-forms with a subspace of the space $\Omega^k_P(\g)$ of $\g$-valued $k$-forms $\omega$
on $P$ satisfying :
\begin{enumerate}
  \item $R_g^*\omega = \Ad_{g\inv}\omega$, where $R_g : P \to P$ denotes the right
  action of $g \in G$.
  \item $\iota_\xi\omega = 0$ for any $\xi \in \g$, where $\iota$ denotes interior
  multiplication, and we identify $\xi$ with the constant vector field
  $\xi$ under the identification of the vertical space with $\g$.
\end{enumerate}
%
We have a maps
\begin{align*}
\Omega^p_M(\g_p) \otimes \Omega^q_M(\g_P) &\to \Omega^{p+q}_M(\g_p \otimes \g_p) \\
(\omega_1 \otimes \xi_1) \otimes (\omega_2 \otimes \xi_2)
&\mapsto (\omega_1 \wedge \omega_2) \otimes (\xi_1 \otimes \xi_2)
\end{align*}
From now on, we will usually omit the tensor symbol for $\g_P$-valued forms in favor of
juxtaposition, i.e. we write $\omega\xi$ instead of $\omega \otimes \xi$. Using the Lie
bracket, we then get
\begin{align*}
\Omega^p_M(\g_p) \otimes \Omega^q_M(\g_P) &\to \Omega^{p+q}_M(\g_p) \\
\omega \otimes \eta \mapsto [\omega,\eta]
\end{align*}
For any semisimple Lie group $G$ (in particular, for any compact Lie group $G$), we have
an inner product $\langle\cdot,\cdot\rangle : \g \times \g \to \R$ that is invariant
under the adjoint action (e.g. the Killing form). Fixing one such inner product
induces a fiber product on the trivial bundle $P \times \g$, and invariance
guarantees that this descends to a fiber product on $\g_P$. This give us pairings
\begin{align*}
\Omega^p_M(\g_P) \otimes \Omega^q_M(\g_P) &\to \Omega^{p+q}_M \\
\omega \otimes \eta &\mapsto \langle \omega, \eta \rangle
\end{align*}
%
We then fix an orientation and Riemannian metric on $M$, which gives us a Hodge
star operator $\star : \Omega^p_M \to \Omega^{n-p}_M$ and a Riemannian volume form
$dV_g$. The Hodge star extends to $\g_P$-valued $k$-forms, where given
$\omega \in \Omega^p_M$ and $\xi \in \Gamma(M,\g_P)$, we define
$\star(\omega\xi) = (\star\omega)\xi$. Then given
$\omega_1\xi_1,\omega_2\xi_2 \in \Omega^p_M(\g_P)$, we have
\[
\langle\omega_1\xi_1,\star\omega_2\xi_2\rangle =
\langle \omega_1,\omega_2\rangle_g\langle\xi_1,\xi_2\rangle
\]
where $\langle\cdot,\cdot\rangle_g$ denotes the fiber metric on $\Lambda^pT^*M$
induced by $g$. This gives us an inner product on each $\Omega^p_M(\g_P)$ defined
by
\[
(\theta,\varphi) = \int_M \langle \theta,\star\varphi\rangle
\]
Which gives us the $L_2$ norm on $\Omega^p_M(\g_P)$ with $\norm{F}_{L^2}^2 = (F,F)$.
\begin{defn}
A connection on a principal bundle $\pi : P \to M$ is a choice of $G$-invariant splitting
of the exact sequence of vector bundles over $P$
\[\begin{tikzcd}
0 \ar[r] & \underline{\g} \ar[r] & TP \ar[r] & \pi^*TM \ar[r] & 0
\end{tikzcd}\]
i.e. a distribution $H \subset TP$ such that
\begin{enumerate}
  \item $(R_g)_*H_p = H_{p\cdot g}$
  \item $H \oplus \underline{\g} = TP$
\end{enumerate}
Equivalently, it is the data of a $\g$-valued $1$-form $A \in \Omega^1_P(\g)$
satisfying
\begin{enumerate}
  \item $R_g^*A = \Ad_{g\inv} A$
  \item $\iota_\xi A = \xi$ for all $\xi \in \g$.
\end{enumerate}
\end{defn}
%
Note in particular that by a dimension count, we have that $\pi_*\vert_H : H \to TM$ is
an isomorphism. This implies that given a tangent vector $v$ at $x$ and a point
$p \in P$ in the fiber over $x$, we get a unique horizontal lift
$\tilde{v} \in H_p$. For a fixed principal $G$-bundle $\pi : P \to M$, let
$\mathscr{A}(P)$ denote the space of all connections on $P$, which is an affine space
over $\Omega^1_M(\g_P)$. \\

Given any distribution $E \subset TP$, we get a Frobenius tensor
$\phi_E : E \otimes E \to TP/E$ given by $X \otimes Y \to [X,Y] \mod E$ where
we extend $X$ and $Y$ to local vector fields. The Frobenius tensor should be
thought of as the obstruction to the existence of an integral submanifold for the
distribution $E$. In the case of a connection $H$ on a principal bundle $P \to M$,
we can extend to all of $TP$ by first projecting onto $H$, and
have an identification of $TP/H \cong \underline{\g}$, and the Frobenius
tensor is given by $X \otimes Y \mapsto A([X,Y])$, where $A$ is the connection $1$-form,
and is called the \ib{curvature form} of the connection, and is denoted $F_A$. In
terms of differential forms, we have that for horizontal vectors $\xi_1,\xi_2$ on
$TP$,
\[
dA(\xi_1,\xi_2) = \xi_1A(\xi_2) - \xi_2A(\xi_1) - A([\xi_1,\xi_2])
\]
The fact that $\xi_1$ and $\xi_2$ are horizontal implies that they are in the
kernel of $A$, which gives us that
\[
dA + \frac{1}{2}[A,A] = -F_A\footnote{Our convention for the sign of the curvature
is opposite from many other conventions, which usually sets
$F_A = dA + \frac{1}{2}[A,A]$}
\]
It can be shown that $F_A$ transforms by the adjoint action under pullback, and
vanishes on vertical vectors, so it descends to a $\g_P$-valued $2$-form on the base
manifold $M$. \\

Another thing to note is that there is a natural action of the gauge group
$\mathscr{G}(P)$ on the space of connections $\mathscr{A}(P)$. Interpreting the
elements of $\mathscr{G}(P)$ as bundle automorphisms $\varphi : P \to P$ and
elements of $\mathscr{A}(P)$ as $\g$-valued $1$-forms $A$ on $P$, the action
is simply pullback, $(\varphi,A) \mapsto \varphi^*A$. To show that
this defines an action, we must check that $\varphi^*A$ satisfies the conditions
%
\begin{enumerate}
  \item $R_g^*\varphi^*A = \Ad_{g\inv} \varphi^*A$
  \item $\iota_\xi \varphi^*A = \xi$ for all $\xi \in \g$.
\end{enumerate}
%
Which are all simple consequences of the $G$-equivariance of $\varphi$ and
the transformation law for $A$. For a specific formula, let
$\varphi : P \to P$ be an element of the gauge group, and let
$g_\varphi : P \to G$ be its associated $G$-equivariant map. Then
\[
\varphi^*A = \Ad_{g_\varphi\inv} A + g_\varphi^*\theta
\]
where $\theta \in \Omega^1_G(\g)$ denotes the \ib{Maurer-Cartan form}
\[
\theta_g(v) = (dL_{g\inv})_g(v)
\]
which satisfies the \ib{Maurer-Cartan equation}
\[
d\theta + \frac{1}{2}[\theta,\theta] = 0
\]
%
\begin{prop}
Let $A \in \mathscr{A}(P)$ be a connection and $\varphi : P \to P$ an element
of $\mathscr{G}(P)$ with associated $G$-equivariant map $g_\varphi : P \to G$.
Then
\[
F_{\varphi^*A} = \Ad_{g_\varphi\inv} F_A
\]
\end{prop}
%
\begin{proof}
Using the transformation law for $\varphi^*A$ we compute
\begin{align*}
F_{\varphi^*A} &= d(\Ad_{g_\varphi\inv} A + g_\varphi^*\theta)
+ \frac{1}{2}[\Ad_{g_\varphi\inv} A + g_\varphi^*\theta,
\Ad_{g_\varphi\inv} A + g_\varphi^*\theta] \\
&= \Ad_{g_\varphi\inv}dA + g_\varphi^*d\theta + \frac{1}{2}
\left([\Ad_{g_\varphi\inv}A, \Ad_{g_\varphi\inv} A] + [\Ad_{g_\varphi\inv} A,
g_\varphi^*\theta] + [g^*_\varphi\theta, \Ad_{g_\varphi\inv}]
+ [g_\varphi^*\theta,g_\varphi^*\theta] \right) \\
&= \Ad_{g_\varphi\inv} dA + \frac{1}{2}[\Ad_{g_\varphi\inv}A, \Ad_{g_\varphi\inv} A]]
\end{align*}
Where we use skew-symmetry and the Maurer-Cartan equation.
\end{proof}
%
\section{The Yang-Mills Functional}
%
With the setup done, we have the ingredients necessary to define the Yang-Mills
functional.
%
\begin{defn}
The \ib{Yang-Mills functional} is the map
$L : \mathscr{A}(P) \to \R$ given by
\[
L(A) = \norm{F_A}_{L^2}^2 = \int_M \langle F_A,F_A \rangle
\]
\end{defn}
%
\begin{rem*}
The bilinear form $\langle \cdot,\cdot\rangle$ should be thought of as a symplectic
form on $\mathscr{A}(P)$, and the mapping $A \mapsto F_A$ should be thought of
as the moment map to some action of $\mathscr{G}(P)$. In this context, the
Yang-Mills functional is the norm-square of the moment map.
\end{rem*}
%
We immediately see that the Yang-Mills equations are invariant under $\mathscr{G}(P)$
in the following sense -- if we have any gauge transformation $\varphi$ with
associated map $g_\varphi : P \to G$, we have that $L(\varphi^*A) = L(A)$,
which follows immeidately from the invariance of $\langle\cdot,\cdot\rangle$
and the transformation law for curvature. \\

Our goal now will be to find the Euler-Lagrange
equations for the Yang-Mills functional by computing the first and second variations.
The connection form $A$ on $P$ induces an exterior covariant derivative on any
associated vector bundle $E = P \times_G V$ arising from a linear representation
$\rho : G \to \GL(V)$. Let $\dt{\rho} : \g \to \End(V)$ be the derivative of
$\rho$ at the identity. Then the exterior covariant derivative is given by
\begin{align*}
d_A : \Omega^p_M(E) &\to \Omega^{p+1}_M(E) \\
\psi &\mapsto d\psi + \dt{\rho}(A) \wedge \psi
\end{align*}
%
In particular, we get an exterior covariant derivative on $\g_P$, which is given by
\[
d_A\psi = d\psi + [A,\psi]
\]
Using the Hodge star operator, we construct the formal adjoint with respect to
the inner product $d_A^* : \Omega^{p}_M(\g_P) \to \Omega^{p-1}_M(\g_P)$ in the same manner
as for classical Hodge theory on a Riemannian manifold. Explicitly, the formula
on $p$-forms is given by
\[
d^*_A = (-1)^{n(p+1) + 1}\star d_A \star
\]
where $n = \dim M$. \\

For any other connection $A + \eta$ with $\eta \in \Omega^1_M(\g_P)$, a quick computation
yields
\[
F_{A+\eta} = F_A + \frac{1}{2}[\eta,\eta] + d_A\eta
\]
This allows us to compute the first variation of $L$.
%
\begin{prop}[\ib{The First Variation}]
For a local extremum $A \in \mathscr{A}(P)$ of the Yang-Mills functional, we have
\[
d_A\star F_A = 0
\]
The local extremum connection $A$ is then callled a \ib{Yang-Mills connection}.
\end{prop}
%
\begin{proof}
Consider a variation $A + t\eta$ with $t \in \R$ and $\eta \in \Omega^1_M(\g_P)$.
We have that the curvature is given by
\[
F_{A+t\eta} = F_A + \frac{t^2}{2}[\eta,\eta] + td_A\eta
\]
This then gives us
\begin{align*}
\norm{F_{A+t\eta}}_{L^2} &= \int_M \langle F_{A+t\eta},F_{A+t\eta}\rangle \\
&= \int_M \langle F_A + \frac{t^2}{2}[\eta,\eta]
+ td_A\eta,\star (F_A + \frac{t^2}{2}[\eta,\eta] + td_A\eta \rangle \\
\end{align*}
Expanding this out, we get that the term that is linear in $t$ is

\[
\int_M \langle F_A, \star d_A\eta\rangle + \langle d_A\eta, \star F_A \rangle
= 2(F_A,d_A\eta)
\]
where we use symmetry of $(\cdot,\cdot)$. Since $A$ is extremal, we have that
this term must vanish, giving us that $(F_A,d_A\eta) = (d^*_A F,\eta) = 0$ for
every $\eta$. Then since we have (up to sign) $d^*_A = \star d_A \star$, and
$\star$ is an isomorphism, this implies $d_A\star F_A = 0$.
\end{proof}
%
\begin{prop}[\ib{The Second Variation}]
At a Yang-Mills connection $A \in \mathscr{A}(P)$, we have
\[
d^*_A d_A\eta + \star[\eta,\star F_A] = 0
\]
\end{prop}
%
\begin{proof}
We differentiate the first variational equation with respect to $t$, i.e.
we compute
\[
\frac{d}{dt}\bigg\vert_{t=0} d_{A+t\eta}^*F_{A+t\eta}
\]
We expand out
\begin{align*}
d_{A+t\eta}^*F_{A+t\eta} &= \pm \star d_{A+t\eta}\star F_{A+t\eta} \\
&= \pm\left(\star d_A\star\left(F_A + td_A\eta + \frac{t^2}{2}[\eta,\eta]\right)
+ t\star
\left[\eta, \star\left(F_A + td_A\eta + \frac{t^2}{2}[\eta,\eta]\right)\right]\right)
\end{align*}
Taking the term linear in $t$ yields
\[
\pm\left( \star d_A\star d_A\eta + \star[\eta,\star F_A] \right)
\]
Giving us that at an extremal connection $A$, we have
\[
d^*_A d_A\eta + \star[\eta,\star F_A] = 0
\]
\end{proof}
%
%TODO Hessian?
%TODO Example with U(1)
%
\section{A Symplectic Viewpoint}
%

%
\section{Yang-Mills Over a Riemann Surface}
%
We now restrict our attention to when $M$ is a surface, i.e. a $2$ dimensional real
manifold. The Hodge star operator maps $\Omega^1_M \to \Omega^1_M$, and satisfies
$\star^2 = -\id$, which induces an almost complex structure on $M$, giving us a
decomposition $\Omega^1_M(\C) = \Omega^{1,0}_M(\C) \oplus \Omega^{0,1}_M(\C)$
into the $\pm i$ eigenspaces of the complexified Hodge star. The operator
$\dbar \defeq \pi^{0,1} \circ d$ (where $\pi^{0,1}$ denotes projection onto
$\Omega^{0,1}_M(\C)$) satisfies $\dbar^2 = 0$ since by dimension reasons,
$\Omega^{0,2}_M(\C) = 0$, so the induced almost complex structure is integrable by the
Newlander-Nirenberg theorem. The same argument with projection onto $\Omega^{1,0}_M(\C)$
gives an operator $\partial$ satisfying $\partial^2 = 0$, and we get a decomposition
$d = \partial + \dbar$. Then given a principal bundle $P \to M$, We get a similar
decomposition for $\Omega^1_M(\g_P)$ after complexification giving a decomposition
$d_A = \partial_A + \dbar_A$ for any connection $A \in \mathscr{A}(P)$.
%
\end{document}