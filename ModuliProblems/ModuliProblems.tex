\documentclass[psamsfonts, 12pt]{amsart}
%
%-------Packages---------
%
\usepackage[h margin=1 in, v margin=1 in]{geometry}
\usepackage{amssymb,amsfonts}
\usepackage{amsmath}
\usepackage{accents}
\usepackage[all,arc]{xy}
\usepackage{tikz-cd}
\usepackage{enumerate}
\usepackage{mathrsfs}
\usepackage{amsthm}
\usepackage{mathpazo}
\usepackage{float}
%\usepackage[backend=biber]{biblatex}
%\addbibresource{bibliography.bib}
\usepackage{yfonts}
\usepackage{mathtools}
\usepackage{enumitem}
\usepackage{mathrsfs}
\usepackage{fourier-orns}
\usepackage[all]{xy}
\usepackage{hyperref}
\usepackage{url}
\usepackage{mathtools}
\usepackage{graphicx}
\usepackage{pdfsync}
\usepackage{mathdots}
\usepackage{calligra}
\usepackage{import}
\usepackage{xifthen}
\usepackage{pdfpages}
\usepackage{transparent}

\usepackage{tgpagella}
\usepackage[T1]{fontenc}
%
\usepackage{listings}
\usepackage{color}

\definecolor{dkgreen}{rgb}{0,0.6,0}
\definecolor{gray}{rgb}{0.5,0.5,0.5}
\definecolor{mauve}{rgb}{0.58,0,0.82}

\lstset{frame=tb,
  language=Matlab,
  aboveskip=3mm,
  belowskip=3mm,
  showstringspaces=false,
  columns=flexible,
  basicstyle={\small\ttfamily},
  numbers=none,
  numberstyle=\tiny\color{gray},
  keywordstyle=\color{blue},
  commentstyle=\color{dkgreen},
  stringstyle=\color{mauve},
  breaklines=true,
  breakatwhitespace=true,
  tabsize=3
  }
%
%--------Theorem Environments--------
%
\newtheorem{thm}{Theorem}[section]
\newtheorem*{thm*}{Theorem}
\newtheorem{cor}[thm]{Corollary}
\newtheorem{prop}[thm]{Proposition}
\newtheorem{lem}[thm]{Lemma}
\newtheorem*{lem*}{Lemma}
\newtheorem{conj}[thm]{Conjecture}
\newtheorem{quest}[thm]{Question}
%
\theoremstyle{definition}
\newtheorem{defn}[thm]{Definition}
\newtheorem*{defn*}{Definition}
\newtheorem{defns}[thm]{Definitions}
\newtheorem{con}[thm]{Construction}
\newtheorem{exmp}[thm]{Example}
\newtheorem{exmps}[thm]{Examples}
\newtheorem{notn}[thm]{Notation}
\newtheorem{notns}[thm]{Notations}
\newtheorem{addm}[thm]{Addendum}
\newtheorem{exer}[thm]{Exercise}
%
\theoremstyle{remark}
\newtheorem{rem}[thm]{Remark}
\newtheorem*{claim}{Claim}
\newtheorem*{aside*}{Aside}
\newtheorem*{rem*}{Remark}
\newtheorem*{hint*}{Hint}
\newtheorem*{note}{Note}
\newtheorem{rems}[thm]{Remarks}
\newtheorem{warn}[thm]{Warning}
\newtheorem{sch}[thm]{Scholium}
%
%--------Macros--------
\renewcommand{\qedsymbol}{$\blacksquare$}
\renewcommand{\sl}{\mathfrak{sl}}
\newcommand{\Bord}{\mathsf{Bord}}
\renewcommand{\hom}{\mathrm{Hom}}
\renewcommand{\emptyset}{\varnothing}
\renewcommand{\O}{\mathcal{O}}
\newcommand{\R}{\mathbb{R}}
\newcommand{\ib}[1]{\textbf{\textit{#1}}}
\newcommand{\Q}{\mathbb{Q}}
\newcommand{\Z}{\mathbb{Z}}
\newcommand{\N}{\mathbb{N}}
\newcommand{\C}{\mathbb{C}}
\newcommand{\A}{\mathbb{A}}
\newcommand{\F}{\mathbb{F}}
\newcommand{\M}{\mathcal{M}}
\newcommand{\dbar}{\overline{\partial}}
\newcommand{\zbar}{\overline{z}}
\renewcommand{\S}{\mathbb{S}}
\newcommand{\V}{\vec{v}}
\newcommand{\RP}{\mathbb{RP}}
\newcommand{\CP}{\mathbb{CP}}
\newcommand{\B}{\mathcal{B}}
\newcommand{\GL}{\mathrm{GL}}
\newcommand{\SL}{\mathrm{SL}}
\newcommand{\SP}{\mathrm{SP}}
\newcommand{\SO}{\mathrm{SO}}
\newcommand{\SU}{\mathrm{SU}}
\newcommand{\gl}{\mathfrak{gl}}
\newcommand{\g}{\mathfrak{g}}
\newcommand{\Bun}{\mathsf{Bun}}
\newcommand*{\dt}[1]{%
   \accentset{\mbox{\large\bfseries .}}{#1}}
\newcommand{\inv}{^{-1}}
\newcommand{\bra}[2]{ \left[ #1, #2 \right] }
\newcommand{\set}[1]{\left\lbrace #1 \right\rbrace}
\newcommand{\abs}[1]{\left\lvert#1\right\rvert}
\newcommand{\norm}[1]{\left\lVert#1\right\rVert}
\newcommand{\transv}{\mathrel{\text{\tpitchfork}}}
\newcommand{\defeq}{\vcentcolon=}
\newcommand{\enumbreak}{\ \\ \vspace{-\baselineskip}}
\let\oldexists\exists
\renewcommand\exists{\oldexists~}
\let\oldL\L
\renewcommand\L{\mathfrak{L}}
\makeatletter
\newcommand{\incfig}[2]{%
    \fontsize{48pt}{50pt}\selectfont
    \def\svgwidth{\columnwidth}
    \scalebox{#2}{\input{#1.pdf_tex}}
}
%
\newcommand{\tpitchfork}{%
  \vbox{
    \baselineskip\z@skip
    \lineskip-.52ex
    \lineskiplimit\maxdimen
    \m@th
    \ialign{##\crcr\hidewidth\smash{$-$}\hidewidth\crcr$\pitchfork$\crcr}
  }%
}
\makeatother
\newcommand{\bd}{\partial}
\newcommand{\lang}{\begin{picture}(5,7)
\put(1.1,2.5){\rotatebox{45}{\line(1,0){6.0}}}
\put(1.1,2.5){\rotatebox{315}{\line(1,0){6.0}}}
\end{picture}}
\newcommand{\rang}{\begin{picture}(5,7)
\put(.1,2.5){\rotatebox{135}{\line(1,0){6.0}}}
\put(.1,2.5){\rotatebox{225}{\line(1,0){6.0}}}
\end{picture}}
\DeclareMathOperator{\id}{id}
\DeclareMathOperator{\im}{Im}
\DeclareMathOperator{\codim}{codim}
\DeclareMathOperator{\coker}{coker}
\DeclareMathOperator{\supp}{supp}
\DeclareMathOperator{\inter}{Int}
\DeclareMathOperator{\sign}{sign}
\DeclareMathOperator{\sgn}{sgn}
\DeclareMathOperator{\indx}{ind}
\DeclareMathOperator{\alt}{Alt}
\DeclareMathOperator{\Aut}{Aut}
\DeclareMathOperator{\trace}{trace}
\DeclareMathOperator{\ad}{ad}
\DeclareMathOperator{\End}{End}
\DeclareMathOperator{\Ad}{Ad}
\DeclareMathOperator{\Lie}{Lie}
\DeclareMathOperator{\spn}{span}
\DeclareMathOperator{\dv}{div}
\DeclareMathOperator{\grad}{grad}
\DeclareMathOperator{\Sym}{Sym}
\DeclareMathOperator{\tr}{tr}
\DeclareMathOperator{\sheafhom}{\mathscr{H}\text{\kern -3pt {\calligra\large om}}\,}
\newcommand*\myhrulefill{%
   \leavevmode\leaders\hrule depth-2pt height 2.4pt\hfill\kern0pt}
\newcommand\niceending[1]{%
  \begin{center}%
    \LARGE \myhrulefill \hspace{0.2cm} #1 \hspace{0.2cm} \myhrulefill%
  \end{center}}
\newcommand*\sectionend{\niceending{\decofourleft\decofourright}}
\newcommand*\subsectionend{\niceending{\decosix}}
\def\upint{\mathchoice%
    {\mkern13mu\overline{\vphantom{\intop}\mkern7mu}\mkern-20mu}%
    {\mkern7mu\overline{\vphantom{\intop}\mkern7mu}\mkern-14mu}%
    {\mkern7mu\overline{\vphantom{\intop}\mkern7mu}\mkern-14mu}%
    {\mkern7mu\overline{\vphantom{\intop}\mkern7mu}\mkern-14mu}%
  \int}
\def\lowint{\mkern3mu\underline{\vphantom{\intop}\mkern7mu}\mkern-10mu\int}
%
%--------Hypersetup--------
%
\hypersetup{
    colorlinks,
    citecolor=black,
    filecolor=black,
    linkcolor=blue
}
%
%--------Solution--------
%
\newenvironment{solution}
  {\begin{proof}[Solution]}
  {\end{proof}}
%
%--------Graphics--------
%
%\graphicspath{ {images/} }
%
\begin{document}
%
\author{Jeffrey Jiang}
%
\title{Modern Moduli Theory Exercises}
%
\maketitle
%
\setcounter{section}{1}
%
\setcounter{thm}{1}
%
\begin{exer}
Show that the restriction
$\mathrm{Fun}(\mathsf{Sch}^{\mathrm{op}},\mathsf{Set}) \to
\mathrm{Fun}(\mathsf{Ring},\mathsf{Set})$ is not fully faithful.
\end{exer}
%
\begin{proof}
Consider the functor $H^1$, which maps $X \mapsto H^1(X,\O_X)$ and the functor
$F$ that maps every scheme to $\set{0}$ and every morphism to the identity map
$\set{0} \to \set{0}$. On affine schemes, $H^1$ and $F$ agree. However, nonisomorphic
schemes (e.g. any two nonisomorphic nonaffine schemes) get mapped to isomorphic
objects in $\mathrm{Set}$, so the functor cannot be fully faithful.
\end{proof}
%
\begin{exer} \enumbreak
\begin{enumerate}
  \item For schemes $T$ and $X$, the assignment $U \subset T \mapsto \mathrm{Map}(U,X)$
  is a sheaf of sets over $T$.
  \item Use this to show that the functor
  $\mathrm{Sch}\to\mathrm{Fun}(\mathsf{Ring}, \mathsf{Set})$ is fully faithful.
\end{enumerate}
\end{exer}
%
\begin{proof} \enumbreak
\begin{enumerate}
  \item The assignment clearly forms a presheaf with the natural restriction maps
  given by restricting a morphism to a subset. To verify that the assignment
  defines a sheaf, we need to verify two more conditions. Let $\set{U_i}$ be
  an open covering of an open set $U$. Then let $f,g \in \mathrm{Map}(U,X)$
  such that $f\vert_{U_i} = g\vert_{U_i}$ for all $i$. Then $f$ and $g$ agree
  on the points of $U$. In addition, $f$ and $g$ induce the same map on stalks,
  so they induce the same map of sheaves. Therefore, $f$ and $g$ define the same
  morphism. For the other condition, suppose we have $f_i \in \mathrm{Map}(U_i,X)$
  such that $f_i$ and $f_j$ agree on $U_i \cap U_j$. Then since functions and
  sheaf morphisms glue, the $f_i$ determine a morphism $f \in \mathrm{Map}(U,X)$
  with $f\vert_{U_i} = f_i$.
  \item To show that the functor is fully faithful, we want to show that for
  schemes $X,T$, the map
  \[
  \mathrm{Map}(T,X) \to
  \mathrm{Map}_{\mathrm{Fun}(\mathsf{Ring},\mathsf{Set})}
  (\mathrm{Map}(\cdot, T), \mathrm{Map}(\cdot, X))
  \]
  is bijective. In the case that $T = \mathrm{Spec}(A)$ is affine, the standard
  proof of the Yoneda lemma works, where given a natural transformation
  $\eta : \mathrm{Map}(\cdot, T) \to \mathrm{Map}(\cdot, X))$, it is induced by
  postcomposition with the map $\eta_T(\id_T) : T \to X$, where we note that $T$
  is a valid input the the functors since it is affine. We then want to verify
  bijectivity when $T$ is not necessarily affine. For injectivity, let $\set{U_i}$
  be an affine open cover of $T$, and let $f,g : T \to X$ be two morphisms that
  induce the same natural transformation
  $\mathrm{Map}(\cdot,T) \to \mathrm{Map}(\cdot,X)$. Then let $f_i,g_i : U_i \to X$
  be the restrictions of $f$ and $g$ to the $U_i$. Since the $U_i$ are affine,
  we know that the natural transformations
  $\mathrm{Map}(\cdot,U_i) \to \mathrm{Map}(\cdot,X)$ induced by the $f_i$ and $g_i$
  are the same, so we know $f_i = g_i$ on $U_i$. Therefore, $f = g$ since
  $U \mapsto \mathrm{Map}(U,X)$ is a sheaf. For surjectivity, let
  $\eta : \mathrm{Map}(\cdot,T) \to \mathrm{Map}(\cdot,X)$ be a natural transformation.
  We have natural inclusions
  $\mathrm{Map}(\cdot, U_i) \hookrightarrow \mathrm{Map}(\cdot, T)$ given by
  composition with the inclusions $U_i \hookrightarrow T$, and $\eta$ restricts
  to natural transformations
  $\eta_i : \mathrm{Map}(\cdot, U_i) \to \mathrm{Map}(\cdot, X)$, which are determined
  by morphisms $f_i : U_i \to X$. By restricting the natural transformation to
  intersections $U_i \cap U_j$, we find that $f_i$ and $f_j$ agree on $U_i \cap U_j$.
  Therefore, the $f_i$ glue to a morphism $T \to X$, which much necessarily induce
  $\eta$. Therefore, the map
  \[
  \mathrm{Map}(T,X) \to
  \mathrm{Map}_{\mathrm{Fun}(\mathsf{Ring},\mathsf{Set})}
  (\mathrm{Map}(\cdot, T), \mathrm{Map}(\cdot, X))
  \]
  is bijective in the non-affine case as well.
\end{enumerate}
\end{proof}
%
\setcounter{thm}{4}
%
\begin{exer}
For any category $\mathcal{C}$ and an object $T \in \mathcal{C}$, the \ib{slice category}
$\mathcal{C}_{/T}$ is the category where the objects are morphisms $X \to T$ in
$\mathcal{C}$, and the morphisms $(X\to T) \to (Y \to T)$ are morphisms $X \to Y$
in $\mathcal{C}$ such that
\[\begin{tikzcd}
X \ar[dr] \ar[rr]  & & Y \ar[dl] \\
& T
\end{tikzcd}\]
is a commutative diagram. Show that the canonical functor
$\mathcal{P}(\mathcal{C}_{/T}) \to \mathcal{P}(\mathcal{C})_{/h_T}$ is an equivalence.
\end{exer}
%
\begin{proof}
Let $A : \mathcal{P}(\mathcal{C}_{/T}) \to \mathcal{P}(\mathcal{C})_{/h_T}$ denote
the canonical functor. Explicitly, given a functor
$F \in \mathcal{P}(\mathcal{C}_{/T})
= \mathrm{Fun}(\mathcal{C}^{\mathrm{op}}_{/T},\mathsf{Set})$, $A(F)$ is the
functor $\mathcal{C}^{\mathrm{op}} \to \mathsf{Set}$ given on objects by
\[
[A(F)](X) = \coprod_{p \in \mathrm{Map}(X,T)} F\left(X \xrightarrow{p} T \right)
\]
Given a morphism $f : X \to Y$, we let
\[
[A(F)](f)  = \coprod_{\substack{p \in \mathrm{Map}(X,T) \\ q \text{ compatible}}}
F\left(\begin{tikzcd}
X \ar[rr, "f"] \ar[dr, "p"'] && Y \ar[dl, "q"] \\
& T
\end{tikzcd}\right)
\]
where $q \in \mathrm{Map}(Y,T)$ is compatible with $p$ if the diagram commutes.
The functor $A(F)$ comes equipped with a canonical natural transformation $\eta$ to
$h_T$ corresponding to forgetting everything except for the morphism $p : X \to T$.
To show that $A$ is an equivalence, we provide an inverse functor
$B : \mathcal{P}(\mathcal{C}_{/h_T}) \to \mathcal{P}(\mathcal{C}_{/T})$. Given a
functor $F : \mathcal{C}^{\mathrm{op}} \to \mathsf{Set}$ with a natural transformation
$\eta$ to $h_T$, define the functor $B(F) \in \mathcal{P}(\mathcal{C}_{/T})$ on
objects by
\[
B(F)(X \xrightarrow{p} T) = \eta_X\inv(p) \subset F(X)
\]
On morphisms, we let
\[
B(F)\left(\begin{tikzcd}
X \ar[rr,"f"] \ar[dr, "p"'] && Y \ar[dl, "q"]\\
& T
\end{tikzcd}\right) = F(f)\vert_{\eta_X\inv(p)}
\]
Where we use the fact that $\eta$ is a natural transformation and the
fact that the diagram commutes to deduce that the image of $\eta_X\inv(p)$ under $F(f)$
is contained in $\eta_Y\inv(q)$. \\

We then show that $A$ and $B$ are inverses of each other. We first consider the
functor $A \circ B:\mathcal{P}(\mathcal{C})_{/h_T}\to\mathcal{P}(\mathcal{C})_{/h_T}$.
Given a functor $F \in \mathcal{P}(\mathcal{C})_{/h_T}$ equipped with a natural
transformation $\eta$ to $h_T$, we have that
\[
[(A \circ B)(F)](X) = \coprod_{p \in \mathrm{Map}(X,T)} \eta_X\inv(p) = F(X)
\]
So $A \circ B$ is equivalent to the identity functor. In the other direction, given
a functor $F \in \mathcal{P}(\mathcal{C}_{/T})$, we have that
\[
[(B \circ A)(F)](X\xrightarrow{p} T) = \eta_X\inv(p)
\]
where $\eta_X : F(X) \to \mathrm{Map}(X,T)$ is the map that takes an element of
\[
F(X \xrightarrow{p} T)
\subset \coprod_{q \in \mathrm{Map}(X,T)} F(X\xrightarrow{q} T)
\]
and maps it to $p$, so $\eta_X\inv(p)$ is exactly $F(X \xrightarrow{p} T)$, so
$B \circ A$ is also equivalent to the identity functor.
\end{proof}
%
\setcounter{section}{2}
%
\setcounter{thm}{1}
%
\begin{exer}
Show that if $X \to Y$ is an \'{e}tale morphism of smooth varieties over $\C$,
the every point of $X$ has a neighborhood in the analytic topology which
is a homeomorphism onto its image.
\end{exer}
%
\begin{proof}
It suffices to check on affine opens, and we may further assume that $X \to Y$ is
standard \'{e}tale. Then the condition that the determinant of $[\partial_jf_i]$ is
a unit implies that the morphism induces an isomorphism on cotangent spaces,
which implies that the map is a local diffeomorphism. Therefore, in the analytic
topology, we may find sufficeintly small open sets such that the map
is a diffeomorphism onto its image.
\end{proof}
%
\begin{exer}
Use the adjunction between $f^*$ and $f_*$ to show that $\mathrm{Qcoh}_{/S}$
is a fibered category over $\mathrm{Sch}/S$ and an arrow $(X,E) \to (Y,F)$ is
cartesian if and only if the homomorphism $F \to f_*(E)$ induces an isomorphism
$f^*(F) \cong E$.
\end{exer}
%
\begin{proof}
Let $p : \mathrm{Qcoh}_{/S} \to \mathrm{Sch}_{/S}$ denote the functor mapping
a pair $(X,E)$ to $X$. To show that $\mathrm{Qcoh}_{/S}$ is a fibered category
over $\mathrm{Sch}_{/S}$, we want to show that given a morphism of $S$-schemes
$f : X \to Y$ and a quasi-coherent sheaf $F$ over $Y$ that there
exists a $p$-cartesian morphism $\phi : (X,E) \to (Y,F)$ with
$E$ quasi-coherent over $X$ and $p(\phi) = f$. \\

Suppose we have a cartesian arrow $(X,E) \to (Y,F)$ given by a morphism $f : X \to Y$ of
schemes and a sheaf morphism $\varphi_f : F \to f_*E$. Then for any other pair
$(Z,G) \in \mathrm{Qcoh}_{/S}$ we know that that a scheme morphism $g : Z \to X$
and an arrow $(Z,G) \to (Y,F)$ given by scheme morphism $h$ and sheaf map $\varphi_h$
satisfying the compatibility condition $h = f \circ g$ is equivalent to an
arrow $(Z,G) \to (X,E)$ which is given by $g$ and a sheaf morphism $\varphi_g$
such that $\varphi_h : F \to h_*G = f_*g_*G$ is equal to the composition of
$\varphi_f$ and the map $f_*E \to f_*g_*G$ induced by $\varphi_g$. Using
the adjunction between $f^*$ and $f^*$, we get an isomorphism
$F \to f_*f^*F$ by applying the adjunction to the identity map
$f^*F \to f^*F$. We claim that the arrow $(X,f^*F) \to (Y,F)$ given by
$f$ and  this isomorphism satisfies the conditions of a cartesian arrow.
Let $g$ and $(h, \varphi_h)$ be as above. We want to show that
this induces a unique arrow $(g,\varphi_g) : (Z,G) \to (X,f^*F)$. We know by
the adjunction that a map $f^*F \to g_*G$ is equivalent to a map
$F \to f_*g_*G$, so we can and must take $\varphi_g$ to be the map corresponding
to $\varphi_h$. \\

Now suppose we are given a cartesian arrow $(f,\varphi_f) : (X,E) \to (Y,F)$.
We claim that the map $f^*F \to E$ induced by $\varphi_f$ is an isomorphism.
It suffices to provide an inverse arrow. Since $(X,f^*F) \to (Y,F)$ is also
cartesian, we have that $(f,\varphi_f)$ induces an arrow
$(X,E) \to (X, f^*F)$. By standard arguments, this must be the
inverse to $(f,\varphi_f)$, since the composition in both directions must
be equal to the identity by the universal property of being cartesian.
\end{proof}
%
\setcounter{section}{3}
%
\setcounter{thm}{0}
%
\begin{exer}
If $\mathcal{F} = \mathrm{Qcoh}_{/S} \to \mathrm{Sch}_{/S}$ and $D$ is a diagram
consisting of two schemes and two non-identity arrows $f,g : X \to Y$, then show That
the category of cartesian sections of $\mathcal{F}$ over $D$ is equivalent to the
category of quasi-coherent sheaves $E$ over $Y$ along with an isomorphism
$f^*(E) \cong g^*(E)$.
\end{exer}
%
\begin{proof}
We first unpack some of the definitions. A cartesian section
$\sigma \in \Gamma^{\mathrm{cart}}_{\mathrm{Sch}_{/S}}(D,\mathrm{Qcoh}_{/S})$ is a
diagram in $\mathrm{Qcoh}_{/S}$ of the form
\[\begin{tikzcd}
(X,F) \ar[r, "{(f,\varphi_f)}", bend left=30] \ar[r, "{(g,\varphi_g)}"', bend right=30]
& (Y,E)
\end{tikzcd}\]
where the arrows $(f,\varphi_f)$ and $(g,\varphi_g)$ are cartesian. By the
previous problem, the arrows being cartesian implies that the maps $\psi_f : f^*E \to F$
and $\psi_g : g^*E \to F$ by applying the adjunction to $\varphi_f$ and $\varphi_g$
are isomorphisms. Given two cartesian sections $\sigma_1$ and $\sigma_2$ where
\[
\sigma_1 = \left(\begin{tikzcd}
(X,F_1) \ar[r, "{(f_1,\varphi_{f_1})}", bend left=30]
\ar[r, "{(g_1,\varphi_{g_1})}"', bend right=30]
& (Y,E_1)
\end{tikzcd}\right) \qquad
\sigma_2 = \left( \begin{tikzcd}
(X,F_2) \ar[r, "{(f_2,\varphi_{f_2})}", bend left=30]
\ar[r, "{(g_2,\varphi_{g_2})}"', bend right=30]
& (Y,E_2)
\end{tikzcd}\right)
\]
a morphism $\sigma_1 \to \sigma_2$ is a natural transformation $\eta$, which
is given by the data of two arrows $(\id_X, \eta_X) : (X,F_1) \to (X,F_2)$ and
$(\id_Y, \eta_Y) : (Y, E_1) \to (Y, E_2)$  along with the condition that
\[
(f_2,\varphi_{f_2}) \circ (\id_X,\eta_X) = (\id_Y,\eta_Y) \circ (f_1, \varphi_{f_1})
\]
Then let $\mathcal{C}$ denote the category where the objects are pairs
$(E, \rho)$ where $E \in \mathrm{Qcoh}_{/S}(Y)$ and $\rho : f^*E \to g^*E$ is
an isomorphism, and the morphisms $(E_1,\rho_1) \to (E_2,\rho_2)$ are
sheaf morphisms $\eta : E_1 \to E_2$ such that the following square commutes:
\[\begin{tikzcd}
f^*E_1 \ar[d, "f^*\eta"']\ar[r, "\rho_1"] & g^*E_1 \ar[d, "g^*\eta"] \\
f^*E_2 \ar[r, "\rho_2"'] & g^*E_2
\end{tikzcd}\]
We want to show that there is an equivalence of categories
$\Gamma^{\mathrm{cart}}_{\mathrm{Sch}_{/S}}(D, \mathrm{Qcoh}_{/S}) \to \mathcal{C}$.
Consider the functor given on objects by mapping a cartesian section
\[\begin{tikzcd}
(X,F) \ar[r, "{(f,\varphi_f)}", bend left=30] \ar[r, "{(g,\varphi_g)}"', bend right=30]
& (Y,E)
\end{tikzcd}\]
to the pair $(E, \psi_g\inv \circ \psi_f)$, where $\psi_g,\psi_f$ are the
maps given by applying the adjunction to $\varphi_g$ and $\varphi_f$ respectively. On
morphisms, we take a natural transformation $\eta$ to the sheaf morphism $\eta_Y$,
noting that the following square commutes:
\[\begin{tikzcd}
f^*E_1 \ar[r, "\psi_{g_1}\inv \circ \psi_{f_1}"]
\ar[d, "f^*\eta_Y"'] & g^*E_1 \ar[d, "g^*\eta_Y"] \\
f^*E_2 \ar[r, "\psi_{g_2}\inv \circ \psi_{f_2}"'] & g^*E_2
\end{tikzcd}\]
To show that this functor is an equivalence, we show it is fully faithful and
essentially surjective. The fact that the functor is essentially surjective is
relatively clear, is we can realize any isomorphism $f^*E\to g^*E$ as going
through some intermediate sheaf $F$ (for example, take $F = f^*E$). Showing that
the functor is fully faithful amounts to showing that a natural transformation $\eta$
\[
\left(\begin{tikzcd}
(X,F_1) \ar[r, "{(f_1,\varphi_{f_1})}", bend left=30]
\ar[r, "{(g_1,\varphi_{g_1})}"', bend right=30]
& (Y,E_1)
\end{tikzcd}\right) \xrightarrow{\eta}
\left( \begin{tikzcd}
(X,F_2) \ar[r, "{(f_2,\varphi_{f_2})}", bend left=30]
\ar[r, "{(g_2,\varphi_{g_2})}"', bend right=30]
& (Y,E_2)
\end{tikzcd}\right)
\]
%TODO fix this?
is equivalent to a sheaf morphism $\alpha : E_1 \to E_2$. This
is clear, since the condition that $\eta$ be a natural transformation is that
\[
(f_2,\varphi_{f_2}) \circ (\id_X,\eta_X) = (\id_Y,\eta_Y) \circ (f_1, \varphi_{f_1})
\]
which implies that we can recover $\eta_X$ from $\eta_Y$. Therefore, declaring
$\eta_Y = \alpha$ specifies $\eta$ uniquely, so the functor is fully faithful,
so it is an equivalence.
\end{proof}
%
\begin{exer}
Show that if the indexing category $\mathcal{I}$ has a terminal object $*$, then
restricting any diagram $D : \mathcal{I} \to \mathcal{C}$ to $*$ defines an equivalence
$\Gamma^{\mathrm{cart}}_{\mathcal{C}}(D,\mathcal{F}) \cong \mathcal{F}(D(*))$.
\end{exer}
%
\begin{proof}
Explicitly, the functor restricting $*$ is given by mapping a cartesian
section $\sigma$ to $\sigma(*)$ and mapping a natural transformation
$\eta : \sigma_1 \to \sigma_2$ to map $\eta_* : \sigma_1(*) \to \sigma_2(*)$. This
is fully faithful by essentially the same argument in the previous exercise, so all
that remains is to show that the functor is essentially surjective. This amounts
to showing that given any object $\xi \in \mathcal{F}$ with $p(\xi) = D(*)$,
there exists a cartesian section $\sigma$ with $\sigma(*) = \xi$. If we fix a
cleavage for $F$, we can construct a cartesian section where an object
$A \in \mathcal{I} $ maps to source of the unique cartesian arrow lifting the map
$D(A) \to D(*)$, which gives us that the functor is essentially surjective.
\end{proof}
%
\begin{exer}
Let $p : \mathcal{F} \to \mathcal{C}$ be a fibered category, and fix a cleavage for
$\mathcal{F}$. Show that for any cover $\mathcal{U} = \set{U_\alpha \to U}$,
the category $\mathrm{Desc}_{\mathcal{F}}(\mathcal{U})$ is equivalent to the
category $\mathcal{D}$ of pairs of collections
$(\set{\xi_\alpha}, \set{\phi_{\alpha\beta}})$,
of objects $\xi_\alpha \in \mathcal{F}(U_\alpha)$ and isomorphisms
$\phi_{\alpha\beta} : \mathrm{pr}_1^*(\xi_\beta) \to \mathrm{pr}_0^*(\xi_\alpha)$
such that for any $\alpha,\beta,\gamma$, we have
\[
\mathrm{pr}_{02}^*(\phi_{\alpha\gamma}) =
\mathrm{pr}_{01}^*(\phi_{\alpha\beta}) \circ \mathrm{pr}_{12}^*(\phi_{\beta\gamma}):
\mathrm{pr}_2^*(\xi_\gamma) \to \mathrm{pr}_0^*(\xi_\alpha)
\]
\end{exer}
%
\begin{proof}
We first note that a morphism in $\mathcal{D}$ between
$(\set{\xi_\alpha}, \set{\phi_{\alpha\beta}})$ and
$(\set{\zeta_\alpha}, \set{\psi_{\alpha\beta}})$ consists of a collection of
morphisms $\set{\rho_{\alpha} : \xi_\alpha \to \zeta_\alpha}$ such that for any
$\alpha,\beta$, the following diagram commutes:
\[\begin{tikzcd}
\mathrm{pr}_1^*(\xi_\beta \ar[r, "\phi_{\alpha\beta}"]\ar[d,"\mathrm{pr}_1^*\rho_\beta"']
& \mathrm{pr}_0^*(\xi_\alpha)\ar[d,"\mathrm{pr}_0^*\rho_\alpha"] \\
\mathrm{pr}_1^*\zeta_\beta \ar[r, "\psi_{\alpha\beta}"'] & \mathrm{pr}_0^*\zeta_\alpha
\end{tikzcd}\]
We then define the functor $\mathrm{Desc}_{\mathcal{F}}(\mathcal{U}) \to \mathcal{D}$
Let $D_{\mathcal{U}} : \mathcal{I}_I \to \mathcal{C}$ denote the diagram
in $\mathcal{C}$, and let $\sigma : \mathcal{I}_I \to \mathcal{F}$ be a cartesian
section. Then we map $\sigma$ to the pair
$(\set{\sigma(J)}_{J \in \mathcal{I}}, \set{\sigma(\varphi)}_{\varphi \in K})$
where $K \subset \mathrm{Mor}(\mathcal{I})$ is the set of arrows between the first
two ``levels" of the descent diagram. We map a natural transformation
$\eta : \sigma_1 \to \sigma_2$ to the collection
$\set{\eta_J : \sigma_1(J) \to \sigma_2(J)}$. We first note that this functor
is essentially surjective, since given an object
$(\set{\xi_\alpha}, \set{\phi_{\alpha\beta}}) \in \mathcal{D}$, we can make a
cartesian section $\sigma$, where the objects on the first ``level" are given
by the $\xi_\alpha$, and we fill in the rest of the objects in the diagram using
the cleavage. The morphisms are then induced using the isomorphisms from the cleavage
and the isomorphism $\phi_{\alpha\beta}$, and the fact that the choice doesn't
matter for the third ``level" amounts to the cocycle condition. \\

We then show the functor is fully faithful. Let
$\set{\rho_\alpha : \xi_\alpha \to \zeta_\alpha}$ be a morphism in $\mathcal{D}$.
Then we can construct a natural transformation $\eta$, where the maps
$\eta_\alpha : \xi_\alpha \to \zeta_\alpha$ is given by the $\rho_\alpha$,
which then uniquely determines the rest of the maps of the diagram by using the cleavage
and the pullbacks of the maps in a similar fashion to $3.2$ and $3.1$.
\end{proof}
%
\begin{exer}
We know that the composition of cartesian fibrations
$\mathcal{F}' \to \mathcal{F} \to \mathcal{C}$ is again a cartesian fibration.
Given a Grothendieck topology on $\mathcal{C}$, we can define a Grothendieck
topology on $\mathcal{F}$ whose coverings consist of families of cartesian arrows
$\set{\xi_i \to \xi}_{i \in I}$ such that $\set{p(\xi_i) \to p(\xi)}_{i \in I}$
is a covering in $\mathcal{C}$. Show that if $\mathcal{F}$ is a stack over
$\mathcal{C}$ and $\mathcal{F}'$ is a stack over $\mathcal{F}$ with the induced
topology, then $\mathcal{F}'$ is a stack over $\mathcal{C}$.
\end{exer}
%
\begin{proof}
We want to show that for any covering $\mathcal{U} = \set{U_\alpha \to U}$ in
$\mathcal{C}$, the restriction functor
$\Gamma^{\mathrm{cart}}_{\mathcal{C}}(D_{\mathcal{U}}^+,\mathcal{F})
\to \Gamma^{\mathrm{cart}}_{\mathcal{C}}(D_{\mathcal{U}},\mathcal{F})$ is an
equivalence. Let $\mathcal{I}$ denote the indexing category for the descent diagram,
and $\mathcal{I}_+$ the indexing category for the augmented descent diagram.
Then let $\sigma$ be a cartesian section of $\mathcal{F}' \to \mathcal{C}$. We
note that $\sigma$ is cartesian over $p \circ \sigma$. Then
$p \circ \sigma$ is a cartesian section of $\mathcal{F} \to \mathcal{C}$, since
any cartesian arrow of $\mathcal{F}' \to \mathcal{C}$ must project to a cartesian
arrow of $\mathcal{F} \to \mathcal{C}$. Since $\mathcal{F} \to \mathcal{C}$ is a
stack, this is equivalent to an augmented cartesian section
$\sigma_+ : \mathcal{I}_+ \to \mathcal{F}$. Since $\mathcal{F}' \to \mathcal{F}$ is
a cartesian fibration, we can lift the diagram $\sigma_+$ to a diagram
$\widetilde{\sigma_+} : \mathcal{I}_+ \to \mathcal{F}'$, which is cartesian
over $\mathcal{F}$, so since $\mathcal{F}' \to \mathcal{F}$ is a stack, this is
equivalent to the original section $\sigma$.
\end{proof}
%
\begin{exer}
Show that if $f : \mathcal{C} \to \mathcal{D}$ and $\mathcal{F}$ is a fibered category
over $\mathcal{D}$, then $f\inv(\mathcal{F}$) is a fibered category over $\mathcal{C}$,
where an arrow in $f\inv(\mathcal{F})$ is cartesian if and only if the corresponding
arrow in $\mathcal{F}$ is cartesian.
\end{exer}
%
\begin{proof}
We first show that arrows in $f\inv(\mathcal{F})$ are cartesian if and only if the
corresponding arrows in $\mathcal{F}$ are cartesian. Suppose we have a cartesian arrow
$(X,\xi) \to (Y,\eta)$ in $f\inv(\mathcal{F})$. Then
$(Z,\alpha) \to (X,\xi)$ is equivalent data to a morphism $Z \to X$ in $\mathcal{C}$
and a morphism $\alpha \to \eta$ in $\mathcal{D}$ such that $\alpha \to \eta$
covers the map $f(Z) \to f(X)$. But this is equivalent to the arrow
$\xi\to\eta$ covering $f(X) \to f(Y)$ being cartesian. To show that $f\inv(\mathcal{F})$
is a fibered category, given a morphism $X \to Y$ in $\mathcal{C}$ and an object
$(Y,\eta)$ lying over $Y$, we can pick any cartesian lift of the morphism
$f(X \to Y)$ in $\mathcal{F}$, which gives us the desired cartesian lift in
$f\inv(\mathcal{F})$
\end{proof}
%
\setcounter{thm}{12}
%
\begin{exer}
Let $\mathcal{F}$ be a fibered category over $\mathrm{Sch}_{/S}$ such that for
any set of schemes $\set{U_i}$, the canonical morphism
$\mathcal{F}(\coprod_iU_i) \to \prod_i \mathcal{F}(U_i)$ is an equivalence of
categories. Show that if $\mathcal{F}$ satisfies descent with respect to the
covering $\set{U_i \to U}$ if and only if it satisfies descent with respect to
the covering $\set{U' \defeq \coprod_i U_i \to U}$. Use this to show that a fibered
category over $\mathrm{Sch}_{/S}$ is a stack if and only if $\mathcal{F}$ maps disjoint
unions of schemes to products of categories, and $\mathcal{F}$ satisfies descent
to all coverings $U' \to U$.
\end{exer}
%
\begin{proof}
For the first part, let $D_{U'}$ denote the descent diagram for the
covering $U'$, and let $D_U$ denote the descent diagram for the covering by
all the $U_i$. Then let
$\sigma \in \Gamma^{\mathrm{cart}}_{\mathrm{Sch}_{/S}}(D_{U'},\mathcal{F})$
be a cartesian section. Then upon fixing a cleavage, we know that $\sigma$ is equivalent
to an object $\xi$ over $\coprod_i U_i$ and an isomorphism
$\phi : \mathrm{pr}_1^*\xi \to \mathrm{pr}_0^*\xi$. Then under canonical functor
$\mathcal{F}(\coprod_i U_i) \to \prod_i \mathcal{F}(U_i)$, we have that
$\xi$ is mapped to the tuple $(\xi_i)$ where $\xi_i$ is the source of the cartesian
lift of the inclusion $U_i \hookrightarrow U$ determined by a cleavage. Similarly,
the isomorphism $\phi$ becomes a tuple $(\phi_{ij})$ of maps
$\phi_{ij} : \mathrm{pr}_1^*\xi_j \to \mathrm{pr}_0^*\xi_i$, which satisfy the
cocycle condition because $\phi$ did. This is then equivalent to a cartesian
section of the diagram $D_U$. Therefore, if $\mathcal{F}$ satisfies descent
with respect to either cover, by going through the equivalence, we get that it
satisfies descent with respect to the other cover, since we have that the
category of cartesian sections over either augmented descent diagram is
equivalent to $\mathcal{F}(U)$. \\

For the second part, let $p : \mathcal{F} \to \mathrm{Sch}_{/S}$ be any fibered
category. 
\end{proof}
%
\end{document}