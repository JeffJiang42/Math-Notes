\documentclass[psamsfonts, 12pt]{amsart}
%
%-------Packages---------
%
\usepackage[h margin=1 in, v margin=1 in]{geometry}
\usepackage{amssymb,amsfonts}
\usepackage[all,arc]{xy}
\usepackage{tikz-cd}
\usepackage{enumerate}
\usepackage{mathrsfs}
\usepackage{amsthm}
\usepackage{mathpazo}
\usepackage{float}
\usepackage[backend=biber]{biblatex}
\addbibresource{bibliography.bib}
%\usepackage{charter} %another font
%\usepackage{eulervm} %Vakil font
\usepackage{yfonts}
\usepackage{mathtools}
\usepackage{enumitem}
\usepackage{mathrsfs}
\usepackage{fourier-orns}
\usepackage[all]{xy}
\usepackage{hyperref}
\usepackage{url}
\usepackage{mathtools}
\usepackage{graphicx}
\usepackage{pdfsync}
\usepackage{mathdots}
\usepackage{calligra}
\usepackage{import}
\usepackage{xifthen}
\usepackage{pdfpages}
\usepackage{transparent}

\usepackage{tgpagella}
\usepackage[T1]{fontenc}
%
\usepackage{listings}
\usepackage{color}

\definecolor{dkgreen}{rgb}{0,0.6,0}
\definecolor{gray}{rgb}{0.5,0.5,0.5}
\definecolor{mauve}{rgb}{0.58,0,0.82}

\lstset{frame=tb,
  language=Matlab,
  aboveskip=3mm,
  belowskip=3mm,
  showstringspaces=false,
  columns=flexible,
  basicstyle={\small\ttfamily},
  numbers=none,
  numberstyle=\tiny\color{gray},
  keywordstyle=\color{blue},
  commentstyle=\color{dkgreen},
  stringstyle=\color{mauve},
  breaklines=true,
  breakatwhitespace=true,
  tabsize=3
  }
%
%--------Theorem Environments--------
%
\newtheorem{thm}{Theorem}[section]
\newtheorem*{thm*}{Theorem}
\newtheorem{cor}[thm]{Corollary}
\newtheorem{prop}[thm]{Proposition}
\newtheorem{lem}[thm]{Lemma}
\newtheorem*{lem*}{Lemma}
\newtheorem{conj}[thm]{Conjecture}
\newtheorem{quest}[thm]{Question}
%
\theoremstyle{definition}
\newtheorem{defn}[thm]{Definition}
\newtheorem*{defn*}{Definition}
\newtheorem{defns}[thm]{Definitions}
\newtheorem{con}[thm]{Construction}
\newtheorem{exmp}[thm]{Example}
\newtheorem{exmps}[thm]{Examples}
\newtheorem{notn}[thm]{Notation}
\newtheorem{notns}[thm]{Notations}
\newtheorem{addm}[thm]{Addendum}
\newtheorem{exer}[thm]{Exercise}
%
\theoremstyle{remark}
\newtheorem{rem}[thm]{Remark}
\newtheorem*{claim}{Claim}
\newtheorem*{aside*}{Aside}
\newtheorem*{rem*}{Remark}
\newtheorem*{hint*}{Hint}
\newtheorem*{note}{Note}
\newtheorem{rems}[thm]{Remarks}
\newtheorem{warn}[thm]{Warning}
\newtheorem{sch}[thm]{Scholium}
%
%--------Macros--------
\renewcommand{\qedsymbol}{$\blacksquare$}
\renewcommand{\sl}{\mathfrak{sl}}
\newcommand{\Bord}{\mathsf{Bord}}
\renewcommand{\hom}{\mathrm{Hom}}
\renewcommand{\emptyset}{\varnothing}
\renewcommand{\O}{\mathcal{O}}
\newcommand{\R}{\mathbb{R}}
\newcommand{\ib}[1]{\textbf{\textit{#1}}}
\newcommand{\Q}{\mathbb{Q}}
\newcommand{\Z}{\mathbb{Z}}
\newcommand{\N}{\mathbb{N}}
\newcommand{\C}{\mathbb{C}}
\newcommand{\A}{\mathbb{A}}
\newcommand{\F}{\mathbb{F}}
\newcommand{\M}{\mathcal{M}}
\newcommand{\dbar}{\overline{\partial}}
\newcommand{\zbar}{\overline{z}}
\renewcommand{\S}{\mathbb{S}}
\newcommand{\V}{\vec{v}}
\newcommand{\RP}{\mathbb{RP}}
\newcommand{\CP}{\mathbb{CP}}
\newcommand{\B}{\mathcal{B}}
\newcommand{\GL}{\mathrm{GL}}
\newcommand{\SL}{\mathrm{SL}}
\newcommand{\SP}{\mathrm{SP}}
\newcommand{\SO}{\mathrm{SO}}
\newcommand{\SU}{\mathrm{SU}}
\newcommand{\gl}{\mathfrak{gl}}
\newcommand{\g}{\mathfrak{g}}
\newcommand{\Bun}{\mathsf{Bun}}
\newcommand{\inv}{^{-1}}
\newcommand{\bra}[2]{ \left[ #1, #2 \right] }
\newcommand{\set}[1]{\left\lbrace #1 \right\rbrace}
\newcommand{\abs}[1]{\left\lvert#1\right\rvert}
\newcommand{\norm}[1]{\left\lVert#1\right\rVert}
\newcommand{\transv}{\mathrel{\text{\tpitchfork}}}
\newcommand{\defeq}{\vcentcolon=}
\newcommand{\enumbreak}{\ \\ \vspace{-\baselineskip}}
\let\oldexists\exists
\renewcommand\exists{\oldexists~}
\let\oldL\L
\renewcommand\L{\mathfrak{L}}
\makeatletter
\newcommand{\incfig}[2]{%
    \fontsize{48pt}{50pt}\selectfont
    \def\svgwidth{\columnwidth}
    \scalebox{#2}{\input{#1.pdf_tex}}
}
%
\newcommand{\tpitchfork}{%
  \vbox{
    \baselineskip\z@skip
    \lineskip-.52ex
    \lineskiplimit\maxdimen
    \m@th
    \ialign{##\crcr\hidewidth\smash{$-$}\hidewidth\crcr$\pitchfork$\crcr}
  }%
}
\makeatother
\newcommand{\bd}{\partial}
\newcommand{\lang}{\begin{picture}(5,7)
\put(1.1,2.5){\rotatebox{45}{\line(1,0){6.0}}}
\put(1.1,2.5){\rotatebox{315}{\line(1,0){6.0}}}
\end{picture}}
\newcommand{\rang}{\begin{picture}(5,7)
\put(.1,2.5){\rotatebox{135}{\line(1,0){6.0}}}
\put(.1,2.5){\rotatebox{225}{\line(1,0){6.0}}}
\end{picture}}
\DeclareMathOperator{\id}{id}
\DeclareMathOperator{\im}{Im}
\DeclareMathOperator{\codim}{codim}
\DeclareMathOperator{\coker}{coker}
\DeclareMathOperator{\supp}{supp}
\DeclareMathOperator{\inter}{Int}
\DeclareMathOperator{\sign}{sign}
\DeclareMathOperator{\sgn}{sgn}
\DeclareMathOperator{\indx}{ind}
\DeclareMathOperator{\alt}{Alt}
\DeclareMathOperator{\Aut}{Aut}
\DeclareMathOperator{\trace}{trace}
\DeclareMathOperator{\ad}{ad}
\DeclareMathOperator{\End}{End}
\DeclareMathOperator{\Ad}{Ad}
\DeclareMathOperator{\Lie}{Lie}
\DeclareMathOperator{\spn}{span}
\DeclareMathOperator{\dv}{div}
\DeclareMathOperator{\grad}{grad}
\DeclareMathOperator{\Sym}{Sym}
\DeclareMathOperator{\sheafhom}{\mathscr{H}\text{\kern -3pt {\calligra\large om}}\,}
\newcommand*\myhrulefill{%
   \leavevmode\leaders\hrule depth-2pt height 2.4pt\hfill\kern0pt}
\newcommand\niceending[1]{%
  \begin{center}%
    \LARGE \myhrulefill \hspace{0.2cm} #1 \hspace{0.2cm} \myhrulefill%
  \end{center}}
\newcommand*\sectionend{\niceending{\decofourleft\decofourright}}
\newcommand*\subsectionend{\niceending{\decosix}}
\def\upint{\mathchoice%
    {\mkern13mu\overline{\vphantom{\intop}\mkern7mu}\mkern-20mu}%
    {\mkern7mu\overline{\vphantom{\intop}\mkern7mu}\mkern-14mu}%
    {\mkern7mu\overline{\vphantom{\intop}\mkern7mu}\mkern-14mu}%
    {\mkern7mu\overline{\vphantom{\intop}\mkern7mu}\mkern-14mu}%
  \int}
\def\lowint{\mkern3mu\underline{\vphantom{\intop}\mkern7mu}\mkern-10mu\int}
%
%--------Hypersetup--------
%
\hypersetup{
    colorlinks,
    citecolor=black,
    filecolor=black,
    linkcolor=blue,
    urlcolor=blacksquare
}
%
%--------Solution--------
%
\newenvironment{solution}
  {\begin{proof}[Solution]}
  {\end{proof}}
%
%--------Graphics--------
%
%\graphicspath{ {images/} }
%
\begin{document}
%
\author{Jeffrey Jiang}
%
\title{Variations of Hodge Structure}
%
\maketitle
%
\tableofcontents
%
\section*{Notation and Conventions}
%
For a complex manifold $X$, we let $\O_X$ denote its sheaf of holomorphic functions.
We let $\Omega^k_X$ denote the sheaf of holomorphic $k$-forms on $X$, and
we let $\mathcal{A}^k_X$ denote the sheaf of smooth complex $k$-forms on $X$.
%
\section{Hodge Structures}
%
The purpose of a Hodge structure is to abstract away the properties of the
cohomology groups of a compact Kahler manifold into a linear algebraic gadget.
%
\section{Flat Bundles and Local Systems}
%
\begin{defn}
Let $X$ be a complex manifold. A \ib{local system} on $X$ is a sheaf $H$ of
abelian groups on $X$ that is isomorphic to the constant sheaf $\underline{A}$
for some abelian group $A$.
\end{defn}
%
In most cases, we will discuss local systems where the abelian group $A$ is
a vector space over $\R$ or $\C$. Given a local system $H$ of $\C$-vector
spaces, we can obtain a sheaf $\mathcal{H}$ of free $\O_X$-modules over $X$ by
taking the tensor product
\[
\mathcal{H} \defeq H \otimes_{\underline{\C}} \O_X
\]
Where we regard $\O_X$ as a sheaf of $\underline{\C}$-algebras via the inclusion
$\underline{\C} \hookrightarrow \O_X$. The same discussion holds if if $H$
is a local system of abelian groups or $\R$ vector spaces by tensoring over
$\underline{\Z}$ and $\underline{\R}$ respectively. We note that the sheaf
$\mathcal{H}$ is isomorphic to $\O_X^n$, which is the sheaf of holomorphic sections of
the trivial rank $n$ vector bundle $X \times \C^n$. However, we also have the
data of our original sheaf $H$, which we can identify as a subsheaf of
$\mathcal{H}$ via the mapping $h \mapsto h \otimes 1$. Our goal will be to identify
$H$ as a subsheaf of distinguished sections of the trivial bundle $X \times \C^n$.
%
\begin{defn}
Let $\mathcal{E}$ be a sheaf of $\O_X$-modules. A \ib{connection} on $\mathcal{E}$
is a sheaf morphism $\nabla : \mathcal{E} \to \mathcal{E} \otimes_{\O_X} \Omega^1_X$
such that
\begin{enumerate}
  \item $\nabla$ is $\underline{\C}$-linear.
  \item For an open set $U$, a function $f \in \O_X(U)$, and a section
  $s \in \mathcal{E}(U)$, we have that $\nabla$ satisfies the Leibniz rule
  \[
  \nabla(fs) = s \otimes df + f\nabla(s)
  \]
\end{enumerate}
\end{defn}
%
In the case that $\mathcal{E}$ is the sheaf of holomorphic sections
of a holomorphic vector bundle $E \to X$, the connection $\nabla$ is corresponds to
the notion of a \emph{holomorphic} connection on $E$. By replacing $\O_X$ with the sheaf
of smooth complex valued functions and $\Omega^1_X$ with $\mathcal{A}^1_X$, the same
definition yields the more traditional definition of a connection on a smooth complex
vector bundle.
%
\begin{defn}
Let $\mathcal{E}$ be a sheaf of $\O_X$-modules equipped with a connection $\nabla$.
A section $s \in \mathcal{E}(U)$ is \ib{flat} if $\nabla(s) = 0$.
\end{defn}
%
Given a sheaf $\mathcal{E}$ with connection $\nabla$, the flat sections
of $\mathcal{E}$ form a subsheaf. In the case that $\mathcal{E}$ comes from a
vector bundle $\pi : E \to X$, the flat sections correspond to the equivalent
interpretation of the connection $\nabla$ as a \ib{horizontal distribution} on
the total space $E$, i.e. a vector subbundle $D \subset TE$ such that
$D \oplus \ker d\pi = TE$. From this perspective, the flat sections of $\nabla$
correspond to sections of $E$ that lie inside $D$. Another thing to note is
that by a dimension count, we have that the rank of $D$ as a vector bundle
over $E$ is the same as a the dimension of the base space $X$, which can be
seen by noting that for any point $e \in E$, the fiber $D_e$ projects isomorphically
onto the tangent space $T_{\pi(e)}X$ under $d\pi_e$ since it is complentary to the
kernel.
%
\begin{prop}
Let $H$ be a local system of $\C$-vector spaces, and
$\mathcal{H} = H \otimes_{\underline{\C}} \O_X$ the sheaf of holomorphic sections of
the associated vector bundle. Then there exists a connection $\nabla$ on $\mathcal{H}$
such that $H$ is the sheaf of flat sections of $\nabla$.
\end{prop}
%
\begin{proof}
Over an open set $U \subset X$, we have that $\mathcal{H}(U)$ is isomorphic to
$\O_X^n(U)$. Furthermore, we may choose the isomorphism $\O^n_X(U) \to \mathcal{H}(U)$
such that the standard basis vector $e_i$ maps to an element $h_i$ of $H$, thought
of as a subsheaf of $\mathcal{H}$. Then given a local section $s$, it can
be written in this trivialization as $s = \sum_i f_ih_i$ for some holomorphic
functions $f_i \in \O_X(U)$. Define $\nabla$ by the formula
\[
\nabla(s) = \sum_i h_i \otimes df_i
\]
Furthermore, we claim that this independent of our choice of trivialization. Given
another trivialization $\set{h'_i}$, we have that $h_i$ and $h_i'$ are related by
some matrix $A \in \GL_n\C$, (i.e. $h_i = A_i^jh'_j$, using Einstein summation
convention) since $H$ is locally constant. Therefore, in the
local trivialization $\set{h'_i}$, the section $s$ can be represented as
$s = \sum_if_iA^j_ih'_j$. Therefore, if we use our definition of $\nabla$
in this trivialization, we obtain
\[
\nabla(s) = \sum_{i,j} h'_j \otimes d(f_iA^j_i) = \sum_{i,j} h'_j \otimes A^j_i df_i
= \sum_{i,j} A^j_ih'_j \otimes df_i = \sum_i h_i \otimes df_i
\]
where we use the fact that the matrix $A$ is locally constant on $U$ and $\C$-linearity
of $d$. Therefore, our local definition of $\nabla$ is well-defined, and glues to a
connection on all of $\mathcal{H}$. Furthermore, we see that a section $s$ is flat if
and only if it is locally of the form $s = \sum_i \lambda_ih_i$ for scalars
$\lambda_i \in \underline{\C}(U)$. Then by identifying $H$ with the subsheaf
$H \otimes_{\underline{\C}} 1 \subset \mathcal{H}$, we immediately see that
the flat sections are exactly the sections of $H$.
\end{proof}
%
The connection $\nabla$ defined in the proof has a special property that
is not shared by all connections. To identify this property, we need to define
a notion of curvature. \\

Let $\mathcal{E}$ be a sheaf of $\O_X$-modules equipped with a connection
$\nabla : \mathcal{E} \to \mathcal{E} \otimes_{\O_X} \Omega^1_X$. This induces a map
$\mathcal{E} \otimes_{\O_X} \Omega^k_X \to \mathcal{E} \otimes_{\O_X} \Omega^{k+1}_X$,
which we also call $\nabla$, by the formula
\[
\nabla(s \otimes \omega) = \nabla(s) \wedge \omega + s \otimes d\omega
\]
Where $\nabla(s) \wedge \omega$ is obtained by writing $\nabla(s)$ locally as a sum
of the form $\sum_i s_i \otimes \omega_i$ for $s_i \in \mathcal{E}(U)$ and
$\omega_i \in \Omega^1_X(U)$ and wedging the $\omega_i$ with $\omega$.
%
\begin{prop}
The extended connection
$\nabla:\mathcal{E}\otimes_{\O_X}\Omega^k_X \to \mathcal{E}\otimes_{\O_X}\Omega^{k+1}_X$
satisfies a graded Leibniz rule, i.e. for a function $f \in \O_X(U)$ and a local
section $\theta \in \mathcal{E}(U) \otimes \Omega^k_X(U)$, we have
\[
\nabla(f\theta) = (-1)^k \theta \wedge df + f\nabla\theta
\]
\end{prop}
%
\begin{proof}
By linearity, it suffices to check this when $\theta$ is of the form $s \otimes \omega$
for a local section $s \in \mathcal{E}(U)$ and $\omega \in \Omega^1_X(U)$. We then
compute
\begin{align*}
\nabla(f\theta) &= \nabla(f(s\otimes\omega)) \\
&= \nabla(fs \otimes \omega) \\
&= \nabla(fs) \wedge \omega + fs \otimes d\omega \\
&= (s \otimes df + f\nabla(s))\wedge \omega + fs \otimes d\omega \\
&= s \otimes(df\wedge\omega) + f\nabla(s)\wedge\omega + fs\otimes d\omega \\
&= s \otimes (df \wedge \omega) + f\nabla(s \otimes \omega) \\
&= (-1)^k s\otimes(\omega \wedge df) + f\nabla(s\otimes\omega) \\
&= (-1)^k \theta\wedge df + f\nabla(\theta)
\end{align*}
\end{proof}
%
\begin{defn}
The \ib{curvature} of a connection $\nabla$ is the map
$\Omega : \mathcal{E} \to \mathcal{E} \otimes \Omega^2_X$ given by
$\Omega \defeq \nabla \circ \nabla$.
\end{defn}
%
\begin{prop}
The curvature $\Omega$ of any connection $\nabla$ is $\O_X$-linear.
\end{prop}
%
\begin{proof}
Fix a local section $s$ and a holomorphic function $f$. Then we compute
\begin{align*}
\Omega(fs) &= \nabla(s \otimes df + f\nabla(s)) \\
&= \nabla(s \otimes df) + \nabla(f\nabla(s)) \\
&= (\nabla(s) \wedge df + s \otimes d^2f) - \nabla(s)\wedge df + f\nabla(\nabla(s)) \\
&= f\Omega(s)
\end{align*}
\end{proof}
%
If a connection has curvature equal to zero, it is said to be \ib{flat}. \\

From the perspective of horizontal distributions, the curvature tensor $\Omega$
has an interpretation as the \ib{Frobenius tensor} of the horizontal distribution $D$,
which measures the failure of $D$ to be \ib{involutive}, i.e. the failure of vector
fields on $E$ lying in $D$ to be closed under the Lie bracket. By Frobenius' theorem,
this is equivalent to $D$ being \ib{integrable}, so $\Omega$ measures the obstruction
to the existence of integral submanifolds of $D$ inside of the total space $E$.
From this perspective, flatness of a connection is exactly integrability of
the horizontal distribution.
%
\begin{prop}
The connection $\nabla$ on the vector bundle $\mathcal{H} \otimes_{\underline{\C}}\O_X$
is flat.
\end{prop}
%
\begin{proof}
Fix a local trivialization of $\mathcal{H}$ such that a local basis of sections
is given by a set $\set{h_i}$ with $h_i \in H(U)$. The local
basis of sections determines an isomorphism $\O_X^n(U) \to \mathcal{H}(U)$,
and identifies $\nabla$ with the operator that applies the de Rham differential $d$
to each component of $\O_X^n(U)$. Then since $d^2 = 0 $, we have that
$\Omega = \nabla\circ\nabla = 0$.
\end{proof}
%
This suggests the following.
%
\begin{thm}
There is an bijective correspondence
\[
\set{\text{Local systems of } \C \text{-vector spaces}} \longleftrightarrow
\set{\text{Holomorphic vector bundles with flat connection}}
\]
\end{thm}
%
\begin{proof}
In one direction, we can associate to a local system $H$ the vector bundle
$\mathcal{H}$ with the connection we defined above. In the
other direction, given a holomorphic vector bundle $\pi : E \to X$ with flat connection
$\nabla$, we claim that the sheaf $H$ of flat sections of $E$ forms a local system.
Once we do so, it's clear that the mappings specified above are inverses to
each other, giving us the desired correspondence. \\

Since the bundle $\pi : E \to X$ has a flat connection $\nabla$, we have that the
horizontal distribution $D$ is integrable, so there exists a integral
submanifolds of $E$ whose tangent spaces correspond to the fibers of $D \to E$.
Fix a basepoint $x \in X$, and consider the zero element $0 \in E_x$ the the fiber
lying over $x$. Then let $Y \subset E$ denote an integral submanifold of $D$ containing
$0$. Since $D$ is complementary to $\ker d\pi$ and $T_0Y = D_0$, we have that
$\pi\vert_Y : Y \to X$ is a local diffeomorphism at $0$, so there exists connected
open neighborhoods $U \subset Y$ and $V \subset X$ such that $\pi\vert_U$ is a
diffeomorphism $U \to V$. Furthermore, we note that by construction, $U$ is the union of
the images of flat sections $V \to E$, since it is an integral submanifold of the
horizontal distribution. Then since $U$ is an open neighborhood of the
zero section $V \to E$, taking fiberwise linear combinations of elements
of $U$ spans the fibers of $E$ lying over $V$. Since linear combinations of flat
sections are flat, we have that the fibers of $E$ lying over $V$ are generated by flat
sections. Therefore, we can identify the flat sections of $E$ over $U$ with elements
of the fiber $E_0$, since a section $\sigma$ whose value at $x$ is a vector
$v \in E_0$ is uniquely extended to a flat section in a neighborhood of $x$
by existence and uniqueness of solutions to the differential equation $\nabla\sigma = 0$.
Putting everything together, we find that the flat sections over $U$ can be
identified with constant functions $U \to E_0$, which tells us that the sheaf
of flat sections can be identified with the constant sheaf $\underline{E_0}$, so
they form a local system.
\end{proof}
%
\section{Derived Pushforwards of Sheaves}
%
To discuss the Gauss-Manin connection, we need to make a short digression into
homological algebra and derived pushforwards of sheaves. For a topological space
$X$, let $\mathrm{Sh}(X)$ denote the abelian category of sheaves of $R$-modules over
$X$, where $R = \Z$, $\R$, or $\C$.
%
\begin{defn}
Let $\pi : X \to Y$ be a continuous map. The \ib{pushforward} (also known as the
\ib{direct image}) along $\pi$ is a functor $\pi_* : \mathrm{Sh}(X) \to \mathrm{Sh}(Y)$,
where given a sheaf $\mathcal{F}$ over $X$, the sheaf $\pi_*\mathcal{F}$ is defined by
the mapping
\[
\pi_*\mathcal{F}(U) \defeq \mathcal{F}(\pi\inv(U))
\]
Given a sheaf morphism $\varphi : \mathcal{F} \to \mathcal{G}$, the sheaf morphism
$\pi_*\varphi : \pi_*\mathcal{F} \to \pi_*\mathcal{G}$ is given by
\[
\pi_*\varphi(U) \defeq \varphi(\pi\inv(U))
\]
\end{defn}
%
\begin{defn}
Given a continuous map $\pi: X \to Y$ and the \ib{pullback} along $\pi$ is a functor
$\pi^* : \mathrm{Sh}(Y) \to \mathrm{Sh}(X)$ where for a sheaf
$\mathcal{F} \in \mathrm{Sh}(Y)$, the sheaf $\pi^*\mathcal{F}$ is the
sheafification of the presheaf
\[
U \mapsto \lim_{V \supset \pi(U)} \mathcal{F}(V)
\]
Given a morphism $\varphi : \mathcal{F} \to \mathcal{G}$, the sheaf morphism
$\pi^*\varphi : \pi^*\mathcal{F} \to \pi^*\mathcal{G}$ is given by the sheaf
morphism induced by the map of presheaves
\[
\lim_{V \supset \pi(U)} \varphi(V) :
\lim_{V \supset \pi(U)} \mathcal{F}(V) \to \lim_{V \supset \pi(U)} \mathcal{G}(V)
\]
\end{defn}
%
It's clear from the definition of the pullback that for a point
$x \in X$, the stalk $(\pi^*\mathcal{F})_x$ is equal to the stalk
$\mathcal{F}_{\pi(x)}$. Furthermore, given a morphism
$\varphi : \mathcal{F} \to \mathcal{G}$ of sheaves over $Y$, we have that the
morphism $\pi^*\varphi$ is given on stalks by the same maps induced by $\varphi$.
Then since injectivity and surjectivity for sheaf maps can be checked on the level
of stalks, this immediately yields the result:
%
\begin{prop}
The functor $\pi^*$ is exact, i.e. for an exact sequence of sheaves over $Y$,
\[\begin{tikzcd}
0 \ar[r] & \mathcal{E} \ar[r] & \mathcal{F} \ar[r] & \mathcal{G} \ar[r] & 0
\end{tikzcd}\]
the sequence  of sheaves over $X$ given by
\[\begin{tikzcd}
0\ar[r]& \pi^*\mathcal{E} \ar[r] & \pi^*\mathcal{F} \ar[r] & \pi^*\mathcal{G} \ar[r]&0
\end{tikzcd}\]
is exact.
\end{prop}
%
The key fact we will need regarding $\pi^*$ and $\pi_*$ is the fact that they
are adjoint functors.
%
\begin{prop}
Let $\pi : X \to Y$ be a continuous map, and let $\mathcal{F} \in \mathrm{Sh}(X)$
and $\mathcal{G} \in \mathrm{Sh}(Y)$. Then there is a bijection
\[
\hom(\pi^*\mathcal{G},\mathcal{F}) \longleftrightarrow \hom(\mathcal{G},\pi_*\mathcal{F})
\]
that is natural in $\mathcal{F}$ and $\mathcal{G}$, i.e. functorial with respect to
maps into and out of $\mathcal{F}$ and $\mathcal{G}$.
\end{prop}
%
\begin{proof}
We provide maps in both directions. For one direction, let
$\varphi \in \hom(\pi^*\mathcal{G},\mathcal{F})$. For an open set $U \subset X$,
this gives us a map $\varphi(U) : \pi^*\mathcal{G}(U) \to \mathcal{F}(U)$.
By definition, we have that
$\pi^*\mathcal{G}(U) = \mathrm{lim}_{V\supset\pi(U)}\mathcal{G}(V)$, so the
map $\varphi(U)$ is equivalent to the data of maps
$\phi_V : \mathcal{G}(V) \to \mathcal{F}(U)$ for all open sets $V \supset \pi(U)$
such that if $W \supset V$, then $\phi_W$ is equal to the composition of $\phi_V$
with the restriction map $\mathcal{G}(W) \to \mathcal{G}(V)$. From this data,
we want to produce a sheaf morphism $\mathcal{G} \to \pi_*\mathcal{F}$. To do this,
for each open set $U \subset Y$,  we want to produce a map
$\mathcal{G}(U) \to \mathcal{F}(\pi\inv(U))$. We note that $\pi(\pi\inv(U)) = U$,
so if we consider the map
$\varphi(\pi\inv(U)) : \lim_{V \supset U}\mathcal{G}(V) \to \mathcal{F}(\pi\inv(U))$,
we may take the map $\phi_U : \mathcal{G}(U) \to \mathcal{F}(\pi\inv(U))$.
Doing this for all open sets gives us the desired sheaf morphism
$\mathcal{G} \to \pi_*\mathcal{F}$. \\

In the other direction, suppose we are given a sheaf map
$\psi \in \hom(\mathcal{G},\pi_*\mathcal{F})$. Then for an open set $U \subset Y$
we have a map $\psi(U) : \mathcal{G}(U) \to \mathcal{F}(\pi\inv(U))$. From this we
want to produce a sheaf map $\pi^*\mathcal{G} \to \mathcal{F}$. Let $W \subset X$
be an open set. We then must give a map $\pi^*\mathcal{G}(W) \to \mathcal{F}(W)$.
By definition, we have that
$\pi^*\mathcal{G}(W) = \lim_{V \supset \pi(W)}\mathcal{G}(V)$. So we
need to give the data of maps $\xi_V : \mathcal{G}(V) \to \mathcal{F}(W)$
for every $V \supset \pi(W)$ that are compatible with the restriction maps of
$\mathcal{G}$ in the same sense as we specified above with the $\phi_V$. We
observe that for $V \supset \pi(W)$, we have that
$\pi\inv(V) \supset \pi\inv(\pi(W)) \supset W$, so we have a restriction map
$\mathcal{F}(\pi\inv(V)) \to \mathcal{F}(W)$. We can then define the maps
$\xi_V$ to the composition of $\psi(V) : \mathcal{G}(V) \to \mathcal{F}(\pi\inv(V))$
with the restriction $\mathcal{F}(\pi\inv(V)) \to \mathcal{F}(W)$.
The fact that these maps are compatible follows from $\psi$ being a sheaf morphism. \\

From here, it is a simple but tedious verification to show that the two constructions
provided above are inverse to each other and are natural in $\mathcal{F}$ and
$\mathcal{G}$, giving us the desired adjunction.
 \end{proof}
%
With the preliminaries out of the way, we can discuss what we need for the
Gauss-Manin connection. The first observation to make is that for a continuous
map $\pi : X \to Y$, the pushforward functor $\pi_*$ is left-exact. This follows
immediately from the definition and the fact that injectivity of a sheaf
morphism can be checked on sections. However, $\pi_*$ is not right exact in
general -- as with most things involving sheaves, the issue arises from the
existence of surjective sheaf morphisms that are not surjective on sections.
%
\begin{defn}
Let $\pi : X \to Y$ be a continuous map, and let $\mathcal{F} \in \mathrm{Sh}(X)$.
The \ib{derived pushforward sheaves} of $\mathcal{F}$ (also referred to as
\ib{higher direct image sheaves}) are the right derived  functors
$R^i\pi_*\mathcal{F}$.
\end{defn}
%
Explicitly, the derived pushforward sheaves of $\mathcal{F}$ can be computed
by taking an injective resolution of $\mathcal{F}$
\[\begin{tikzcd}
0 \ar[r] & \mathcal{F} \ar[r, "d^0"] &
\mathcal{I}^1 \ar[r, "d^1"] & \mathcal{I}^2 \ar[r, "d^2"] & \cdots
\end{tikzcd}\]
Applying $\pi_*$ to this resolution gives a complex of sheaves over $Y$
\[\begin{tikzcd}
0 \ar[r] & \pi_*\mathcal{F} \ar[r, "\pi_*d^0"] &
\pi_*\mathcal{I}^1 \ar[r, "\pi_*d^1"] & \pi_*\mathcal{I}^2 \ar[r, "\pi_*d^2"] & \cdots
\end{tikzcd}\]
and we have that the sheaf $R^i\pi_*\mathcal{F}$ is the $i^{th}$ cohomology
sheaf in this sequence, i.e.
\[
R^i\pi_*\mathcal{F} \defeq \frac{\ker \pi_*d^i}{\im \pi_*d^{i-1}}
\]
where we let $\pi_*d^{-1}$ denote the zero map $0 \to \pi_*\mathcal{F}$. \\

To give a more usable characterization of the derived pushforward sheaves,
we first prove a lemma.
%
\begin{lem}
Let $\pi : X \to Y$ be a continuous map, and $\mathcal{I} \in \mathrm{Sh}(X)$
an injective sheaf. Then $\pi_*\mathcal{I}$ is an injective sheaf.
\end{lem}
%
\begin{proof}
We want to show that given a sheaf morphism
$\varphi : \mathcal{F} \to \pi_*\mathcal{I}$ and an injection
$\mathcal{F} \to \mathcal{G}$, there exists a sheaf map
$\widetilde{\varphi} : \mathcal{G} \to \pi_*\mathcal{I}$ such that the follwing
diagram commutes:
\[\begin{tikzcd}
\mathcal{F} \ar[r, "\varphi"] \ar[d, hookrightarrow] & \pi_*\mathcal{I}  \\
\mathcal{G} \ar[ur, "\widetilde{\varphi}"',dashed]
\end{tikzcd}\]
Using the adjunction of $\pi^*$ with $\pi_*$, this is equivalent to finding
a map $\widehat{\varphi} : \pi^*\mathcal{G} \to \mathcal{I}$ such that
the following diagram commutes :
\[\begin{tikzcd}
\pi^*\mathcal{F} \ar[d, hookrightarrow]\ar[r] & \mathcal{I} \\
\pi^*\mathcal{G} \ar[ur, "\widehat{\varphi}"',dashed]
\end{tikzcd}\]
where the map $\pi^*\mathcal{F} \to \mathcal{I}$ is the composition of
$\pi^*\varphi : \pi^*\mathcal{F} \to \pi^*\pi_*\mathcal{I}$ with the
natural map $\pi^*\pi_*\mathcal{I} \to \mathcal{I}$ obtained by taking the image of
$\id_{\pi_*\mathcal{I}}$ under the map
$\hom(\pi_*\mathcal{I},\pi_*\mathcal{I}) \to \hom(\pi^*\pi_*\mathcal{I},\mathcal{I})$
given by the adjunction. In addition, we use the fact that $\pi^*$ is exact to
conclude that the map $\pi^*\mathcal{F} \to \pi^*\mathcal{G}$ is injective. We
then note that since $\mathcal{I}$ is injective, such a $\widehat{\varphi}$ exists.
\end{proof}
%
The lemma then lets us find the nice characterization we desire.
%
\begin{prop}
For a continuous map $\pi : X \to Y$ and a sheaf $\mathcal{F} \in \mathrm{Sh}(X)$,
the derived pushforward sheaves $R^i\pi_*\mathcal{F}$ are the sheafification
of the presheaves defined by
\[
U \mapsto H^i(\pi\inv(U), \mathcal{F}\vert_{\pi\inv(U)})
\]
\end{prop}
%
\begin{proof}
Consider the complex of sheaves over $Y$ obtained by applying $\pi_*$ to an
injective resolution of $\mathcal{F}$.
\[\begin{tikzcd}
0 \ar[r] & \pi_*\mathcal{F} \ar[r, "\pi_*d^0"] &
\pi_*\mathcal{I}^1 \ar[r, "\pi_*d^1"] & \pi_*\mathcal{I}^2 \ar[r, "\pi_*d^2"] & \cdots
\end{tikzcd}\]
By the previous lemma, this is an injective resolution of the sheaf
$\pi_*\mathcal{F}$. For an open set $U \subset Y$, we have that the sections
of $R^i\pi_*\mathcal{F}$ over $U$ are
\[
(R^i\pi_*\mathcal{F})(U) \defeq \left(\frac{\ker \pi_*d^i}{\im\pi_*d^{i-1}}\right)(U)
\]
By definition, the quotient sheaf $\ker\pi_*d^i/\im\pi_*d^{i-1}$ is the sheafification
of the presheaf
\[
U \mapsto \frac{(\ker\pi_*d^i)(U)}{(\im\pi_*d^{i-1})(U)}
\]
We then note that since the complex of sheaves is an injective resolution for
the sheaf $\pi_*\mathcal{F}$, the $R$-module given by
$(\ker\pi_*d^i)(U)/(\im\pi_*d^{i-1})(U)$ is the $i^{th}$ right derived functor
for the functor $\Gamma(U,-)$ that takes a sheaf over $Y$ to its sections over $U$.
We then note that this is exactly the $i^{th}$ sheaf cohomology group
$H^i(U,\pi_*\mathcal{F}\vert_U)$,  which is the same as
$H^i(\pi\inv(U),\mathcal{F}\vert_{\pi\inv(U)})$.
\end{proof}
%
In this way we see that the derived pushforwards behave like the sheaf cohomology
groups for a sheaf relative to the map $\pi : X \to Y$. Indeed, if $Y$ is a point,
then the functor $\pi_*$ is just taking global sections, and the derived
pushforwards $R^i\pi_*\mathcal{F}$ become the constant sheaves associated
to the sheaf cohomology groups $H^i(X,\mathcal{F})$.
%
\section{The Gauss-Manin Connection}
%
To study variations of Hodge structure, we first need a notion of what it means to vary
the complex structure on a smooth manifold $X$.
%
\begin{defn}
A \ib{family of complex manifolds} is a proper holomorphic submersion
$\pi : \mathcal{X} \to B$.
\end{defn}
%
If we fix a basepoint $b \in B$, then the fiber $X_b \defeq \pi\inv(b)$ is a
complex submanifold of $\mathcal{X}$. The idea is that fibers near $X_b$ should
be deformations of $X_b$. This is made precise by a theorem due to Ehresmann.
%
\begin{thm}[\ib{Ehresmann}]
Let $\pi : \mathcal{X} \to B$ be a smooth proper submersion, where $B$ is contractible.
Let $b \in B$ be a basepooint, and $X_b$ the fiber over $b$. Then there
exists a difeomorphism $\mathcal{X} \to X_b \times B$ such that the following
diagram commutes:
\[\begin{tikzcd}
\mathcal{X} \ar[dr, "\pi"']\ar[rr] && X_b \times B \ar[dl] \\
& B
\end{tikzcd}\]
where the map $X_b \times B \to B$ is projection onto the second factor.
\end{thm}
%
In particular, if $\pi$ is a family of complex manifolds, this statement
can be refined further.
%
\begin{thm}
Let $\pi : \mathcal{X} \to B$ be a proper holomorphic submersion where $B$ is
contractible. Fix a basepoint $b \in B$. Then there exists a smooth map
$T_b : \mathcal{X} \to X_b$ such that the map
$(T_b,\pi) : \mathcal{X} \to X_b \times B$ is a trivialization of $\mathcal{X}$ and
the fibers of $T_b$ are complex submanifolds of $\mathcal{X}$.
\end{thm}
%
Note that the map $T_b$ need not be (and often is not) holomorphic. However, this
tells us that in some sense, the complex structure is varying holomorphically. To
make this more precise, we note that by forgetting the complex structure, we can view
$X_b$ as a smooth manifold. From this perspective, a complex structure on $X_b$ can be
interpreted as a diffeomorphism $X \to X_b$ where $X$ is a complex manifold -- by
declaring this map to be an isomorphism of complex manfolds, this uniquely determines
a complex structure on the smooth manifold $X_b$. Using this perspective,
the map $T_b : \mathcal{X} \to X_b$ can be interpreted as a family of maps
$X_p \to X_b$ parameterized by the points $p \in B$ via restriction to the fibers
of $\pi$. If we pick a point $x$ in some fiber $X_p$, it lies in some fiber
of the map $T_b\vert_{X_p}$, which is a complex manifold diffeomorphic to $B$.
Moving in the ``$B$-direction" takes us though different fibers of $\pi$ (in other
words, different complex structures on $X_b$), and since the fibers of $T_b$ are
complex manifolds, this movement is ``holomorphic." Since all the points in the
fiber  $T_b\vert_{X_p}$ map to the same point as $x$ under $T_b$, this can be seen as
viewing the same point $T_b(x)$ of the underlying smooth manifold $X_b$ using the
different complex structures holomorphically parameterized by $B$. \\

We now make use of our detour into derived pushforwards. Let $\pi : \mathcal{X} \to B$
be a family of complex manifolds over a contractible base $B$ with basepoint $b$.
Letting $\underline{R}$ denote the constant sheaf over $\mathcal{X}$ with stalk
$R$ where $R = \Z$, $\R$, or $\C$, we get the derived pushforward
sheaves $R^i\pi_*\underline{R} \in \mathrm{Sh}(B)$. Over a neighborhood $U$ of $b$,
we get that
\[
(R^i\pi_*\underline{R})(U) = H^i(\pi\inv(U),\underline{R}\vert_{\pi\inv(U)})
\]
Since $\underline{R}\vert_{\pi\inv(U)}$ is the same as the constant sheaf
over $\pi\inv(U)$, these are the same as the singular cohomology groups
$H^i(\pi\inv(U),R)$, which are topological invariants.
Then using Ehresmann's theorem, we know that $\mathcal{X} \to B$ can be
trivialized to $X_b \times B \to B$, so $\pi\inv(U) \cong X_b \times U$.
By restricting our attention to contractible neighbrhoods $U$ of $b$
(which we can do since $B$ is locally contractible), we have that
$X_b \times U$ is homotopy equivalent to $X_b$. Putting this all together, we
find that the sheaves $R^i\pi_*\underline{R}$ are local systems with
stalks isomorphic to the singular cohomology groups $H^i(X_b,R)$.
%
\begin{defn}
Let $\pi : \mathcal{X} \to B$ be a family of complex manifolds, and let $b \in B$
be a basepoint. Let $\mathcal{H}^i$ denote the vector bundle obtained from the local
system $R^i\pi_*\underline{R}$. Then the \ib{Gauss-Manin connection} on
$\mathcal{H}^i$ is the flat connection $\nabla$ induced by the local system.
\end{defn}
%
A section of $\mathcal{H}^i$ can be thought of as a family of cohomology classes
$\alpha_t \in H^i(X_b,R)$ parameterized by $B$. From this perspective, the flat
sections are exactly the ones that define the same cohomology class as $\alpha_b$,
where we use the isomorphism $X_t \to X_b$ induced by the map
$T_b : \mathcal{X} \to X_b$ obtained by trivializing $\pi : \mathcal{X} \to B$
to identify the cohomology groups $H^i(X_t,R) \cong H^i(X_b,R)$.
%
\newpage
%
\nocite{*}
%
\printbibliography
%
\end{document}