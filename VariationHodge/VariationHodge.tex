\documentclass[psamsfonts, 12pt]{amsart}
%
%-------Packages---------
%
\usepackage[h margin=1 in, v margin=1 in]{geometry}
\usepackage{amssymb,amsfonts}
\usepackage[all,arc]{xy}
\usepackage{tikz-cd}
\usepackage{enumerate}
\usepackage{mathrsfs}
\usepackage{amsthm}
\usepackage{mathpazo}
\usepackage{float}
\usepackage[backend=biber]{biblatex}
\addbibresource{bibliography.bib}
%\usepackage{charter} %another font
%\usepackage{eulervm} %Vakil font
\usepackage[mathcal]{eucal}
\usepackage{yfonts}
\usepackage{mathtools}
\usepackage{enumitem}
\usepackage{mathrsfs}
\usepackage{fourier-orns}
\usepackage[all]{xy}
\usepackage{hyperref}
\usepackage{url}
\usepackage{mathtools}
\usepackage{graphicx}
\usepackage{pdfsync}
\usepackage{mathdots}
\usepackage{calligra}
\usepackage{import}
\usepackage{xifthen}
\usepackage{pdfpages}
\usepackage{transparent}

\usepackage{tgpagella}
\usepackage[T1]{fontenc}
%
\usepackage{listings}
\usepackage{color}

\definecolor{dkgreen}{rgb}{0,0.6,0}
\definecolor{gray}{rgb}{0.5,0.5,0.5}
\definecolor{mauve}{rgb}{0.58,0,0.82}

\lstset{frame=tb,
  language=Matlab,
  aboveskip=3mm,
  belowskip=3mm,
  showstringspaces=false,
  columns=flexible,
  basicstyle={\small\ttfamily},
  numbers=none,
  numberstyle=\tiny\color{gray},
  keywordstyle=\color{blue},
  commentstyle=\color{dkgreen},
  stringstyle=\color{mauve},
  breaklines=true,
  breakatwhitespace=true,
  tabsize=3
  }
%
%--------Theorem Environments--------
%
\newtheorem{thm}{Theorem}[section]
\newtheorem*{thm*}{Theorem}
\newtheorem{cor}[thm]{Corollary}
\newtheorem{prop}[thm]{Proposition}
\newtheorem{lem}[thm]{Lemma}
\newtheorem*{lem*}{Lemma}
\newtheorem{conj}[thm]{Conjecture}
\newtheorem{quest}[thm]{Question}
%
\theoremstyle{definition}
\newtheorem{defn}[thm]{Definition}
\newtheorem*{defn*}{Definition}
\newtheorem{defns}[thm]{Definitions}
\newtheorem{con}[thm]{Construction}
\newtheorem{exmp}[thm]{Example}
\newtheorem{exmps}[thm]{Examples}
\newtheorem{notn}[thm]{Notation}
\newtheorem{notns}[thm]{Notations}
\newtheorem{addm}[thm]{Addendum}
\newtheorem{exer}[thm]{Exercise}
%
\theoremstyle{remark}
\newtheorem{rem}[thm]{Remark}
\newtheorem*{claim}{Claim}
\newtheorem*{aside*}{Aside}
\newtheorem*{rem*}{Remark}
\newtheorem*{hint*}{Hint}
\newtheorem*{note}{Note}
\newtheorem{rems}[thm]{Remarks}
\newtheorem{warn}[thm]{Warning}
\newtheorem{sch}[thm]{Scholium}
%
%--------Macros--------
\renewcommand{\qedsymbol}{$\blacksquare$}
\renewcommand{\sl}{\mathfrak{sl}}
\newcommand{\Bord}{\mathsf{Bord}}
\renewcommand{\hom}{\mathrm{Hom}}
\renewcommand{\emptyset}{\varnothing}
\renewcommand{\O}{\mathcal{O}}
\newcommand{\R}{\mathbb{R}}
\newcommand{\ib}[1]{\textbf{\textit{#1}}}
\newcommand{\Q}{\mathbb{Q}}
\newcommand{\Z}{\mathbb{Z}}
\newcommand{\N}{\mathbb{N}}
\newcommand{\C}{\mathbb{C}}
\newcommand{\A}{\mathbb{A}}
\newcommand{\F}{\mathbb{F}}
\newcommand{\M}{\mathcal{M}}
\newcommand{\dbar}{\overline{\partial}}
\newcommand{\zbar}{\overline{z}}
\renewcommand{\S}{\mathbb{S}}
\newcommand{\V}{\vec{v}}
\newcommand{\RP}{\mathbb{RP}}
\newcommand{\CP}{\mathbb{CP}}
\newcommand{\B}{\mathcal{B}}
\newcommand{\GL}{\mathrm{GL}}
\newcommand{\SL}{\mathrm{SL}}
\newcommand{\SP}{\mathrm{SP}}
\newcommand{\SO}{\mathrm{SO}}
\newcommand{\SU}{\mathrm{SU}}
\newcommand{\gl}{\mathfrak{gl}}
\newcommand{\g}{\mathfrak{g}}
\newcommand{\Bun}{\mathsf{Bun}}
\newcommand{\inv}{^{-1}}
\newcommand{\bra}[2]{ \left[ #1, #2 \right] }
\newcommand{\set}[1]{\left\lbrace #1 \right\rbrace}
\newcommand{\abs}[1]{\left\lvert#1\right\rvert}
\newcommand{\norm}[1]{\left\lVert#1\right\rVert}
\newcommand{\transv}{\mathrel{\text{\tpitchfork}}}
\newcommand{\defeq}{\vcentcolon=}
\newcommand{\enumbreak}{\ \\ \vspace{-\baselineskip}}
\let\oldexists\exists
\renewcommand\exists{\oldexists~}
\let\oldL\L
\renewcommand\L{\mathfrak{L}}
\makeatletter
\newcommand{\incfig}[2]{%
    \fontsize{48pt}{50pt}\selectfont
    \def\svgwidth{\columnwidth}
    \scalebox{#2}{\input{#1.pdf_tex}}
}
%
\newcommand{\tpitchfork}{%
  \vbox{
    \baselineskip\z@skip
    \lineskip-.52ex
    \lineskiplimit\maxdimen
    \m@th
    \ialign{##\crcr\hidewidth\smash{$-$}\hidewidth\crcr$\pitchfork$\crcr}
  }%
}
\makeatother
\newcommand{\bd}{\partial}
\newcommand{\lang}{\begin{picture}(5,7)
\put(1.1,2.5){\rotatebox{45}{\line(1,0){6.0}}}
\put(1.1,2.5){\rotatebox{315}{\line(1,0){6.0}}}
\end{picture}}
\newcommand{\rang}{\begin{picture}(5,7)
\put(.1,2.5){\rotatebox{135}{\line(1,0){6.0}}}
\put(.1,2.5){\rotatebox{225}{\line(1,0){6.0}}}
\end{picture}}
\DeclareMathOperator{\id}{id}
\DeclareMathOperator{\im}{Im}
\DeclareMathOperator{\codim}{codim}
\DeclareMathOperator{\coker}{coker}
\DeclareMathOperator{\supp}{supp}
\DeclareMathOperator{\inter}{Int}
\DeclareMathOperator{\sign}{sign}
\DeclareMathOperator{\sgn}{sgn}
\DeclareMathOperator{\indx}{ind}
\DeclareMathOperator{\alt}{Alt}
\DeclareMathOperator{\Aut}{Aut}
\DeclareMathOperator{\trace}{trace}
\DeclareMathOperator{\ad}{ad}
\DeclareMathOperator{\End}{End}
\DeclareMathOperator{\Ad}{Ad}
\DeclareMathOperator{\Lie}{Lie}
\DeclareMathOperator{\spn}{span}
\DeclareMathOperator{\dv}{div}
\DeclareMathOperator{\grad}{grad}
\DeclareMathOperator{\Sym}{Sym}
\DeclareMathOperator{\sheafhom}{\mathscr{H}\text{\kern -3pt {\calligra\large om}}\,}
\newcommand*\myhrulefill{%
   \leavevmode\leaders\hrule depth-2pt height 2.4pt\hfill\kern0pt}
\newcommand\niceending[1]{%
  \begin{center}%
    \LARGE \myhrulefill \hspace{0.2cm} #1 \hspace{0.2cm} \myhrulefill%
  \end{center}}
\newcommand*\sectionend{\niceending{\decofourleft\decofourright}}
\newcommand*\subsectionend{\niceending{\decosix}}
\def\upint{\mathchoice%
    {\mkern13mu\overline{\vphantom{\intop}\mkern7mu}\mkern-20mu}%
    {\mkern7mu\overline{\vphantom{\intop}\mkern7mu}\mkern-14mu}%
    {\mkern7mu\overline{\vphantom{\intop}\mkern7mu}\mkern-14mu}%
    {\mkern7mu\overline{\vphantom{\intop}\mkern7mu}\mkern-14mu}%
  \int}
\def\lowint{\mkern3mu\underline{\vphantom{\intop}\mkern7mu}\mkern-10mu\int}
%
%--------Hypersetup--------
%
\hypersetup{
    colorlinks,
    citecolor=black,
    filecolor=black,
    linkcolor=blue,
}
%
%--------Solution--------
%
\newenvironment{solution}
  {\begin{proof}[Solution]}
  {\end{proof}}
%
%--------Graphics--------
%
%\graphicspath{ {images/} }
%
\begin{document}
%
\author{Jeffrey Jiang}
%
\title{Variations of Hodge Structure}
%
\maketitle
%
\tableofcontents
%
\section*{Notation and Conventions}
%
For a complex manifold $X$, we let $\O_X$ denote its sheaf of holomorphic functions.
We let $\Omega^k_X$ denote the sheaf of holomorphic $k$-forms on $X$, and
we let $\mathcal{A}^k_X$ denote the sheaf of smooth complex $k$-forms on $X$.
We let $TX$ denote the smooth tangent bundle of $X$, and let $T_X = T^{1,0}X$
denote the holomorphic tangent bundle.

%
\section{Pure Hodge Structures}
%
The purpose of a Hodge structure is to abstract away the properties of the
cohomology groups of a compact K\"ahler manifold into a linear algebraic gadget.
%
\begin{defn}
An \ib{(pure) integral Hodge structure of weight $k$} is a finitely
generated abelian group $V$ along with a decomposition of
$V_\C \defeq V \otimes_\Z \C$ as
\[
V_\C = \bigoplus_{p+q=k} V^{p,q}
\]
satisfying the condition that $\overline{V^{p,q}} = V^{q,p}$. Such a decomposition
is called the \ib{Hodge decomposition}. A \ib{morphism} of Hodge structures is a map
of abelian groups $\varphi : V \to W$ such that the complexified map
$\varphi_\C : V_\C \to W_\C$ preserves the Hodge decomposition, i.e.
$\varphi_\C(V^{p,q}) \subset W^{p,q}$.
\end{defn}
%
We see that this is an abstraction of the $k^{th}$ integral cohomology group of a
K\"ahler manifold, i.e. for a K\"ahler manifold $X$, we have that
$H^k(X,\Z)$ is an integral Hodge structure of weight $k$, where the decomposition
of $H^k(X,\Z)\otimes_\Z \C \cong H^k(X,\C)$ is given by the Hodge decomposition.
Rational, real and complex Hodge structures are defined analogously. Note
that given an integral Hodge structure $V_\Z$, we get a rational
Hodge structure $V_\Q = V_\Z \otimes_\Z \Q$, and similarly for real and
complex Hodge structures, so the level of generality is opposite to the
inclusions $\Z \subset \Q \subset \R \subset \C$.
%
\begin{defn}
Let $V$ be a Hodge structure of weight $k$. Then the \ib{Hodge filtration} is
the filtration $F^\bullet V$ on $V_\C$ defined by
\[
F^pV_\C \defeq \bigoplus_{r \geq p} V^{r,k-1}
\]
\end{defn}
%
The first thing to note is how the Hodge filtration behaves with the Hodge
decomposition, which results from a few simply computations.
%
\begin{prop}
Let $V$ be a Hodge structure of weight $k$.
\begin{enumerate}
  \item $V_\C = F^pV_\C \oplus \overline{F^{k-p+1}V_\C}$.
  \item $V^{p,q} = F^pV_\C \cap \overline{F^qV_\C}$.
\end{enumerate}
\end{prop}
%
\begin{proof} \enumbreak
\begin{enumerate}
  \item We have
  \begin{align*}
  F^pV_\C \oplus \overline{F^{k-p+1}V_\C}
  &= \left(\bigoplus_{r \geq p} V^{r,k-r}\right) \oplus
  \left(\bigoplus_{s \geq k-p+1}\overline{V^{s,k-s}}\right) \\
  &= \left(\bigoplus_{r \geq p} V^{r,k-r}\right) \oplus
  \left(\bigoplus_{s \geq k-p+1}V^{k-s,s}\right) \\
  &= \bigoplus_{p+q = k} V^{p,q} \\
  &= V_\C
  \end{align*}
  an easy way to see the second to last equality is to keep an eye on the first
  term of the bidegree. For the first summand, we have that as $r$ increases
  from $p$ to $k$, the first term of the bidegree follows in suit. For the
  other term, as $s$ ranges from $k-p+1$ to $k$, the first term of the bidegree
  decreases from $p-1$ to $0$, so we have all the $(p,q)$ accounted for.
  \item We have
  \begin{align*}
  F^pV_\C \cap \overline{F^qV_\C}
  &= \left(\bigoplus_{r \geq p} V^{r,k-r}\right)
  \cap \left(\bigoplus_{s\geq q} \overline{V^{s,k-s}}\right) \\
  &= \left(\bigoplus_{r \geq p} V^{r,k-r}\right)
  \cap \left(\bigoplus_{s\geq q} V^{k-s,s}\right) \\
  &= V^{p,q}
  \end{align*}
  Again, if we just keep an eye on the first term of the bidegree, a
  summand is in the intersection if and only if these terms coincide,
  which we see only happens when $r = p$ and $s = q$.
\end{enumerate}
\end{proof}
%
In the case of K\"ahler manifolds, recall that the cohomology groups
$H^{p,q}(X,\C)$ can be realized as the de Rham cohomology classes that admit
representatives that are smooth closed $(p,q)$-forms. Applying this to the
Hodge filtration on $H^k(X,\C)$, we have that the subspaces $F^pH^k(X,\C)$
are generated by the cohomology classes with representatives with
``at least $p$ many $dz^i$ terms." In fact, more can be said.
%
\begin{prop}
Let $F^p\mathcal{A}^k_X$ denote the subspace of smooth $k$-forms spanned by
the $k$-forms of type $(r,k-r)$ for $r \geq p$. Then
\[
F^pH^k(X,\C) = \frac{\ker(d : F^p \mathcal{A}^k_X \to F^p \mathcal{A}^{k+1}_X)}
{\im(d : F^p \mathcal{A}^{k-1}_X \to F^p \mathcal{A}^k_X)}
\]
\end{prop}
%
\begin{proof}
Let $Z^p$ and $B^p$ denote the kernel and image specified in the statement respectively,
so we want to show that $F^pH^k(X,\C) = Z^p/B^p$. We have a map $Z \to H^k(X,\C)$
taking a differential form to its cohomology class, and the image clearly contains
$F^pH^k(X,\C)$. Conversely, suppose we have that $\alpha$ is closed form whose
cohomology class is contained in $F^pH^k(X,\C)$. Let $\Delta = dd^* + d^*d$ be
the Hodge Laplacian, and can write $\alpha$ uniquely as $\alpha = \beta + \Delta\gamma$,
where $\beta$ is harmonic. Then since $\Delta$ preserves the bidegree of forms, we
get that $\beta$ and $\gamma$ are both in $F^p\mathcal{A}^k_X$. Then since both
$\beta$ and $\gamma$ are $d$-closed, we have
\[
0 = d\Delta\gamma = d(dd^*\gamma + d^*d\gamma) = dd^*d\gamma = 0
\]
By orthogonality of the image of $d^*$ and the kernel of $d$ (since they are
adjoint operators), we have that $d^*d\gamma = 0$. Therefore, we can write
$\alpha = \beta + dd^*\gamma$, so $\alpha$ defines the same cohomology class as
$\beta$. Therefore, using the identification of $H^{p,q}(X,\C)$ with harmonic
$(p,q)$ forms, we get that $[\alpha] = [\beta] \in F^pH^k(X,\C)$. \\

We now want to show that the kernel of the map specified above is $B^p$. We
note that $B^p$ is clearly contained in the kernel, so we only need to verify the
the opposite inclusion. We do this by decreasing induction on $p$, so the base case
is when $p = k$. When $p = k$, we have that $F^pH^k(X,\C) = H^{k,0}(X,\C)$. Then
suppose we have a closed form $\alpha$ with $[\alpha] \in H^{k,0}(X,\C)$, so
$\alpha$ can be assumed to be type $(k,0)$. Therefore, we have that $\dbar\alpha = 0$.
Therefore, we have that $d\alpha = \dbar\alpha = 0$, so if $\alpha$ is
$d$-exact, we may apply $\partial\dbar$-lemma to write $\alpha = \partial\dbar\xi$.
But since forms cannot have a negative term in the bidegree, we conclude that
$\alpha = 0$, which is what we wanted since $F^k \mathcal{A}^{k-1}_X = 0$. \\

We now assume we have proven the fact for $p+1$ to prove it for $p$. Suppose
we have $\alpha \in F^p\mathcal{A}^k_X$ such that $d\alpha = 0$. We want to
show that $\alpha = d\xi$ for some $\xi \in F^p \mathcal{A}^{k-1}_X$. Then
we can write $\alpha = \Delta\beta$, since the harmonic representative of
$[\alpha]$ is $0$. Then let $q$ such that $p+q = k$. We then get that
$\alpha^{p,q} = \Delta\beta^{p,q} = 2\Delta_{\dbar}\beta^{p,q}$, where
$\Delta_{\dbar}$ is the $\dbar$-Laplacian $\dbar\dbar^* + \dbar^*\dbar$ and we
are using the K\"ahler identity $2\Delta_{\dbar} = \Delta$. Therefore, we
have that $\alpha^{p,q} = 2\Delta_{\dbar}\beta^{p,q}$ is $\dbar$-closed.
Again using orthogonality of the kernel of $\dbar$ with the image of
$\dbar^*$, we find that $\dbar^*\dbar\beta^{p,q} = 0$, so
$\alpha^{p,q} = 2\dbar\dbar^*\beta^{p,q}$. Then we have that
$\alpha - 2d\dbar^*\beta^{p,q}$ is exact. Furthermore, we have that it is
contained in $F^{p+1}\mathcal{A}^k_X$, since we have zeroed out the only
component of $\alpha$ contained in $F^p\mathcal{A}^k_X$ by subtracting a form
of type $(p+1, q-1)$. Applying the induction hypothesis to this form, we
find that $\alpha - 2d\dbar^*\beta^{p,q} = d\gamma$ for some
$\gamma \in F^{p+1}\mathcal{A}^{k-1}$, so we get
$\alpha = 2d\dbar^*\beta^{p,q} + d\gamma$, which shows the desired result.
\end{proof}
%
Given two K\"ahler manifolds $X$ and $Y$ and a holomorphic map $\varphi : X \to Y$, we
get a pullback map on cohomology $\varphi^* : H^k(Y,\Z) \to H^k(X,\Z)$ which
when complexified, corresponds to the pullback of complex differential forms.
Then since the pullback of differential forms under a holomorphic map
preserves the type, we see that the pullback map on the $k^{th}$ cohomology
groups is a morphism of Hodge structures of weight $k$. \\
%
\section{Polarized Hodge Structures}
%
The K\"ahler class $[\omega] \in H^2(X,\R)$ defines a Lefschetz operator
$L : H^k(X,\R) \to H^{k+2}(X,\R)$ where $L[\alpha] = [\omega \wedge \alpha]$.
This gives rise to the \ib{Lefschetz decomposition} of $H^k(X,\R)$ as
\[
H^k(X,\R) = \bigoplus_{r\geq 0}L^rH^{k-2r}(X,\R)_{\mathrm{Prim}}
\]
where the primitive cohomology $H^i(X,\R)_{\mathrm{Prim}} \subset H^i(X,\R)$
denotes the subspace primitive cohomology classes, which are the classes annhilated
by $L^{n-i+1}$. This tells us that we can understand the cohomology of $X$ in
terms of $L$ and the primitive pieces. Complexifying gives us an analogous
decomposition on the complex cohomology, and the Lefschetz decomposition on
$H^k(X,\C)$ is compatible with the Hodge decomposition in the sense that
\[
H^k(X,\C)_{\mathrm{Prim}} =
\bigoplus_{p,q}(H^{p,q}(X,\C) \cap H^k(X,\C)_{\mathrm{Prim}})
\]
So we can further break down the primitive pieces of the cohomology in terms
of the Hodge decomposition. The K\"ahler class also gives a bilinear
pairing $Q_k$ on the cohomology group $H^k(X,\R)$ by
\[
Q_k([\alpha],[\beta]) = \int_X \omega^{n-k}\wedge\alpha\wedge\beta
\]
The form $Q_k$ is symmetric when $k$ is even and alternating if $k$ is odd.
This then induces a Hermitian $H_k$ form on $H^k(X,\C)$ given by
\[
H_k([\alpha],[\beta]) = i^kQ([\alpha],[\overline{\beta}])
\]
In addition, the complexified Lefschetz decomposition is orthogonal with respect
to $H_k$, and the restriction of $H_k$ to the primitive component
$L^rH^{k-2r}(X,\C)_{\mathrm{Prim}}$ is the same as the
restriction of the form $(-1)^rH_{k-2r}$, where we are using the fact that
$L^r$ is injective on $H^{k-2r}(X,\C)$ to identify
$L^rH^{k-2r}(X,\C)_{\mathrm{Prim}}$ and $H^{k-2r}(X,\C)_{\mathrm{Prim}}$.
Furthermore, the Hodge decomposition is orthogonal with respect to
the $H_k$, and the restriction of
$(-1)^{k(k-1)/2}i^{p-q-k}H_k$ to $H^{p,q}(X,\C)_{\mathrm{Prim}}$
is positive definite. Since $p-q$ always has the same parity as $k$, this
amounts to saying that the restriction of $H_k$ to $H^{p,q}(X,\C)_{\mathrm{Prim}}$
alternates between being positive and negative definite as we increment $p$.\\


Suppose the manifold $X$ is not only K\"ahler, but is a complex submanifold
of some $\CP^N$. The restriction of the Fubini-Study metric on $\CP^N$ gives a
K\"ahler metric $\omega$ on $X$ whose cohomology class $[\omega]$ is integral, i.e.
$[\omega] \in H^2(X,\Z)$, which corresponds to the curvature form of the Chern
connection on a positive Hermitian line bundle on $X$. Kodaira's embedding theorem
tells us that the opposite is also true -- given an integral K\"ahler class $[\omega]$,
we can find a positive Hermitian holomorphic line bundle $L \to X$ where the curvature
of the Chern connection is $[\omega]$ such that $L$ defines an embedding
$X \hookrightarrow \CP^N$ where $\O_{\CP^N}(1)\vert_X \cong L$ and
$[\omega]$ is represented by the restriction of the Fubini-Study metric.
Then the operator $L$ corresponds to intersecting with the hyperplane class, giving
a geometric interpretation of the $H_k$ as a notion of intersection forms on
cohomology. Finally, $[\omega]$ being an integral class implies that the above
discussion restricts to the integral cohomology, giving us bilinear forms
$Q_k$ on $H^k(X,\Z)$ and primitive integral classes $H^k(X,\Z)_{\mathrm{Prim}}$
Putting everything together, this motivates the following:
%
\begin{defn}
A \ib{polarized integral Hodge structure} of weight $k$ is a pure integral
Hodge structure $V$ of weight $k$ along with a bilinear form $Q$ on $V$ such that
$Q$ is alternating if $k$ is odd, and symmetric if $k$ is even, such that
the Hermitian form $H$
\[
H(\alpha,\beta) = i^kQ(\alpha,\bar{\beta})
\]
on $V_\C$ satisfies:
\begin{enumerate}
  \item The Hodge decomposition is orthogonal with respect to $H$.
  \item the form $(-1)^{k(k-1)/2}i^{p-q-k}H$ is positive definite on $V^{p,q}$.
\end{enumerate}
\end{defn}
%
Taking the above discussion into account, this abstracts the primitive
integral cohomology of a smooth projective algebraic variety. The abelian
group $V$ represents $H^k(X,\Z)_{\mathrm{Prim}}$, the $V^{p,q}$ in the Hodge
decomposition represent $H^{p,q}(X,\C)_{\mathrm{Prim}}$, and $Q$
represents the intersection pairing $Q_k$ constructed above.
%
\section{Flat Bundles and Local Systems}
%
\begin{defn}
Let $X$ be a complex manifold. A \ib{local system} on $X$ is a sheaf $H$ of
abelian groups on $X$ that is isomorphic to the constant sheaf $\underline{A}$
for some abelian group $A$.
\end{defn}
%
In most cases, we will discuss local systems where the abelian group $A$ is
a vector space over $\R$ or $\C$. Given a local system $H$ of $\C$-vector
spaces, we can obtain a sheaf $\mathcal{H}$ of free $\O_X$-modules over $X$ by
taking the tensor product
\[
\mathcal{H} \defeq H \otimes_{\underline{\C}} \O_X
\]
Where we regard $\O_X$ as a sheaf of $\underline{\C}$-algebras via the inclusion
$\underline{\C} \hookrightarrow \O_X$. The same discussion holds if if $H$
is a local system of abelian groups or $\R$ vector spaces by tensoring over
$\underline{\Z}$ and $\underline{\R}$ respectively. We note that the sheaf
$\mathcal{H}$ is isomorphic to $\O_X^n$, which is the sheaf of holomorphic sections of
the trivial rank $n$ vector bundle $X \times \C^n$. However, we also have the
data of our original sheaf $H$, which we can identify as a subsheaf of
$\mathcal{H}$ via the mapping $h \mapsto h \otimes 1$. Our goal will be to identify
$H$ as a subsheaf of distinguished sections of the trivial bundle $X \times \C^n$.
%
\begin{defn}
Let $\mathcal{E}$ be a sheaf of $\O_X$-modules. A \ib{connection} on $\mathcal{E}$
is a sheaf morphism $\nabla : \mathcal{E} \to \mathcal{E} \otimes_{\O_X} \Omega^1_X$
such that
\begin{enumerate}
  \item $\nabla$ is $\underline{\C}$-linear.
  \item For an open set $U$, a function $f \in \O_X(U)$, and a section
  $s \in \mathcal{E}(U)$, we have that $\nabla$ satisfies the Leibniz rule
  \[
  \nabla(fs) = s \otimes df + f\nabla(s)
  \]
\end{enumerate}
\end{defn}
%
In the case that $\mathcal{E}$ is the sheaf of holomorphic sections
of a holomorphic vector bundle $E \to X$, the connection $\nabla$ is corresponds to
the notion of a \emph{holomorphic} connection on $E$. By replacing $\O_X$ with the sheaf
of smooth complex valued functions and $\Omega^1_X$ with $\mathcal{A}^1_X$, the same
definition yields the more traditional definition of a connection on a smooth complex
vector bundle.
%
\begin{defn}
Let $\mathcal{E}$ be a sheaf of $\O_X$-modules equipped with a connection $\nabla$.
A section $s \in \mathcal{E}(U)$ is \ib{flat} if $\nabla(s) = 0$.
\end{defn}
%
Given a sheaf $\mathcal{E}$ with connection $\nabla$, the flat sections
of $\mathcal{E}$ form a subsheaf. In the case that $\mathcal{E}$ comes from a
vector bundle $\pi : E \to X$, the flat sections correspond to the equivalent
interpretation of the connection $\nabla$ as a \ib{horizontal distribution} on
the total space $E$, i.e. a vector subbundle $D \subset TE$ such that
$D \oplus \ker d\pi = TE$. From this perspective, the flat sections of $\nabla$
correspond to sections of $E$ that lie inside $D$. Another thing to note is
that by a dimension count, we have that the rank of $D$ as a vector bundle
over $E$ is the same as a the dimension of the base space $X$, which can be
seen by noting that for any point $e \in E$, the fiber $D_e$ projects isomorphically
onto the tangent space $T_{\pi(e)}X$ under $d\pi_e$ since it is complentary to the
kernel.
%
\begin{prop}
Let $H$ be a local system of $\C$-vector spaces, and
$\mathcal{H} = H \otimes_{\underline{\C}} \O_X$ the sheaf of holomorphic sections of
the associated vector bundle. Then there exists a connection $\nabla$ on $\mathcal{H}$
such that $H$ is the sheaf of flat sections of $\nabla$.
\end{prop}
%
\begin{proof}
Over an open set $U \subset X$, we have that $\mathcal{H}(U)$ is isomorphic to
$\O_X^n(U)$. Furthermore, we may choose the isomorphism $\O^n_X(U) \to \mathcal{H}(U)$
such that the standard basis vector $e_i$ maps to an element $h_i$ of $H$, thought
of as a subsheaf of $\mathcal{H}$. Then given a local section $s$, it can
be written in this trivialization as $s = \sum_i f_ih_i$ for some smooth
functions $f_i \in \O_X(U)$. Define $\nabla$ by the formula
\[
\nabla(s) = \sum_i h_i \otimes df_i
\]
Furthermore, we claim that this independent of our choice of trivialization. Given
another trivialization $\set{h'_i}$, we have that $h_i$ and $h_i'$ are related by
some matrix $A \in \GL_n\C$, (i.e. $h_i = A_i^jh'_j$, using Einstein summation
convention) since $H$ is locally constant. Therefore, in the
local trivialization $\set{h'_i}$, the section $s$ can be represented as
$s = \sum_if_iA^j_ih'_j$. Therefore, if we use our definition of $\nabla$
in this trivialization, we obtain
\[
\nabla(s) = \sum_{i,j} h'_j \otimes d(f_iA^j_i) = \sum_{i,j} h'_j \otimes A^j_i df_i
= \sum_{i,j} A^j_ih'_j \otimes df_i = \sum_i h_i \otimes df_i
\]
where we use the fact that the matrix $A$ is locally constant on $U$ and $\C$-linearity
of $d$. Therefore, our local definition of $\nabla$ is well-defined, and glues to a
connection on all of $\mathcal{H}$. Furthermore, we see that a section $s$ is flat if
and only if it is locally of the form $s = \sum_i \lambda_ih_i$ for scalars
$\lambda_i \in \underline{\C}(U)$. Then by identifying $H$ with the subsheaf
$H \otimes_{\underline{\C}} 1 \subset \mathcal{H}$, we immediately see that
the flat sections are exactly the sections of $H$.
\end{proof}
%
The connection $\nabla$ defined in the proof has several special properties that
are not shared by all connections. One thing to observe is that the
$(0,1)$ part of $\nabla$ is exactly $\dbar$ in a trivialization of
$\mathcal{H}$ mapping the standard basis of $\O^n_X$ to sections of
$H \subset \mathcal{H}$. This completely characterizes the holomorphic structure
on the associated smooth complex vector bundle. To identify the second property,
we need to define a notion of curvature. \\

Let $\mathcal{E}$ be a sheaf of $\O_X$-modules equipped with a connection
$\nabla : \mathcal{E} \to \mathcal{E} \otimes_{\O_X} \Omega^1_X$. This induces a map
$\mathcal{E} \otimes_{\O_X} \Omega^k_X \to \mathcal{E} \otimes_{\O_X} \Omega^{k+1}_X$,
which we also call $\nabla$, by the formula
\[
\nabla(s \otimes \omega) = \nabla(s) \wedge \omega + s \otimes d\omega
\]
Where $\nabla(s) \wedge \omega$ is obtained by writing $\nabla(s)$ locally as a sum
of the form $\sum_i s_i \otimes \omega_i$ for $s_i \in \mathcal{E}(U)$ and
$\omega_i \in \Omega^1_X(U)$ and wedging the $\omega_i$ with $\omega$.
%
\begin{prop}
The extended connection
$\nabla:\mathcal{E}\otimes_{\O_X}\Omega^k_X \to \mathcal{E}\otimes_{\O_X}\Omega^{k+1}_X$
satisfies a graded Leibniz rule, i.e. for a function $f \in \O_X(U)$ and a local
section $\theta \in \mathcal{E}(U) \otimes \Omega^k_X(U)$, we have
\[
\nabla(f\theta) = (-1)^k \theta \wedge df + f\nabla\theta
\]
\end{prop}
%
\begin{proof}
By linearity, it suffices to check this when $\theta$ is of the form $s \otimes \omega$
for a local section $s \in \mathcal{E}(U)$ and $\omega \in \Omega^1_X(U)$. We then
compute
\begin{align*}
\nabla(f\theta) &= \nabla(f(s\otimes\omega)) \\
&= \nabla(fs \otimes \omega) \\
&= \nabla(fs) \wedge \omega + fs \otimes d\omega \\
&= (s \otimes df + f\nabla(s))\wedge \omega + fs \otimes d\omega \\
&= s \otimes(df\wedge\omega) + f\nabla(s)\wedge\omega + fs\otimes d\omega \\
&= s \otimes (df \wedge \omega) + f\nabla(s \otimes \omega) \\
&= (-1)^k s\otimes(\omega \wedge df) + f\nabla(s\otimes\omega) \\
&= (-1)^k \theta\wedge df + f\nabla(\theta)
\end{align*}
\end{proof}
%
\begin{defn}
The \ib{curvature} of a connection $\nabla$ is the map
$\Omega : \mathcal{E} \to \mathcal{E} \otimes \Omega^2_X$ given by
$\Omega \defeq \nabla \circ \nabla$.
\end{defn}
%
\begin{prop}
The curvature $\Omega$ of any connection $\nabla$ is $\O_X$-linear.
\end{prop}
%
\begin{proof}
Fix a local section $s$ and a holomorphic function $f$. Then we compute
\begin{align*}
\Omega(fs) &= \nabla(s \otimes df + f\nabla(s)) \\
&= \nabla(s \otimes df) + \nabla(f\nabla(s)) \\
&= (\nabla(s) \wedge df + s \otimes d^2f) - \nabla(s)\wedge df + f\nabla(\nabla(s)) \\
&= f\Omega(s)
\end{align*}
\end{proof}
%
If a connection has curvature equal to zero, it is said to be \ib{flat}. \\

From the perspective of horizontal distributions, the curvature tensor $\Omega$
has an interpretation as the \ib{Frobenius tensor} of the horizontal distribution $D$,
which measures the failure of $D$ to be \ib{involutive}, i.e. the failure of vector
fields on $E$ lying in $D$ to be closed under the Lie bracket. By Frobenius' theorem,
this is equivalent to $D$ being \ib{integrable}, so $\Omega$ measures the obstruction
to the existence of integral submanifolds of $D$ inside of the total space $E$.
From this perspective, flatness of a connection is exactly integrability of
the horizontal distribution.
%
\begin{prop}
The connection $\nabla$ on the vector bundle $\mathcal{H} \otimes_{\underline{\C}}\O_X$
is flat.
\end{prop}
%
\begin{proof}
Fix a local trivialization of $\mathcal{H}$ such that a local basis of sections
is given by a set $\set{h_i}$ with $h_i \in H(U)$. The local
basis of sections determines an isomorphism $\O_X^n(U) \to \mathcal{H}(U)$,
and identifies $\nabla$ with the operator that applies the de Rham differential $d$
to each component of $\O_X^n(U)$. Then since $d^2 = 0 $, we have that
$\Omega = \nabla\circ\nabla = 0$.
\end{proof}
%
This suggests the following.
%
\begin{thm}
There is an bijective correspondence
\[
\set{\text{Local systems of } \C \text{-vector spaces}} \longleftrightarrow
\set{\text{Holomorphic vector bundles with flat connection}}
\]
\end{thm}
%
\begin{proof}
In one direction, we can associate to a local system $H$ the vector bundle
$\mathcal{H}$ with the connection we defined above. In the
other direction, given a holomorphic vector bundle $\pi : E \to X$ with flat connection
$\nabla$, we claim that the sheaf $H$ of flat sections of $E$ forms a local system.
Once we do so, it's clear that the mappings specified above are inverses to
each other, giving us the desired correspondence. \\

Since the bundle $\pi : E \to X$ has a flat connection $\nabla$, we have that the
horizontal distribution $D$ is integrable, so there exists a integral
submanifolds of $E$ whose tangent spaces correspond to the fibers of $D \to E$.
Fix a basepoint $x \in X$, and consider the zero element $0 \in E_x$ the the fiber
lying over $x$. Then let $Y \subset E$ denote an integral submanifold of $D$ containing
$0$. Since $D$ is complementary to $\ker d\pi$ and $T_0Y = D_0$, we have that
$\pi\vert_Y : Y \to X$ is a local diffeomorphism at $0$, so there exists connected
open neighborhoods $U \subset Y$ and $V \subset X$ such that $\pi\vert_U$ is a
diffeomorphism $U \to V$. Furthermore, we can find a neighborhood $U'$ of $Y$
diffeomorphic to $Y \times (-\varepsilon,\varepsilon)$ such that the slice
$Y \times \set{t}$ projects diffeomorphically onto $V$. Then $U'$ is the union of
the images of flat sections $V \to E$, since it is fibered by integral submanifolds
of the horizontal distribution. Then since $U'$ is an open neighborhood of the
zero section $V \to E$, taking fiberwise linear combinations of elements
of $U'$ spans the fibers of $E$ lying over $V$. Since linear combinations of flat
sections are flat, we have that the fibers of $E$ lying over $V$ are generated by flat
sections. Therefore, we can identify the flat sections of $E$ over $U'$ with elements
of the fiber $E_0$, since a section $\sigma$ whose value at $x$ is a vector
$v \in E_0$ is uniquely extended to a flat section in a neighborhood of $x$
by existence and uniqueness of solutions to the differential equation
$\nabla\sigma = 0$. Putting everything together, we find that the flat sections over
$U'$ can be identified with constant functions $U \to E_x$, which tells us that the
sheaf of flat sections can be identified with the constant sheaf $\underline{E_x}$,
so they form a local system.
\end{proof}
%
\section{Derived Pushforwards of Sheaves}
%
To discuss the Gauss-Manin connection, we need to make a short digression into
homological algebra and derived pushforwards of sheaves. For a topological space
$X$, let $\mathrm{Sh}(X)$ denote the abelian category of sheaves of $R$-modules over
$X$, where $R = \Z$, $\R$, or $\C$.
%
\begin{defn}
Let $\pi : X \to Y$ be a continuous map. The \ib{pushforward} (also known as the
\ib{direct image}) along $\pi$ is a functor $\pi_* : \mathrm{Sh}(X) \to \mathrm{Sh}(Y)$,
where given a sheaf $\mathcal{F}$ over $X$, the sheaf $\pi_*\mathcal{F}$ is defined by
the mapping
\[
\pi_*\mathcal{F}(U) \defeq \mathcal{F}(\pi\inv(U))
\]
Given a sheaf morphism $\varphi : \mathcal{F} \to \mathcal{G}$, the sheaf morphism
$\pi_*\varphi : \pi_*\mathcal{F} \to \pi_*\mathcal{G}$ is given by
\[
\pi_*\varphi(U) \defeq \varphi(\pi\inv(U))
\]
\end{defn}
%
\begin{defn}
Given a continuous map $\pi: X \to Y$ and the \ib{pullback} along $\pi$ is a functor
$\pi^* : \mathrm{Sh}(Y) \to \mathrm{Sh}(X)$ where for a sheaf
$\mathcal{F} \in \mathrm{Sh}(Y)$, the sheaf $\pi^*\mathcal{F}$ is the
sheafification of the presheaf
\[
U \mapsto \lim_{V \supset \pi(U)} \mathcal{F}(V)
\]
Given a morphism $\varphi : \mathcal{F} \to \mathcal{G}$, the sheaf morphism
$\pi^*\varphi : \pi^*\mathcal{F} \to \pi^*\mathcal{G}$ is given by the sheaf
morphism induced by the map of presheaves
\[
\lim_{V \supset \pi(U)} \varphi(V) :
\lim_{V \supset \pi(U)} \mathcal{F}(V) \to \lim_{V \supset \pi(U)} \mathcal{G}(V)
\]
\end{defn}
%
It's clear from the definition of the pullback that for a point
$x \in X$, the stalk $(\pi^*\mathcal{F})_x$ is equal to the stalk
$\mathcal{F}_{\pi(x)}$. Furthermore, given a morphism
$\varphi : \mathcal{F} \to \mathcal{G}$ of sheaves over $Y$, we have that the
morphism $\pi^*\varphi$ is given on stalks by the same maps induced by $\varphi$.
Then since injectivity and surjectivity for sheaf maps can be checked on the level
of stalks, this immediately yields the result:
%
\begin{prop}
The functor $\pi^*$ is exact, i.e. for an exact sequence of sheaves over $Y$,
\[\begin{tikzcd}
0 \ar[r] & \mathcal{E} \ar[r] & \mathcal{F} \ar[r] & \mathcal{G} \ar[r] & 0
\end{tikzcd}\]
the sequence  of sheaves over $X$ given by
\[\begin{tikzcd}
0\ar[r]& \pi^*\mathcal{E} \ar[r] & \pi^*\mathcal{F} \ar[r] & \pi^*\mathcal{G} \ar[r]&0
\end{tikzcd}\]
is exact.
\end{prop}
%
The key fact we will need regarding $\pi^*$ and $\pi_*$ is the fact that they
are adjoint functors.
%
\begin{prop}
Let $\pi : X \to Y$ be a continuous map, and let $\mathcal{F} \in \mathrm{Sh}(X)$
and $\mathcal{G} \in \mathrm{Sh}(Y)$. Then there is a bijection
\[
\hom(\pi^*\mathcal{G},\mathcal{F}) \longleftrightarrow
\hom(\mathcal{G},\pi_*\mathcal{F})
\]
that is natural in $\mathcal{F}$ and $\mathcal{G}$, i.e. functorial with respect to
maps into and out of $\mathcal{F}$ and $\mathcal{G}$.
\end{prop}
%
\begin{proof}
We provide maps in both directions. For one direction, let
$\varphi \in \hom(\pi^*\mathcal{G},\mathcal{F})$. For an open set $U \subset X$,
this gives us a map $\varphi(U) : \pi^*\mathcal{G}(U) \to \mathcal{F}(U)$.
By definition, we have that
$\pi^*\mathcal{G}(U) = \mathrm{lim}_{V\supset\pi(U)}\mathcal{G}(V)$, so the
map $\varphi(U)$ is equivalent to the data of maps
$\phi_V : \mathcal{G}(V) \to \mathcal{F}(U)$ for all open sets $V \supset \pi(U)$
such that if $W \supset V$, then $\phi_W$ is equal to the composition of $\phi_V$
with the restriction map $\mathcal{G}(W) \to \mathcal{G}(V)$. From this data,
we want to produce a sheaf morphism $\mathcal{G} \to \pi_*\mathcal{F}$. To do this,
for each open set $U \subset Y$,  we want to produce a map
$\mathcal{G}(U) \to \mathcal{F}(\pi\inv(U))$. We note that $\pi(\pi\inv(U)) = U$,
so if we consider the map
$\varphi(\pi\inv(U)) : \lim_{V \supset U}\mathcal{G}(V) \to \mathcal{F}(\pi\inv(U))$,
we may take the map $\phi_U : \mathcal{G}(U) \to \mathcal{F}(\pi\inv(U))$.
Doing this for all open sets gives us the desired sheaf morphism
$\mathcal{G} \to \pi_*\mathcal{F}$. \\

In the other direction, suppose we are given a sheaf map
$\psi \in \hom(\mathcal{G},\pi_*\mathcal{F})$. Then for an open set $U \subset Y$
we have a map $\psi(U) : \mathcal{G}(U) \to \mathcal{F}(\pi\inv(U))$. From this we
want to produce a sheaf map $\pi^*\mathcal{G} \to \mathcal{F}$. Let $W \subset X$
be an open set. We then must give a map $\pi^*\mathcal{G}(W) \to \mathcal{F}(W)$.
By definition, we have that
$\pi^*\mathcal{G}(W) = \lim_{V \supset \pi(W)}\mathcal{G}(V)$. So we
need to give the data of maps $\xi_V : \mathcal{G}(V) \to \mathcal{F}(W)$
for every $V \supset \pi(W)$ that are compatible with the restriction maps of
$\mathcal{G}$ in the same sense as we specified above with the $\phi_V$. We
observe that for $V \supset \pi(W)$, we have that
$\pi\inv(V) \supset \pi\inv(\pi(W)) \supset W$, so we have a restriction map
$\mathcal{F}(\pi\inv(V)) \to \mathcal{F}(W)$. We can then define the maps
$\xi_V$ to the composition of $\psi(V) : \mathcal{G}(V) \to \mathcal{F}(\pi\inv(V))$
with the restriction $\mathcal{F}(\pi\inv(V)) \to \mathcal{F}(W)$.
The fact that these maps are compatible follows from $\psi$ being a sheaf morphism. \\

From here, it is a simple but tedious verification to show that the two constructions
provided above are inverse to each other and are natural in $\mathcal{F}$ and
$\mathcal{G}$, giving us the desired adjunction.
 \end{proof}
%
With the preliminaries out of the way, we can discuss what we need for the
Gauss-Manin connection. The first observation to make is that for a continuous
map $\pi : X \to Y$, the pushforward functor $\pi_*$ is left-exact. This follows
immediately from the definition and the fact that injectivity of a sheaf
morphism can be checked on sections. However, $\pi_*$ is not right exact in
general -- as with most things involving sheaves, the issue arises from the
existence of surjective sheaf morphisms that are not surjective on sections.
%
\begin{defn}
Let $\pi : X \to Y$ be a continuous map, and let $\mathcal{F} \in \mathrm{Sh}(X)$.
The \ib{derived pushforward sheaves} of $\mathcal{F}$ (also referred to as
\ib{higher direct image sheaves}) are the right derived  functors
$R^i\pi_*\mathcal{F}$.
\end{defn}
%
Explicitly, the derived pushforward sheaves of $\mathcal{F}$ can be computed
by taking an injective resolution of $\mathcal{F}$
\[\begin{tikzcd}
0 \ar[r] & \mathcal{F} \ar[r, "d^0"] &
\mathcal{I}^1 \ar[r, "d^1"] & \mathcal{I}^2 \ar[r, "d^2"] & \cdots
\end{tikzcd}\]
Applying $\pi_*$ to this resolution gives a complex of sheaves over $Y$
\[\begin{tikzcd}
0 \ar[r] & \pi_*\mathcal{F} \ar[r, "\pi_*d^0"] &
\pi_*\mathcal{I}^1 \ar[r, "\pi_*d^1"] & \pi_*\mathcal{I}^2 \ar[r, "\pi_*d^2"] & \cdots
\end{tikzcd}\]
and we have that the sheaf $R^i\pi_*\mathcal{F}$ is the $i^{th}$ cohomology
sheaf in this sequence, i.e.
\[
R^i\pi_*\mathcal{F} \defeq \frac{\ker \pi_*d^i}{\im \pi_*d^{i-1}}
\]
where we let $\pi_*d^{-1}$ denote the zero map $0 \to \pi_*\mathcal{F}$. \\

To give a more usable characterization of the derived pushforward sheaves,
we first prove a lemma.
%
\begin{lem}
Let $\pi : X \to Y$ be a continuous map, and $\mathcal{I} \in \mathrm{Sh}(X)$
an injective sheaf. Then $\pi_*\mathcal{I}$ is an injective sheaf.
\end{lem}
%
\begin{proof}
We want to show that given a sheaf morphism
$\varphi : \mathcal{F} \to \pi_*\mathcal{I}$ and an injection
$\mathcal{F} \to \mathcal{G}$, there exists a sheaf map
$\widetilde{\varphi} : \mathcal{G} \to \pi_*\mathcal{I}$ such that the follwing
diagram commutes:
\[\begin{tikzcd}
\mathcal{F} \ar[r, "\varphi"] \ar[d, hookrightarrow] & \pi_*\mathcal{I}  \\
\mathcal{G} \ar[ur, "\widetilde{\varphi}"',dashed]
\end{tikzcd}\]
Using the adjunction of $\pi^*$ with $\pi_*$, this is equivalent to finding
a map $\widehat{\varphi} : \pi^*\mathcal{G} \to \mathcal{I}$ such that
the following diagram commutes :
\[\begin{tikzcd}
\pi^*\mathcal{F} \ar[d, hookrightarrow]\ar[r] & \mathcal{I} \\
\pi^*\mathcal{G} \ar[ur, "\widehat{\varphi}"',dashed]
\end{tikzcd}\]
where the map $\pi^*\mathcal{F} \to \mathcal{I}$ is the composition of
$\pi^*\varphi : \pi^*\mathcal{F} \to \pi^*\pi_*\mathcal{I}$ with the
natural map $\pi^*\pi_*\mathcal{I} \to \mathcal{I}$ obtained by taking the image of
$\id_{\pi_*\mathcal{I}}$ under the map
$\hom(\pi_*\mathcal{I},\pi_*\mathcal{I}) \to \hom(\pi^*\pi_*\mathcal{I},\mathcal{I})$
given by the adjunction. In addition, we use the fact that $\pi^*$ is exact to
conclude that the map $\pi^*\mathcal{F} \to \pi^*\mathcal{G}$ is injective. We
then note that since $\mathcal{I}$ is injective, such a $\widehat{\varphi}$ exists.
\end{proof}
%
The lemma then lets us find the nice characterization we desire.
%
\begin{prop}
For a continuous map $\pi : X \to Y$ and a sheaf $\mathcal{F} \in \mathrm{Sh}(X)$,
the derived pushforward sheaves $R^i\pi_*\mathcal{F}$ are the sheafifications
of the presheaves defined by
\[
U \mapsto H^i(\pi\inv(U), \mathcal{F}\vert_{\pi\inv(U)})
\]
\end{prop}
%
\begin{proof}
Consider the complex of sheaves over $Y$ obtained by applying $\pi_*$ to an
injective resolution of $\mathcal{F}$.
\[\begin{tikzcd}
0 \ar[r] & \pi_*\mathcal{F} \ar[r, "\pi_*d^0"] &
\pi_*\mathcal{I}^1 \ar[r, "\pi_*d^1"] & \pi_*\mathcal{I}^2 \ar[r, "\pi_*d^2"] & \cdots
\end{tikzcd}\]
By the previous lemma, this is an injective resolution of the sheaf
$\pi_*\mathcal{F}$. For an open set $U \subset Y$, we have that the sections
of $R^i\pi_*\mathcal{F}$ over $U$ are
\[
(R^i\pi_*\mathcal{F})(U) \defeq \left(\frac{\ker \pi_*d^i}{\im\pi_*d^{i-1}}\right)(U)
\]
By definition, the quotient sheaf $\ker\pi_*d^i/\im\pi_*d^{i-1}$ is the sheafification
of the presheaf
\[
U \mapsto \frac{(\ker\pi_*d^i)(U)}{(\im\pi_*d^{i-1})(U)}
\]
We then note that since the complex of sheaves is an injective resolution for
the sheaf $\pi_*\mathcal{F}$, the $R$-module given by
$(\ker\pi_*d^i)(U)/(\im\pi_*d^{i-1})(U)$ is the $i^{th}$ right derived functor
for the functor $\Gamma(U,-)$ that takes a sheaf over $Y$ to its sections over $U$.
We then note that this is exactly the $i^{th}$ sheaf cohomology group
$H^i(U,\pi_*\mathcal{F}\vert_U)$,  which is the same as
$H^i(\pi\inv(U),\mathcal{F}\vert_{\pi\inv(U)})$.
\end{proof}
%
In this way we see that the derived pushforwards behave like the sheaf cohomology
groups for a sheaf relative to the map $\pi : X \to Y$. Indeed, if $Y$ is a point,
then the functor $\pi_*$ is just taking global sections, and the derived
pushforwards $R^i\pi_*\mathcal{F}$ become the constant sheaves associated
to the sheaf cohomology groups $H^i(X,\mathcal{F})$.
%
\section{The Gauss-Manin Connection}
%
To study variations of Hodge structure, we first need a notion of what it means to vary
the complex structure on a smooth manifold $X$.
%
\begin{defn}
A \ib{family of complex manifolds} is a proper holomorphic submersion
$\pi : \mathcal{X} \to B$.
\end{defn}
%
If we fix a basepoint $b \in B$, then the fiber $X_b \defeq \pi\inv(b)$ is a
complex submanifold of $\mathcal{X}$. The idea is that fibers near $X_b$ should
be deformations of $X_b$. This is made precise by a theorem due to Ehresmann.
%
\begin{thm}[\ib{Ehresmann}]
Let $\pi : \mathcal{X} \to B$ be a smooth proper submersion, where $B$ is contractible.
Let $b \in B$ be a basepooint, and $X_b$ the fiber over $b$. Then there
exists a difeomorphism $\mathcal{X} \to X_b \times B$ such that the following
diagram commutes:
\[\begin{tikzcd}
\mathcal{X} \ar[dr, "\pi"']\ar[rr] && X_b \times B \ar[dl] \\
& B
\end{tikzcd}\]
where the map $X_b \times B \to B$ is projection onto the second factor.
\end{thm}
%
In particular, if $\pi$ is a family of complex manifolds, this statement
can be refined further.
%
\begin{thm}
Let $\pi : \mathcal{X} \to B$ be a proper holomorphic submersion where $B$ is
contractible. Fix a basepoint $b \in B$. Then there exists a smooth map
$T_b : \mathcal{X} \to X_b$ such that the map
$(T_b,\pi) : \mathcal{X} \to X_b \times B$ is a trivialization of $\mathcal{X}$ and
the fibers of $T_b$ are complex submanifolds of $\mathcal{X}$.
\end{thm}
%
Note that the map $T_b$ need not be (and often is not) holomorphic. However, this
tells us that in some sense, the complex structure is varying holomorphically. To
make this more precise, we note that by forgetting the complex structure, we can view
$X_b$ as a smooth manifold. From this perspective, a complex structure on $X_b$ can be
interpreted as a diffeomorphism $X \to X_b$ where $X$ is a complex manifold -- by
declaring this map to be an isomorphism of complex manfolds, this uniquely determines
a complex structure on the smooth manifold $X_b$. Using this perspective,
the map $T_b : \mathcal{X} \to X_b$ can be interpreted as a family of diffeomorphisms
$X_p \to X_b$ parameterized by the points $p \in B$ via restriction to the fibers
of $\pi$. If we pick a point $x$ in some fiber $X_p$, it lies in some fiber
of the map $T_b\vert_{X_p}$, which is a complex manifold diffeomorphic to $B$.
Moving in the ``$B$-direction" takes us though different fibers of $\pi$ (in other
words, different complex structures on $X_b$), and since the fibers of $T_b$ are
complex manifolds, this movement is ``holomorphic." Since all the points in the
fiber  $T_b\vert_{X_p}$ map to the same point as $x$ under $T_b$, this can be seen as
viewing the same point $T_b(x)$ of the underlying smooth manifold $X_b$ using the
different complex structures holomorphically parameterized by $B$. \\

Given a family of complex manifolds $\pi : \mathcal{X} \to B$
manifolds over a contractible base $B$ such that the fiber $X_b$ over a fixed
basepoint $b \in B$ is isomorphic to $X$ as a complex manifold, we get
an exact sequence of holomorphic vector bundles over $X$
\[\begin{tikzcd}
0 \ar[r] & T_X \ar[r] & T_\mathcal{X}\vert_X \ar[r] & (\pi^*T_B)\vert_X \ar[r] & 0
\end{tikzcd}\]
Since $B$ is contractible, the bundle $T_B$ is isomorphic to the trivial bundle
$B \times T_{B,b}$, so the bundle $(\pi^*T_B)\vert_X$ is isomorphic to the trivial
$X \times T_{B,b}$. The long exact sequence on sheaf cohomology induced by the short
exact sequence then gives us a boundary map
$\rho : H^0(X, X\times T_{B,b}) \to H^1(X,T_X)$. The cohomology group
$H^0(X, X\times T_{B,b})$ is equal to the space of global sections of
$X \times T_{B,b}$, which is simply the space of holomorphic maps $X \to T_{B,b}$.
Then since $X$ is compact, any such map is constant, so we may interpret $\rho$ as
a map $T_{B,b} \to H^1(X,T_X)$. This map is called the \ib{Kodaira-Spencer map}.
Though we won't get into it here, the Kodaira-Spencer map classifies the
infinitesimal deformation of complex structure in the family
$\pi : \mathcal{X} \to B$. \\

We now make use of our detour into derived pushforwards. Let $\pi : \mathcal{X} \to B$
be a family of complex manifolds over a contractible base $B$ with basepoint $b$.
Letting $\underline{R}$ denote the constant sheaf over $\mathcal{X}$ with stalk
$R$ where $R = \Z$, $\R$, or $\C$, we get the derived pushforward
sheaves $R^i\pi_*\underline{R} \in \mathrm{Sh}(B)$. Over a neighborhood $U$ of $b$,
we get that
\[
(R^i\pi_*\underline{R})(U) = H^i(\pi\inv(U),\underline{R}\vert_{\pi\inv(U)})
\]
Since $\underline{R}\vert_{\pi\inv(U)}$ is the same as the constant sheaf
over $\pi\inv(U)$, these are the same as the singular cohomology groups
$H^i(\pi\inv(U),R)$, which are topological invariants.
Then using Ehresmann's theorem, we know that $\mathcal{X} \to B$ can be
trivialized to $X_b \times B \to B$, so $\pi\inv(U) \cong X_b \times U$.
By restricting our attention to contractible neighborhoods $U$ of $b$
(which we can do since $B$ is locally contractible), we have that
$X_b \times U$ is homotopy equivalent to $X_b$. Putting this all together, we
find that the sheaves $R^i\pi_*\underline{R}$ are local systems with
stalks isomorphic to the singular cohomology groups $H^i(X_b,R)$.
%
\begin{defn}
Let $\pi : \mathcal{X} \to B$ be a family of complex manifolds, and let $b \in B$
be a basepoint. Let $\mathcal{H}^i$ denote the vector bundle obtained from the local
system $R^i\pi_*\underline{R}$. Then the \ib{Gauss-Manin connection} on
$\mathcal{H}^i$ is the flat connection $\nabla$ induced by the local system.
\end{defn}
%
A section of $\mathcal{H}^i$ can be thought of as a family of cohomology classes
$\alpha_t \in H^i(X_b,R)$ parameterized by $B$. From this perspective, the flat
sections are exactly the ones that define the same cohomology class as $\alpha_b$,
where we use the isomorphism $X_t \to X_b$ induced by the map
$T_b : \mathcal{X} \to X_b$ obtained by trivializing $\pi : \mathcal{X} \to B$
to identify the cohomology groups $H^i(X_t,R) \cong H^i(X_b,R)$.
%
\section{Variations of Hodge Structure}
%
Fix a compact K\"ahler manifold $X$ and a family of deformations
$\pi : \mathcal{X} \to B$ over a contractible base $B$, i.e. a family
where the fiber over a distinguished basepoint $b \in B$ is $X$. We then
have the following result.
%
\begin{thm}
For a sufficiently small neighborhood of the basepoint $b \in B$, all the
fibers are K\"ahler manifolds.
\end{thm}
%
Because of this, up to restricting $B$ to a smaller neighborhood, we can
assume that all the fibers are K\"ahler manifolds diffeomorphic to $X$. In
particular, this tell us that all the Betti numbers $b^i(X_t)$ are the
same for all $t \in B$. However, more is true.
%
\begin{thm}
For a sufficiently small neighborhood of the basepoint $b$, the Hodge numbers
$h^{p,q}(X_t) \defeq \dim H^{p,q}(X_t,\C)$ are equal to $h^{p,q}(X)$.
\end{thm}
%
Though we won't prove it here, the result can be deduced from upper semicontinuity
of the dimensions of the kernels of a smoothly varying elliptic differential
operator. \\

Since $B$ is contractible, we get isomorphisms $H^k(\mathcal{X},\C) \cong H^k(X,\C)$
and $H^k(\mathcal{X},\C) \cong H^k(X_t,\C)$ given by the restriction of differential
forms, so we get a canonical identification $H^k(X_t,\C) \cong H^k(X,\C)$.
for all $t \in B$. The fact that the Hodge numbers are constant might
suggest that the K\"ahler structure is not varying, but this isn't true.
One way to see this is through the period maps.
%
\begin{defn}
Let $b^{p,k} \defeq \dim F^pH^k(X,\C)$ and let $G = \mathrm{Gr}(b^{p,k}, H^k(X,\C))$.
denote the Grasmannian of complex $b^{p,k}$-dimensional subspaces of $H^k(X,\C$).
The \ib{period maps} of the family $\pi : \mathcal{X} \to B$ are the maps
\begin{align*}
\mathcal{P}^{p,k} : B &\to G \\
t &\mapsto F^pH^k(X_t,\C) \subset H^k(X,\C)
\end{align*}
where  we use the identification
$H^k(X_t,\C) \cong H^k(X,\C)$.
\end{defn}
%
There are two key facts about the period maps that we will need, but
won't prove.
%
\begin{prop} \enumbreak
\begin{enumerate}
  \item The $\mathcal{P}^{p,k}$ are holomorphic.
  \item The image of differential of $\mathcal{P}^{p,k}$ at $t \in B$
  \[
  d\mathcal{P}^{p,k}_t : T_{B,t} \to T_{G,F^pH^k(X_t,\C)}
  = \mathrm{Hom}(F^pH^k(X_t,\C), H^k(X,\C)/F^pH^k(X_t,\C))
  \]
  is contained in $\mathrm{Hom}(F^pH^k(X_t,\C), F^{p-1}H^k(X_t,\C)/F^pH^k(X_t,\C))$.
\end{enumerate}
\end{prop}
%
Altogether, we may package the period maps $\mathcal{P}^{p,k}$ into a
single map $\mathcal{P}^k$ from $B$ into a product of Grassmannians,
whose $p^{th}$ component is $\mathcal{P}^{p,k}$.
%
The first fact tells us that the Hodge filtration varies holomorphically
in a family. It is less obvious as to how one should interpret the second
fact -- we will need it to discuss Griffiths transversality. \\

We now put everything together to discuss variations of Hodge structure. Let
$\mathcal{H}^k$ denote the vector bundles over $B$ obtained from the local systems
$R^i\pi_*\underline{\C}$, equipped with the Gauss-Manin connection $\nabla$.
The Grassmannians $G^{p,k} \defeq \mathrm{Gr}(b^{p,k},H^k(X,\C))$ have natural
holomorphic vector bundles $\mathcal{S}^{p,k} \to G$ called the
\ib{tautological bundles} where the fiber over a subspace $W \subset H^k(X,\C)$ is $W$.
%
\begin{defn}
The \ib{Hodge bundles} of the family of deformations $\pi : \mathcal{X} \to B$
are the holomorphic vector bundles
$F^p\mathcal{H}^k \defeq (\mathcal{P}^{p,k})^*\mathcal{S}^{p,k}$.
\end{defn}
%
As the name suggests, the fiber of $F^p\mathcal{H}^k$ over $t \in B$ is the fiber
of $\mathcal{S}^{p,k}$ over $\mathcal{P}^{p,k}(t) = F^pH^k(X_t,\C)$. The
Hodge bundles give a decreasing filtration of $\mathcal{H}^k$
\[
\mathcal{H}^k = F^0\mathcal{H}^k \supset \cdots \supset F^k\mathcal{H}^k
\supset F^{k+1}\mathcal{H}^k = 0
\]
where over each $t \in B$, taking the fibers over $t$ gives the Hodge filtration on
$X_t$. Furthermore, the quotient bundles
$\mathcal{H}^{p,q} \defeq F^p\mathcal{H}^k/F^{p+1}\mathcal{H}$ (where $p+q = k$)
have the property that the fibers over $T$ are given by
$F^pH^k(X_t,\C)/F^{p+1}H^k(X_t,\C) = H^{p,q}(X_t,\C)$, which tells us that
the associated graded vector bundle of this filtration gives us the
Hodge decompositions over each fiber. Furthermore, the Hodge subbundles satisfy
a condition called \ib{Griffiths transversality}.
%
\begin{thm}[\ib{Griffiths Transversality}]
The Hodge bundles satisfy
\[
\nabla F^p\mathcal{H}^k \subset F^{p-1}\mathcal{H}^k \otimes \Omega^1_B
\]
\end{thm}
%
\begin{proof}
Since the flat sections of $\mathcal{H}^k$ are a local system with constant
stalk $H^k(X,\C)$, we can trivialize the bundle $\mathcal{H}^k$ to
obtain an isomorphism of holomorphic vector bundles
$\mathcal{H}^k \cong B \times H^k(X,\C)$ which identifies the Gauss-Manin
connection $\nabla$ with the ordinary derivative on the trivial bundle. Putting
things together, $\nabla$ can be interpreted as follows: a section
$\sigma : B \to \mathcal{H}^k$ maps a point $t \in B$ to a cohomology class
$\sigma(t) \in H^k(X_t,\C)$, which is identified with a cohomology class
$\alpha_t$ in $H^k(X,\C)$. Differentiating $\sigma$ in the direction
$v \in T_{B,t}$ amounts to taking the infinitesimal change in the cohomology
class identified with $\alpha_t$ as we vary the Hodge filtration by considering
the fibers of $\mathcal{X} \to B$ in the direction $v$, which
is encoded by the image of $v$ under the differential of the period map
$d\mathcal{P}^k_t(v)$, which is an element of
$\bigoplus_p \mathrm{Hom}(F^pH^k(X_t,\C), H^k(X,\C)/F^p(X_t,\C))$.
If the class $\sigma(t)$ lies in $F^pH^k(X_t,\C)$, then applying
the $p^{th}$ component of $d\mathcal{P}^k_t(v)$ to $\sigma(t)$ gives us
$\nabla_v\sigma(t)$. Griffiths transversality then follows immediately
from the second fact in Proposition 5.4.
\end{proof}
%
Griffiths transversality tells us that the restrictions of $\nabla$
to the Hodge bundles $F^p\mathcal{H}^k$ descend to the quotients
$\mathcal{H}^{p,q}$ to maps
$\theta^{p,q} : \mathcal{H}^{p,q} \to \mathcal{H}^{p-1,q+1} \otimes \Omega^1_B$.
%
\begin{prop}
The maps $\theta^{p,q} : \mathcal{H}^{p,q} \to \mathcal{H}^{p-1,q+1} \otimes \Omega^1_B$
are holomorphic bundle homomorphisms.
\end{prop}
%
\begin{proof}
We want to show that the $\theta^{p,q}$ are $\O_B$-linear. Let
$f$ be a holomorphic function and $\sigma$ a section of
$\mathcal{H}^{p,q} \subset \mathcal{H}^k$, which we can represent with
a section of $F^p\mathcal{H}^k$, which we also call $\sigma$.
Since $\nabla$ is a connection, we have that
\[
\nabla(f\sigma) = f\nabla(\sigma) + \sigma \otimes df
\]
We then note that $\sigma \otimes df$ is a section of
$F^p\mathcal{H}^k\otimes \Omega^1_B$, and that
$\nabla(\sigma)$ is a section of $F^{p-1}\mathcal{H}^k$ by Griffiths transversality,
so $f\nabla(\sigma)$ and $\nabla(f\sigma)$ define the same element of
$\mathcal{H}^{p-1,q+1} \otimes \Omega^1_B$. Therefore, $\theta^{p,q}$ is
$\O_B$-linear.
\end{proof}
%
Working fiberwise over $t \in B$, we have that $\theta^{p,q}$ restricted
to the fiber over $t$ is a map
\[
\mathcal{H}^{p,q}_t = H^{p,q}(X,\C) \to
(\mathcal{H}^{p-1,q+1} \otimes \Omega^1_B)_t = H^{p-1,q+1}(X_t,\C) \otimes T^*_{B,t}
\]
Given a section $\sigma$ of $\mathcal{H}^{p,q}$ whose value at
$t$ is the cohomology class $[\sigma(t)] \in H^{p,q}(X_t,\C)$ applying
$\theta^{p,q}$ to $\sigma$ and evaluating the result at a tangent vector
$v \in T_{B,b}$ gives us a cohomology class in $H^{p-1,q+1}(X_t,\C)$, which
can be interpreted as the infinitesimal change in $[\sigma(t)]$ as we vary the
fibers of $\mathcal{X} \to B$ in the direction $v$. The maps $\theta^{p,q}$ can
explicitly be realized as the differentials of the period map.
For this reason, the collection of maps $\set{\theta^{p,q}}$  evaluated at $t$ is
called the \ib{infinitesimal variation of Hodge structure} at $t$.
Another useful characterization of the $\theta^{p,q}$ comes from
second fundamental forms with respect to $\nabla$ of the short exact sequences
%
\[\begin{tikzcd}
0\ar[r]& F^p\mathcal{H}\ar[r]& \mathcal{H} \ar[r] & \mathcal{H}/F^p \mathcal{H} \ar[r]&0
\end{tikzcd}\]
The second fundamental form gives a map
$F^p \mathcal{H} \to \mathcal{H}/F^p \mathcal{H} \otimes \mathcal{A}^{1,0}_B$,
which is obtained by composing $\nabla$ with the map
$\mathcal{H} \otimes \mathcal{A}^{1,0}_B\to\mathcal{H}/F^p\mathcal{H}
\otimes \mathcal{A}^{1,0}_B$ induced by the quotient map. By Griffiths
transversality, the image of $F^p \mathcal{H}$ under the second fundamental form
is contained in $F^{p-1}\mathcal{H}/F^p\mathcal{H} = \mathcal{H}^{p-1,q}$
and $F^{p+1}\mathcal{H}$ is mapped to $F^p \mathcal{H}$, so the second fundamental
form descends to the quotient $F^p \mathcal{H}/F^{p+1} = \mathcal{H}^{p,q}$ to give
the map $\theta^{p,q}$. A nontrivial computation shows that the
maps $\theta^{p,q}$ are given by composing interior multiplication with
wedgeing with the Kodaira-Spencer map.\\

The above discussion motivates the definition of an abstract variation of
Hodge structure.
%
\begin{defn}
A \ib{variation of Hodge structure} of weight $k$ over a complex manifold $B$ is the
data of a local system $V$ of abelian groups over $B$, and a decreasing filtration
of the vector bundle $\mathcal{V} \defeq V \otimes_{\underline{\C}} \O_B$
by holomorphic subbundles $F^p\mathcal{V}$ such that
\begin{enumerate}
  \item $\mathcal{V} = F^p\mathcal{V} \oplus \overline{F^{k-p+1}\mathcal{V}}$
  as smooth vector bundles, where conjugation is taken from the real
  structure defined by $V \otimes_{\underline{\R}} \R$.
  \item The subbundles $F^p \mathcal{V}$ satisfy the \ib{Griffiths transversality}
  condition
  \[
  \nabla(F^p \mathcal{V}) \subset \mathcal{F}^{p-1}\otimes_{\O_B} \Omega^1_B
  \]
  where $\nabla$ is the Gauss-Manin connection on $\mathcal{V}$ obtained
  from the local system $V$.
\end{enumerate}
\end{defn}
%
The prototypical example of a variation of Hodge structure of weight $k$ is given by
the local system $R^k\pi_*\underline{\Z}$ obtained from a family
$\pi : \mathcal{X} \to B$. A variation of Hodge structure can also be
defined in terms of the Hodge decomposition instead of the Hodge filtration,
i.e. by specifying a direct sum decomposition of $\mathcal{H}$ by
subbundles $\mathcal{H}^{p,q}$ giving a Hodge decomposition of each fiber.
One can go from one viewpoint to the other in one direction by taking quotients
$F^p \mathcal{V}/F^{p+1} \mathcal{V}$, and in the other direction by taking
direct sums of the $\mathcal{H}^{p,q}$.\\

For polarized Hodge structures the prototypical example will be a family of
smooth projective varieties $\pi : \mathcal{X} \to B$. The fibers of $\pi$
come equipped with integral K\"ahler classes, giving us a section
of the local system $R^2\pi_*\underline{\Z}$, which we use to define the
Lefschetz operator fiberwise, giving us a bilinear form $Q$ on each of the
stalks of the local system $R^k\pi_*\underline{\Z}$ and a sub Hodge structure
on each fiber given by the primitive component of $H^k(X_t,\Z)$, which forms
a local system $V_{\mathrm{Prim}} \subset R^k\pi_*\underline{\Z}$.
Complexifying, we get a Hermitian fiber metric on the vector bundle
$R^k\pi_*\underline{\Z} \otimes_{\underline{\C}} \O_B$, and the restrictions
of $Q$ to $V_{\mathrm{Prim}}$ and $H$ to
$\mathcal{H}_{\mathrm{Prim}} \defeq V_{\mathrm{Prim}} \otimes_{\C} \O_B$
give the stalks of $V_{\mathrm{Prim}}$ the structure of polarized Hodge
structures. Furthermore, the compatibility of the Hodge decomposition
with the primitive components tells us that the local system $V_{\mathrm{Prim}}$
defines a variation of Hodge structure. Putting things together, we get
%
\begin{defn}
A \ib{polarized variation of Hodge structure} of weight $k$ over $B$ is
a variation of Hodge structure $V$ of weight $k$, along with a bilinear form
$Q$ on each stalk of $V$ such that
\begin{enumerate}
  \item $Q$ is symmetric if $k$ is even and alternating if $k$ is odd.
  \item The Hodge decomposition on each fiber of the associated vector
  bundle is orthogonal with respect to the Hermitian fiber metric $H$ induced by $Q$.
  \item The hermitian form $H$ is flat with respect to the Gauss-Manin connection,
  i.e. $\nabla H = 0$.
  \item The form $(-1)^{k(k-1)/2}i^{p-q-k}H$ is positive definite on each
  fiber of $V^{p,q}$.
\end{enumerate}
\end{defn}
%
The Hermitian fiber metric associated to a polarized variation of Hodge structure is
often called the \ib{Hodge metric}.
%
\section{Higgs Bundles and Complex Variations of Hodge Structure}
%
Polarized complex variations of Hodge structure are in correspondence with objects
called \ib{Higgs bundles}, which is the shadow of a the
\ib{nonabelian Hodge theorem}\cite{SimpsonHiggsBundles},
which provides an equivalence between the category of local systems with a certain
subcategory of the category of Higgs bundles. For this section, we
will assume that we are working with vector bundles over a smooth complex
projective variety $X$.
%
\begin{defn}
A \ib{Higgs bundle} is a holomorphic vector bundle $E \to X$ equipped with
a holomorphic $\End(E)$ valued $1$-form $\theta \in \Omega^1_X(\End(E))$
satisfying $\theta \wedge \theta = 0$. The form $\theta$ is called
the \ib{Higgs field}.
\end{defn}
%
Given a complex variation of Hodge structure, we want to construct a Higgs bundle.
Suppose we have a complex variation of Hodge structure of weight $k$, which is given
by a holomorphic vector bundle $E \to X$ along with a filtration of $E$
by holomorphic subbundles $F^pE$ and a flat connection $\nabla$ on $E$
satisfying Griffiths transversality. We let
$E_{\mathrm{Hodge}} = \bigoplus_p F^pE/F^{p+1}E$ denote
the associated graded bundle coming from the filtration, and let $E^{p,q}$
(where $p+q = k$) denote the summands. We have that $E$ is isomorphic to
$E_{\mathrm{Hodge}}$ as smooth bundles, but not necessarily as holomorphic
bundles. The infinitesimal variations of Hodge structures give us holomorphic
maps $\theta^{p,q} : E^{p,q} \to E^{p-1,q}\otimes \Omega^1_X$,
and taking their sum gives a holomorphic map $\theta : E \to E \otimes \Omega^1_X$,
i.e. an element $\theta \in \Omega^1_X(\End(E))$. Furthermore, since $\theta$
is induced by the Gauss-Manin connection on $E$, which is flat, we get
that $\theta^2 = 0$, so $\theta$ defines a Higgs field. Putting things together,
a complex variation of Hodge structure $E$ gives us a Higgs bundle
$(E_{\mathrm{Hodge}}, \theta)$. \\

Suppose further that $E \to X$ comes from a polarized variation of
Hodge structure, so we have a Hodge metric on $E_{\mathrm{Hodge}}$
that is flat with respect to the Gauss-Manin connection $\nabla$, and
whose restrictions to the $E^{p,q}$ are definite. By changing the signs
of the Hodge metric on the $E^{p,q}$ to make it positive definite, we obtain
a Hermitian metric on $E_{\mathrm{Hodge}}$, which for simplicity we will
also refer to as the Hodge metric -- from now on ``Hodge metric" will always
refer to the positive definite form unless specified otherwise. The
Hodge metric on $E_{\mathrm{Hodge}}$ gives us a Chern connection
$D$, and the isomorphism of $E$ and $E_{\mathrm{Hodge}}$ as smooth bundles
allows us to transport $\nabla$ to a flat connection on $E_{\mathrm{Hodge}}$.
Furthermore, we can use the Hodge metric to define the splitting
of the filtration of $E$ by the bundles $F^pE$ to obtain our
isomorphism of $E$ with $E_{\mathrm{Hodge}}$ as a smooth bundles.
The difference $\nabla - D$ is an element of
$\mathcal{A}^1_B(\End(E_\mathrm{Hodge}))$, and it turns out to be a familiar friend.
%
\begin{prop}
Let $\theta \in \Omega^1_X(\End(E_\mathrm{Hodge}))$ be the map obtained by summing the
infinitesimal variations of Hodge structure. Then
\[
\nabla - D = \theta + \theta^*
\]
Where $\theta^*$ is the Hermitian adjoint of $\theta$ with respect to the Hodge
metric.
\end{prop}
%
The proof of this amounts to understanding the second fundamental form
of a holomorphic subbundle.
%
\begin{lem}
Let $E \to X$ be a Hermitian holomorphic vector bundle, $S \subset E$ a
holomorphic subbundle, and $Q \defeq E/S$ the quotient bundle. Equip
$S$ with the Hermitian metric by restricting the Hermitian metric on $E$,
and equip $Q$ with the Hermitian metric obtained by identifying the
underlying smooth bundle with $S^\perp$. Let $\nabla_E$,$\nabla_S,$ and $\nabla_Q$
denote the respective Chern connections, and
$\eta \in \mathcal{A}^{1,0}_X(\hom(S,Q))$ the second fundamental form.
Then under the identification of the underlying smooth bundle of $E$
with $S \oplus Q$, we have that
\[
\nabla_E - (\nabla_S + \nabla_Q) = \eta + \eta^*
\]
where $\eta^*$ is the Hermitian adjoint of $\eta$, and $\nabla_S + \nabla_Q$
is the direct sum connection on $S \oplus Q$.
\end{lem}
%
\begin{proof}
It suffices to verify this locally. Fix an orthonormal frame $e_1,\ldots, e_n$
for $E$ such that $e_1, \ldots, e_s$ forms an orthonormal frame for
$S$, so $e_{s+1},\ldots e_n$ is a local orthonormal frame for $S^\perp$. Then by
the definition of the second fundamental form we have that the connection matrix
$\alpha_E$ for $\nabla_E$ in this frame is given in block form by
\[
\begin{pmatrix}
\alpha_S & \eta^* \\
\eta & \alpha_Q
\end{pmatrix}
\]
where $\alpha_S$ and $\alpha_Q$ are the connection matrices for $\nabla_S$ and
$\nabla_Q$ respectively. Since the connection matrix for $\nabla_S + \nabla_Q$
is block diagonal with $\alpha_S$ and $\alpha_Q$ on the diagonal, this
shows the desired result.
\end{proof}
%
\begin{proof}[Proof of proposition]
Since $\nabla$ is the Gauss-Manin connection coming from a local system,
we know that the $(0,1)$ part defines the holomorphic structure on $E$.
Furthermore, since the Hodge metric is flat with respect to $\nabla$,
we get that $\nabla$ is the Chern connection on $E$ with respect to the
holomorphic structure obtained from $\nabla$. We have an canonical identification
of $E^{k,0}$ with the bundle $F^kE$ as holomorphic bundles, where $k$ is the
weight of the variation of Hodge structure. Then we can identify
$(F^kE)^\perp$ with the quotient $E/F^kE$ and $F^{k-1}E/F^k = E^{k-1,1}$
with a subbundle of $F^kE^\perp$ as smooth bundles. Iterating this
construction with $E^{k-1,1}$ and $F^{k-2}E$ and so on, we get an
orthogonal splitting identifying $F^pE$ with $\bigoplus_{r\leq p}E^{p,q}$,
identifying the underlying smooth bundles of $E$ and the Hodge bundle
$E_{\mathrm{Hodge}}$. Then using the fact that the Chern connection on
$E_{\mathrm{Hodge}}$ is the sum of the Chern connections on the $E^{p,q}$,
an iterative application of the previous lemma gives the desired
result.
\end{proof}
%
We then want to identify the Higgs bundles that arise from variations
of Hodge structure. The set of isomorphism classes of Higgs bundles admits
a $\C^\times$-action by scaling the Higgs field, i.e. the action of
$t \in \C^\times$ on a Higgs bundle $(E,\theta)$ is $(E,t\theta)$. Our
first observation relates the fixed points of this $\C^\times$ action
with what Simpson calls a \ib{system of Hodge bundles}, which is
slightly weaker notion than a variation of Hodge structure.
%
\begin{prop}
Suppose a Higgs bundle $(E,\theta)$ admits the structure of a
\ib{system of Hodge bundles}, i.e. there exists some $k \in \Z^{\geq 0}$ such
that $E$ admits a direct sum decomposition $E = \bigoplus_{p+q=k}E^{p,q}$ such
that $\theta$ maps $E^{p,q}$ to $E^{p-1,q+1}\otimes\Omega^1_X$. Then $E$ is
fixed under the $\C^\times$ action, i.e. $(E,\theta) \cong (E,t\theta)$ for all
$t \in \C^\times$. Conversely, any Higgs bundle that is fixed by the $\C^\times$
action admits such a decomposition.
\end{prop}
%
\begin{rem*}
The difference between a variation of Hodge structure and a system of Hodge
bundles is somewhat subtle. Given a variation of Hodge structure, one
obtains a system of Hodge bundles by taking the associated graded
bundle and the infinitesimal variations of Hodge structure. In general,
one cannot go the other direction, since we have forgotten the data
of the Gauss-Manin connection and have only kept the ``infinitesimal data."
\end{rem*}
%
\begin{proof}
First suppose $(E,\theta)$ admits the structure of a system of Hodge bundles.
We want to produce isomorphisms $(E,\theta) \to (E,t\theta)$ for all
$t \in \C^\times$. Fix such a $t$, and consider the map $\varphi_t : E \to E$ given
by scaling $E^{p,q}$ by $t^q$. The map is a holomorphic bundle isomorphism
$E \to E$, and we have that for $v \in E^{p,q}$
\[
(t\theta\circ \varphi_t)(v) = t\theta(t^qv) = t^{q+1}\theta(v) =
((\varphi_t \otimes \id_{\Omega^1_X}) \circ \theta)(v)
\]
where we use the fact that $\theta(v) \in E^{p-1,q+1}$. \\

For the other direction, suppose we have a Higgs bundle $(E,\theta)$ of rank $k$
that is fixed by the $\C^\times$-action. In particular, for a $t \in \C^\times$
where $t$ is not a root of unity, we have that there exists an isomorphism
$\varphi_t : E \to E$ such that $\varphi_t$ intertwines $\theta$ and $t\theta$
as above. Taking the characteristic polynomials of $\varphi_t$
fiberwise on $E$, we get holmomorphic functions on $B$ mapping $b \in B$
to the $i^{th}$ coefficient of the characteristic polynomial of $\varphi_t$
restricted to $E_b$. Since $B$ is compact, these functions are constant,
so the eigenvalues of $\varphi_t$ are constant functions on $B$, so we
can decompose $E = \bigoplus_{\lambda}E_\lambda$ into $\lambda$-eigenbundles
of $\varphi_t$, so $E_\lambda \defeq \ker(f-\lambda)^n$, where $n$ is the multiplicity
of the eigenvalue $\lambda$. By assumption, we have that
$t\theta \circ \varphi_t = (\varphi_t \otimes \id_{\Omega^1_X}) \circ\theta$,
so we get that
$t^n\theta\circ (f-\lambda)^n=((\varphi-t\lambda)^n\otimes\id_{\Omega^1_X})\circ \theta$, which tells us that $\theta$ maps $E_\lambda$ to
$E_{t\lambda} \otimes \Omega^1_X$. Since $t$ is not a root of unity, we
may decompose the set of eigenvalues as a disjoint uniion of sets of the form
$\set{\lambda_i, t\lambda_i, t^2\lambda_i, \ldots t^r\lambda_i}$ for some subset
of eigenvalues $\lambda_i$. Then we get a system of Hodge bundles by
defining $E^{i,k-i}$ as the direct sum $\bigoplus_i E_{\lambda_i}$.
\end{proof}
%
This suggests that Higgs bundles are intimately related to complex variations
of Hodge structure, and if we restrict ourselves to polarized complex
variations of Hodge structure, we do in fact get a correspondence, which is a
special case of the \ib{nonabelian Hodge theorem}. To state the
nonabelian Hodge theorem, we use the fact that a local system $V$
is equivalent to the data of a representation $\pi_1(X) \to \GL_k\C$.
In one direction, you take the holonomy of the Gauss-Manin connection,
which is independent of the homotopy class of a loop from flatness. In
the other direction, you obtain a flat bundle from a represetation
$\pi_1(M) \to \GL_k\C$ by taking the associated bundle
\[
\widetilde{X} \times_{\pi_1(M)} \C^n \defeq (\widetilde{X}\times\C^n)/\pi_1(M)
\]
with $\pi_1(M)$ acting diagonally. The second notion we'll need
is that notion of stability of Higgs bundles.
%
\begin{defn}
The \ib{slope} $\mu(E)$ of a Higgs bundle $(E,\theta)$ is
$\mu(E) \defeq \mathrm{deg}(E)/\mathrm{rank}(E)$. A Higgs bundle $(E,\theta)$
is :
\begin{enumerate}
  \item \ib{Stable} if for every Higgs subbundle $S \subset E$, i.e. a holomorphic
  subbundle such that $\theta(S) \subset S \otimes \Omega^1_X$, we have
  $\mu(S) < \mu(E)$.
  \item \ib{Semistable} if for every Higgs subbundle $S$, we have
  $\mu(S) \leq \mu(E)$.
  \item \ib{Polystable} if $E$ is a direct sum of stable Higgs subbundles
  of the same slope.
\end{enumerate}
\end{defn}
%
This allows us to state the nonabelian Hodge theorem.
%
\begin{thm}[\ib{Nonabelian Hodge Theorem}]
There is an equivalence of categories between the category of
local systems arising from semisimple representations of $\pi_1(X)$
and polystable Higgs bundles with vanishing Chern classes.
\end{thm}
%
To apply this theorem, we must identify the systems of Hodge bundles that
arise from complex variations of Hodge structure. This was done by
Simpson\cite{SimpsonYangMills}, building upon work of Donaldson and Uhlenbeck-Yau
on the existence of Hermitian-Yang-Mills connections on holomorphic vector
bundles.
%
\begin{thm}
Every stable Higgs bundle $(E,\theta)$ admits a Hermitian-Yang-Mills metric.
In particular, if the Chern classes of $E$ vanish, then $E$ admits a flat
connection. If $E$ comes from a system of Hodge bundles, $(E,\theta)$,
then the flat connection satisfies Griffiths transversality, and the
maps $\theta^{p,q}$ are induced by the flat connection.
\end{thm}
%
This theorem tells us that the construction taking a variation of Hodge
structure to a system of Hodge bundles can be inverted in the case that
the system of Hodge bundles forms a stable Higgs bundle. \\

Furthermore, we have the following results:
%
\begin{prop} \enumbreak
\begin{enumerate}
  \item The system of Hodge bundles arising from an complex variation
  of Hodge structure with irreducible holonomy $\pi_1(M) \to \mathrm{GL}_k\C$
  is a stable Higgs bundle.
  \item The representations $\pi_1(M) \to \GL_k\C$ arising from polarized
  complex variations of Hodge structure are semisimple, i.e. they decompose
  into direct sums of irreducible representations.
\end{enumerate}
\end{prop}
%
Combinining the first part of the proposition with Theorem 7.7, we get
that the stable systems of Hodge bundles are exactly in correspondence
with variations of Hodge structure with irreducible holonomy. The
second part of the proposition tells us that any complex variation
of Hodge structure decomposes into a direct sum of irreducible ones,
so the corresponding system of Hodge bundles decomposes into
a direct sum of stable systems. Putting everything together, we
get:
%
\begin{prop}
Polarized complex variations of Hodge structure are in bijection with
semisimple representations of $\pi_1(X)$ whose corresponding Higgs bundles
are fixed by the $\C^\times$ action.
\end{prop}
%
It's worth noting how this correspondence fits into the proof of
the nonabelian Hodge theorem. The proof consists of proving that
the categories of semisimple representations of $\pi_1(X)$ and
polystable Higgs bundles with vanishing Chern classes are equivalent
to the same category -- the category of \ib{harmonic bundles}.
The idea of harmonic bundles concerns the subtle interplay
between a flat connection $\nabla$ on a Hermitian vector bundle $E$
and the Chern connection on $E$ with respect to the holomorphic
structure defined by the $(0,1)$ part of $\nabla$. \\

Suppose we have a flat connection $\nabla$ on a vector bundle $E$ with Hermitian
metric $K$. Let $\nabla'$ and $\nabla"$ denote the $(1,0)$ and $(0,1)$ parts
respectively, so $\nabla = \nabla' + \nabla"$. Let $D$ be the Chern connection with
respect to the holomorphic structure $\nabla"$ and the Hermitian metric,
and let $\delta_K'$ denote the $(1,0)$ part of $D$, so $D = \delta_K' + \nabla"$.
Let $\theta_K = (\nabla' - \delta_K')/2$. We then define the \ib{pseudocurvature}
to be the operator $G_K \defeq (\dbar + \theta_K)^2$. The pseudocurvature vanishes
precisely when $\theta_K\wedge\theta_K = 0$, i.e. $\theta_K$ defines a Higgs bundle
structure on $E$.\\

Suppose instead we have a Higgs bundle $(E,\theta)$. Then we get
a connection $\nabla$ from the formula
$\nabla \defeq \partial + \dbar + \theta + \theta^*$. In general, the
connection $\nabla$ need not be flat. However, when it is, the
construction above recovers the Higgs field $\theta$.
%
\begin{defn}
A \ib{harmonic bundle} is a holomorphic bundle $E$ with flat connection
$\nabla$ such that there exists a Hermitian metric $K$ such that
$\theta_K \defeq (\nabla' - \delta'_K)/2$ defines a Higgs bundle structure on $E$.
Such a Hermitian metric is a \ib{harmonic metric}.
\end{defn}
%
Therefore the idea of the proof of the nonabelian Hodge theorem involves
proving the existence of Harmonic metrics on polystable Higgs bundles.
The proof of this is in the spirit of, and draws upon the proofs of
Narasimhan-Seshadri and Uhlenbeck-Yau.
In the case of polarized complex variations of Hodge structure, we can
reinterpret our observation as the statement that the Hodge metric is a
harmonic metric.
%
\iffalse
Fix a smooth complex vector bundle $E \to B$ and a flat connection
$\nabla : E \to E \otimes \mathcal{A}^1_X$. We decompose $\nabla$ by
type to get $\nabla = \nabla'+\nabla"$ where
\begin{align*}
\nabla' : E &\to E\otimes \mathcal{A}^{1,0}_B \\
\nabla" : E &\to E \otimes \mathcal{A}^{0,1}_B
\end{align*}
%
By comparing types, the fact that $\nabla^2 = 0$ gives us the following facts.
%
\begin{enumerate}
  \item $(\nabla')^2 = 0$.
  \item $\nabla'\nabla" + \nabla"\nabla' = 0$.
  \item $(\nabla")^2 = 0$.
\end{enumerate}
%
The third fact gives us that $\nabla"$ defines a holomorphic structure on
$E$, where the holomorphic sections are the smooth sections $\sigma$ satisfying
$\nabla"(\sigma) = 0$. The second fact tells us that $\nabla'$ defines
a holomorphic connection on $E$ with respect to holomorphic structure $\nabla"$,
i.e. for any holomorphic section $\sigma$, we have $\nabla'(\sigma)$ is also
holomorphic, since we have
\[
\nabla"(\nabla'\sigma) = -\nabla'(\nabla"(\sigma)) = 0
\]
The first fact gives us that $\nabla'$ is a flat holomorphic connection. \\

Suppose further that our flat bundle is equipped with a Hermitian fiber metric
$H$. Then there exists a unique smooth connection $D$ on $E$ called the
\ib{Chern connection} such that $D" = \nabla"$ and
$dH(\sigma_1,\sigma_2) = H(D(\sigma_1),\sigma_2) + H(\sigma_1,D(\sigma_2))$
where $D"$ is the $(0,1)$ part of $D$, and $\sigma_1$ and $\sigma_2$
are smooth sections of $E$. In a smooth local trivialization of $E$,
the Chern connection can be written as $d + A$ for some Hermitian
$A \in \mathcal{A}^1_B(\End E)$, which we can write uniquely as
$A = \theta + \theta^*$, where $\theta \in \mathcal{A}^1_B(\End E)$ and
$\theta^*$ is the Hermitian adjoint of $\theta$ with respect to the fiber metric.
Since the connection $\nabla$ is  flat, it can be identified with $d$ is a local
trivialization, so we find that locally, the difference $\nabla - D$ is equal
to $\theta + \theta^*$. These local definitiions glue together, since the
difference of any two connections is an element of $\mathcal{A}^1_B(\End E)$,
so we may globally write $\nabla - D = \theta + \theta^*$. Furthermore,
since $D" = \nabla"$, we have that $\theta+\theta^*$ is type $(1,0)$, i.e.
$\theta+\theta^* \in \mathcal{A}^{1,0}_B(\End E)$.\\

Suppose further that our flat bundle $E$ comes from a polarized
complex variation of Hodge structure of weight $k$, so we have a decomposition
$E = \bigoplus_{p+q=k} E^{p,q}$ of $E$ (as a smooth bundle) into holomorphic
subbundles $E^{p,q}$ and a Hodge metric $Q$, which is definite on the $E^{p,q}$.
By modifying the signs of $Q$ on the $E^{p,q}$, we constrct a positive
definite Hermitian form $H$ on $E$ whose restrictions to the $E^{p,q}$
are positive definite.
\fi
%
\iffalse
\section{The Kodaira-Spencer Map}
The Kodaira-Spencer map has a fairly explicit geometric description in terms of a
trivialization of the family $\pi : \mathcal{X} \to B$ and the Dolbeault isomorphism
$H^{0,1}_{\dbar}(X,T_X) \cong H^1(X,T_X)$. From theorem 4.3, we obtain
a smooth map $T_b : \mathcal{X} \to X_b$ such that the fibers are complex submanifolds
of $\mathcal{X}$ and $(T_b,\pi)$ is a trivialization of $\mathcal{X}$. Fix
a vector $v \in T_{B,b}$, and consider the constant vector field
$V$ on $TB \cong B \times T_{B,b}$ where $V_b = v$. We can then extend this
to $TX \times TB$ by making the second component $0$, and then taking
the pushforward along the inverse of the trivialization gives a
complex vector field on $\mathcal{X}$ of type $(1,0)$, which we will also
call $V$. Restricting $V$ to the fiber $X_b = X$ then gives a smooth section of
$T_X$, and then the class represented by $\dbar (V\vert_{X})$ is the
value of $\rho(v)$
\fi
%
\newpage
%
\nocite{*}
%
\printbibliography
%
\end{document}