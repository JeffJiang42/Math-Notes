\documentclass[psamsfonts, 12pt]{amsart}
%
%-------Packages---------
%
\usepackage[h margin=1 in, v margin=1 in]{geometry}
\usepackage{amssymb,amsfonts}
\usepackage[all,arc]{xy}
\usepackage{tikz-cd}
\usepackage{enumerate}
\usepackage{mathrsfs}
\usepackage{amsthm}
\usepackage{mathpazo}
\usepackage{float}
%\usepackage[backend=biber]{biblatex}
%\addbibresource{bibliography.bib}
%\usepackage{charter} %another font
%\usepackage{eulervm} %Vakil font
\usepackage[mathcal]{eucal}
\usepackage{yfonts}
\usepackage{mathtools}
\usepackage{enumitem}
\usepackage{mathrsfs}
\usepackage{fourier-orns}
\usepackage[all]{xy}
\usepackage{hyperref}
\usepackage{url}
\usepackage{mathtools}
\usepackage{graphicx}
\usepackage{pdfsync}
\usepackage{mathdots}
\usepackage{calligra}
\usepackage{import}
\usepackage{xifthen}
\usepackage{pdfpages}
\usepackage{transparent}

\usepackage{tgpagella}
\usepackage[T1]{fontenc}
%
\usepackage{listings}
\usepackage{color}

\definecolor{dkgreen}{rgb}{0,0.6,0}
\definecolor{gray}{rgb}{0.5,0.5,0.5}
\definecolor{mauve}{rgb}{0.58,0,0.82}

\lstset{frame=tb,
  language=Matlab,
  aboveskip=3mm,
  belowskip=3mm,
  showstringspaces=false,
  columns=flexible,
  basicstyle={\small\ttfamily},
  numbers=none,
  numberstyle=\tiny\color{gray},
  keywordstyle=\color{blue},
  commentstyle=\color{dkgreen},
  stringstyle=\color{mauve},
  breaklines=true,
  breakatwhitespace=true,
  tabsize=3
  }
%
%--------Theorem Environments--------
%
\newtheorem{thm}{Theorem}[section]
\newtheorem*{thm*}{Theorem}
\newtheorem{cor}[thm]{Corollary}
\newtheorem{prop}[thm]{Proposition}
\newtheorem{lem}[thm]{Lemma}
\newtheorem*{lem*}{Lemma}
\newtheorem{conj}[thm]{Conjecture}
\newtheorem{quest}[thm]{Question}
%
\theoremstyle{definition}
\newtheorem{defn}[thm]{Definition}
\newtheorem*{defn*}{Definition}
\newtheorem{defns}[thm]{Definitions}
\newtheorem{con}[thm]{Construction}
\newtheorem{exmp}[thm]{Example}
\newtheorem{exmps}[thm]{Examples}
\newtheorem{notn}[thm]{Notation}
\newtheorem{notns}[thm]{Notations}
\newtheorem{addm}[thm]{Addendum}
\newtheorem{exer}[thm]{Exercise}
%
\theoremstyle{remark}
\newtheorem{rem}[thm]{Remark}
\newtheorem*{claim}{Claim}
\newtheorem*{aside*}{Aside}
\newtheorem*{rem*}{Remark}
\newtheorem*{hint*}{Hint}
\newtheorem*{note}{Note}
\newtheorem{rems}[thm]{Remarks}
\newtheorem{warn}[thm]{Warning}
\newtheorem{sch}[thm]{Scholium}
%
%--------Macros--------
\renewcommand{\qedsymbol}{$\blacksquare$}
\renewcommand{\sl}{\mathfrak{sl}}
\newcommand{\Bord}{\mathsf{Bord}}
\renewcommand{\hom}{\mathrm{Hom}}
\renewcommand{\emptyset}{\varnothing}
\renewcommand{\O}{\mathcal{O}}
\newcommand{\R}{\mathbb{R}}
\newcommand{\ib}[1]{\textbf{\textit{#1}}}
\newcommand{\Q}{\mathbb{Q}}
\newcommand{\Z}{\mathbb{Z}}
\newcommand{\N}{\mathbb{N}}
\newcommand{\C}{\mathbb{C}}
\newcommand{\A}{\mathbb{A}}
\newcommand{\F}{\mathbb{F}}
\newcommand{\M}{\mathcal{M}}
\newcommand{\dbar}{\overline{\partial}}
\newcommand{\zbar}{\overline{z}}
\renewcommand{\S}{\mathbb{S}}
\newcommand{\V}{\vec{v}}
\newcommand{\RP}{\mathbb{RP}}
\newcommand{\CP}{\mathbb{CP}}
\newcommand{\B}{\mathcal{B}}
\newcommand{\GL}{\mathrm{GL}}
\newcommand{\SL}{\mathrm{SL}}
\newcommand{\SP}{\mathrm{SP}}
\newcommand{\SO}{\mathrm{SO}}
\newcommand{\SU}{\mathrm{SU}}
\newcommand{\gl}{\mathfrak{gl}}
\newcommand{\g}{\mathfrak{g}}
\newcommand{\Bun}{\mathsf{Bun}}
\newcommand{\inv}{^{-1}}
\newcommand{\bra}[2]{ \left[ #1, #2 \right] }
\newcommand{\set}[1]{\left\lbrace #1 \right\rbrace}
\newcommand{\abs}[1]{\left\lvert#1\right\rvert}
\newcommand{\norm}[1]{\left\lVert#1\right\rVert}
\newcommand{\transv}{\mathrel{\text{\tpitchfork}}}
\newcommand{\defeq}{\vcentcolon=}
\newcommand{\enumbreak}{\ \\ \vspace{-\baselineskip}}
\let\oldexists\exists
\renewcommand\exists{\oldexists~}
\let\oldL\L
\renewcommand\L{\mathfrak{L}}
\makeatletter
\newcommand{\incfig}[2]{%
    \fontsize{48pt}{50pt}\selectfont
    \def\svgwidth{\columnwidth}
    \scalebox{#2}{\input{#1.pdf_tex}}
}
%
\newcommand{\tpitchfork}{%
  \vbox{
    \baselineskip\z@skip
    \lineskip-.52ex
    \lineskiplimit\maxdimen
    \m@th
    \ialign{##\crcr\hidewidth\smash{$-$}\hidewidth\crcr$\pitchfork$\crcr}
  }%
}
\makeatother
\newcommand{\bd}{\partial}
\newcommand{\lang}{\begin{picture}(5,7)
\put(1.1,2.5){\rotatebox{45}{\line(1,0){6.0}}}
\put(1.1,2.5){\rotatebox{315}{\line(1,0){6.0}}}
\end{picture}}
\newcommand{\rang}{\begin{picture}(5,7)
\put(.1,2.5){\rotatebox{135}{\line(1,0){6.0}}}
\put(.1,2.5){\rotatebox{225}{\line(1,0){6.0}}}
\end{picture}}
\DeclareMathOperator{\id}{id}
\DeclareMathOperator{\im}{Im}
\DeclareMathOperator{\codim}{codim}
\DeclareMathOperator{\coker}{coker}
\DeclareMathOperator{\supp}{supp}
\DeclareMathOperator{\inter}{Int}
\DeclareMathOperator{\sign}{sign}
\DeclareMathOperator{\sgn}{sgn}
\DeclareMathOperator{\indx}{ind}
\DeclareMathOperator{\alt}{Alt}
\DeclareMathOperator{\Aut}{Aut}
\DeclareMathOperator{\trace}{trace}
\DeclareMathOperator{\ad}{ad}
\DeclareMathOperator{\End}{End}
\DeclareMathOperator{\Ad}{Ad}
\DeclareMathOperator{\Lie}{Lie}
\DeclareMathOperator{\spn}{span}
\DeclareMathOperator{\dv}{div}
\DeclareMathOperator{\grad}{grad}
\DeclareMathOperator{\Sym}{Sym}
\DeclareMathOperator{\sheafhom}{\mathscr{H}\text{\kern -3pt {\calligra\large om}}\,}
\newcommand*\myhrulefill{%
   \leavevmode\leaders\hrule depth-2pt height 2.4pt\hfill\kern0pt}
\newcommand\niceending[1]{%
  \begin{center}%
    \LARGE \myhrulefill \hspace{0.2cm} #1 \hspace{0.2cm} \myhrulefill%
  \end{center}}
\newcommand*\sectionend{\niceending{\decofourleft\decofourright}}
\newcommand*\subsectionend{\niceending{\decosix}}
\def\upint{\mathchoice%
    {\mkern13mu\overline{\vphantom{\intop}\mkern7mu}\mkern-20mu}%
    {\mkern7mu\overline{\vphantom{\intop}\mkern7mu}\mkern-14mu}%
    {\mkern7mu\overline{\vphantom{\intop}\mkern7mu}\mkern-14mu}%
    {\mkern7mu\overline{\vphantom{\intop}\mkern7mu}\mkern-14mu}%
  \int}
\def\lowint{\mkern3mu\underline{\vphantom{\intop}\mkern7mu}\mkern-10mu\int}
%
%--------Hypersetup--------
%
\hypersetup{
    colorlinks,
    citecolor=black,
    filecolor=black,
    linkcolor=blue,
    urlcolor=blacksquare
}
%
%--------Solution--------
%
\newenvironment{solution}
  {\begin{proof}[Solution]}
  {\end{proof}}
%
%--------Graphics--------
%
%\graphicspath{ {images/} }
%
\begin{document}
%
\author{Jeffrey Jiang}
%
\title{$(G,X)$-Structures}
%
\maketitle
%
\section{The Statement of Local Rigidity}
%
Before discussing $(G,X)$-structures, we first state the result we plan on
working towards, which is local rigidity for semisimple Lie groups. Fix a
semisimple lie group $G$ and $\Gamma \subset G$ an irreducible lattice
Heuristically, $\Gamma$ being irreducible means that $\Gamma$ is not locally a product
of lattices. More explicity, it is defined as follows: in the case that $G$ is
simply connected, $G$ being semisimple implies that it is a product
$G = \prod_i G_i$ of simple Lie groups $G_i$. Then $\Gamma \subset G_i$ being
irreducible simply means that it is not a product $\Gamma = \prod_i\Gamma_i$ of
lattices $\Gamma_i \subset G_i$. When $G$ isn't simply connected, $\Gamma$ being
irreducible means that the preimage of $\Gamma$ under the universal covering
$\widetilde{G} \to G$ is irreducible.Then let $\mathcal{R}(\Gamma,G)$ denote the
space of maps $\Gamma \to G$ with the topology of pointwise convergence.
%
\begin{defn}
A lattice $\Gamma$ is \ib{locally rigid} if there is a neighborhood $U$ of
the inclusion map $\Gamma \hookrightarrow G$ in $\mathcal{R}(\Gamma,G)$
such that any other homomorphism $\Gamma \to G$ contained in $U$ is conjugate
to the inclusion.
\end{defn}
%
Our goal will be to prove the following statement (at least for the rank $1$ case)
%
\begin{thm}
Under some conditions on $G$ and $\Gamma$, if $\rho \in \mathcal{R}(\Gamma,G)$ is
sufficiently close to the inclusion $\Gamma \hookrightarrow G$, then
$\rho(\Gamma)$ is a lattice in $G$ and $\rho\vert_\Gamma$ is an isomorphism, i.e.
$\Gamma$ is locally rigid.
\end{thm}
%
The proof of this will use the notion of $(G,X)$ structures, where $G$ is
the semisimple Lie group, and $X$ is a symmetric space.
%
\section{$(G,X)$-Structures}
%
Fix once and for all a manifold $X$ with a transitive action of a Lie group $G$
by real analytic homeomorphisms. The idea of a $(G,X)$ structure is to use $X$
as a ``model geometry," from which we can construct other manifolds that locally
``look like" $X$, and have symmetries that ``look like the action" of $G$.
%
\begin{defn}
A \ib{$(G,X)$-manifold} is a manifold $M$ that admits an open cover
$\set{U_i}$ with charts $\varphi_i : U_i \to X$ (i.e. diffeomorphisms onto their images)
such that the transition functions
$\varphi_i \circ \varphi_j\inv : \varphi_j(U_j \cap U_j) \to \varphi_i(U_i)$
are given by restrictions of the actions of group elements $g_{ij} \in G$. Two such
charts will be called \ib{compatible}. Given a smooth manifold $M$, a $(G,X)$-stucture
on $M$ is a choice of \ib{atlas}, which is a maximal set of compatible charts
$\set{\varphi_i : U_i \to X}$ giving $M$ the structure of a $(G,X)$-manifold.
\end{defn}
%
The discussion of the $(G,X)$-structure on a manifold $M$ involves fiber bundles
and connections on them.
%
\begin{defn}
Let $M$ and $F$ be smooth manifolds. A \ib{fiber bundle} with model fiber $F$ over
$M$ is a smooth manifold $E$ equipped with a submersion $\pi : E \to M$ such that
\begin{enumerate}
  \item The fibers $E_p \defeq \pi\inv(p)$ are diffeomorphic to $F$
  \item For every point $p$, there eixsts a \ib{local trivialization} of $E$,
  i.e. a diffeomorphism $\varphi : \pi\inv(U) \to U \times F$ for a neighborhood
  $U$ of $p$ such that the following diagram commutes:
  \[\begin{tikzcd}
  \pi\inv(U) \ar[rr,"\varphi"] \ar[dr, "\pi"']&& U \times F \ar[dl] \\
  & U
  \end{tikzcd}\]
  where the map $U \times F \to U$ is projection onto the first factor.
\end{enumerate}
We refer to $E$ as the \ib{total space} and $M$ as the \ib{base space}.
\end{defn}
%
For this discussion, we will be concerned with fiber bundles with model fiber $X$,
so ``fiber bundle" without further modifiers will mean ``fiber bundle with model
fiber $X$." Our goal will be to relate the $(G,X)$-structure on a manifold $M$
to a fiber bundle equipped with the data of a connection. Given
any fiber bundle $\pi : E \to M$, we can find an open cover $\set{U_i}$ of $M$ such
that the fiber bundle $E\vert_{U_i} \defeq \pi\inv(U_i)$ admits local trivializations
$\varphi_i : E\vert_{U_i} \to U_i \times X$. If we consider the maps
$\varphi_i \circ \varphi_j\inv$, then we have that they are given in components
by the mapping
\[
(u,x) \mapsto (u, [\psi_{ij}(u)](x))
\]
for \ib{transition functions} $\psi_{ij} : U_i \cap U_j \to \mathrm{Diff}(X)$.
We can recover the original bundle $E \to M$ by gluing these bundles together
using the transition functions $\psi_{ij}$, which is well defined because they
agree in triple intersections and satisfy a cocycle condition. \\

In our situation, we have a distinguished group of symmetries of $X$ coming
from the action of $G$, which can be interpreted as a group
homomorphism $G \to \mathrm{Diff}(X)$, so the fiber bundle $E \to M$ we want to
construct from the $(G,X)$-structure on $M$ should have transition functions
valued in $G$. Furthermore, we will impose the condition
that the transition functions $\psi_{ij}$ are \emph{constant}, so
the second components of the maps $\varphi_i\circ\varphi_j\inv$ are given by the
action of a single group element $g_{ij} \in G$. We also want a notion of compatibility
with the charts of the $(G,X)$-structure. Specifically, we want the domains $U_i$ of
the local trivilizations to be the domains of charts $U_i \to X$, so we must be able to
trivialize the bundle $E$ over the charts in the atlas defining the $(G,X)$-structure.
Given a chart $\kappa : U \to X$ such that $E$ is trivialized
over $U$, i.e. there is a local trivialization
$\varphi_i : E\vert_{U_i} \to U_i \times X$. Then we get a section
$s : U \to E\vert_{U_i}$, where $(\varphi_i \circ s)(p) = (p, \kappa(p))$, which
embeds $U$ in $E\vert_{U_i}$, and is given as the graph of $\kappa$ under this
choice of trivialization. Our next restriction is to require that
two compatible charts $\kappa : U \to X$ and $\kappa' : U' \to X$ with intersecting
domains have sections $s : U \to E\vert_U$ and $s' : U \to E\vert_{U'}$
that agree on $U \cap U'$. Putting everything together, given a $(G,X)$-manifold
$M$, we can construct a fiber bundle $E \to M$ where a point of $E$ is an equivalence
class $[p,\kappa, x]$ where $p$ is a point of $M$, $\kappa : U \to X$ is a chart
containing $p$, and $x$ is a point of $X$, where $[p,\kappa, x] = [p', \kappa', x']$
if $p = p'$ and there exists a group element $g \in G$ such that
$\kappa = g \circ \kappa'$ and $x' = gx$ on some neighborhood containing $p$. \\

For those in the know, the requirement that the transition functions are constant
suggests that the fiber bundles we constructed is analogous to a \ib{local system},
which has an identification with vector bundles equipped with a flat connection.
A similar story will hold here. To discuss this, we will use Ehresmann's notion
of a connection on a fiber bundle, which generalizes the notion for vector
bundles to general fiber bundles. \\

Let $\pi : E \to M$ be a fiber bundle. This determines a rank $n$
(where $n \defeq \dim X$) distribution $V \subset TE$ (i.e. a rank $n$ vector subbundle
of $TE \to E$.) called the \ib{vertical distribution}, where the fiber $V_e$
over $e \in E$ is $\ker d\pi_e$. Geometrically, one can identify
$V_e$ as the tangent vectors in $T_eE$ that are tangent to the fiber
$E_{\pi(e)}$.
%
\begin{defn}
A \ib{connection} on a fiber bundle $E \to M$ is a choice of complementary
subbundle to $V$, i.e. a distribution $H \subset TE$ such that
$TP = V \oplus H$. The distribution $H$ is also referred to as the
\ib{horizontal distribution}.
\end{defn}
%
The horizontal distribution should be thought of as distinguished
``manifold directions" on $E$, which is complementary to the ``fiber directions"
specified by the vertical distribution. By a dimension count, we get
that $d\pi_e\vert_{H_e} : H_e \to T_{\pi(e)}M$ is an isomorphism, so given
$e \in E$, every tangent vector $v \in T_{\pi(e)}M$ has a unique \ib{horizontal lift}
to a tangent vector $\tilde{v} \in H_e \subset T_eE$. This the allows
us to uniquely lift paths on $M$ going through $\pi(e)$ to a path passing through $e$
by taking horizontal lifts of velocity vectors, giving us a vector field, and then
taking the integral curve passing through $e$. Using path lifting, we can then
define the \ib{holonomy} of a connection. Fix a point $p \in M$, and let $\gamma$
be a loop based at $p$. Then for a point $e \in E_p$, we can lift $\gamma$
to a path $\widetilde{\gamma}$ based starting at $e$, whose endpoint
is another point $e' \in E_p$. Doing this for all points $e \in E_p$, we obtain
a diffeomorphism $E_p \to E_p$ (for the inverse, do the same construction with
the reversed loop) called the \ib{holonomy} of loop $\gamma$. \\

We then want to restrict our attention to a specific subset of connections
that are compatible with the action of $G$ on $X$.
%
\begin{defn}
Let $M$ be a smooth manifold (not necessarily with a $(G,X)$-structure).
A \ib{$(G,X)$-connection} on $M$ is the data of
\begin{enumerate}
  \item A fiber bundle $E \to M$ with model fiber $X$.
  \item A section $s : M \to E$.
  \item A connection on $E$ such that the holonomy about any closed curve
  $\gamma$ is given by the action of an element of $G$.
\end{enumerate}
A $(G,X)$-connection is \ib{flat} if the horizontal distribution is
integrable (e.g. closed under Lie brackets), equivalently, if any contractible
closed curve has trivial holonomy. For convenience, we might say ``let $E \to M$
be a $(G,X)$-connection", and leave the section and connection implicit.
\end{defn}
%
From our definition, we conveniently get the following:
%
\begin{prop}
Let $M$ be a $(G,X)$-manifold, and $E \to M$ the fiber bundle we constructed
above using the $(G,X)$-structure. Then $E$ can be equipped with a connection
that gives it the structure of a flat $(G,X)$-connection.
\end{prop}
%
\begin{proof}
In the construction we made earlier, we got a distinguished section
$s : M \to E$ given locally by taking charts $\kappa : U \to X$, and
embedding $U \to E$ by taking the graph of $\kappa$ with respect to a local
trivialization of $E$ over $U$. Explicitly, using the description of points
of $E$ as equivalence classes of the form $[m,\kappa, x]$ with $m \in M$,
$\kappa : U \to X$ a chart, and $x \in X$, the section $s$ is given by
\[
s(m) = [m,\kappa, \kappa(m)]
\]
for any choice of chart $\kappa$ about $m$. Furthermore, we have a foliation
of $E$ by leaves diffeomorphic to $M$, where a leaf of the foliation is
given by the points $\set{[m,\kappa,x] ~:~ m \in M}$. This gives rise to
a horizontal distribution $H$ by taking the space $H_e$ at a point $e \in E$
to be the tangent space of the leaf passing through $e$, and $H$ is integrable
since it admits integral submanifolds, namely the leaves of the foliation. Therefore,
$H$ is a flat connection. The final thing we need to check is that the holonomy
is given by the action of $G$, and this follows from the equivalence relation
we used to define $E$, namely identifying points that were related by the action of $G$.
Having identified all of the pieces, this gives $E$ the structure of a
flat $(G,X)$-connection.
\end{proof}
%
The punchline is that the converse is also true!
%
\begin{prop}
Let $M$ be manifold equipped with a flat $(G,X)$-connection $E \to M$. Then
$M$ can be equipped with the structure of a $(G,X)$-manifold such that
$E \to M$ is the bundle coming from the construction described above.
\end{prop}
%
\begin{proof}
The proof of this should be familiar to those familiar with principal bundles
and local systems. Given a flat $(G,X)$-connection $E \to M$, we have that
the holonomy of a loops $\gamma$ is determined by its homotopy class, since
nullhomotopic maps have trivial holonomy. Therefore, the map taking
a homotopy class $[\gamma] \in \pi_1(M, m_0)$ its holonomy $g \in G$ defines
a group homomorphism $\pi_1(M,m_0) \to G$, which then determines
an action of $\pi_1(M,m_0)$ on $X$. Then if we let $\widetilde{M}$ denote
the universal cover of $M$, we can take the associated bundle
\[
\widetilde{M} \times_{\pi_1(M)} X = (\widetilde{M} \times X)/\pi_1(M)
\]
where the action on the product is the diagonal action. The associated
bundle is a fiber bundle with model fiber $X$. Furthermore, it comes equipped
with a natural connection, where the horizontal distribution is obtained
by taking the image of $T\widetilde{M} \subset T(\widetilde{M} \times X)$ under the
differential of the quotient map
$\widetilde{M}\times X \to \widetilde{M} \times_{\pi_1(M)} X$. Furthermore, this
connection is easily seen to be flat, since
$T\widetilde{M} \subset T(\widetilde{M} \times X)$ is integrable,
integrability of a distribution is a local property, and the fact that the quotient
map is a covering map. The holonomy of this connection then corresponds exactly with
the original homomorphism $\pi_1(M) \to G$, up to conjugation on $G$.
Finally, one can show that the bundle $\widetilde{M} \times_{\pi_1(M)} X$ with the
connection defined above is isomorphic to the $(G,X)$-connection $E \to M$,
which allows us to identify a section $s : M \to \widetilde{M} \times_{\pi_1(M)} X$
corresponding to the section $M \to E$. From this, we want to recover an atlas of
charts on $M$, which we can do by taking an evenly covered
neighborhood of a point $m_0 \in M$, which necessarily trivializes
the associated bundle $\widetilde{M} \times_{\pi_1(M)} X$, and then
taking the second component of the section $s$ with repect to this trivialization
to obtain a chart about $m_0$.
\end{proof}
%
\end{document}